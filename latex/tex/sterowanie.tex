\section{Algorytmy sterowania}

W~niniejszym rozdziale przedstawiono szczegółowy opis algorytmów sterowania
zaimplementowanych i~przeanalizowanych w~ramach pracy. Kod sterowników został
zrealizowany w~języku Python w~postaci klas dziedziczących wspólną strukturę,
co zapewnia modularność i~łatwą wymienność w~pętli symulacyjnej. Każdy
regulator wyznacza sygnał sterujący $u(t)$ (siłę przyłożoną do wózka)
na~podstawie aktualnego wektora stanu
$x(t) = [\theta, \dot{\theta}, x, \dot{x}]^T$ oraz wartości zadanych
$x_{\mathrm{ref}}$.

W~literaturze problem sterowania wahadłem odwróconym jest szeroko omawiany
jako klasyczny problem testowy dla metod sterowania liniowego i~nieliniowego
\cite{art1, art3}. Poniżej opisano teoretyczne podstawy oraz szczegóły
implementacyjne zbadanych struktur sterowania.

\subsection{Równoległy regulator PD-PD}

Pierwszym zaimplementowanym układem jest regulator o~strukturze kaskadowej lub
równoległej, wykorzystujący klasyczne sprzężenie zwrotne typu PD
(Proporcjonalno-Różniczkujące). W~literaturze podejście to jest często
stosowane jako punkt odniesienia dla bardziej zaawansowanych metod
\cite{moreno2023, art1}.

W~klasie \texttt{PDPDController} zastosowano strukturę równoległą, w~której
całkowity sygnał sterujący jest sumą reakcji na~błąd kąta oraz błąd pozycji.
Jest to podejście intuicyjne, dekomponujące problem na~dwa podzadania:
stabilizację wahadła w~pozycji pionowej oraz doprowadzenie wózka do~zadanej
pozycji.

Prawo sterowania wyraża się wzorem:
\begin{equation}
    u(t) = \mathrm{sat}_{u_{\mathrm{max}}} \left( u_{\theta}(t) + u_{x}(t) \right),
\end{equation}
gdzie funkcja nasycenia $\mathrm{sat}(\cdot)$ wynika z~ograniczeń fizycznych
Definiując uchyby regulacji jako $e_{\theta}(t) = \theta_{\mathrm{ref}} - \theta(t)$ oraz $e_x(t) = x_{\mathrm{ref}} - x(t)$, prawo sterowania dla poszczególnych pętli można zapisać w~ogólnej postaci regulatora PID:
\begin{align}
    u_{\theta}(t) &= K_{p,\theta} e_{\theta}(t) + K_{i,\theta} \int_0^t e_{\theta}(\tau)\,d\tau + K_{d,\theta} \frac{d e_{\theta}(t)}{dt}, \label{eq:pd_theta} \\
    u_{x}(t) &= K_{p,x} e_x(t) + K_{i,x} \int_0^t e_x(\tau)\,d\tau + K_{d,x} \frac{d e_x(t)}{dt}. \label{eq:pd_x}
\end{align}
W~powyższych równaniach przyjęto upraszczające założenie, że docelowe
prędkości ($\dot{\theta}_{\mathrm{ref}}, \dot{x}_{\mathrm{ref}}$) wynoszą zero.

W~implementacji programowej (plik \texttt{pd\_pd.py}) przyjęto następujące
nastawy dobrane eksperymentalnie:
\begin{itemize}
    \item Tor stabilizacji kąta: $K_{p,\theta} = -95.0$, $K_{d,\theta} = -14.0$.
    Ujemne znaki wynikają z~przyjętej konwencji układu współrzędnych i~zwrotu siły.
    \item Tor pozycji: $K_{p,x} = -16.0$, $K_{d,x} = -14.0$.
\end{itemize}
Mimo iż klasa umożliwia włączenie członu całkującego (PID), w~badaniach
\cite{art2} często wskazuje się, że dla obiektów tej klasy człon różniczkujący
(PD) jest kluczowy dla tłumienia oscylacji, a~całkowanie może wprowadzać
niestabilność w~stanach nieustalonych bez odpowiednich mechanizmów
$\mathrm{anti\text{-}windup}$.

\subsubsection{Dobór nastaw regulatora PD-PD}
Dobór nastaw przeprowadzono metodą eksperymentalną, kierując się zasadą dekompozycji problemu. W~pierwszej kolejności nastrojono pętlę stabilizacji wahadła ($\theta$), traktując wózek jako swobodny, a~następnie pętlę pozycji ($x$) jako nadrzędną, o~znacznie wolniejszej dynamice.
Kluczowym aspektem badań, który zostanie przeprowadzony w~dalszej części pracy, jest weryfikacja wpływu członów całkujących ($K_i$) na~uchyb ustalony oraz stabilność układu.

\begin{itemize}
    \item \textbf{Eksperyment 1:} Porównanie odpowiedzi skokowej dla $K_{i,\theta} = 0$ oraz $K_{i,\theta} \neq 0$.
    \item \textbf{Cel:} Sprawdzenie, czy wprowadzenie akcji całkującej eliminuje uchyb ustalony bez wprowadzania cykli granicznych.
\end{itemize}

TODO: Wstawić tutaj wykresy porównawcze z eksperymentu oraz tabelę jakości regulacji.

\subsection{Układ hybrydowy PD-LQR}

Kolejnym analizowanym algorytmem jest regulator liniowo-kwadratowy (LQR),
będący standardem w~sterowaniu optymalnym systemów wielowymiarowych MIMO
\cite{jezierski2017}. Klasa \texttt{PDLQRController} implementuje sterowanie
oparte na~pełnym wektorze stanu, wspomagane dodatkowym członem PD dla uchybu
pozycji, co tworzy strukturę hybrydową opisaną m.in. w~\cite{art1} oraz
\cite{art3} (w~kontekście porównawczym).

Problem LQR polega na~znalezieniu prawa sterowania
$u(t) = -K x(t)$, które minimalizuje wskaźnik jakości:
\begin{equation}
    J = \int_{0}^{\infty} \left( x(t)^T Q x(t) + u(t)^T R u(t) \right) dt,
\end{equation}
gdzie $Q \succeq 0$ jest macierzą wag stanu, a~$R > 0$ wagą
sterowania. Optymalna macierz wzmocnień $K$ wyznaczana jest poprzez
rozwiązanie algebraicznego równania Riccatiego (CARE):
\begin{equation}
    A^T P + P A - P B R^{-1} B^T P + Q = 0,
\end{equation}
skąd $K = R^{-1} B^T P$. Macierze $A$
i~$B$ pochodzą z~linearyzacji modelu wahadła wokół punktu równowagi
górnej ($\theta = 0$).

W~zaimplementowanym rozwiązaniu (plik \texttt{pd\_lqr.py}), sygnał sterujący
składa się z~dwóch komponentów:
\begin{equation}
    u(t) = u_{\mathrm{LQR}}(t) + u_{\mathrm{PD,pos}}(t).
\end{equation}
Składnik LQR realizuje stabilizację wokół punktu pracy:
\begin{equation}
    u_{\mathrm{LQR}}(t) = -K \cdot (x(t) - x_{\mathrm{ref}}).
\end{equation}
Zastosowane wagi optymale to:
\begin{equation}
    Q = \text{diag}([69.44,\; 76.98,\; 17.70,\; 14.17]), \quad R = 8.028.
\end{equation}
Dodatkowy człon PD na~pętli pozycji (zrealizowany analogicznie do wzoru
\ref{eq:pd_x}) ma na~celu poprawę śledzenia skokowych zmian wartości zadanej
$x_{\mathrm{ref}}$, co jest częstą praktyką w~aplikacjach praktycznych, gdzie LQR
zapewnia stabilność, a~regulator zewnętrzny dba o~uchyb w~stanie ustalonym
\cite{art2}.

\subsubsection{Dobór wag macierzy Q i R}
Podstawowym wyzwaniem w~projektowaniu regulatora LQR jest odpowiedni dobór macierzy wag $\mathbf{Q}$ i~$R$, które determinują kompromis między szybkością redukcji uchybu a~energią sterowania. W~pracy przyjęto podejście polegające na~iteracyjnym zwiększaniu kar za~odchylenie kąta, przy zachowaniu stałej kary za~pozycję.

\begin{itemize}
    \item \textbf{Eksperyment 2:} Analiza wpływu stosunku $Q_{\theta}/R$ na~czas regulacji.
    \item \textbf{Cel:} Wyznaczenie granicy stabilności dla bardzo agresywnych nastaw LQR.
\end{itemize}

TODO: Dodać tabelę przedstawiającą czasy regulacji i przesterowania dla różnych zestawów macierzy Q.

\subsection{Nieliniowe sterowanie predykcyjne (MPC)}

Algorytm MPC (Model Predictive Control) stanowi zaawansowaną metodę sterowania,
która w~odróżnieniu od~LQR, uwzględnia wprost ograniczenia sygnału sterującego
oraz nieliniową dynamikę obiektu \cite{camacho2007, rawlings2017}.
Zaimplementowany w~klasie \texttt{MPCController} (plik \texttt{mpc.py})
algorytm rozwiązuje w~każdym kroku symulacji problem optymalizacji dynamicznej
nieliniowej (NMPC).

Zadanie optymalizacji, rozwiązywane numerycznie metodą SQP (Sequential
Quadratic Programming) przy użyciu solwera \texttt{SLSQP}, zdefiniowane jest
następująco:
\begin{equation}
    \min_{\Delta U} J = \sum_{k=1}^{N_p} (\hat{x}_k - x_{\mathrm{ref}})^T Q (\hat{x}_k - x_{\mathrm{ref}}) + R \sum_{k=0}^{N_c-1} (\Delta u_k)^2,
\end{equation}
przy ograniczeniach:
\begin{align}
    \hat{x}_{k+1} &= f(\hat{x}_k, u_k), \quad k=0,\dots,N_p-1 \\
    u_{\mathrm{min}} &\le u_k \le u_{\mathrm{max}}, \\
    u_k &= u_{k-1} + \Delta u_k.
\end{align}
Gdzie:
\begin{itemize}
    \item $N_p = 12$ -- horyzont predykcji,
    \item $N_c = 4$ -- horyzont sterowania (blokowanie sterowania dla $k \ge N_c$),
    \item $f(\cdot)$ -- nieliniowy model dyskretny obiektu (całkowanie metodą
    Rungego-Kutt 4. rzędu),
    \item $Q = \text{diag}([158.4,\; 36.8,\; 43.4,\; 19.7])$ -- macierz kar stanu,
    \item $R = 0.086$ -- współczynnik kary za~zmianę sterowania ($\Delta u$).
\end{itemize}
Kluczową zaletę MPC, podkreślaną w~pracach \cite{mills2009} oraz
\cite{jezierski2017}, jest możliwość bezpośredniego uwzględnienia ograniczeń
(saturacji) już na~etapie wyliczania sterowania, co zapobiega zjawisku
,,obcinania'' sygnału sterującego i~pogorszenia jakości regulacji, co może
mieć miejsce w~przypadku LQR.

\subsubsection{Dobór horyzontu i wag funkcji celu}
Wydajność regulatora MPC jest ściśle powiązana z~długością horyzontu predykcji $N_p$ oraz horyzontu sterowania $N_c$. Zbyt krótki horyzont może prowadzić do~niestabilności (brak widoczności ,,przyszłych'' ograniczeń), natomiast zbyt długi drastycznie zwiększa koszt obliczeniowy.

\begin{itemize}
    \item \textbf{Eksperyment 3:} Wpływ długości horyzontu $N_p \in \{5, 10, 15, 20\}$ na~jakość sterowania.
    \item \textbf{Cel:} Znalezienie minimalnego horyzontu zapewniającego stabilność przy ograniczonej mocy obliczeniowej.
\end{itemize}

\noindent \textit{TODO: Wykres wpływu długości horyzontu na średni czas wykonania kroku symulacji.}

\subsection{MPC z~rozszerzonym wskaźnikiem jakości (MPC-J2)}

Jako wariant badawczy algorytmu predykcyjnego zaimplementowano sterownik
\texttt{MPCControllerJ2} (plik \texttt{mpc\_J2.py}). Jego struktura jest
zbliżona do~podstawowego MPC, jednak funkcja kosztu została rozbudowana
o~dodatkowy składnik karzący bezwzględną wartość sygnału sterującego
(energię), a~nie tylko jego przyrosty.

Zmodyfikowana funkcja celu przyjmuje postać:
\begin{equation}
    J = \sum_{k=1}^{N_p} (x_k - x_{\mathrm{ref}})^T Q (x_k - x_{\mathrm{ref}}) \;+\; R_{\Delta} \sum_{k=0}^{N_c-1} (\Delta u_k)^2 \;+\; R_{\mathrm{abs}} \sum_{k=0}^{N_c-1} (u_k)^2.
\end{equation}
Wprowadzenie parametru $R_{\mathrm{abs}}$ pozwala na~bezpośrednie minimalizowanie
zużycia energii sterowania, co może być istotne w~zastosowaniach mobilnych
zasilanych bateryjnie. W~badaniach przyjęto wagi: $q_{\theta}=80.0$,
$q_x=120.0$ (elementy macierzy diagonalnej $Q$), kładąc większy
nacisk na~precyzję pozycjonowania wózka w~porównaniu do~standardowego MPC.

\subsubsection{Analiza wpływu kary za energię}
Dodatkowy składnik funkcji kosztu $R_{\mathrm{abs}}$ ma na~celu minimalizację zużycia energii, co jest kluczowe w~systemach autonomicznych.

\begin{itemize}
    \item \textbf{Eksperyment 4:} Porównanie całkowitego zużycia energii $\int u^2 dt$ dla różnych wartości $R_{\mathrm{abs}}$.
    \item \textbf{Cel:} Ilościowa ocena oszczędności energetycznych w~porównaniu do~klasycznego MPC.
\end{itemize}

\noindent \textit{TODO: Wykres słupkowy zużycia energii w funkcji parametru R\_abs.}

\subsection{Regulator rozmyty wspomagany LQR (Fuzzy-LQR)}

Ostatnim zbadanym układem jest sterownik hybrydowy \texttt{TSFuzzyController}
(plik \texttt{fuzzy\_lqr.py}), łączący liniowy regulator LQR z~systemem
wnioskowania rozmytego typu Takagi-Sugeno (T-S). Koncepcja ta, opisana szerzej
w~\cite{art3} oraz \cite{roose2017}, ma na~celu adaptację wzmocnień regulatora
w~zależności od~punktu pracy, co pozwala na~agresywniejszą reakcję w~przypadku
dużych odchyleń od~pionu.

Sygnał sterujący jest sumą:
\begin{equation}
    u(t) = u_{\mathrm{LQR}}(t) + u_{\mathrm{Fuzzy}}(t).
\end{equation}
Część rozmyta $u_{\mathrm{Fuzzy}}(t)$ wykorzystuje bazę reguł postaci:
\begin{quote}
    JEŚLI $e_\theta$ jest $A_i$ ORAZ $\dot{\theta}$ jest $B_i$ ... TO $u_i = f_i(x)$,
\end{quote}
gdzie $f_i(x)$ jest liniową funkcją stanu (lokalny regulator
liniowy). Zastosowano funkcje przynależności trójkątne dla zmiennych stanu,
dzieląc przestrzeń na~obszary ,,Mały błąd'' i~,,Duży błąd''.
Baza wiedzy składa się z~16 reguł ($2^4$ kombinacji dla 4 zmiennych stanu).
Wyjście sterownika obliczane jest jako średnia ważona:
\begin{equation}
    u_{\mathrm{Fuzzy}} = G \cdot \frac{\sum_{i=1}^{16} w_i(x) \cdot u_i}{\sum_{i=1}^{16} w_i(x)},
\end{equation}
gdzie $w_i$ to stopień aktywacji $i$-tej reguły, a~$G = 0.9$ to globalne
wzmocnienie skalujące.

Zastosowany mechanizm ,,Gain Scheduling'' pozwala na:
\begin{enumerate}
    \item Zachowanie łagodnej charakterystyki LQR w~pobliżu punktu równowagi
    (małe wzmocnienia w~regułach dla ,,Małych błędów'').
    \item Zwiększenie sztywności układu w~sytuacjach krytycznych (duże
    wzmocnienia zdefiniowane w~zmiennej \texttt{F\_rules} dla ,,Dużych
    błędów'').
\end{enumerate}
Takie podejście pozwala na~rozszerzenie obszaru stabilności regulatora
w~porównaniu do~klasycznego LQR, co potwierdzają wyniki badań w~pracy
\cite{art3}.

\subsubsection{Dobór reguł i funkcji przynależności}
Struktura sterownika rozmytego została zaczerpnięta z~pracy \cite{art3}, jednak kluczowym etapem implementacji była adaptacja parametrów funkcji przynależności (trójkątnych) oraz macierzy reguł do~dynamiki konkretnego modelu wahadła.

Zastosowano podział przestrzeni błędów na~strefy:
\begin{itemize}
    \item \textit{Mały błąd}: Sterowanie łagodne, zbliżone do~LQR.
    \item \textit{Duży błąd}: Sterowanie agresywne (Gain Scheduling), mające na~celu szybki powrót do~punktu pracy.
\end{itemize}

\begin{itemize}
    \item \textbf{Eksperyment 5:} Weryfikacja skuteczności przełączania reguł w~obecności silnych zakłóceń zewnętrznych.
    \item \textbf{Cel:} Pokazanie przewagi Fuzzy-LQR nad klasycznym LQR w~sytuacjach krytycznych.
\end{itemize}

\noindent \textit{TODO: Wykres trajektorii fazowej dla Fuzzy-LQR i LQR przy dużym wychyleniu początkowym.}
