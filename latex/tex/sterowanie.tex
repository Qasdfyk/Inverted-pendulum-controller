\section{Algorytmy sterowania}

W~niniejszym rozdziale przedstawiono szczegółowy opis algorytmów sterowania
zaimplementowanych i~przeanalizowanych w~ramach pracy. Kod sterowników został
zrealizowany w~języku Python w~postaci klas dziedziczących wspólną strukturę,
co zapewnia modularność i~łatwą wymienność w~pętli symulacyjnej. Każdy
regulator wyznacza sygnał sterujący $u(t)$ (siłę przyłożoną do wózka)
na~podstawie aktualnego wektora stanu
$x(t) = [\theta, \dot{\theta}, x, \dot{x}]^T$ oraz wartości zadanych
$x_{\mathrm{ref}}$.

W~literaturze problem sterowania wahadłem odwróconym jest szeroko omawiany
jako klasyczny problem testowy dla metod sterowania liniowego i~nieliniowego
\cite{Prasad2014, Nguyen2024}. Poniżej opisano teoretyczne podstawy oraz szczegóły
implementacyjne zbadanych struktur sterowania.

\subsection{Równoległy regulator PD-PD}

Pierwszym zaimplementowanym układem jest regulator o~strukturze kaskadowej lub
równoległej, wykorzystujący klasyczne sprzężenie zwrotne typu PD
(Proporcjonalno-Różniczkujące). W~literaturze podejście to jest często
stosowane jako punkt odniesienia dla bardziej zaawansowanych metod
\cite{Moreno2023, Prasad2014}.

W~klasie \texttt{PDPDController} zastosowano strukturę równoległą, w~której
całkowity sygnał sterujący jest sumą reakcji na~błąd kąta oraz błąd pozycji.
Jest to podejście intuicyjne, dekomponujące problem na~dwa podzadania:
stabilizację wahadła w~pozycji pionowej oraz doprowadzenie wózka do~zadanej
pozycji.

Prawo sterowania wyraża się wzorem:
\begin{equation}
    u(t) = \mathrm{sat}_{u_{\mathrm{max}}} \left( u_{\theta}(t) + u_{x}(t) \right),
\end{equation}
gdzie funkcja nasycenia $\mathrm{sat}(\cdot)$ wynika z~ograniczeń fizycznych
Definiując uchyby regulacji jako $e_{\theta}(t) = \theta_{\mathrm{ref}} - \theta(t)$ oraz $e_x(t) = x_{\mathrm{ref}} - x(t)$, prawo sterowania dla poszczególnych pętli można zapisać w~ogólnej postaci regulatora PID:
\begin{align}
    u_{\theta}(t) &= K_{p,\theta} e_{\theta}(t) + K_{i,\theta} \int_0^t e_{\theta}(\tau)\,d\tau + K_{d,\theta} \frac{d e_{\theta}(t)}{dt}, \label{eq:pd_theta} \\
    u_{x}(t) &= K_{p,x} e_x(t) + K_{i,x} \int_0^t e_x(\tau)\,d\tau + K_{d,x} \frac{d e_x(t)}{dt}. \label{eq:pd_x}
\end{align}
W~powyższych równaniach przyjęto upraszczające założenie, że docelowe
prędkości ($\dot{\theta}_{\mathrm{ref}}, \dot{x}_{\mathrm{ref}}$) wynoszą zero.

W~implementacji programowej (plik \texttt{pd\_pd.py}) przyjęto następujące
nastawy dobrane eksperymentalnie:
\begin{itemize}
    \item Tor stabilizacji kąta: $K_{p,\theta} = -95.0$, $K_{d,\theta} = -14.0$.
    Ujemne znaki wynikają z~przyjętej konwencji układu współrzędnych i~zwrotu siły.
    \item Tor pozycji: $K_{p,x} = -16.0$, $K_{d,x} = -14.0$.
\end{itemize}
Mimo iż klasa umożliwia włączenie członu całkującego (PID), w~badaniach
\cite{Varghese2017} często wskazuje się, że dla obiektów tej klasy człon różniczkujący
(PD) jest kluczowy dla tłumienia oscylacji, a~całkowanie może wprowadzać
niestabilność w~stanach nieustalonych bez odpowiednich mechanizmów
$\mathrm{anti\text{-}windup}$.

\subsubsection{Proces doboru nastaw oraz analiza PID}
Dobór nastaw dla regulatora PD-PD został zrealizowany wieloetapowo, ewoluując
od metod heurystycznych do pełnej optymalizacji numerycznej. Wstępne próby
doboru metodą ,,prób i~błędów'', oparte na~dekompozycji problemu (najpierw
stabilizacja wahadła, potem pozycja wózka), pozwoliły uzyskać stabilność,
jednak jakość regulacji była niezadowalająca. Układ charakteryzował się
powolnym dochodzeniem do~punktu pracy i~znacznymi oscylacjami
(\ref{fig:pdpd_manual}).

Aby wyeliminować subiektywność strojenia ręcznego, zastosowano algorytm
Ewolucji Różnicowej (Differential Evolution), zaimplementowany w~module
\texttt{scipy.optimize}. Zdefiniowano globalną funkcję kosztu $J$, która
gwarantuje ,,uczciwe'' porównanie wszystkich badanych regulatorów:
\begin{equation}
    J = w_{\theta} \cdot \text{MSE}(\theta) + w_{x} \cdot \text{MSE}(x) + w_{u} \cdot \text{RMS}(u),
\end{equation}
gdzie przyjęto wagi $w_{\theta}=4.0$ (priorytet stabilizacji), $w_{x}=1.0$
(dokładność pozycjonowania) oraz $w_{u}=0.01$ (koszt energii). Algorytm operował
na populacji 10 osobników przez 20 generacji, co pozwoliło uniknąć minimów
lokalnych i~znaleźć optymalny zestaw wzmocnień (\ref{fig:pdpd_opt}).

\paragraph{Analiza porównawcza struktur PD i PID}
W~literaturze przedmiotu \cite{Varghese2017, Nguyen2024} często podkreśla się, że dla
obiektów niestabilnych statycznie, takich jak wahadło odwrócone, kluczowa jest
szybka reakcja na~zmiany kąta, którą zapewnia człon różniczkujący (D).
Włączenie członu całkującego (I), tworzącego regulator PID, wprowadza dodatkowe
przesunięcie fazowe (opóźnienie), co w~układzie o~dynamice astatycznej
i~nieliniowej prowadzi do znacznego pogorszenia zapasu stabilności.

Przeprowadzone badania symulacyjne potwierdziły te wnioski. Próba dodania akcji
całkującej ($K_i \neq 0$) skutkowała zjawiskiem \textit{integral wind-up} --
akumulacją błędu w~fazie rozruchu, co powodowało przeregulowania wykraczające
poza obszar przyciągania stabilnego punktu równowagi. Jak widać na~rysunku
\ref{fig:pid_bad}, regulator PID wpada w~niegasnące oscylacje lub doprowadza
do przewrócenia wahadła, podczas gdy ,,czysty'' regulator PD zapewnia sztywne
i~szybkie sterowanie.

\begin{figure}[H]
    \centering
    \includegraphics[width=1.0\textwidth]{images/tuning/pid_1_integral_bad.png}
    \caption{Destabilizujący wpływ członu całkującego (PID) - widoczne narastające oscylacje i utrata stabilności.}
    \label{fig:pid_bad}
\end{figure}

\begin{figure}[H]
    \centering
    \includegraphics[width=1.0\textwidth]{images/tuning/pdpd_2_manual.png}
    \caption{Stabilna, lecz oscylacyjna praca regulatora PD-PD przy strojeniu ręcznym.}
    \label{fig:pdpd_manual}
\end{figure}

\begin{figure}[H]
    \centering
    \includegraphics[width=1.0\textwidth]{images/tuning/pdpd_3_opt.png}
    \caption{Przebiegi czasowe dla zoptymalizowanych nastaw regulatora PD-PD (algorytm Differential Evolution).}
    \label{fig:pdpd_opt}
\end{figure}

\subsection{Układ hybrydowy PD-LQR}

Kolejnym analizowanym algorytmem jest regulator liniowo-kwadratowy (LQR),
będący standardem w~sterowaniu optymalnym systemów wielowymiarowych MIMO
\cite{Jezierski2017}. Klasa \texttt{PDLQRController} implementuje sterowanie
oparte na~pełnym wektorze stanu, wspomagane dodatkowym członem PD dla uchybu
pozycji, co tworzy strukturę hybrydową opisaną m.in. w~\cite{Prasad2014} oraz
\cite{Nguyen2024} (w~kontekście porównawczym).

Problem LQR polega na~znalezieniu prawa sterowania
$u(t) = -K x(t)$, które minimalizuje wskaźnik jakości:
\begin{equation}
    J = \int_{0}^{\infty} \left( x(t)^T Q x(t) + u(t)^T R u(t) \right) dt,
\end{equation}
gdzie $Q \succeq 0$ jest macierzą wag stanu, a~$R > 0$ wagą
sterowania. Optymalna macierz wzmocnień $K$ wyznaczana jest poprzez
rozwiązanie algebraicznego równania Riccatiego (CARE):
\begin{equation}
    A^T P + P A - P B R^{-1} B^T P + Q = 0,
\end{equation}
skąd $K = R^{-1} B^T P$. Macierze $A$
i~$B$ pochodzą z~linearyzacji modelu wahadła wokół punktu równowagi
górnej ($\theta = 0$).

W~zaimplementowanym rozwiązaniu (plik \texttt{pd\_lqr.py}), sygnał sterujący
składa się z~dwóch komponentów:
\begin{equation}
    u(t) = u_{\mathrm{LQR}}(t) + u_{\mathrm{PD,pos}}(t).
\end{equation}
Składnik LQR realizuje stabilizację wokół punktu pracy:
\begin{equation}
    u_{\mathrm{LQR}}(t) = -K \cdot (x(t) - x_{\mathrm{ref}}).
\end{equation}
Zastosowane wagi optymalne to:
\begin{equation}
    Q = \text{diag}([1.0,\; 1.0,\; 500.0,\; 250.0]), \quad R = 1.0.
\end{equation}
Dodatkowy człon PD na~pętli pozycji (zrealizowany analogicznie do wzoru
\ref{eq:pd_x}) ma na~celu poprawę śledzenia skokowych zmian wartości zadanej
$x_{\mathrm{ref}}$, co jest częstą praktyką w~aplikacjach praktycznych, gdzie LQR
zapewnia stabilność, a~regulator zewnętrzny dba o~uchyb w~stanie ustalonym
\cite{Varghese2017}.

\subsubsection{Dobór wag macierzy Q i R}
Dobór wag dla regulatora LQR również charakteryzował się ewolucyjnym podejściem
do~problemu optymalizacji wskaźnika jakości.

W~pierwszej fazie przyjęcie jednostkowej macierzy diagonalnej $Q=I$ oraz $R=1$
okazało się niewystarczające. Mimo teoretycznej stabilności wynikającej z~rozwiązania
równania CARE, wahadło wykonywało bardzo duże wychylenia, a~wózek wielokrotnie
wyjeżdżał poza dopuszczalny zakres roboczy toru. Świadczyło to o~zbyt małej karze
nałożonej na~uchyb kątowy.

\begin{figure}[H]
    \centering
    \includegraphics[width=1.0\textwidth]{images/tuning/pdlqr_1_bad.png}
    \caption{Próba sterowania LQR z wagami jednostkowymi ($Q=I, R=1$). Widoczna duża bezwładność układu.}
    \label{fig:lqr_bad}
\end{figure}

Następnie przeprowadzono strojenie ręczne metodą prób i~błędów (zgodnie z~regułą
Brysona). Ręczne zwiększanie kar za~wychylenie kąta ($Q_{\theta}$) poprawiło
sztywność wahadła. Udało się ustalić zestaw wag zapewniający stabilną pracę, choć
czas regulacji był wciąż niezadowalający, a~reakcja na~zakłócenia powolna.

\begin{figure}[H]
    \centering
    \includegraphics[width=1.0\textwidth]{images/tuning/pdlqr_2_manual.png}
    \caption{Wyniki strojenia ręcznego LQR metodą Brysona.}
    \label{fig:lqr_manual}
\end{figure}

W~ostatnim etapie zastosowano optymalizację numeryczną. Algorytm genetyczny
poszukiwał optymalnych elementów diagonalnych macierzy $Q$ oraz skalara $R$,
    minimalizując czas regulacji. Zoptymalizowane wagi (w szczególności wysoka kara
    $Q_{x} = 500$) sprawiają, że regulator bardzo agresywnie pilnuje pozycji wózka,
    co pośrednio wymusza stabilne trzymanie wahadła (wymagane do kontroli pozycji).

\begin{figure}[H]
    \centering
    \includegraphics[width=1.0\textwidth]{images/tuning/pdlqr_3_opt.png}
    \caption{Optymalne przebiegi regulatora PD-LQR po zastosowaniu algorytmu genetycznego.}
    \label{fig:lqr_opt}
\end{figure}

\subsection{Nieliniowe sterowanie predykcyjne (MPC)}

Algorytm MPC (Model Predictive Control) stanowi zaawansowaną metodę sterowania,
która w~odróżnieniu od~LQR, uwzględnia wprost ograniczenia sygnału sterującego
oraz nieliniową dynamikę obiektu \cite{Camacho2007, Rawlings2017}.
Zaimplementowany w~klasie \texttt{MPCController} (plik \texttt{mpc.py})
algorytm rozwiązuje w~każdym kroku symulacji problem optymalizacji dynamicznej
nieliniowej (NMPC).

Zadanie optymalizacji, rozwiązywane numerycznie metodą SQP (Sequential
Quadratic Programming) przy użyciu solwera \texttt{SLSQP}, zdefiniowane jest
następująco:
\begin{equation}
    \min_{\Delta U} J = \sum_{k=1}^{N_{\mathrm{p}}} (\hat{x}_k - x_{\mathrm{ref}})^T Q (\hat{x}_k - x_{\mathrm{ref}}) + R \sum_{k=0}^{N_{\mathrm{c}}-1} (\Delta u_k)^2,
\end{equation}
przy ograniczeniach:
\begin{align}
    \hat{x}_{k+1} &= f(\hat{x}_k, u_k), \quad k=0,\dots,N_{\mathrm{p}}-1 \\
    u_{\mathrm{min}} &\le u_k \le u_{\mathrm{max}}, \\
    u_k &= u_{k-1} + \Delta u_k.
\end{align}
Gdzie:
\begin{itemize}
    \item $N_{\mathrm{p}} = 12$ -- horyzont predykcji,
    \item $N_{\mathrm{c}} = 4$ -- horyzont sterowania (blokowanie sterowania dla $k \ge N_{\mathrm{c}}$),
    \item $f(\cdot)$ -- nieliniowy model dyskretny obiektu (całkowanie metodą
    Rungego-Kutt 4. rzędu),
    \item $Q = \text{diag}([158.4,\; 36.8,\; 43.4,\; 19.7])$ -- macierz kar stanu,
    \item $R = 0.086$ -- współczynnik kary za~zmianę sterowania ($\Delta u$).
\end{itemize}
Kluczową zaletę MPC, podkreślaną w~pracach \cite{Mills2009} oraz
\cite{Jezierski2017}, jest możliwość bezpośredniego uwzględnienia ograniczeń
(saturacji) już na~etapie wyliczania sterowania, co zapobiega zjawisku
nasycenia elementu wykonawczego, które mogłoby mieć miejsce w~przypadku LQR.

Analiza wykazała, że bezpośrednie przeniesienie macierzy wag $Q$ i $R$
z~regulatora LQR do~sterownika MPC prowadziło do~znaczącego pogorszenia jakości
sterowania (wydłużenie czasu regulacji z~ok. 3s do~ponad 9s). Wynika to z~faktu,
że model MPC, dzięki jawnemu uwzględnieniu ograniczeń sygnału sterującego, pozwala
na~zastosowanie znacznie bardziej agresywnych nastaw (większych kar za~błędy stanu),
które w~liniowym regulatorze LQR powodowałyby nasycenie i~potencjalną niestabilność.
Dlatego zdecydowano się na~niezależną optymalizację parametrów obu regulatorów,
aby porównywać ich najlepsze możliwe konfiguracje, a~nie identyczne, ale
nieoptymalne nastawy.

\subsubsection{Dobór horyzontu i wag funkcji celu}
Dla regulatora MPC kluczowym wyzwaniem był dobór horyzontu predykcji oraz
macierzy wag, determinujących zachowanie układu w~stanie nieustalonym.

Początkowe ustawienie zbyt krótkiego horyzontu predykcji ($N_{\mathrm{p}} < 5$)
prowadziło do~niestabilności układu zamkniętego. Regulator ,,nie widział'', że
rozpędzając wózek w~celu korekcji kąta, nie zdąży wyhamować przed upadkiem
wahadła lub osiągnięciem końca toru. Zwiększenie horyzontu do $N_{\mathrm{p}}=10$
w~ramach korekty ręcznej ustabilizowało proces. Dodatkowa manipulacja wagami $Q$
pozwoliła na~uzyskanie poprawnego sterowania, jednak koszt obliczeniowy był
wysoki, a~przebiegi wciąż wykazywały niepożądane przeregulowania.

Automatyzacja procesu strojenia przy użyciu skryptu \texttt{tune\_mpc.py} pozwoliła
na~znalezienie kompromisu między długością horyzontu a~wagami. Algorytm
optymalizacyjny wskazał $N_{\mathrm{p}}=12$ jako optimum dla tego modelu
dyskretnego, zapewniając stabilność przy akceptowalnym czasie obliczeń.

\begin{figure}[H]
    \centering
    \includegraphics[width=1.0\textwidth]{images/tuning/mpc_3_opt.png}
    \caption{Zoptymalizowany regulator MPC ($N_p=12$) - szybka i gładka stabilizacja.}
    \label{fig:mpc_opt}
\end{figure}

\subsection{MPC z~rozszerzonym wskaźnikiem jakości (MPC-J2)}

Zaimplementowano sterownik \texttt{MPCControllerJ2} jako wariant badawczy algorytmu predykcyjnego
(plik \texttt{mpc\_J2.py}). Jego struktura jest
zbliżona do~podstawowego MPC, jednak funkcja kosztu została rozbudowana
o~dodatkowy składnik karzący bezwzględną wartość sygnału sterującego
(energię), a~nie tylko jego przyrosty.

Zmodyfikowana funkcja celu przyjmuje postać:
\begin{equation}
    J = \sum_{k=1}^{N_{\mathrm{p}}} (x_k - x_{\mathrm{ref}})^T Q (x_k - x_{\mathrm{ref}}) \;+\; R_{\Delta} \sum_{k=0}^{N_{\mathrm{c}}-1} (\Delta u_k)^2 \;+\; R_{\mathrm{abs}} \sum_{k=0}^{N_{\mathrm{c}}-1} (u_k)^2.
\end{equation}
Wprowadzenie parametru $R_{\mathrm{abs}}$ pozwala na~bezpośrednie minimalizowanie
zużycia energii sterowania, co jest podejściem powszechnie stosowanym
w~praktycznych implementacjach algorytmów predykcyjnych \cite{Camacho2007, Rawlings2017}.
Ograniczenie amplitudy sygnału sterującego nie tylko redukuje wydatek energetyczny
(istotny w~aplikacjach mobilnych), ale także zmniejsza obciążenie mechaniczne
elementów wykonawczych, co wpływa na~żywotność napędu.

W~badaniach przyjęto wagi: $q_{\theta}=80.0$,
$q_x=120.0$ (elementy macierzy diagonalnej $Q$), kładąc większy
nacisk na~precyzję pozycjonowania wózka w~porównaniu do~standardowego MPC.

\subsubsection{Analiza wpływu kary za energię}
W~przypadku wariantu MPC-J2 analizowano nieliniowy wpływ parametru $R_{\mathrm{abs}}$
na~zachowanie układu.

Przyjęcie zbyt dużej wartości kary za~sterowanie bezwzględne ($R_{\mathrm{abs}}$)
spowodowało, że regulator wykazywał tendencję do~pasywności. Wahadło przewracało się,
ponieważ koszt energetyczny utrzymania go w~pionie przewyższał zysk wynikający
z~małego błędu kąta w~funkcji celu.

Stopniowe, ręczne zmniejszanie parametru $R_{\mathrm{abs}}$ pozwoliło znaleźć punkt
pracy, w~którym układ odzyskał stabilność przy zachowaniu relatywnej oszczędności
energetycznej. Odpowiedź dynamiczna była jednak powolna i~zbyt asekuracyjna dla
większych zakłóceń.

Algorytm optymalizacyjny precyzyjnie dostroił $R_{\mathrm{abs}}$, minimalizując złożony
wskaźnik kosztu (błąd + energia). Znaleziono ,,złoty środek'', w~którym układ
stabilizuje się szybko, ale sterowanie pozbawione jest zbędnych oscylacji
wysokoczęstotliwościowych, co przekłada się na~oszczędność energii.

\begin{figure}[H]
    \centering
    \includegraphics[width=1.0\textwidth]{images/tuning/mpcJ2_3_opt.png}
    \caption{Optymalny kompromis między jakością regulacji a energią w MPC-J2.}
    \label{fig:mpcj2_opt}
\end{figure}

\subsection{Regulator rozmyty wspomagany LQR (Fuzzy-LQR)}

Ostatnim zbadanym układem jest sterownik hybrydowy \texttt{TSFuzzyController}
(plik \texttt{fuzzy\_lqr.py}), łączący liniowy regulator LQR z~systemem
wnioskowania rozmytego typu Takagi-Sugeno (T-S). Koncepcja ta, opisana szerzej
w~\cite{Nguyen2024} oraz \cite{Roose2017}, ma na~celu adaptację wzmocnień regulatora
w~zależności od~punktu pracy, co pozwala na~agresywniejszą reakcję w~przypadku
dużych odchyleń od~pionu.

Sygnał sterujący jest sumą:
\begin{equation}
    u(t) = u_{\mathrm{LQR}}(t) + u_{\mathrm{Fuzzy}}(t).
\end{equation}
Część rozmyta $u_{\mathrm{Fuzzy}}(t)$ wykorzystuje bazę reguł postaci:
\begin{quote}
    JEŚLI $e_\theta$ jest $A_i$ ORAZ $\dot{\theta}$ jest $B_i$ ... TO $u_i = f_i(x)$,
\end{quote}
gdzie $f_i(x)$ jest liniową funkcją stanu (lokalny regulator
liniowy). Zastosowano funkcje przynależności trójkątne dla zmiennych stanu,
dzieląc przestrzeń na~obszary ,,Mały błąd'' i~,,Duży błąd''.
Baza wiedzy składa się z~16 reguł ($2^4$ kombinacji dla 4 zmiennych stanu).
Wyjście sterownika obliczane jest jako średnia ważona:
\begin{equation}
    u_{\mathrm{Fuzzy}} = G \cdot \frac{\sum_{i=1}^{16} w_i(x) \cdot u_i}{\sum_{i=1}^{16} w_i(x)},
\end{equation}
gdzie $w_i$ to stopień aktywacji $i$-tej reguły, a~$G = 0.9$ to globalne
wzmocnienie skalujące.

Zastosowany mechanizm ,,Gain Scheduling'' pozwala na:
\begin{enumerate}
    \item Zachowanie łagodnej charakterystyki LQR w~pobliżu punktu równowagi
    (małe wzmocnienia w~regułach dla ,,Małych błędów'').
    \item Zwiększenie sztywności układu w~sytuacjach krytycznych (duże
    wzmocnienia zdefiniowane w~zmiennej \texttt{F\_rules} dla ,,Dużych
    błędów'').
\end{enumerate} 
Takie podejście pozwala na~rozszerzenie obszaru stabilności regulatora
w~porównaniu do~klasycznego LQR, co potwierdzają wyniki badań w~pracy
\cite{Nguyen2024}.

\subsubsection{Dobór reguł i funkcji przynależności}
Strojenie rozmytego regulatora Fuzzy-LQR jest zadaniem złożonym ze względu na~dużą
liczbę parametrów definiujących bazę reguł i~funkcje przynależności.

Błędne zdefiniowanie zbyt wąskich funkcji przynależności dla strefy ,,małego błędu''
skutkowało gwałtownym przełączaniem się regulatora na~agresywne reguły (chatter).
Prowadziło to do~silnych drgań wokół punktu równowagi, co jest zjawiskiem
niepożądanym w~rzeczywistych układach napędowych.
Opierając się na~literaturze \cite{Nguyen2024}, dobrano ręcznie szerokości trójkątnych
funkcji przynależności tak, aby przejście między strefami było płynne. Układ uzyskał
stabilność asymptotyczną, jednak nie wykorzystywał w~pełni potencjału szybkiej
reakcji na~duże zakłócenia, działając zachowawczo.

Ostatecznie, dedykowany skrypt \texttt{tune\_fuzzy\_lqr.py} posłużył do~optymalizacji
wag pojedynczych reguł oraz parametrów kształtu funkcji przynależności. Uzyskano
nieliniową powierzchnię sterowania, która łączy zalety miękkiego sterowania LQR
w~pobliżu zera z~maksymalną siłą reakcji przy dużych wychyleniach.

\begin{figure}[H]
    \centering
    \includegraphics[width=1.0\textwidth]{images/tuning/fuzzy_3_opt.png}
    \caption{Efektywne sterowanie Fuzzy-LQR po optymalizacji bazy reguł.}
    \label{fig:fuzzy_opt}
\end{figure}
