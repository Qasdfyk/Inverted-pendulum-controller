\section{Algorytmy sterowania}

W~niniejszym rozdziale przedstawiono szczegółowy opis algorytmów sterowania
zaimplementowanych i~przeanalizowanych w~ramach pracy. Kod sterowników został
zrealizowany w~języku Python w~postaci klas dziedziczących wspólną strukturę,
co zapewnia modularność i~łatwą wymienność w~pętli symulacyjnej. Każdy
regulator wyznacza sygnał sterujący $u(t)$ (siłę przyłożoną do wózka)
na~podstawie aktualnego wektora stanu
$x(t) = [\theta, \dot{\theta}, x, \dot{x}]^T$ oraz wartości zadanych
$x_{\mathrm{ref}}$.

W~literaturze problem sterowania wahadłem odwróconym jest szeroko omawiany
jako klasyczny problem testowy dla metod sterowania liniowego i~nieliniowego
\cite{Prasad2014, Nguyen2024}. Poniżej opisano teoretyczne podstawy oraz szczegóły
implementacyjne zbadanych struktur sterowania.

\subsection{Równoległy regulator PD}

Pierwszym zaimplementowanym układem jest regulator o~strukturze równoległej,
wykorzystujący klasyczne sprzężenie zwrotne typu PD
(Proporcjonalno-Różniczkujące)~\cite{Ogata2010}. W~literaturze podejście to jest często
stosowane jako punkt odniesienia dla bardziej zaawansowanych metod
\cite{Moreno2023, Prasad2014}.

W~klasie \texttt{PDPDController} zastosowano strukturę równoległą, w~której
całkowity sygnał sterujący jest sumą reakcji na~błąd kąta oraz błąd pozycji.
Jest to podejście intuicyjne, dekomponujące problem na~dwa podzadania:
stabilizację wahadła w~pozycji pionowej oraz doprowadzenie wózka do~zadanej
pozycji. Schemat blokowy regulatora przedstawiono na~Rys.~\ref{fig:diagram_pdpd}.

\begin{figure}[H]
    \centering
    \begin{tikzpicture}[auto, node distance=1.5cm, >=Stealth]
        % Wejścia (po lewej)
        \node [input] (in_theta) {};
        \node [below=1.2cm of in_theta, input] (in_x) {};
        
        % Sumy błędów
        \node [sum, right=0.8cm of in_theta] (sum_theta) {\tiny $+$};
        \node [sum, right=0.8cm of in_x] (sum_x) {\tiny $+$};
        
        % Regulatory PD
        \node [block, right=0.8cm of sum_theta] (pd_theta) {$\mathrm{PD}_\theta$};
        \node [block, right=0.8cm of sum_x] (pd_x) {$\mathrm{PD}_x$};
        
        % Suma sterowania (między nimi)
        \node [sum, right=1.8cm of pd_theta, yshift=-0.6cm] (sum_u) {\tiny $+$};
        
        % Saturacja i obiekt
        \node [block, right=0.7cm of sum_u, minimum width=2.5em] (sat) {sat};
        \node [bigblock, right=0.8cm of sat] (plant) {Wahadło};
        
        % Wyjście
        \node [output, right=1.2cm of plant] (out) {};
        
        % --- Połączenia ---
        % Wejścia do sum
        \draw [arrow] (in_theta) -- node[above, font=\scriptsize] {$\theta_{\mathrm{ref}}$} (sum_theta);
        \draw [arrow] (in_x) -- node[above, font=\scriptsize] {$x_{\mathrm{ref}}$} (sum_x);
        
        % Sumy do PD
        \draw [arrow] (sum_theta) -- (pd_theta);
        \draw [arrow] (sum_x) -- (pd_x);
        
        % PD do sumy sterowania (rozdzielone wejścia: góra i dół)
        \draw [arrow] (pd_theta.east) -| (sum_u.north);
        \draw [arrow] (pd_x.east) -| (sum_u.south);
        
        % Suma -> sat -> plant -> wyjście
        \draw [arrow] (sum_u) -- (sat);
        \draw [arrow] (sat) -- node[above, font=\scriptsize] {$u$} (plant);
        \draw [arrow] (plant) -- node[above, font=\scriptsize] {$\theta,x$} (out);
        
        % Sprzężenie zwrotne (na dole) - ciągła linia
        \draw [arrow] ($(plant.east) + (0.3,0)$) -- ++(0,+1.8) -| (sum_theta.north);
        \draw [arrow] ($(plant.east) + (0.3,0)$) -- ++(0,-1.8) -| (sum_x.south);
        
        % Znak minus na sumach
        \node [above=3pt of sum_theta.north, font=\tiny] {$-$};
        \node [below=3pt of sum_x.south, font=\tiny] {$-$};
    \end{tikzpicture}
    \caption{Schemat blokowy regulatora PD o~strukturze równoległej.}
    \label{fig:diagram_pdpd}
\end{figure}

Prawo sterowania wyraża się wzorem:
\begin{equation}
    u(t) = \mathrm{sat}_{u_{\mathrm{max}}} \left( u_{\theta}(t) + u_{x}(t) \right),
\end{equation}
gdzie funkcja nasycenia $\mathrm{sat}(\cdot)$ wynika z~ograniczeń fizycznych
siłownika. Definiując uchyby regulacji jako $e_{\theta}(t) = \theta_{\mathrm{ref}} - \theta(t)$ 
oraz $e_x(t) = x_{\mathrm{ref}} - x(t)$, prawo sterowania dla poszczególnych pętli 
można zapisać w~ogólnej postaci regulatora PD (będącej szczególnym przypadkiem PID z~$K_i=0$):
\begin{align}
    u_{\theta}(t) &= K_{p,\theta} e_{\theta}(t) + K_{i,\theta} \int_0^t e_{\theta}(\tau)\,d\tau + K_{d,\theta} \frac{d e_{\theta}(t)}{dt}, \label{eq:pd_theta} \\
    u_{x}(t) &= K_{p,x} e_x(t) + K_{i,x} \int_0^t e_x(\tau)\,d\tau + K_{d,x} \frac{d e_x(t)}{dt}. \label{eq:pd_x}
\end{align}
W~powyższych równaniach przyjęto upraszczające założenie, że docelowe
prędkości ($\dot{\theta}_{\mathrm{ref}}, \dot{x}_{\mathrm{ref}}$) wynoszą zero.

W~implementacji programowej (plik \texttt{pd\_pd.py}) przyjęto następujące
nastawy dobrane eksperymentalnie:
\begin{itemize}
    \item Tor stabilizacji kąta: $K_{p,\theta} = -95.0$, $K_{d,\theta} = -14.0$.
    Ujemne znaki wynikają z~przyjętej konwencji układu współrzędnych i~zwrotu siły.
    \item Tor pozycji: $K_{p,x} = -16.0$, $K_{d,x} = -14.0$.
\end{itemize}
Mimo iż klasa umożliwia włączenie członu całkującego (PID), w~badaniach
\cite{Varghese2017} często wskazuje się, że dla obiektów tej klasy człon różniczkujący
(PD) jest kluczowy dla tłumienia oscylacji, a~całkowanie może wprowadzać
niestabilność w~stanach nieustalonych bez odpowiednich mechanizmów
$\mathrm{anti\text{-}windup}$.

\subsubsection{Proces doboru nastaw oraz analiza PD}
Dobór nastaw dla regulatora PD został zrealizowany wieloetapowo, ewoluując
od metod heurystycznych do pełnej optymalizacji numerycznej.

\begin{figure}[H]
    \centering
    \includegraphics[width=1.0\textwidth]{images/tuning/pdpd_1_manual.png}
    \caption{Regulator PD nastawiony ręcznie ($K_{p,\theta}=-40$, $K_{d,\theta}=-8$, 
    $K_{p,x}=-1$, $K_{d,x}=-3$).}
    \label{fig:pid_manual}
\end{figure}

Wstępne próby doboru metodą ,,prób i~błędów'' (Rys.~\ref{fig:pid_manual}), oparte 
na~dekompozycji problemu (najpierw stabilizacja wahadła, potem pozycja wózka), 
pozwoliły uzyskać stabilność, jednak jakość regulacji była niezadowalająca. 
Układ charakteryzował się powolnym dochodzeniem do~punktu pracy i~znacznymi 
oscylacjami. Wzmocnienia proporcjonalne ($K_p=-40$ dla kąta) były niewystarczające,
aby szybko tłumić odchylenia od~pionu.

\paragraph{Analiza porównawcza struktur PD i PID}
W~literaturze przedmiotu \cite{Varghese2017, Nguyen2024} często podkreśla się, że dla
obiektów niestabilnych statycznie, takich jak wahadło odwrócone, kluczowa jest
szybka reakcja na~zmiany kąta, którą zapewnia człon różniczkujący (D).
Włączenie członu całkującego (I), tworzącego regulator PID, wprowadza dodatkowe
przesunięcie fazowe (opóźnienie), co w~układzie o~dynamice astatycznej
i~nieliniowej prowadzi do znacznego pogorszenia zapasu stabilności.

\begin{figure}[H]
    \centering
    \includegraphics[width=1.0\textwidth]{images/tuning/pdpd_2_integral_bad.png}
    \caption{Regulator PID z~członem całkującym ($K_{p,\theta}=-40$, $K_{i,\theta}=-2$, 
    $K_{d,\theta}=-8$, $K_{p,x}=-1$, $K_{i,x}=-2$, $K_{d,x}=-3$).}
    \label{fig:pdpd_integral}
\end{figure}

Przeprowadzone badania symulacyjne potwierdziły te wnioski (Rys.~\ref{fig:pdpd_integral}). 
Próba dodania akcji całkującej ($K_i \neq 0$) skutkowała zjawiskiem 
\textit{integral wind-up} -- akumulacją błędu w~fazie rozruchu, co powodowało 
przeregulowania wykraczające poza obszar przyciągania stabilnego punktu równowagi. 
Regulator PID wpadał w~niegasnące oscylacje lub doprowadzał do~przewrócenia wahadła, 
podczas gdy ,,czysty'' regulator PD (z~$K_i=0$) zapewnia sztywne i~szybkie sterowanie.
Warto jednak zauważyć, że w~strukturze hybrydowej PD-LQR, gdzie LQR zapewnia 
dodatkowe tłumienie, człon całkujący może być bezpiecznie zastosowany -- 
zagadnienie to omówiono szczegółowo w~Podrozdziale~4.2.

Aby wyeliminować subiektywność strojenia ręcznego, zastosowano algorytm
Ewolucji Różnicowej (Differential Evolution)~\cite{StornPrice1997}, zaimplementowany 
w~module \texttt{scipy.optimize}. Zdefiniowano globalną funkcję kosztu $J$, która
umożliwia porównanie wszystkich badanych regulatorów w~ujednoliconych warunkach:
\begin{equation}
    J = w_{\theta} \cdot \text{MSE}(\theta) + w_{x} \cdot \text{MSE}(x) + w_{u} \cdot \text{RMS}(u),
\end{equation}
gdzie przyjęto wagi $w_{\theta}=4.0$ (priorytet stabilizacji), $w_{x}=1.0$
(dokładność pozycjonowania) oraz $w_{u}=0.01$ (koszt energii). Algorytm operował
na~populacji 10 osobników przez 20 generacji, co pozwoliło uniknąć minimów
lokalnych.

\begin{figure}[H]
    \centering
    \includegraphics[width=1.0\textwidth]{images/tuning/pdpd_3_opt.png}
    \caption{Zoptymalizowany regulator PD ($K_{p,\theta}=-95$, $K_{d,\theta}=-14$, 
    $K_{p,x}=-16$, $K_{d,x}=-14$).}
    \label{fig:pdpd_opt}
\end{figure}

Zoptymalizowane nastawy (Rys.~\ref{fig:pdpd_opt}) charakteryzują się znacznie wyższymi
wzmocnieniami niż dobrane ręcznie: $K_{p,\theta}=-95$ (vs. $-40$) oraz $K_{p,x}=-16$ 
(vs. $-1$). W~wyniku optymalizacji ustalono, że intensywna reakcja na~błąd pozycji wózka pośrednio
stabilizuje wahadło, ponieważ wymusza szybkie korekty trajektorii.

\subsection{Układ hybrydowy PD-LQR}

Regulator liniowo-kwadratowy (LQR) stanowi fundamentalną metodę sterowania optymalnego
dla systemów liniowych wielowymiarowych MIMO~\cite{Lewis2012, Jezierski2017}. 
W~odróżnieniu od~regulatorów PD, które wymagają empirycznego doboru wzmocnień 
dla każdej zmiennej stanu, LQR wyznacza optymalne wzmocnienia automatycznie 
na~podstawie modelu liniowego obiektu oraz macierzy wag $Q$ i~$R$ definiujących 
kompromis między jakością regulacji a~zużyciem energii.

Problem LQR polega na~znalezieniu prawa sterowania $u(t) = -K x(t)$, które 
minimalizuje wskaźnik jakości:
\begin{equation}
    J = \int_{0}^{\infty} \left( x(t)^T Q x(t) + u(t)^T R u(t) \right) dt,
\end{equation}
gdzie $Q \succeq 0$ jest macierzą wag stanu, a~$R > 0$ wagą sterowania. 
Optymalna macierz wzmocnień $K$ wyznaczana jest poprzez rozwiązanie algebraicznego 
równania Riccatiego (CARE):
\begin{equation}
    A^T P + P A - P B R^{-1} B^T P + Q = 0,
\end{equation}
skąd $K = R^{-1} B^T P$. Macierze $A$ i~$B$ pochodzą z~linearyzacji modelu wahadła 
wokół górnego punktu równowagi ($\theta = 0$).

\subsubsection{Ograniczenia czystego regulatora LQR}

Przed przystąpieniem do~analizy struktury hybrydowej, zbadano działanie czystego 
regulatora LQR (klasa \texttt{LQRController}, plik \texttt{lqr.py}).

\begin{figure}[H]
    \centering
    \includegraphics[width=1.0\textwidth]{images/tuning/lqr_1_baseline.png}
    \caption{Czysty regulator LQR z~wagami jednostkowymi ($Q=I$, $R=1$).}
    \label{fig:lqr_baseline}
\end{figure}

Na~Rysunku~\ref{fig:lqr_baseline} przedstawiono odpowiedź układu z~czystym regulatorem 
LQR przy zastosowaniu jednostkowych macierzy wag ($Q=I$, $R=1$). Mimo poprawnej 
stabilizacji wahadła (mały błąd kątowy), widoczne są istotne ograniczenia tej 
konfiguracji. Jednostkowe wagi traktują 1~rad błędu kąta tak samo jak 1~m błędu 
pozycji, co jest fizycznie nieuzasadnione i~prowadzi do~powolnego osiągania zadanej 
pozycji wózka.

\begin{figure}[H]
    \centering
    \includegraphics[width=1.0\textwidth]{images/tuning/lqr_2_opt.png}
    \caption{Czysty regulator LQR z~wagami zoptymalizowanymi ($Q=\mathrm{diag}([1, 1, 500, 250])$, $R=1$).}
    \label{fig:lqr_opt_2}
\end{figure}

Zwiększenie wag dla pozycji i~prędkości wózka ($Q_x=500$, $Q_{\dot{x}}=250$) prowadzi 
do~znaczącej poprawy jakości śledzenia wartości zadanej (Rys.~\ref{fig:lqr_opt_2}). 
Wyższe kary za~błąd pozycji wymuszają na~regulatorze intensywniejszą reakcję, 
co skutkuje szybszym osiągnięciem zadanego punktu. Jedank sygnał sterujący dla 
tak skonfigurowanego regulatora LQR jest bardzo słabej jakości. Widoczny są gwałtowne, 
nieciągłe zmiany sygnału sterującego, które są kosztowne energetycznie i mogą być niebezpieczne dla układu.
Motywuje to rozszerzenie struktury o~dodatkową pętlę PD z~możliwością włączenia członu całkującego.

\subsubsection{Struktura hybrydowa PD-LQR}

Klasa \texttt{PDLQRController} (plik \texttt{pd\_lqr.py}) implementuje sterowanie
oparte na~pełnym wektorze stanu, wspomagane dodatkowym członem PD dla uchybu
pozycji, co tworzy strukturę hybrydową opisaną m.in. w~\cite{Prasad2014} oraz
\cite{Nguyen2024} (w~kontekście porównawczym). Schemat blokowy tego układu 
przedstawiono na~Rys.~\ref{fig:diagram_pdlqr}.

\begin{figure}[H]
    \centering
    \begin{tikzpicture}[auto, node distance=1.4cm, >=Stealth]
        % Wejście
        \node [input] (in) {};
        
        % Suma błędu LQR
        \node [sum, right=0.8cm of in] (sum_lqr) {\tiny $+$};
        
        % Blok LQR
        \node [block, right=0.8cm of sum_lqr] (lqr) {LQR $(-K)$};
        
        % Suma sterowania
        \node [sum, right=1.2cm of lqr] (sum_u) {\tiny $+$};
        
        % Saturacja i obiekt
        \node [block, right=0.7cm of sum_u, minimum width=2.5em] (sat) {sat};
        \node [bigblock, right=0.8cm of sat] (plant) {Wahadło};
        
        % Wyjście
        \node [output, right=1.2cm of plant] (out) {};
        
        % PD pozycji (poniżej)
        \node [sum, below=1.4cm of sum_lqr] (sum_pd) {\tiny $+$};
        \node [block, right=0.8cm of sum_pd] (pd_pos) {$\mathrm{PD}_x$};
        
        % --- Połączenia ---
        % Wejście LQR
        \draw [arrow] (in) -- node[above, font=\scriptsize] {$x_{\mathrm{ref}}$} (sum_lqr);
        \draw [arrow] (sum_lqr) -- (lqr);
        \draw [arrow] (lqr) -- node[above, font=\scriptsize] {$u_{\mathrm{LQR}}$} (sum_u);
        
        % PD -> suma sterowania
        \draw [arrow] (pd_pos.east) -| (sum_u.south);
        
        % Reszta toru
        \draw [arrow] (sum_u) -- (sat);
        \draw [arrow] (sat) -- node[above, font=\scriptsize] {$u$} (plant);
        \draw [arrow] (plant) -- node[above, font=\scriptsize] {$x$} (out);
        
        % Wejście referencyjne PD
        \node [input, left=0.8cm of sum_pd] (in_pd) {};
        \draw [arrow] (in_pd) -- node[above, font=\scriptsize] {$x_{\mathrm{ref}}$} (sum_pd);
        \draw [arrow] (sum_pd) -- (pd_pos);
        
        % Sprzężenie zwrotne LQR (na dole) - ciągła strzałka
        \draw [arrow] ($(plant.east) + (0.3,0)$) -- ++(0,1.4) -| (sum_lqr.north);
        \node [above=3pt of sum_lqr.north, font=\tiny] {$-$};
        
        % Sprzężenie zwrotne PD (x) - od głównej linii sprzężenia
        \draw [arrow] ($(plant.east) + (0.3,0)$) -- ++(0,-3.4) -| (sum_pd.south);
        \node [below=3pt of sum_pd.south, font=\tiny] {$-$};
        
        % Ramka LQR
        \begin{scope}[on background layer]
            \node [draw, dashed, rounded corners, fill=green!5, 
                   fit=(sum_lqr) (lqr), inner sep=6pt] {};
        \end{scope}
    \end{tikzpicture}
    \caption{Schemat blokowy hybrydowego regulatora PD-LQR.}
    \label{fig:diagram_pdlqr}
\end{figure}

Problem LQR polega na~znalezieniu prawa sterowania
$u(t) = -K x(t)$, które minimalizuje wskaźnik jakości:
\begin{equation}
    J = \int_{0}^{\infty} \left( x(t)^T Q x(t) + u(t)^T R u(t) \right) dt,
\end{equation}
gdzie $Q \succeq 0$ jest macierzą wag stanu, a~$R > 0$ wagą
sterowania~\cite{Lewis2012}. Optymalna macierz wzmocnień $K$ wyznaczana jest poprzez
rozwiązanie algebraicznego równania Riccatiego (CARE):
\begin{equation}
    A^T P + P A - P B R^{-1} B^T P + Q = 0,
\end{equation}
skąd $K = R^{-1} B^T P$. Macierze $A$
i~$B$ pochodzą z~linearyzacji modelu wahadła wokół punktu równowagi
górnej ($\theta = 0$).

W~zaimplementowanym rozwiązaniu (plik \texttt{pd\_lqr.py}), sygnał sterujący
składa się z~dwóch komponentów:
\begin{equation}
    u(t) = u_{\mathrm{LQR}}(t) + u_{\mathrm{PD,pos}}(t).
\end{equation}
Składnik LQR realizuje stabilizację wokół punktu pracy:
\begin{equation}
    u_{\mathrm{LQR}}(t) = -K \cdot (x(t) - x_{\mathrm{ref}}).
\end{equation}
Zastosowane wagi optymalne to:
\begin{equation}
    Q = \text{diag}([1.0,\; 1.0,\; 500.0,\; 250.0]), \quad R = 1.0.
\end{equation}
Dodatkowy człon PD na~pętli pozycji (zrealizowany analogicznie do wzoru
\ref{eq:pd_x}) ma na~celu poprawę śledzenia skokowych zmian wartości zadanej
$x_{\mathrm{ref}}$, co jest częstą praktyką w~aplikacjach praktycznych, gdzie LQR
zapewnia stabilność, a~regulator zewnętrzny dba o~uchyb w~stanie ustalonym
\cite{Varghese2017}.

\subsubsection{Dobór wag macierzy Q i R}
Dobór wag dla regulatora LQR również charakteryzował się ewolucyjnym podejściem
do~problemu optymalizacji wskaźnika jakości.

\begin{figure}[H]
    \centering
    \includegraphics[width=1.0\textwidth]{images/tuning/pdlqr_1_bad.png}
    \caption{Regulator LQR z~wagami jednostkowymi ($Q=I$, $R=1$, 
    PD: $K_{p,x}=-4.5$, $K_{d,x}=-3$).}
    \label{fig:lqr_bad}
\end{figure}

W~pierwszej fazie przyjęcie jednostkowej macierzy diagonalnej $Q=I$ oraz $R=1$
(Rys.~\ref{fig:lqr_bad}) okazało się niewystarczające. Mimo teoretycznej stabilności 
wynikającej z~rozwiązania równania CARE, wahadło wykonywało bardzo duże wychylenia, 
a~wózek wielokrotnie wyjeżdżał poza dopuszczalny zakres roboczy toru. Problem wynikał
z~faktu, że jednostkowe wagi traktują 1~rad błędu kąta tak samo jak 1~m błędu pozycji
i~1~N$^2$ kosztu sterowania -- co jest fizycznie nieuzasadnione.

\begin{figure}[H]
    \centering
    \includegraphics[width=1.0\textwidth]{images/tuning/pdlqr_2_manual.png}
    \caption{Regulator LQR strojony metodą Brysona ($Q=\mathrm{diag}([25, 1, 4, 1])$, 
    $R=10$, PD: $K_{p,x}=-4.5$, $K_{d,x}=-3$).}
    \label{fig:lqr_manual}
\end{figure}

Następnie przeprowadzono strojenie ręczne metodą prób i~błędów, inspirując się
regułą Brysona (Rys.~\ref{fig:lqr_manual}). Reguła ta postuluje, że elementy 
diagonalne macierzy $Q$ i~$R$ powinny być odwrotnie proporcjonalne do~kwadratów 
maksymalnych dopuszczalnych wartości odpowiednich zmiennych stanu i~sterowania, 
tj. $Q_{ii} = 1/x_{i,\max}^2$ oraz $R = 1/u_{\max}^2$. Ręczne zwiększanie kar 
za~wychylenie kąta ($Q_{\theta}=25$) poprawiło sztywność wahadła. Udało się ustalić 
zestaw wag zapewniający stabilną pracę, choć czas regulacji był wciąż niezadowalający, 
a~reakcja na~zakłócenia powolna.

\begin{figure}[H]
    \centering
    \includegraphics[width=1.0\textwidth]{images/tuning/pdlqr_3_opt.png}
    \caption{Zoptymalizowany regulator PD-LQR bez członu całkującego 
    ($Q=\mathrm{diag}([1, 1, 500, 250])$, $R=1$, PD: $K_{p,x}=-1{,}5$, 
    $K_{i,x}=0$, $K_{d,x}=-5$).}
    \label{fig:lqr_opt}
\end{figure}

W~ostatnim etapie zastosowano optymalizację numeryczną (Rys.~\ref{fig:lqr_opt}). 
Algorytm genetyczny poszukiwał optymalnych elementów diagonalnych macierzy $Q$ 
oraz skalara $R$, minimalizując wskaźnik jakości. Zoptymalizowane wagi 
(w~szczególności wysoka kara $Q_{x} = 500$) sprawiają, że regulator bardzo 
intensywnie koryguje pozycję wózka, co pośrednio wymusza stabilne utrzymanie wahadła. 
Należy zauważyć, że optymalizator wyznaczył strategię przeciwną do~intuicyjnej: 
zamiast karać głównie kąt ($Q_{\theta}$), kładzie nacisk na~pozycję ($Q_x$), 
co zmusza wózek do~szybkich korekt stabilizujących wahadło.

\subsubsection{Analiza wpływu członu całkującego w~układzie PD-LQR}

W~literaturze \cite{Prasad2014} sugeruje się, że struktura hybrydowa PD-LQR 
może bezpiecznie wykorzystywać akcję całkującą ($K_i \neq 0$), ponieważ LQR 
zapewnia wystarczające tłumienie. Przeprowadzono eksperymentalną weryfikację 
tej tezy.

\begin{figure}[H]
    \centering
    \includegraphics[width=1.0\textwidth]{images/tuning/pdlqr_4_with_ki.png}
    \caption{Regulator PD-LQR z~członem całkującym ($K_{p,x}=-2{,}5$, 
    $K_{i,x}=-1{,}0$, $K_{d,x}=-5$, $Q=\mathrm{diag}([1, 1, 500, 250])$, $R=1$).}
    \label{fig:lqr_with_ki}
\end{figure}

Na Rysunku~\ref{fig:lqr_with_ki} przedstawiono odpowiedź układu z~włączonym 
członem całkującym ($K_i = -1{,}0$). Mimo że układ zachowuje stabilność, 
widoczny jest wyraźny uchyb ustalony pozycji wózka. Paradoksalnie, człon 
całkujący, który teoretycznie powinien eliminować uchyb ustalony, w~tej 
konfiguracji powoduje jego powstanie. Zjawisko to wynika z~interakcji między 
pętlą PD a~regulatorem LQR -- akumulowany błąd w~integratorze interferuje 
z~optymalnym sterowaniem LQR, prowadząc do~przesunięcia punktu równowagi.

Z~tego powodu w~finalnej konfiguracji regulatora PD-LQR zdecydowano się 
na~wyłączenie członu całkującego ($K_i = 0$), pozostawiając strukturę PD-LQR 
(Rys.~\ref{fig:lqr_opt}). Takie rozwiązanie zapewnia:
\begin{itemize}
    \item \textbf{Brak uchybu ustalonego} -- regulator LQR z~wysoką wagą 
    $Q_x = 500$ skutecznie eliminuje błąd pozycji bez potrzeby całkowania.
    \item \textbf{Szybszą odpowiedź dynamiczną} -- brak opóźnienia fazowego 
    wprowadzanego przez integrator.
    \item \textbf{Prostszą strukturę} -- mniej parametrów do~strojenia.
\end{itemize}

Wyniki te wskazują, że w~przypadku dobrze nastrojonego regulatora LQR 
z~wysokimi wagami dla błędu pozycji, dodatkowy człon całkujący nie tylko 
nie poprawia jakości regulacji, ale może ją pogorszyć.

\subsection{Nieliniowe sterowanie predykcyjne (MPC)}

Algorytm MPC (Model Predictive Control) stanowi zaawansowaną metodę sterowania,
która w~odróżnieniu od~LQR, uwzględnia wprost ograniczenia sygnału sterującego
oraz nieliniową dynamikę obiektu \cite{Camacho2007, Rawlings2017}.
Zaimplementowany w~klasie \texttt{MPCController} (plik \texttt{mpc.py})
algorytm rozwiązuje w~każdym kroku symulacji problem optymalizacji dynamicznej
nieliniowej (NMPC). Zasadę działania regulatora MPC ilustruje Rys.~\ref{fig:diagram_mpc}.

\begin{figure}[H]
    \centering
    \begin{tikzpicture}[auto, node distance=1.4cm, >=Stealth]
        % Wejście
        \node [input] (in) {};
        
        % Optymalizator MPC
        \node [draw, fill=orange!15, rectangle, minimum height=3em, minimum width=5em, 
               right=1cm of in, rounded corners=3pt, align=center, font=\small] (opt) {
            Optymalizator\\[-2pt]
            \scriptsize $\min J(\Delta U)$
        };
        
        % Obiekt
        \node [bigblock, right=1.5cm of opt] (plant) {Wahadło};
        
        % Wyjście
        \node [output, right=1.2cm of plant] (out) {};
        
        % Model predykcyjny (poniżej optymalizatora)
        \node [draw, fill=yellow!20, rectangle, minimum height=2em, minimum width=4em,
               below=1cm of opt, rounded corners=2pt, font=\small] (model) {Model $f(x,u)$};
        
        % --- Połączenia ---
        % Wejście -> Optymalizator
        \draw [arrow] (in) -- node[above, font=\scriptsize] {$x_{\mathrm{ref}}$} (opt);
        
        % Optymalizator -> Obiekt
        \draw [arrow] (opt) -- node[above, font=\scriptsize] {$u^*_0$} (plant);
        
        % Obiekt -> Wyjście
        \draw [arrow] (plant) -- node[above, font=\scriptsize] {$x$} (out);
        
        % Sprzężenie zwrotne - ciągła strzałka
        \draw [arrow] ($(plant.east) + (0.3,0)$) -- ++(0,-0.8) -| node[pos=0.25, right, font=\scriptsize] {$x(t)$} ($(opt.south) + (0.6,0)$);
        
        % Predykcja: Optymalizator <-> Model (dwukierunkowe, obok siebie)
        \draw [arrow, dashed] ($(opt.south) + (-0.3,0)$) -- ($(model.north) + (-0.3,0)$);
        \draw [arrow, dashed] ($(model.north) + (0.3,0)$) -- node[right, font=\scriptsize] {$\hat{x}$} ($(opt.south) + (0.3,0)$);
        
        % Ramka MPC
        \begin{scope}[on background layer]
            \node [draw, dashed, rounded corners, fill=blue!5, 
                   fit=(opt) (model), inner sep=8pt, 
                   label={[font=\scriptsize]below:Regulator MPC}] {};
        \end{scope}
    \end{tikzpicture}
    \caption{Schemat blokowy regulatora MPC z~wewnętrznym modelem predykcyjnym.}
    \label{fig:diagram_mpc}
\end{figure}

Zadanie optymalizacji rozwiązywane jest numerycznie metodą SQP (Sequential
Quadratic Programming)~\cite{Nocedal2006} przy użyciu solwera \texttt{SLSQP} z~biblioteki 
\texttt{scipy.optimize}. Wybór tego solwera podyktowany był jego dostępnością 
w~popularnych dystrybucjach środowisk naukowych (Anaconda, pip) oraz zdolnością 
do~obsługi ograniczeń nierównościowych (wymaganych dla saturacji sterowania). Zadanie zdefiniowane jest
następująco:
\begin{equation}
    \min_{\Delta U} J = \sum_{k=1}^{N_{\mathrm{p}}} (\hat{x}_k - x_{\mathrm{ref}})^T Q (\hat{x}_k - x_{\mathrm{ref}}) + R \sum_{k=0}^{N_{\mathrm{c}}-1} (\Delta u_k)^2,
\end{equation}
przy ograniczeniach:
\begin{align}
    \hat{x}_{k+1} &= f(\hat{x}_k, u_k), \quad k=0,\dots,N_{\mathrm{p}}-1 \\
    u_{\mathrm{min}} &\le u_k \le u_{\mathrm{max}}, \\
    u_k &= u_{k-1} + \Delta u_k.
\end{align}
Gdzie:
\begin{itemize}
    \item $N_{\mathrm{p}} = 12$ -- horyzont predykcji,
    \item $N_{\mathrm{c}} = 4$ -- horyzont sterowania. Dla kroków $k \ge N_{\mathrm{c}}$ 
    stosowane jest tzw.~\textit{blokowanie sterowania}, co oznacza, że przyrosty 
    sterowania $\Delta u_k = 0$ i~sygnał sterujący pozostaje stały: $u_k = u_{N_{\mathrm{c}}-1}$. 
    Technika ta redukuje liczbę zmiennych decyzyjnych z~$N_{\mathrm{p}}$ do~$N_{\mathrm{c}}$, 
    przyspieszając obliczenia przy zachowaniu długiego horyzontu predykcji,
    \item $f(\cdot)$ -- nieliniowy model dyskretny obiektu (całkowanie metodą
    Rungego-Kutt 4. rzędu),
    \item $Q = \text{diag}([158.4,\; 40.8,\; 43.4,\; 19.7])$ -- macierz kar stanu,
    \item $R = 0.086$ -- współczynnik kary za~zmianę sterowania ($\Delta u$).
\end{itemize}
Kluczową zaletę MPC, podkreślaną w~pracach \cite{Mills2009} oraz
\cite{Jezierski2017}, jest możliwość bezpośredniego uwzględnienia ograniczeń
(saturacji) już na~etapie wyliczania sterowania, co zapobiega zjawisku
nasycenia elementu wykonawczego, które mogłoby mieć miejsce w~przypadku LQR.

Analiza wykazała, że bezpośrednie przeniesienie macierzy wag $Q$ i $R$
z~regulatora LQR do~sterownika MPC prowadziło do~znaczącego pogorszenia jakości
sterowania (wydłużenie czasu regulacji z~ok. 3s do~ponad 9s). Wynika to z~faktu,
że model MPC, dzięki jawnemu uwzględnieniu ograniczeń sygnału sterującego, pozwala
na~zastosowanie nastaw o~znacznie wyższych wzmocnieniach (większych kar za~błędy stanu),
które w~liniowym regulatorze LQR powodowałyby nasycenie i~potencjalną niestabilność.
Dlatego zdecydowano się na~niezależną optymalizację parametrów obu regulatorów,
aby porównywać ich najlepsze możliwe konfiguracje, a~nie identyczne, ale
nieoptymalne nastawy.

\subsubsection{Dobór horyzontu i wag funkcji celu}
Dla regulatora MPC kluczowym zagadnieniem był dobór horyzontu predykcji oraz
macierzy wag, determinujących zachowanie układu w~stanie nieustalonym.

\begin{figure}[H]
    \centering
    \includegraphics[width=1.0\textwidth]{images/tuning/mpc_1_bad.png}
    \caption{Regulator MPC z~krótkim horyzontem ($N_p=5$, $N_c=2$, 
    $Q=\mathrm{diag}([10, 1, 10, 1])$, $R=0.1$).}
    \label{fig:mpc_bad}
\end{figure}

Początkowe ustawienie zbyt krótkiego horyzontu predykcji ($N_{\mathrm{p}} = 5$,
Rys.~\ref{fig:mpc_bad}) prowadziło do~niestabilności układu zamkniętego. Horyzont 
predykcji był zbyt krótki, aby regulator mógł uwzględnić, że rozpędzając wózek 
w~celu korekcji kąta, nie zdąży wyhamować przed upadkiem wahadła lub osiągnięciem końca toru. Horyzont pięciu kroków (przy
$\Delta t=0{,}1$~s daje zaledwie $0{,}5$~s predykcji) jest niewystarczający, aby uchwycić pełną
dynamikę wahadła i~zaplanować odpowiedni manewr powrotny.

\begin{figure}[H]
    \centering
    \includegraphics[width=1.0\textwidth]{images/tuning/mpc_2_manual.png}
    \caption{Regulator MPC z~ręcznie dobranymi wagami ($N_p=10$, $N_c=3$, 
    $Q=\mathrm{diag}([50, 10, 50, 10])$, $R=0.1$).}
    \label{fig:mpc_manual}
\end{figure}

Zwiększenie horyzontu do~$N_{\mathrm{p}}=10$ w~ramach korekty ręcznej 
(Rys.~\ref{fig:mpc_manual}) ustabilizowało proces. Dłuższy horyzont umożliwił 
predykcję na~$1{,}0$~s, co pozwoliło na~antycypację skutków podejmowanych działań 
sterujących. Dodatkowa manipulacja wagami $Q$ pozwoliła na~uzyskanie poprawnego 
sterowania, jednak odpowiedź dynamiczna była powolna, a~przebiegi wykazywały
przeregulowania.

\begin{figure}[H]
    \centering
    \includegraphics[width=1.0\textwidth]{images/tuning/mpc_3_opt.png}
    \caption{Zoptymalizowany regulator MPC ($N_p=12$, $N_c=4$, 
    $Q=\mathrm{diag}([158.4, 40.8, 43.4, 19.7])$, $R=0.086$).}
    \label{fig:mpc_opt}
\end{figure}

Automatyzacja procesu strojenia przy użyciu skryptu \texttt{tune\_mpc.py} pozwoliła
na~znalezienie kompromisu między długością horyzontu a~wagami (Rys.~\ref{fig:mpc_opt}).
Algorytm optymalizacyjny wskazał $N_{\mathrm{p}}=12$ (co odpowiada $1{,}2$~s predykcji) 
jako optimum dla tego modelu dyskretnego, zapewniając stabilność przy akceptowalnym czasie obliczeń. Zoptymalizowane
wagi $Q$ znacząco różnią się od~intuicyjnych proporcji -- wysoka kara za~kąt 
($Q_{\theta}=158.4$) w~połączeniu z~niską karą za~zmianę sterowania ($R=0.086$)
zapewnia szybką, ale gładką stabilizację.

\subsection{MPC z~rozszerzonym wskaźnikiem jakości (MPC-J2)}

Zaimplementowano sterownik \texttt{MPCControllerJ2} jako wariant badawczy algorytmu predykcyjnego
(plik \texttt{mpc\_J2.py}). Jego struktura jest
zbliżona do~podstawowego MPC, jednak funkcja kosztu została rozbudowana
o~dodatkowy składnik karzący bezwzględną wartość sygnału sterującego
(energię), a~nie tylko jego przyrosty.

Zmodyfikowana funkcja celu przyjmuje postać:
\begin{equation}
    J = \sum_{k=1}^{N_{\mathrm{p}}} (x_k - x_{\mathrm{ref}})^T Q (x_k - x_{\mathrm{ref}}) \;+\; R_{\Delta} \sum_{k=0}^{N_{\mathrm{c}}-1} (\Delta u_k)^2 \;+\; R_{\mathrm{abs}} \sum_{k=0}^{N_{\mathrm{c}}-1} (u_k)^2.
\end{equation}
Wprowadzenie parametru $R_{\mathrm{abs}}$ pozwala na~bezpośrednie minimalizowanie
zużycia energii sterowania, co jest podejściem powszechnie stosowanym
w~praktycznych implementacjach algorytmów predykcyjnych \cite{Camacho2007, Rawlings2017}.
Ograniczenie amplitudy sygnału sterującego nie tylko redukuje wydatek energetyczny
(istotny w~aplikacjach mobilnych), ale także zmniejsza obciążenie mechaniczne
elementów wykonawczych, co wpływa na~żywotność napędu.

W~badaniach przyjęto zoptymalizowane wagi: $q_{\theta}=40.0$,
$q_x=40.0$, $q_{\dot{\theta}}=5.0$, $q_{\dot{x}}=5.0$ (elementy macierzy diagonalnej $Q$),
przy czym $R_{\Delta}=0.0001$ i~$R_{\mathrm{abs}}=0$ dla wariantu priorytetyzującego 
jakość regulacji.

\subsubsection{Analiza wpływu kary za energię}
W~przypadku wariantu MPC-J2 analizowano nieliniowy wpływ parametru $R_{\mathrm{abs}}$
na~zachowanie układu. Eksperymenty przeprowadzono przy stałych wagach stanu
($q_{\theta}=40$, $q_x=40$, $q_{\dot{\theta}}=5$, $q_{\dot{x}}=5$), zmieniając 
jedynie wartość kary za~bezwzględną wartość sterowania.

\begin{figure}[H]
    \centering
    \includegraphics[width=1.0\textwidth]{images/tuning/mpcJ2_1_bad.png}
    \caption{Regulator MPC-J2 z~wysoką karą za~energię ($R_{\mathrm{abs}}=10$).}
    \label{fig:mpcj2_bad}
\end{figure}

Przyjęcie zbyt dużej wartości kary za~sterowanie bezwzględne ($R_{\mathrm{abs}}=10$,
Rys.~\ref{fig:mpcj2_bad}) spowodowało, że regulator wykazywał tendencję do~pasywności.
Funkcja kosztu penalizowała każdy niuton siły na~tyle intensywnie, że wartość funkcji celu
faworyzowała dopuszczenie do~upadku wahadła kosztem uniknięcia wysokiego wydatku energetycznego. W~efekcie układ
nie był w~stanie ustabilizować się -- wahadło przewracało się, ponieważ koszt 
energetyczny utrzymania go w~pionie przewyższał zysk wynikający z~małego błędu 
kątowego w~funkcji celu.

\begin{figure}[H]
    \centering
    \includegraphics[width=1.0\textwidth]{images/tuning/mpcJ2_2_manual.png}
    \caption{Regulator MPC-J2 z~ręcznie zmniejszoną karą ($R_{\mathrm{abs}}=1$).}
    \label{fig:mpcj2_manual}
\end{figure}

Stopniowe, ręczne zmniejszanie parametru $R_{\mathrm{abs}}$ do~wartości $1.0$
(Rys.~\ref{fig:mpcj2_manual}) pozwoliło na~początkową stabilizację, jednak układ
ostatecznie ulegał destabilizacji. Regulator działał zbyt zachowawczo -- oszczędzając
energię, pozwalał wózkowi na~zbyt duży dryf od~pozycji zadanej. Gdy błąd pozycji
narastał, próba korekcji wymagała gwałtownego ruchu wózka, co z~kolei destabilizowało
wahadło. Taki scenariusz ilustruje typowy problem źle dostrojonego kompromisu
między jakością regulacji a~oszczędnością energii.

\begin{figure}[H]
    \centering
    \includegraphics[width=1.0\textwidth]{images/tuning/mpcJ2_3_opt.png}
    \caption{Zoptymalizowany regulator MPC-J2 ($R_{\mathrm{abs}}=0$).}
    \label{fig:mpcj2_opt}
\end{figure}

Algorytm optymalizacyjny wskazał, że dla tego zadania najlepszym rozwiązaniem jest
całkowite wyłączenie kary za~bezwzględną wartość sterowania ($R_{\mathrm{abs}}=0$,
Rys.~\ref{fig:mpcj2_opt}). Pozostała jedynie kara za~przyrosty sterowania 
$R_{\Delta}=0.0001$, która zapewnia gładkość sygnału sterującego bez ograniczania
zdolności regulatora do~szybkiej reakcji. Uzyskano optymalny kompromis, w~którym układ
stabilizuje się szybko, a~sterowanie pozbawione jest zbędnych oscylacji
wysokoczęstotliwościowych.

\subsection{Regulator rozmyty wspomagany LQR (Fuzzy-LQR)}

Ostatnim zbadanym układem jest sterownik hybrydowy \texttt{TSFuzzyController}
(plik \texttt{fuzzy\_lqr.py}), łączący liniowy regulator LQR z~systemem
wnioskowania rozmytego typu Takagi-Sugeno (T-S). Koncepcja ta, opisana szerzej
w~\cite{Nguyen2024} oraz \cite{Roose2017}, ma na~celu adaptację wzmocnień regulatora
w~zależności od~punktu pracy, co pozwala na~silniejszą reakcję w~przypadku
dużych odchyleń od~pionu. Strukturę regulatora przedstawiono na~Rys.~\ref{fig:diagram_fuzzy}.

\begin{figure}[H]
    \centering
    \begin{tikzpicture}[auto, node distance=1.4cm, >=Stealth]
        % Wejście
        \node [input] (in) {};
        
        % Suma błędu
        \node [sum, right=0.8cm of in] (sum_e) {\tiny $+$};
        
        % Blok LQR (góra)
        \node [block, right=1.8cm of sum_e, yshift=0.7cm] (lqr) {LQR $(-K)$};
        
        % Blok Fuzzy (dół)
        \node [draw, fill=purple!15, rectangle, minimum height=2.2em, minimum width=4em,
               right=1.8cm of sum_e, yshift=-0.7cm, rounded corners=3pt, font=\small] (fuzzy) {Fuzzy T-S};
        
        % Suma sterowania
        \node [sum, right=1.5cm of lqr, yshift=-0.7cm] (sum_u) {\tiny $+$};
        
        % Saturacja i obiekt
        \node [block, right=0.7cm of sum_u, minimum width=2.5em] (sat) {sat};
        \node [bigblock, right=0.8cm of sat] (plant) {Wahadło};
        
        % Wyjście
        \node [output, right=1.2cm of plant] (out) {};
        
        % --- Połączenia ---
        % Wejście -> Suma błędu
        \draw [arrow] (in) -- node[above, font=\scriptsize] {$x_{\mathrm{ref}}$} (sum_e);
        
        % Suma błędu -> rozgałęzienie do LQR i Fuzzy
        \draw [line] (sum_e.east) -- ++(0.5,0) coordinate (branch);
        \draw [arrow] (branch) |- (lqr.west);
        \draw [arrow] (branch) |- (fuzzy.west);
        
        % LQR i Fuzzy -> Suma sterowania
        \draw [arrow] (lqr.east) -| (sum_u.north);
        \draw [arrow] (fuzzy.east) -| (sum_u.south);
        
        % Reszta toru
        \draw [arrow] (sum_u) -- (sat);
        \draw [arrow] (sat) -- node[above, font=\scriptsize] {$u$} (plant);
        \draw [arrow] (plant) -- node[above, font=\scriptsize] {$x$} (out);
        
        % Sprzężenie zwrotne - ciągła strzałka
        \draw [arrow] ($(plant.east) + (0.3,0)$) -- ++(0,-1.8) -| (sum_e.south);
        \node [below=3pt of sum_e.south, font=\tiny] {$-$};
    \end{tikzpicture}
    \caption{Schemat blokowy regulatora Fuzzy-LQR z~równoległą strukturą hybrydową.}
    \label{fig:diagram_fuzzy}
\end{figure}

Sygnał sterujący jest sumą:
\begin{equation}
    u(t) = u_{\mathrm{LQR}}(t) + u_{\mathrm{Fuzzy}}(t).
\end{equation}
Część rozmyta $u_{\mathrm{Fuzzy}}(t)$ wykorzystuje bazę reguł postaci:
\begin{quote}
    JEŚLI $e_\theta$ jest $A_i$ ORAZ $\dot{\theta}$ jest $B_i$ ... TO $u_i = f_i(x)$,
\end{quote}
gdzie $f_i(x)$ jest liniową funkcją stanu (lokalny regulator
liniowy). Zastosowano funkcje przynależności trójkątne dla zmiennych stanu,
dzieląc przestrzeń na~obszary ,,Mały błąd'' i~,,Duży błąd''.
Baza wiedzy składa się z~16 reguł ($2^4$ kombinacji dla 4 zmiennych stanu).
Wyjście sterownika obliczane jest jako średnia ważona:
\begin{equation}
    u_{\mathrm{Fuzzy}} = G \cdot \frac{\sum_{i=1}^{16} w_i(x) \cdot u_i}{\sum_{i=1}^{16} w_i(x)},
\end{equation}
gdzie $w_i$ to stopień aktywacji $i$-tej reguły, a~$G = 0.36$ to globalne
wzmocnienie skalujące (po optymalizacji).

Zastosowany mechanizm ,,Gain Scheduling'' pozwala na:
\begin{enumerate}
    \item Zachowanie łagodnej charakterystyki LQR w~pobliżu punktu równowagi
    (małe wzmocnienia w~regułach dla ,,Małych błędów'').
    \item Zwiększenie sztywności układu w~sytuacjach krytycznych (duże
    wzmocnienia zdefiniowane w~zmiennej \texttt{F\_rules} dla ,,Dużych
    błędów'').
\end{enumerate} 
Takie podejście pozwala na~rozszerzenie obszaru stabilności regulatora
w~porównaniu do~klasycznego LQR, co potwierdzają wyniki badań w~pracy
\cite{Nguyen2024}.

\subsubsection{Dobór reguł i funkcji przynależności}
Strojenie rozmytego regulatora Fuzzy-LQR jest zadaniem złożonym ze względu na~dużą
liczbę parametrów definiujących bazę reguł i~funkcje przynależności.

\begin{figure}[H]
    \centering
    \includegraphics[width=1.0\textwidth]{images/tuning/fuzzy_1_bad.png}
    \caption{Regulator Fuzzy-LQR z~wąskimi funkcjami przynależności 
    (zakres ,,mały błąd'' dla $\theta$: $[-0.02, 0.02]$~rad).}
    \label{fig:fuzzy_bad}
\end{figure}

Błędne zdefiniowanie zbyt wąskich funkcji przynależności dla strefy ,,małego błędu''
(Rys.~\ref{fig:fuzzy_bad}) skutkowało gwałtownym przełączaniem się regulatora 
na~reguły o~wysokich wzmocnieniach (drgania przełączeniowe). Zakres $[-0.02, 0.02]$~rad 
oznacza, że już przy wychyleniu wahadła o~około $1^{\circ}$ regulator przeskakiwał 
z~trybu o~niskich wzmocnieniach (LQR) na~tryb o~wysokich wzmocnieniach (reguły rozmyte), 
a~przy powrocie do~pionu natychmiast wracał. Prowadziło to do~silnych drgań wokół punktu równowagi,
co jest zjawiskiem niepożądanym w~rzeczywistych układach napędowych ze~względu na~
zużycie mechaniczne i~hałas.

\begin{figure}[H]
    \centering
    \includegraphics[width=1.0\textwidth]{images/tuning/fuzzy_2_manual.png}
    \caption{Regulator Fuzzy-LQR z~ręcznie dobranymi parametrami 
    ($F_{\theta}=20$, $F_{\dot{\theta}}=5$, $F_x=10$, $F_{\dot{x}}=2$, 
    zakres: $[-0.2, 0.2]$~rad).}
    \label{fig:fuzzy_manual}
\end{figure}

Opierając się na~literaturze \cite{Nguyen2024}, dobrano ręcznie szerokości trójkątnych
funkcji przynależności tak, aby przejście między strefami było płynne 
(Rys.~\ref{fig:fuzzy_manual}). Rozszerzenie zakresu do~$[-0.2, 0.2]$~rad wyeliminowało
drgania przełączeniowe, jednak wzmocnienia reguł były zbyt słabe ($F_{\theta}=20$ vs. optymalne $100.0$).
Układ uzyskał stabilność asymptotyczną, jednak nie wykorzystywał w~pełni potencjału 
szybkiej reakcji na~duże zakłócenia, działając zachowawczo.

\begin{figure}[H]
    \centering
    \includegraphics[width=1.0\textwidth]{images/tuning/fuzzy_3_opt.png}
    \caption{Zoptymalizowany regulator Fuzzy-LQR ($F_{\theta}=100.0$, 
    $F_{\dot{\theta}}=5.27$, $F_x=19.82$, $F_{\dot{x}}=19.25$, $G=0.36$).}
    \label{fig:fuzzy_opt}
\end{figure}

Ostatecznie, dedykowany skrypt \texttt{tune\_fuzzy\_lqr.py} posłużył do~optymalizacji
wag pojedynczych reguł oraz parametrów kształtu funkcji przynależności 
(Rys.~\ref{fig:fuzzy_opt}). Algorytm wyznaczył znacznie wyższą wartość wzmocnienia
dla kąta ($F_{\theta}=100.0$), przy jednoczesnym zmniejszeniu globalnego wzmocnienia
($G=0.36$). Uzyskano nieliniową powierzchnię sterowania, która łączy zalety miękkiego 
sterowania LQR w~pobliżu zera z~maksymalną siłą reakcji przy dużych wychyleniach.

\paragraph{Trudności w~strojeniu regulatora rozmytego}
Strojenie regulatora Fuzzy-LQR okazało się zadaniem znacznie bardziej wymagającym 
niż w~przypadku pozostałych badanych algorytmów. Wynika to z~kilku czynników:
\begin{itemize}
    \item \textbf{Wysoka wymiarowość przestrzeni parametrów} --- dla 16 reguł, 
    z~których każda definiuje 4 wzmocnienia, plus parametry funkcji przynależności 
    i~wzmocnienie globalne, łączna liczba stopni swobody sięga kilkudziesięciu.
    \item \textbf{Silne sprzężenia między parametrami} --- zmiana jednego wzmocnienia 
    wpływa na~zachowanie całej bazy reguł poprzez mechanizm interpolacji rozmytej, 
    co utrudnia intuicyjne strojenie metodą prób i~błędów.
    \item \textbf{Wrażliwość na~warunki początkowe optymalizacji} --- algorytm 
    ewolucji różnicowej wielokrotnie zbiegał do~różnych minimów lokalnych, 
    dając rozwiązania o~znacząco odmiennych charakterystykach.
    \item \textbf{Zależność od~scenariusza testowego} --- parametry zoptymalizowane 
    dla warunków nominalnych mogą dawać gorsze wyniki przy zakłóceniach i~odwrotnie.
\end{itemize}
Z~powyższych powodów proces strojenia regulatora rozmytego wymagał wielokrotnego 
uruchamiania optymalizacji z~różnymi punktami startowymi oraz manualnej weryfikacji 
uzyskanych rozwiązań. Stanowi to istotną wadę praktyczną w~porównaniu z~regulatorami 
LQR czy MPC, gdzie przestrzeń parametrów jest znacznie mniejsza i~bardziej interpretowalna.
