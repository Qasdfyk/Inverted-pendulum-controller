\section{Algorytmy sterowania}

W~niniejszym rozdziale przedstawiono szczegółowy opis algorytmów sterowania
zaimplementowanych i~przeanalizowanych w~ramach pracy. Kod regulatorów został
zrealizowany w~języku Python w~postaci klas dziedziczących wspólną strukturę,
co zapewnia modularność i~łatwą wymienność w~pętli symulacyjnej. Każdy
regulator wyznacza sygnał sterujący $u(k)$ (siłę przyłożoną do wózka)
na~podstawie aktualnego, dyskretnego wektora stanu
$x(k) = [x_1(k), x_2(k), x_3(k), x_4(k)]^T$ oraz wartości zadanych
$x_{\mathrm{ref}}$. Przyjęto następujące przyporządkowanie zmiennych stanu, 
gdzie $t = k \cdot T$ ($T$ oznacza okres próbkowania):
\begin{itemize}
    \item $x_1(k)$ -- kąt wychylenia wahadła ($\theta$),
    \item $x_2(k)$ -- prędkość kątowa wahadła (odpowiednik $\dot{\theta}$),
    \item $x_3(k)$ -- pozycja wózka ($x$),
    \item $x_4(k)$ -- prędkość liniowa wózka (odpowiednik $\dot{x}$).
\end{itemize}

W~literaturze problem sterowania wahadłem odwróconym jest szeroko omawiany
jako klasyczny problem testowy dla metod sterowania liniowego i~nieliniowego
\cite{Prasad2014, Nguyen2024}. Poniżej opisano teoretyczne podstawy oraz szczegóły
implementacyjne zbadanych struktur sterowania.

\subsection{Równoległy regulator PID}

Pierwszym zaimplementowanym układem jest regulator o~strukturze równoległej,
wykorzystujący klasyczne sprzężenie zwrotne typu PID~\cite{Prasad2014}. 
Zastosowano strukturę równoległą, w~której
całkowity sygnał sterujący jest sumą reakcji na~błąd kąta oraz błąd pozycji.

\begin{figure}[H]
    \centering
    \begin{tikzpicture}[auto, node distance=1.5cm, >=Stealth]
        % Wejścia (po lewej)
        \node [input] (in_theta) {};
        \node [below=1.2cm of in_theta, input] (in_x) {};
        
        % Sumy błędów
        \node [sum, right=0.8cm of in_theta] (sum_theta) {\tiny $+$};
        \node [sum, right=0.8cm of in_x] (sum_x) {\tiny $+$};
        
        % Regulatory PD
        \node [block, right=0.8cm of sum_theta] (pd_theta) {$\mathrm{PID}_\theta$};
        \node [block, right=0.8cm of sum_x] (pd_x) {$\mathrm{PID}_x$};
        
        % Suma sterowania (między nimi)
        \node [sum, right=1.8cm of pd_theta, yshift=-0.6cm] (sum_u) {\tiny $+$};
        
        % Saturacja i obiekt
        \node [bigblock, right=1.5cm of sum_u] (plant) {Wahadło};
        
        % Wyjście
        \node [output, right=1.2cm of plant] (out) {};
        
        % --- Połączenia ---
        % Wejścia do sum
        \draw [arrow] (in_theta) -- node[above, font=\scriptsize] {$\theta_{\mathrm{ref}}$} (sum_theta);
        \draw [arrow] (in_x) -- node[above, font=\scriptsize] {$x_{\mathrm{ref}}$} (sum_x);
        
        % Sumy do PD
        \draw [arrow] (sum_theta) -- (pd_theta);
        \draw [arrow] (sum_x) -- (pd_x);
        
        % PD do sumy sterowania (rozdzielone wejścia: góra i dół)
        \draw [arrow] (pd_theta.east) -| node[above, font=\scriptsize] {$\mathrm{u_{\theta}}$} (sum_u.north);
        \draw [arrow] (pd_x.east) -| node[below, font=\scriptsize] {$\mathrm{u_x}$} (sum_u.south);
        
        % Suma -> sat -> plant -> wyjście
        \draw [arrow] (sum_u) -- node[above, font=\scriptsize] {$\mathrm{u}$} (plant);
        \draw [arrow] (plant) -- node[above, font=\scriptsize] {$\mathrm{\theta,x}$} (out);
        
        % Sprzężenie zwrotne (na dole) - ciągła linia
        \draw [arrow] ($(plant.east) + (0.3,0)$) -- ++(0,+1.8) -| (sum_theta.north);
        \draw [arrow] ($(plant.east) + (0.3,0)$) -- ++(0,-1.8) -| (sum_x.south);
        
        % Znak minus na sumach
        \node [above=3pt of sum_theta.north, font=\tiny] {$-$};
        \node [below=3pt of sum_x.south, font=\tiny] {$-$};
    \end{tikzpicture}
    \caption{Schemat blokowy regulatora PID o~strukturze równoległej.}
    \label{fig:diagram_pdpd}
\end{figure}

Prawo sterowania w~chwili $k$ wyraża się sumą sygnałów z~dwóch pętli regulacji:
\begin{equation}
    u(k) =  u_{\theta}(k) + u_{x}(k).
\end{equation}
W implementacji cyfrowej algorytm został zdyskretyzowany z~okresem próbkowania $T=0{,}1$~s. Dla uchybów regulacji zdefiniowanych jako $e_{\theta}(k) = \theta_{\mathrm{ref}} - x_1(k)$ oraz $e_x(k) = x_{\mathrm{ref}} - x_3(k)$, człon proporcjonalny realizowany jest jako proste wzmocnienie uchybu. Człon całkujący przybliżono numerycznie metodą prostokątów (sumowanie uchybów w~czasie), wyposażając go dodatkowo w~mechanizm \textit{anti-windup}, który ogranicza wartość całki do zakresu $\pm 5{,}0$. 

Istotną modyfikację wprowadzono w~członie różniczkującym. Zamiast klasycznego obliczania różnicy uchybów ($\frac{\Delta e}{\Delta t}$), co może prowadzić do wzmacniania szumów kwantyzacji, zastosowano technikę różniczkowania sygnału wyjściowego (ang. \textit{derivative on measurement}). Wykorzystując fakt, że przy stałych wartościach zadanych pochodna uchybu odpowiada ujemnej prędkości obiektu ($\dot{e} \approx -\dot{x}_{\mathrm{pomiar}}$), w~równaniach sterowania użyto bezpośrednio zmiennych stanu prędkości. Ostateczne równania różnicowe regulatora przyjmują postać:
\begin{align}
    u_{\theta}(k) &= K_{p,\theta} e_{\theta}(k) + K_{i,\theta} \sum_{j=0}^{k} e_{\theta}(j) T + K_{d,\theta} \left( -x_2(k) \right), \label{eq:pd_theta} \\
    u_{x}(k) &= K_{p,x} e_x(k) + K_{i,x} \sum_{j=0}^{k} e_x(j) T + K_{d,x} \left( -x_4(k) \right). \label{eq:pd_x}
\end{align}
W powyższych zależnościach $x_2(k)$ oznacza prędkość kątową wahadła, a~$x_4(k)$ prędkość liniową wózka w~bieżącym kroku symulacji. Zastosowanie takiej struktury pozwala na uzyskanie gładszego przebiegu sygnału sterującego oraz eliminuje gwałtowne skoki sterowania przy zmianach wartości zadanej (tzw. \textit{derivative kick}).

\subsubsection{Proces doboru nastaw oraz analiza PID}
Dobór nastaw dla regulatora PID został zrealizowany wieloetapowo.
Wstępny dobór nastaw przeprowadzono metodą eksperymentalną prób i błędów, 
jednak nie pozwoliła ona na uzyskanie zadowalających wskaźników jakości
, gdyż obiekt wahadła na wózku wymaga bardzo precyzyjnych nastaw, 
a dobór parametrów metodą prób i błędów jest czasochłonny i nie gwarantuje optymalnych rezultatów. 

Wstępne próby doboru metodą prób i~błędów (Rys.~\ref{fig:combined_pdpd}, linia czerwona) 
wskazały, że początkowe wzmocnienia ($K_{p,\theta}=-60$ dla kąta) nie zapewniały stabilności,
widoczne są oscylacje rosnące na trjektorii pozycji wózka.
Następnie aby wyeliminować subiektywność strojenia ręcznego, zastosowano metodę 
przeszukiwania siatki (ang.~\textit{grid search}).
Przeszukano siatkę wartości dla każdego z~sześciu parametrów.
Wyniki dla najlepszej konfiguracji z siatki przedstawiono linią niebieską na Rys.~\ref{fig:combined_pdpd}.
Charakteryzują się one znacznie wyższymi wzmocnieniami ($K_{p,\theta}=-95$, $K_{p,x}=-60$).
Jednak w tym przypadku jakość sygnału sterującego była niezadowalająca.
Ostateczne strojenie (linia zielona) wykonano ręcznie, biorąc pod uwagę wyniki z siatki.
Ostateczny regulator PID charakteryzuje się szybkim osiąganiem wartości zadanej oraz wysoką jakością sygnału sterującego. 
Ujemne znaki nastaw regulatora PID wynikają z~przyjętej konwencji układu współrzędnych i~zwrotu siły.
Analiza porównawcza wykazuje, że redukcja wzmocnienia $K_{p,\theta}$ w~strojeniu ręcznym, mimo zmniejszenia sztywności układu, jest kluczowa dla ograniczenia oscylacji.
Pozwala to na~uzyskanie gładkiego przebiegu sygnału sterującego, co jest pożądane ze~względu na~trwałość elementów mechanicznych.

\begin{table}[H]
    \centering
    \caption{Zestawienie parametrów strojenia regulatora PID-PID.}
    \label{tab:params_pid}
    \begin{tabular}{|l|c|c|}
        \hline
        \textbf{Wariant} & \textbf{Tor Kąta ($K_p, K_i, K_d$)} & \textbf{Tor Pozycji ($K_p, K_i, K_d$)} \\
        \hline
        Wstępne & $-60, -2, -9$ & $-2, -1, -1$ \\ \hline
        Siatka & $-95, 0, -14$ & $-16, 0, -14$ \\ \hline
        Optymalne & $-40, -1, -8$ & $-1, -0{,}1, -3$ \\ \hline
    \end{tabular}
\end{table}

W~implementacji programowej przyjęto następujące nastawy:
\begin{itemize}
    \item Tor stabilizacji kąta: $K_{p,\theta} = -40,0$, $K_{i,\theta} = -1,0$, $K_{d,\theta} = -8,0$.
    \item Tor pozycji: $K_{p,x} = -1,0$, $K_{i,x} = -0,1$, $K_{d,x} = -3,0$.
\end{itemize}

\begin{figure}[H]
    \centering
    \includegraphics[width=1.0\textwidth]{images/tuning/combined/combined_pdpd.png}
    \caption{Zestawienie przebiegów regulatora PID.}
    \label{fig:combined_pdpd}
\end{figure}






\subsection{Układ hybrydowy PID-LQR}

Regulator liniowo-kwadratowy (LQR) stanowi fundamentalną metodę sterowania optymalnego
dla systemów liniowych wielowymiarowych MIMO~\cite{Jezierski2017}. 
Ze względu na cyfrową realizację sterowania (czas dyskretny), w~pracy zastosowano 
wariant dyskretny algorytmu (DLQR -- \textit{Discrete Linear Quadratic Regulator}).
W~odróżnieniu od~regulatorów PID, które wymagają empirycznego doboru wzmocnień 
dla każdej zmiennej stanu, LQR wyznacza optymalny wektor wzmocnień automatycznie 
na~podstawie dyskretnego modelu obiektu oraz macierzy wag $Q$ i~$R$, definiujących 
kompromis między jakością regulacji a~zużyciem energii.

\begin{figure}[H]
    \centering
    \begin{tikzpicture}[auto, node distance=1.4cm, >=Stealth]
        % Wejście
        \node [input] (in) {};
        
        % Suma błędu LQR
        \node [sum, right=0.8cm of in] (sum_lqr) {\tiny $+$};
        
        % Blok LQR
        \node [block, right=0.8cm of sum_lqr] (lqr) {DLQR};
        
        % Suma sterowania
        \node [sum, right=1.2cm of lqr] (sum_u) {\tiny $+$};
        
        % Saturacja i obiekt
        \node [bigblock, right=1.5cm of sum_u] (plant) {Wahadło};
        
        % Wyjście
        \node [output, right=1.2cm of plant] (out) {};
        
        % PD pozycji (poniżej)
        \node [sum, below=1.4cm of sum_lqr] (sum_pd) {\tiny $+$};
        \node [block, right=0.8cm of sum_pd] (pd_pos) {$\mathrm{PID}_x$};
        
        % --- Połączenia ---
        % Wejście LQR
        \draw [arrow] (in) -- node[above, font=\scriptsize] {$x_{\mathrm{ref}}$} (sum_lqr);
        \draw [arrow] (sum_lqr) -- (lqr);
        \draw [arrow] (lqr) -- node[above, font=\scriptsize] {$u_{\mathrm{LQR}}$} (sum_u);
        
        % PD -> suma sterowania
        \draw [arrow] (pd_pos.east) -| node[above left, font=\scriptsize] {$u_{\mathrm{PID}}$} (sum_u.south);
        
        % Reszta toru
        \draw [arrow] (sum_u) -- node[above, font=\scriptsize] {$u$} (plant);
        \draw [arrow] (plant) -- node[above, font=\scriptsize] {$x$} (out);
        
        % Wejście referencyjne PD
        \node [input, left=0.8cm of sum_pd] (in_pd) {};
        \draw [arrow] (in_pd) -- node[above, font=\scriptsize] {$x_{\mathrm{ref}}$} (sum_pd);
        \draw [arrow] (sum_pd) -- (pd_pos);
        
        % Sprzężenie zwrotne LQR (na dole) - ciągła strzałka
        \draw [arrow] ($(plant.east) + (0.3,0)$) -- ++(0,1.4) -| (sum_lqr.north);
        \node [above=3pt of sum_lqr.north, font=\tiny] {$-$};
        
        % Sprzężenie zwrotne PD (x) - od głównej linii sprzężenia
        \draw [arrow] ($(plant.east) + (0.3,0)$) -- ++(0,-3.4) -| (sum_pd.south);
        \node [below=3pt of sum_pd.south, font=\tiny] {$-$};
        
        % Ramka LQR
        \begin{scope}[on background layer]
            \node [draw, dashed, rounded corners, fill=green!5, 
                   fit=(sum_lqr) (lqr), inner sep=6pt] {};
        \end{scope}
    \end{tikzpicture}
    \caption{Schemat blokowy hybrydowego regulatora PID-LQR.}
    \label{fig:diagram_pdlqr}
\end{figure}

Problem sterowania optymalnego polega na~znalezieniu prawa sterowania 
$u(k) = -K x(k)$, które minimalizuje wskaźnik jakości zdefiniowany jako nieskończona suma:
\begin{equation}
    J = \sum_{k=0}^{\infty} \left( x(k)^T Q x(k) + u(k)^T R u(k) \right),
\end{equation}
gdzie $Q \succeq 0$ jest macierzą wag stanu, a~$R > 0$ wagą sterowania. 
Punktem wyjścia jest model dyskretny układu: $x(k+1) = A_d x(k) + B_d u(k)$, 
gdzie macierze $A_d$ i~$B_d$ wyznaczono z~modelu ciągłego metodą ekstrapolatora 
rzędu zerowego (ZOH) dla czasu próbkowania $T=0{,}1$~s.

Optymalna macierz wzmocnień $K$ wyznaczana jest na~podstawie rozwiązania 
Dyskretnego Algebraicznego Równania Riccatiego (DARE):
\begin{equation}
    P = A_d^T P A_d - A_d^T P B_d \left( R + B_d^T P B_d \right)^{-1} B_d^T P A_d + Q.
\end{equation}
Macierz $P$ jest unikalnym, symetrycznym i~dodatnio określonym rozwiązaniem tego równania. 
Ostateczna postać wzmocnienia regulatora wyraża się wzorem:
\begin{equation}
    K = \left( R + B_d^T P B_d \right)^{-1} B_d^T P A_d.
\end{equation}

Zaimplementowano sterowanie oparte na~pełnym wektorze stanu, 
wspomagane dodatkowym członem PID dla uchybu pozycji, co tworzy strukturę hybrydową 
opisaną m.in. w~\cite{Prasad2014} oraz \cite{Nguyen2024}. 
Schemat blokowy tego układu przedstawiono na~Rys.~\ref{fig:diagram_pdlqr}.

W~zaimplementowanym rozwiązaniu, sygnał sterujący w~chwili $k$
składa się z~dwóch komponentów:
\begin{equation}
    u(k) = u_{\mathrm{LQR}}(k) + u_{\mathrm{PID,pos}}(k).
\end{equation}
Składnik LQR realizuje stabilizację wokół punktu pracy:
\begin{equation}
    u_{\mathrm{LQR}}(k) = -K \cdot (x(k) - x_{\mathrm{ref}}).
\end{equation}
Dodatkowy człon PID na~pętli pozycji (zrealizowany analogicznie do równań przedstawionych w~sekcji dotyczącej regulatora PID) 
ma na~celu poprawę śledzenia skokowych zmian wartości zadanej
$x_{\mathrm{ref}}$ oraz eliminację uchybu ustalonego wynikającego z zakłóceń i niedokładności modelu.
\cite{Prasad2014}.

\subsubsection{Dobór wag macierzy Q i R}
Dobór wartości elementów macierzy wag $Q$ i~$R$ zrealizowano w~sposób
wieloetapowy, ewoluując od~konfiguracji bazowej do~rozwiązania
zoptymalizowanego. W~pierwszej kolejności zbadano zachowanie układu dla nastaw
jednostkowych, co pozwoliło zidentyfikować podstawowe ograniczenia stabilności.
Następnie, wykorzystując wiedzę o~dynamice obiektu, przeprowadzono strojenie
ręczne, by~w~końcowym etapie wykorzystać algorytm przeszukiwania po siatce do~finalnego
dostrojenia parametrów i~maksymalizacji wskaźnika jakości regulacji.

W~pierwszej fazie (Rys.~\ref{fig:combined_pdlqr}, linia czerwona) przyjęcie jednostkowej macierzy diagonalnej $Q=I$ oraz $R=1$ okazało się niewystarczające.
Takie podejście traktuje wszystkie uchyby (kąta w~radianach i~pozycji w~metrach) jednakowo, co jest fizycznie nieuzasadnione, gdyż dopuszczalne odchylenia kąta są znacznie mniejsze niż pozycji.
Skutkowało to wolną reakcją regulatora, który nie priorytetyzował stabilizacji wahadła, pozwalając na~duże wychylenia.

Następnie przeprowadzono strojenie ręczne oparte na~regule Brysona (linia niebieska).
Założono, że maksymalne dopuszczalne wychylenie kąta wynosi ok. $0{,}2$~rad, co zgodnie ze~wzorem $Q_{ii} = 1/x_{i,\text{max}}^2$ dało wagę $Q_{\theta}=25$.
Jednocześnie zwiększono karę za~sterowanie ($R=10$), aby uniknąć nasycenia sygnału.
Zabieg ten znacząco poprawił sztywność wahadła, jednak czas ustalania wciąż nie był satysfakcjonujący.

W~ostatnim etapie zastosowano optymalizację numeryczną metodą przeszukiwania siatki (linia zielona).
Wyniki wskazały na~konieczność drastycznego zwiększenia kary za~błąd kąta ($Q_{\theta}=200$) oraz prędkości ($Q_{\dot{\theta}}=3, Q_{\dot{x}}=40$) przy jednoczesnym obniżeniu kary za~sterowanie do~$R=1$.
Taka konfiguracja wymusza bardzo agresywną reakcję układu na~nawet najmniejsze odchylenia, wykorzystując pełną dynamikę napędu do~szybkiej stabilizacji.
Należy zauważyć, że optymalizator wyznaczył strategię dość intuicyjną: 
kara głównie kąt ($Q_{\theta}$), zamiast kłaść nacisk na~pozycję ($Q_x$), 
co zmusza obiekt do~szybkich korekt stabilizujących wahadło.
\begin{table}[H]
    \centering
    \caption{Zestawienie parametrów strojenia regulatora PID-LQR.}
    \label{tab:params_pdlqr}
    \begin{tabular}{|l|c|c|c|}
        \hline
        \textbf{Wariant} & \textbf{Macierz Q} & \textbf{Waga R} & \textbf{PID ($K_p, K_i, K_d$)} \\
        \hline
        Jednostkowe & $\mathrm{diag}([1, 1, 1, 1])$ & $1{,}0$ & $-4{,}5, 0, -3$ \\ \hline
        Bryson & $\mathrm{diag}([25, 1, 4, 1])$ & $10{,}0$ & $-4{,}5, 0, -3$ \\ \hline
        Optymalne & $\mathrm{diag}([200, 3, 35, 40])$ & $1{,}0$ & $-7{,}0, 0{,}1, -3$ \\ \hline
    \end{tabular}
\end{table}
Ostatecznie zastosowane wagi to:
\begin{equation}
    Q = \text{diag}([200,\; 3,\; 35,\; 40]), \quad R = 1,  K_p=-7{,}0, K_i=0{,}1, K_d=-3{,}0.
\end{equation}
\begin{figure}[H]
    \centering
    \includegraphics[width=1.0\textwidth, height=15cm]{images/tuning/combined/combined_pdlqr.png}
    \caption{Porównanie nastaw regulatora PID-LQR.}
    \label{fig:combined_pdlqr}
\end{figure}




\subsection{Nieliniowe sterowanie predykcyjne (MPC)}

Algorytm MPC (Model Predictive Control) stanowi zaawansowaną metodę sterowania,
która w~odróżnieniu od~LQR, uwzględnia wprost ograniczenia sygnału sterującego
oraz nieliniową dynamikę obiektu \cite{Camacho2007}.
Zaimplementowany algorytm rozwiązuje w~każdym kroku symulacji problem optymalizacji dynamicznej
nieliniowej (NMPC). Model predykcyjny wykorzystuje numeryczne całkowanie równań ruchu metodą 
Rungego-Kutty 4. rzędu (RK4), co pozwala na dokładne odwzorowanie nieliniowej dynamiki obiektu. 
Zasadę działania regulatora MPC ilustruje Rys.~\ref{fig:diagram_mpc}.

\begin{figure}[H]
    \centering
    \begin{tikzpicture}[auto, node distance=1.4cm, >=Stealth]
        % Wejście
        \node [input] (in) {};
        
        % Optymalizator MPC
        \node [draw, fill=orange!15, rectangle, minimum height=3em, minimum width=5em, 
               right=1cm of in, rounded corners=3pt, align=center, font=\small] (opt) {
            Optymalizator\\[-2pt]
            \scriptsize $\min J(\Delta U)$
        };
        
        % Obiekt
        \node [bigblock, right=1.5cm of opt] (plant) {Wahadło};
        
        % Wyjście
        \node [output, right=1.2cm of plant] (out) {};
        
        % Model predykcyjny (poniżej optymalizatora)
        \node [draw, fill=yellow!20, rectangle, minimum height=2em, minimum width=4em,
               below=1cm of opt, rounded corners=2pt, font=\small] (model) {Model $f(x,u)$};
        
        % --- Połączenia ---
        % Wejście -> Optymalizator
        \draw [arrow] (in) -- node[above, font=\scriptsize] {$x_{\mathrm{ref}}$} (opt);
        
        % Optymalizator -> Obiekt
        \draw [arrow] (opt) -- node[above, font=\scriptsize] {$u^*_0$} (plant);
        
        % Obiekt -> Wyjście
        \draw [arrow] (plant) -- node[above, font=\scriptsize] {$x$} (out);
        
        % Sprzężenie zwrotne - ciągła strzałka
        \draw [arrow] ($(plant.east) + (0.3,0)$) -- ++(0,-0.8) -| node[pos=0.25, below, font=\scriptsize] {$x(t)$} ($(opt.south) + (0.6,0)$);
        
        % Predykcja: Optymalizator <-> Model (dwukierunkowe, obok siebie)
        \draw [arrow, dashed] ($(opt.south) + (-0.3,0)$) -- ($(model.north) + (-0.3,0)$);
        \draw [arrow, dashed] ($(model.north) + (0.3,0)$) -- node[right, font=\scriptsize] {$\hat{x}$} ($(opt.south) + (0.3,0)$);
        
        % Ramka MPC
        \begin{scope}[on background layer]
            \node [draw, dashed, rounded corners, fill=blue!5, 
                   fit=(opt) (model), inner sep=8pt, 
                   label={[font=\scriptsize]below:Regulator MPC}] {};
        \end{scope}
    \end{tikzpicture}
    \caption{Schemat blokowy regulatora MPC z~wewnętrznym modelem predykcyjnym.}
    \label{fig:diagram_mpc}
\end{figure}

% --- DODANY FRAGMENT O RK4 I DYSKRETYZACJI ---
W przypadku nieliniowego sterowania predykcyjnego (NMPC), predykcja przyszłych stanów wymaga dyskretnego modelu obiektu. Ze względu na nieliniowy charakter dynamiki wahadła $\dot{x}(t) = f_c(x(t), u(t))$, nie jest możliwe wyznaczenie analitycznej postaci modelu dyskretnego (jak w~przypadku ZOH dla układów liniowych). Dlatego też, przejście z~czasu ciągłego do~dyskretnego realizowane jest poprzez numeryczne całkowanie równań stanu w~przedziale próbkowania $T_s$. Wartość wektora stanu w~kolejnej chwili dyskretnej $k+1$ wyznaczana jest zgodnie z~zależnością:
\begin{equation}
x(k+1) = x(k) + \int_{t_k}^{t_k + T_s} f_c(x(\tau), u(k)) \, d\tau
\end{equation}
W implementacji algorytmu całka ta jest przybliżana metodą Rungego-Kutty czwartego rzędu (RK4) ze~stałym krokiem całkowania równym okresowi próbkowania regulatora. Przyjęto założenie (ZOH), że sterowanie $u(k)$ jest stałe w~całym przedziale całkowania $[t_k, t_{k+1})$.
% ------------------------------------------------

W każdym kroku algorytmu regulacji predykcyjnej zostaje wyznaczona przyszła sekwencja sygnału sterującego, przedstawiona jako wektor przyrostów sygnału sterującego:
\begin{equation}
\Delta u(k) = \left[ \Delta u(k|k) \quad \Delta u(k+1|k) \quad \dots \quad \Delta u(k+N_{u}-1|k) \right]^{T}
\end{equation}
\begin{equation} %\tag{15} % Opcjonalnie usuń tagowanie ręczne jeśli używasz numeracji automatycznej
\Delta u_{p}(k)=\Delta u(k+p|k)
\end{equation}
Klasyczna funkcja kosztu dla MPC prezentuje się w~następujący sposób:
\begin{equation}
e_{p}(k)={x}_{\mathrm{ref}}(k+p|k) - \hat{x}(k+p|k)
\end{equation}
\begin{equation} 
J(k)=\sum_{p=1}^{N}e_{p}(k)^{T}Q~e_{p}(k)+\sum_{p=0}^{N_{u}-1}\Delta u_{p}(k)^{T}R\Delta u_{p}(k)
\end{equation}
przy ograniczeniach:
\begin{align}
    \hat{x}_{k+1} &= f(\hat{x}_k, u_k), \quad k=0,\dots,N-1 \\
    u_{\mathrm{min}} &\le u_k \le u_{\mathrm{max}},
\end{align}
gdzie:
\begin{itemize}
    \item $N $ -- horyzont predykcji,
    \item $N_u$ -- horyzont sterowania,
    \item $f(\cdot)$ -- nieliniowy model dyskretny obiektu (realizowany poprzez całkowanie numeryczne metodą RK4 opisaną wyżej),
    \item $Q $ -- macierz kar stanu,
    \item $R$ -- współczynnik kary za~zmianę sterowania ($\Delta u$).
\end{itemize}
Kluczową zaletę MPC, podkreślaną w~pracach \cite{Mills2009} oraz
\cite{Jezierski2017}, jest możliwość bezpośredniego uwzględnienia ograniczeń
(saturacji) już na~etapie wyliczania sterowania, co zapobiega zjawisku
nasycenia elementu wykonawczego, które mogłoby mieć miejsce w~przypadku LQR.

Zadanie optymalizacji rozwiązywane jest numerycznie metodą SQP (Sequential
Quadratic Programming) przy użyciu solwera \texttt{SLSQP} z~biblioteki 
\texttt{scipy.optimize}. Wybór tego solwera podyktowany był jego dostępnością 
w~popularnych dystrybucjach środowisk naukowych (Anaconda, pip) oraz zdolnością 
do~obsługi ograniczeń nierównościowych.


\subsubsection{Dobór horyzontu i wag funkcji celu}
Dla regulatora MPC kluczowym zagadnieniem był dobór horyzontu predykcji $N$,
horyzontu sterowania $N_u$ oraz macierzy wag, determinujących zachowanie układu
w~stanie nieustalonym.
Początkowe ustawienie zbyt krótkiego horyzontu predykcji ($N = 7$) przy małym
horyzoncie sterowania ($N_u=3$, Rys.~\ref{fig:combined_mpc}, linia czerwona)
prowadziło do oscylacji układu zamkniętego. Horyzont był zbyt krótki,
aby regulator mógł uwzględnić, że rozpędzając wózek w~celu korekcji kąta,
nie zdąży wyhamować przed odchyleniem się wahadła.

Zwiększenie horyzontu do~$N=10$ (przy $N_u=3$) w~ramach korekty ręcznej
(linia niebieska) bardzo polepszyło jakość trajektorii. Dłuższy horyzont umożliwił dalszą predykcję.
Dodatkowa manipulacja wagami $Q$ pozwoliła na~uzyskanie poprawnego
sterowania.

Aby zweryfikować dobrane nastawy znów zastosowano algorytm do automatyzacji procesu strojenia, 
który pozwolił na~znalezienie nowych nastaw (linia zielona).
Algorytm wyznaczył jeszcze lepsze nastawy regulatora MPC $N=12$ przy $N_u=4$ oraz nowe wartości macierzy $Q$ i skalara $R$.

\begin{table}[H]
    \centering
    \caption{Zestawienie parametrów strojenia regulatora MPC.}
    \label{tab:params_mpc}
    \begin{tabular}{|l|c|c|c|c|}
        \hline
        \textbf{Wariant} & \textbf{Horyzont N} & \textbf{Ster. $N_u$} & \textbf{Macierz Q} & \textbf{Waga R} \\
        \hline
        Krótki & 7 & 3 & $\mathrm{diag}([10, 1, 10, 1])$ & $0{,}1$ \\ \hline
        Ręczne & 10 & 3 & $\mathrm{diag}([50, 10, 50, 10])$ & $0{,}1$ \\ \hline
        Optymalne & 12 & 4 & $\mathrm{diag}([158, 41, 43, 20])$ & $0{,}086$ \\ \hline
    \end{tabular}
\end{table}
Ostateczne nastawy regulatora MPC to:
$Q=\mathrm{diag}([158, 41, 43, 20])$, $R=0,086$.
\begin{figure}[H]
    \centering
    \includegraphics[width=1.0\textwidth]{images/tuning/combined/combined_mpc.png}
    \caption{Porównanie nastaw regulatora MPC.}
    \label{fig:combined_mpc}
\end{figure}

\subsection{MPC z~rozszerzonym wskaźnikiem jakości (MPC-alt)}

Dodatkowo zaimplementowano regulator MPC z alternatywną funkcją kosztu. 
Jego struktura oraz nieliniowy model predykcyjny (oparty na całkowaniu RK4) są identyczne jak w~podstawowym wariancie MPC, jednak funkcja kosztu została rozbudowana
o~dodatkowy składnik karzący bezwzględną wartość sygnału sterującego (energię).

Zmodyfikowana funkcja celu:
\begin{equation}
e_{p}(k)={x}_{\mathrm{ref}}(k+p|k) - \hat{x}(k+p|k)
\end{equation}
\begin{equation} 
J(k)=\sum_{p=1}^{N}e_{p}(k)^{T}Q~e_{p}(k)+\sum_{p=0}^{N_{u}-1}\Delta u_{p}(k)^{T}R\Delta u_{p}(k) + \sum_{p=0}^{N_{u}-1} u_{p}(k)^{T}R_{abs} u_{p}(k)
\end{equation}

Wprowadzenie parametru $R_{\mathrm{abs}}$ pozwala na~bezpośrednie minimalizowanie
zużycia energii sterowania, co jest podejściem powszechnie stosowanym
w~praktycznych implementacjach algorytmów predykcyjnych \cite{Camacho2007}.
Ograniczenie amplitudy sygnału sterującego nie tylko redukuje wydatek energetyczny
(istotny w~aplikacjach mobilnych), ale także zmniejsza obciążenie mechaniczne
elementów wykonawczych, co wpływa na~żywotność napędu.

\subsubsection{Dobór parametrów i analiza wpływu kary za energię}
W~przypadku wariantu MPC-alt analizowano nieliniowy wpływ parametru $R_{\mathrm{abs}}$
na~zachowanie układu. Eksperymenty przeprowadzono przy stałych wagach stanu takich samych jak dla MPC:
$Q=\mathrm{diag}([158, 41, 43, 20])$
oraz przy bardzo niskiej karze za~przyrosty
sterowania $R_{\Delta} = 0{,}001$, zmieniając jedynie wartość kary za~bezwzględną 
wartość sterowania.

Wprowadzenie członu $\sum_{p=0}^{N_{u}-1} u_{p}(k)^{T}R_{abs} u_{p}(k)$ do~funkcji celu ma 
fundamentalnie inny charakter niż kara za~przyrosty sterowania $R_{\Delta}$. 
Podczas gdy $R_{\Delta}$ promuje gładkość sygnału sterującego (ogranicza szybkie
zmiany), parametr $R_{\mathrm{abs}}$ bezpośrednio kara za amplitudę siły, 
co przekłada się na~redukcję zużycia energii.

Przyjęcie zbyt dużej wartości kary za~sterowanie bezwzględne ($R_{\mathrm{abs}}=10$,
Rys.~\ref{fig:combined_mpcj2}, linia czerwona) spowodowało, że regulator wykazywał tendencję do~pasywności.
Funkcja kosztu karała każdy niuton siły na~tyle intensywnie, że wartość funkcji celu
faworyzowała bardzo wolną regulację.

Stopniowe, ręczne zmniejszanie parametru $R_{\mathrm{abs}}$ do~wartości $5$
(linia niebieska) pozwoliło delikatnie polepszyć jakość regulacji, 
jednak układ działał zbyt zachowawczo.

Kolejne próby zmniejszania kary za bezwzględną wartość wskazały, że dla tego zadania najlepszym rozwiązaniem jest
zastosowanie niewielkiej, ale niezerowej kary ($R_{\mathrm{abs}}=1$, linia zielona).
Tak dobrana waga pozwala uniknąć bierności regulatora, jednocześnie ograniczając nadmierne zużycie energii.

\begin{table}[H]
    \centering
    \caption{Analiza wpływu kary energetycznej w MPC-alt (przy stałych $Q$ i $R_{\Delta}$).}
    \label{tab:params_mpcj2}
    \begin{tabular}{|l|c|c|}
        \hline
        \textbf{Wariant} & \textbf{Kara absolutna $R_{abs}$} & \textbf{Uwagi} \\
        \hline
        Wysoka & $10{,}0$ & Silne tłumienie sterowania \\ \hline
        Średnia & $5{,}0$ & Kompromis \\ \hline
        Optymalna & $1{,}0$ & Najlepsza dynamika \\ \hline
    \end{tabular}
\end{table}
Ostateczne nastawy regulatora MPC-alt to:
\begin{equation}
    Q=\mathrm{diag}([158, 41, 43, 20]), R=0,001, R_{abs}=1.
\end{equation}
\begin{figure}[H]
    \centering
    \includegraphics[width=1.0\textwidth]{images/tuning/combined/combined_mpcj2.png}
    \caption{Analiza wpływu kary za energię $R_{abs}$ w sterowaniu MPC-alt.}
    \label{fig:combined_mpcj2}
\end{figure}

\subsection{Liniowy regulator MPC (LMPC)}

Zaimplementowano również wariant Liniowego MPC (LMPC), którego implementacja 
została oparta na~strukturze regulatora NMPC, z~zastąpieniem modelu nieliniowego 
modelem zlinearyzowanym. Główna idea polega na~wykorzystaniu zlinearyzowanego modelu 
obiektu wokół punktu równowagi górnej ($\theta = 0$) w~celu uproszczenia 
obliczeń optymalizacyjnych.

Linearyzacja modelu wahadła odwróconego prowadzi do~układu w~postaci 
przestrzeni stanów:
\begin{equation}
    \dot{x} = A_c x + B_c u,
\end{equation}
gdzie macierze $A_c$ i~$B_c$ wyznaczono analitycznie (metodą Jacobiego) dla~punktu 
równowagi $\theta = 0$, $\dot{\theta} = 0$. Model ciągły poddano następnie 
dyskretyzacji metodą ekstrapolatora zerowego rzędu (ZOH -- Zero-Order Hold), 
uzyskując równanie stanu w~postaci dyskretnej:
\begin{equation}
    x_{k+1} = A_d x_k + B_d u_k.
\end{equation}
Dyskretyzacja ZOH zakłada, że~sygnał sterujący $u$ pozostaje stały w~przedziale 
$[k \cdot dt, (k+1) \cdot dt]$, co odpowiada rzeczywistemu działaniu cyfrowego 
układu sterowania. Do~wyznaczenia macierzy $A_d$ i~$B_d$ wykorzystano funkcję 
\texttt{cont2discrete} z~biblioteki \texttt{scipy.signal}.

Dzięki zastosowaniu modelu liniowego, predykcja trajektorii stanu (w~przeciwieństwie 
do~NMPC) sprowadza się do~szybkich operacji macierzowych, bez konieczności 
numerycznego całkowania równań różniczkowych metodą Rungego-Kutty. Dla~zadanej 
sekwencji sterowania $\{u_0, u_1, \ldots, u_{N-1}\}$ predykcja ma postać:
\begin{equation}
    \hat{x}_{k+1} = A_d \hat{x}_k + B_d u_k, \quad k = 0, \ldots, N-1.
\end{equation}

funkcja kosztu przyjmuje postać analogiczną do nieliniowego MPC:
\begin{equation}
e_{p}(k)={x}_{\mathrm{ref}}(k+p|k) - \hat{x}(k+p|k)
\end{equation}
\begin{equation} 
J(k)=\sum_{p=1}^{N}e_{p}(k)^{T}Q~e_{p}(k)+\sum_{p=0}^{N_{u}-1}\Delta u_{p}(k)^{T}R\Delta u_{p}(k)
\end{equation}

Do rozwiązania problemu sterowania użyto solvera \texttt{SLSQP} z~biblioteki 
\texttt{scipy.optimize}, identycznego jak w~przypadku nieliniowego MPC. 
Zastosowanie tego samego solvera dla obu wariantów MPC pozwala na~bezpośrednie 
porównanie wpływu uproszczenia modelu.

\subsubsection{Dobór parametrów i~analiza działania}

Regulator LMPC dostrojono ręcznie metodą prób i błędów (Rys.~\ref{fig:combined_lmpc}).
Najpierw przetestowano zestaw wag $Q=\text{diag}([1, 1, 1, 1])$, $R=0,1$ (linia czerwona). 
Widać, że pozycja wózka nie stabilizuje się wystarczająco szybko.
Zdecydowano się w kolejnym kroku zwiększyć wagi w macierzy $Q$ (linia zielona).

Na~Rys.~\ref{fig:combined_lmpc} przedstawiono porównanie obu konfiguracji.
Układ zoptymalizowany zachowuje się stabilnie, płynnie dochodząc do~wartości
zadanej.

\begin{table}[H]
    \centering
    \caption{Zestawienie parametrów strojenia LMPC.}
    \label{tab:params_lmpc}
    \begin{tabular}{|l|c|c|}
        \hline
        \textbf{Wariant} & \textbf{Macierz Q} & \textbf{Waga R} \\
        \hline
        Wagi jedn. & $\mathrm{diag}([1, 1, 1, 1])$ & $0{,}1$ \\ \hline
        Optymalne & $\mathrm{diag}([15, 1, 15, 1])$ & $0{,}1$ \\ \hline
    \end{tabular}
\end{table}

\begin{figure}[H]
    \centering
    \includegraphics[width=1.0\textwidth]{images/tuning/combined/combined_lmpc.png}
    \caption{Strojenie liniowego regulatora MPC.}
    \label{fig:combined_lmpc}
\end{figure}


Mimo zastosowania uproszczonego modelu liniowego, regulator poprawnie radzi sobie
ze~stabilizacją w~otoczeniu punktu równowagi.
Wybrane parametry dla regulatora LMPC:
horyzont predykcji: $N=12$, horyzont sterowania: $N_u=4$,
macierz wag stanu: $Q = \text{diag}([15,0,\; 1,0,\; 15,0,\; 1,0])$, waga sterowania: $R = 0,1$.

\subsection{Regulator rozmyty wspomagany LQR (Fuzzy-LQR)}

Ostatnim zbadanym układem jest sterownik hybrydowy o~strukturze równoległej, 
łączący klasyczny, liniowy regulator LQR z~nieliniowym systemem
wnioskowania rozmytego typu Takagi-Sugeno (T-S). W~przeciwieństwie do~układów typu
,,Gain Scheduling'' modyfikujących parametry jednego regulatora, tutaj zastosowano
bezpośrednie sumowanie sygnałów sterujących z~dwóch niezależnych bloków (Rys.~\ref{fig:diagram_fuzzy}).
LQR zapewnia optymalną stabilizację w~pobliżu punktu pracy, natomiast człon rozmyty
generuje dodatkowy sygnał korekcyjny, aktywujący się silniej przy większych uchybach.

\begin{figure}[H]
    \centering
    \begin{tikzpicture}[auto, node distance=1.4cm, >=Stealth]
        % Wejście
        \node [input] (in) {};
        
        % Suma błędu
        \node [sum, right=0.8cm of in] (sum_e) {\tiny $+$};
        
        % Blok LQR (góra)
        \node [block, right=1.8cm of sum_e, yshift=0.7cm] (lqr) {LQR};
        
        % Blok Fuzzy (dół)
        \node [draw, fill=purple!15, rectangle, minimum height=2.2em, minimum width=4em,
               right=1.8cm of sum_e, yshift=-0.7cm, rounded corners=3pt, font=\small] (fuzzy) {Fuzzy T-S};
        
        % Suma sterowania
        \node [sum, right=1.5cm of lqr, yshift=-0.7cm] (sum_u) {\tiny $+$};
        
        % Saturacja i obiekt
        \node [bigblock, right=1.8cm of sum_u] (plant) {Wahadło};
        
        % Wyjście
        \node [output, right=1.2cm of plant] (out) {};
        
        % --- Połączenia ---
        % Wejście -> Suma błędu
        \draw [arrow] (in) -- node[above, font=\scriptsize] {$x_{\mathrm{ref}}$} (sum_e);
        
        % Suma błędu -> rozgałęzienie do LQR i Fuzzy
        \draw [line] (sum_e.east) -- ++(0.5,0) coordinate (branch);
        \draw [arrow] (branch) |- (lqr.west);
        \draw [arrow] (branch) |- (fuzzy.west);
        
        % LQR i Fuzzy -> Suma sterowania
        \draw [arrow] (lqr.east) -| (sum_u.north);
        \draw [arrow] (fuzzy.east) -| (sum_u.south);
        
        % Reszta toru
        \draw [arrow] (sum_u) -- node[above, font=\scriptsize] {$u$} (plant);
        \draw [arrow] (plant) -- node[above, font=\scriptsize] {$x$} (out);
        
        % Sprzężenie zwrotne - ciągła strzałka
        \draw [arrow] ($(plant.east) + (0.3,0)$) -- ++(0,-1.8) -| (sum_e.south);
        \node [below=3pt of sum_e.south, font=\tiny] {$-$};
    \end{tikzpicture}
    \caption{Schemat blokowy regulatora Fuzzy-LQR z~równoległą strukturą hybrydową.}
    \label{fig:diagram_fuzzy}
\end{figure}

Sygnał sterujący opisuje równanie:
\begin{equation}
    u(t) =  u_{LQR}(k) + G \cdot u_{Fuzzy}(k),
\end{equation}
gdzie $G$ to globalne wzmocnienie skalujące część rozmytą.

Część rozmyta $u_{\mathrm{Fuzzy}}(k)$ wykorzystuje bazę reguł postaci:
\begin{quote}
    JEŚLI $e_\theta$ jest $A_i$ ORAZ $\dot{\theta}$ jest $B_i$ ... TO $u_i = -K^{(i)} \cdot x$,
\end{quote}
gdzie $K^{(i)}$ jest wektorem wzmocnień lokalnego regulatora liniowego. 
Zastosowano trójkątne funkcje przynależności (Rys.~\ref{fig:fuzzy_membership}), 
przyjmując zakresy strefy ,,Małego błędu'' na poziomie $\pm 0.15$ rad dla kąta oraz $\pm 0.30$ m dla pozycji.

\begin{figure}[H]
    \centering
    \includegraphics[width=0.95\textwidth]{images/diagrams/fuzzy_membership.png}
    \caption{Trójkątne funkcje przynależności dla czterech zmiennych stanu 
    regulatora Fuzzy-LQR. Każda zmienna posiada dwa zbiory rozmyte: 
    ,,Mały błąd'' (aktywny w~pobliżu zera) oraz ,,Duży błąd'' 
    (aktywny przy większych odchyleniach od~punktu równowagi).}
    \label{fig:fuzzy_membership}
\end{figure}

Baza wiedzy składa się z~16 reguł ($2^4$ kombinacji). Wyjście sterownika obliczane jest jako średnia ważona:
\begin{equation}
    u_{\mathrm{Fuzzy}} = G \cdot \frac{\sum_{i=1}^{16} w_i(x) \cdot u_i}{\sum_{i=1}^{16} w_i(x)},
\end{equation}
gdzie $w_i$ to stopień aktywacji $i$-tej reguły, a~$G = 0.36$ to globalne wzmocnienie skalujące.

Zastosowana struktura równoległa pozwala na~uzyskanie efektu nieliniowego kształtowania wzmocnienia. 
W~zaimplementowanym algorytmie wzmocnienia $K^{(i)}$ nie są stałe, lecz zależą od stopnia krytyczności danej reguły:
\begin{equation}
    K^{(i)} = K_{base} + \alpha \cdot L_i,
\end{equation}
gdzie $L_i \in \{0, \dots, 4\}$ to liczba zmiennych stanu znajdujących się w strefie ,,Dużego błędu''.
Przykładowo, dla kąta $\theta$ wzmocnienie bazowe wynosi $100.0$, ale każda zmienna w stanie ,,Dużym'' dodaje do niego wartość $\alpha=20.0$.
Dzięki temu, w sytuacjach awaryjnych (duże wychylenia i prędkości), efektywne wzmocnienie układu wzrasta drastycznie (nawet o 80\%), zapewniając silną reakcję powrotną, podczas gdy wokół zera układ zachowuje się łagodnie.

\subsubsection{Dobór reguł i funkcji przynależności}
Strojenie rozmytego regulatora Fuzzy-LQR jest zadaniem złożonym ze względu na~dużą
liczbę parametrów definiujących bazę reguł i~funkcje przynależności. 
W~przeciwieństwie do~klasycznego regulatora LQR, gdzie dobór sprowadza się do~ustalenia 
wag w~macierzach $Q$ i~$R$, tutaj należy podjąć szereg decyzji 
dotyczących kształtu, rozmieszczenia i~liczby funkcji przynależności dla każdej 
ze~zmiennych stanu. Ponadto, konieczne jest zdefiniowanie reguł wnioskowania, 
które determinują zachowanie układu w~poszczególnych obszarach przestrzeni stanu. 
Niewłaściwy dobór tych parametrów może prowadzić do~niestabilności, gwałtownych 
przełączeń sterowania lub braku pożądanej poprawy dynamiki względem regulatora bazowego.

Błędne zdefiniowanie zbyt wąskich funkcji przynależności dla strefy ,,małego błędu''
(Rys.~\ref{fig:combined_fuzzy}, linia czerwona) skutkowało gwałtownym przełączaniem się regulatora 
na~reguły o~wysokich wzmocnieniach (drgania przełączeniowe).
Opierając się na~literaturze \cite{Nguyen2024}, dobrano ręcznie szerokości funkcji przynależności (linia niebieska).
Rozszerzenie zakresu wyeliminowało drgania, jednak wzmocnienia reguł były zbyt słabe.
Ostatecznie, algorytm wyznaczył wartość wzmocnienia
dla kąta ($F_{\theta}=40.0$) i zmniejszył globalne wzmocnienie
($G=0.36$), co przedstawiono linią zieloną na wykresie.

Strojenie regulatora Fuzzy-LQR okazało się zadaniem znacznie bardziej wymagającym 
niż w~przypadku pozostałych badanych algorytmów. Wynika to z~kilku czynników:
\begin{itemize}
    \item Wysoka wymiarowość przestrzeni parametrów --- dla 16 reguł, 
    z~których każda definiuje 4 wzmocnienia, plus parametry funkcji przynależności.
    \item Silne sprzężenia między parametrami --- zmiana jednego wzmocnienia 
    wpływa na~zachowanie całej bazy reguł.
    \item Zależność od~scenariusza testowego --- parametry zoptymalizowane 
    dla warunków nominalnych mogą dawać gorsze wyniki przy zakłóceniach i~odwrotnie.
\end{itemize}

\begin{table}[H]
    \centering
    \caption{Parametry regulatora Fuzzy-LQR.}
    \label{tab:params_fuzzy}
    \begin{tabular}{|l|c|c|}
        \hline
        \textbf{Wariant} & \textbf{Zakres $\theta_{small}$} & \textbf{Wzmocnienia Reguł} \\
        \hline
        Wąski & $[-0{,}02; 0{,}02]$ & Standardowe \\ \hline
        Słabe & $[-0{,}2; 0{,}2]$ & $F_{\theta}=20$ (bazowe) \\ \hline
        Optymalne & $[-0{,}2; 0{,}2]$ & $F_{\theta}=40, G=0{,}36$ \\ \hline
    \end{tabular}
\end{table}

Ostatecznie użyty zestaw parametrów to: $F_{\theta}=40$, 
    $F_{\dot{\theta}}=1$, $F_x=15$, $F_{\dot{x}}=20$, $G=0.36$.
\begin{figure}[H]
    \centering
    \includegraphics[width=1.0\textwidth]{images/tuning/combined/combined_fuzzy.png}
    \caption{Strojenie regulatora Fuzzy-LQR.}
    \label{fig:combined_fuzzy}
\end{figure}


