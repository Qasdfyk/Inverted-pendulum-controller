% !TEX encoding = utf8
\section{Wstęp}
Odwrócone wahadło na~wózku jest klasycznym przykładem nieliniowego, 
niestabilnego układu mechanicznego, wykorzystywanym powszechnie 
zarówno w~dydaktyce, jak i~w~badaniach nad zaawansowanymi technikami 
sterowania. Mimo że geometria i~parametry fizyczne obiektu są stosunkowo 
proste, układ ten pozostaje wyzwaniem w~zakresie stabilizacji i~wymaga 
precyzyjnej regulacji w~czasie rzeczywistym. Jego charakterystyczna cecha 
--- podwzbudność (ang. \textit{underactuated system}), oznaczająca mniejszą 
liczbę wejść sterujących niż wyjść --- oraz silna wrażliwość 
na~zakłócenia sprawiają, że nawet niewielkie odchylenia mogą prowadzić 
do~gwałtownego narastania błędów i~utraty równowagi.

Znaczenie tego modelu wykracza daleko poza cele czysto akademickie. 
Odwrócone wahadło na~wózku służy jako kanoniczny \textit{benchmark} 
dla metod sterowania i~estymacji stanu, ponieważ łączy w~sobie trudności 
typowe dla systemów rzeczywistych: nieliniowość, niestabilność 
w~otwartym układzie sterowania, ograniczenia aktuatora oraz niepewność 
parametrów. Umożliwia to weryfikację algorytmów w~sytuacjach, 
w~których klasyczne założenia teorii liniowej przestają obowiązywać, 
a~układ wymaga adaptacji lub podejścia optymalnego.

Model ten posiada liczne analogie w~praktyce inżynierskiej. Jego dynamika 
odwzorowuje wiele złożonych zjawisk fizycznych i~konstrukcji technicznych, 
w~tym:
\begin{itemize}
  \item stabilizację robotów dwukołowych (np. typu Segway) oraz robotów 
  mobilnych balansujących na~jednej osi~\cite{segway2018};
  \item sterowanie rakietami nośnymi i~pociskami w~fazie startu, gdzie wektor 
  ciągu pełni rolę siły napędowej wózka, a~bezwładność korpusu odpowiada 
  dynamice wahadła~\cite{nasa2013};
  \item sterowanie ruchem ramion manipulatorów i~suwnic, w~których kluczowe 
  jest tłumienie oscylacji przenoszonego ładunku~\cite{crane2012};
  \item równoważenie platform i~pojazdów samobalansujących, wymagające 
  ciągłej korekty siły napędowej względem położenia środka masy~\cite{moreno2023}.
\end{itemize}

Ze~względu na~powyższe zastosowania, problem stabilizacji odwróconego wahadła 
traktowany jest jako uproszczony model systemów rzeczywistych o~zbliżonej 
dynamice. Badania symulacyjne na~tym obiekcie pozwalają na~wstępną walidację 
skuteczności algorytmów sterowania przed ich implementacją w~bardziej 
złożonych lub kosztownych systemach. Z~tego powodu układ ten od~dziesięcioleci 
stanowi punkt odniesienia w~rozwoju nowoczesnych metod regulacji --- 
od~klasycznych regulatorów PID (regulator proporcjonalno-całkująco-różniczkujący) i~LQR (liniowy regulator kwadratowy), po~sterowanie predykcyjne (MPC), 
adaptacyjne i~rozmyte.

Prostota modelu matematycznego w~połączeniu z~łatwością interpretacji wyników 
(analiza kąta wychylenia i~pozycji wózka) sprawiają, że odwrócone wahadło 
łączy elegancję analityczną z~praktycznymi wyzwaniami inżynierskimi. 
Stanowi tym samym uniwersalne narzędzie do~nauki, testowania i~rozwijania 
metod stabilizacji systemów nieliniowych.

\subsection{Cel i~zakres pracy}

Zasadniczym celem pracy jest zaprojektowanie i~realizacja autorskiego 
środowiska symulacyjnego oraz przeprowadzenie w~nim wielokryterialnej 
analizy porównawczej algorytmów sterowania dla nieliniowego układu 
odwróconego wahadła.

Aby zrealizować ten cel, zdefiniowano zestaw wymagań funkcjonalnych 
i~niefunkcjonalnych, które musi spełniać opracowany system:

Aby zrealizować ten cel, zdefiniowano szereg wymagań funkcjonalnych 
i~niefunkcjonalnych stawianych opracowywanemu systemowi. W~warstwie 
funkcjonalnej środowisko musi wiernie odwzorowywać dynamikę układu wózka 
z~wahadłem w~oparciu o~nieliniowe równania ruchu, a~także umożliwiać 
elastyczne przełączanie pomiędzy zaimplementowanymi strategiami sterowania 
(PID, LQR, MPC, Fuzzy). Kluczowa jest również możliwość weryfikacji 
odporności regulatorów poprzez generowanie addytywnych zakłóceń 
zewnętrznych (np. siły wiatru). System powinien oferować narzędzia 
do~wizualizacji przebiegów oraz animacji ruchu obiektu, a~także 
automatycznie archiwizować wyniki eksperymentów w~celu późniejszego 
wyznaczenia wskaźników jakości (MSE, MAE, energia sterowania).

W~zakresie wymagań niefunkcjonalnych przyjęto, że oprogramowanie zostanie 
zrealizowane w~języku Python 3 z~wykorzystaniem bibliotek numerycznych 
\texttt{NumPy} i~\texttt{SciPy}. Projekt musi charakteryzować się wysoką 
modularnością, separującą logikę sterowania od~silnika symulacyjnego, 
co ułatwi jego dalszą rozbudowę. Ponadto, zaimplementowane algorytmy 
powinny cechować się wydajnością umożliwiającą pracę w~czasie zbliżonym 
do~rzeczywistego.

Większość dostępnych publikacji skupia się zazwyczaj tylko na~jednej grupie metod. 
W~niniejszej pracy postanowiono porównać ze~sobą trzy zupełnie różne podejścia:
\begin{itemize}
  \item klasyczne (PID, kaskady),
  \item optymalne i~predykcyjne (LQR, MPC),
  \item inteligentne (sterowanie rozmyte).
\end{itemize}

Wyróżnikiem pracy jest to, że wszystkie te algorytmy zostały uruchomione 
w~tym samym środowisku symulacyjnym. Dzięki temu możliwe było ich uczciwe 
porównanie w~identycznych warunkach. Sprawdzono nie tylko, jak precyzyjnie 
utrzymują wahadło w~pionie, ale także jak radzą sobie z~trudnymi zakłóceniami 
(np. porywistym wiatrem) oraz ile energii zużywają podczas pracy.

W~ramach projektu zaimplementowano i~poddano badaniom następujące struktury:
\begin{enumerate}
    \item \textbf{Regulator PID} --- reprezentujący klasyczne 
    podejście inżynierskie. Układ składa się z~dwóch pętli sprzężenia 
    zwrotnego (stabilizacja kąta i~pozycji), wyposażonych w~mechanizmy 
    zapobiegające nasyceniu członów całkujących (ang. \textit{anti-windup}) 
    oraz ograniczenia sygnału sterującego.
    
    \item \textbf{Hybrydowy układ PID-LQR} --- struktura łącząca prostotę 
    PID (w~pętli regulacji pozycji) z~optymalnym regulatorem stanu LQR 
    (w~pętli stabilizacji wahadła). Metoda ta stanowi punkt odniesienia 
    dla oceny skuteczności metod bazujących na~modelu liniowym.
    
    \item \textbf{Regulator MPC} --- wariant 
    podstawowy sterowania predykcyjnego z~kwadratową funkcją kosztu. 
    Służy on w~pracy jako nowoczesny \textit{benchmark}, pozwalający ocenić, 
    jakie korzyści daje uwzględnienie ograniczeń sterowania i~dynamiki 
    układu bezpośrednio w~procesie optymalizacji.
    
    \item \textbf{Regulator MPC z~rozszerzonym wskaźnikiem jakości} --- 
    wariant badawczy metody predykcyjnej, w~którym przeanalizowano wpływ 
    dodatkowych kar w~funkcji celu (za~gwałtowne zmiany sterowania 
    oraz jego amplitudę) na~płynność regulacji i~zużycie energii.
    
    \item \textbf{Regulator rozmyty (Fuzzy Logic)} --- metoda sterowania 
    inteligentnego, wykorzystująca zbiór reguł wnioskowania, stanowiąca 
    alternatywę dla metod analitycznych.
\end{enumerate}

Kluczowym elementem pracy jest weryfikacja działania regulatorów w~zadaniu 
stabilizacji wahadła w~pozycji pionowej (punkt pracy) w~obecności zakłóceń 
zewnętrznych. W~celu obiektywnego porównania metod przyjęto zestaw wskaźników 
ilościowych. Jakość stabilizacji weryfikowana jest w~oparciu o~metryki całkowe 
błędu (ISE, IAE) dla kąta wychylenia oraz pozycji wózka. Równolegle ocenie 
poddano ekonomię sterowania, wyznaczając koszt energetyczny poprzez normy 
\(L_1\) i~\(L_2\) sygnału sterującego. Całość uzupełnia analiza odporności 
układu na~zakłócenia (szumy, siły zewnętrzne), wyrażona m.in. wskaźnikiem SNR. 
Tak dobrane kryteria pozwalają na~wszechstronne wskazanie mocnych i~słabych 
stron badanych metod.


\subsection{Przegląd literatury}

W~literaturze przedmiotu odwrócone wahadło na~wózku traktowane jest 
powszechnie jako wzorcowy układ testowy (ang. \textit{testbed}) 
dla weryfikacji algorytmów sterowania układami niestabilnymi. 
W~pracy~\cite{art1} przedstawiono kompletny model nieliniowy obiektu, 
uwzględniający zakłócenia stochastyczne, na~podstawie którego 
przeprowadzono analizę porównawczą regulatorów PID, LQR oraz ich 
konfiguracji hybrydowych. Wyniki te wskazują, że włączenie komponentu 
LQR znacząco poprawia szybkość i~płynność odpowiedzi w~stosunku 
do~klasycznej regulacji PID, szczególnie w~zakresie stabilizacji 
kątowej wahadła.

Rozszerzenie zakresu badań o~sterowanie predykcyjne (MPC) zaprezentowano 
w~pozycji~\cite{art2}. Autorzy stworzyli jednorodne środowisko symulacyjne, 
zestawiając przebiegi zmiennych stanu dla metod PID, LQR oraz MPC. 
Uzyskane rezultaty potwierdziły przewagę rozwiązań opartych na~modelu 
(LQR, MPC) nad klasycznym PID w~kontekście jakości regulacji, 
podkreślając jednocześnie kluczową zaletę MPC --- możliwość bezpośredniego 
uwzględniania ograniczeń fizycznych nałożonych na~wielkości sterujące.

Współczesne prace badawcze coraz częściej integrują metody optymalne 
z~metodami sztucznej inteligencji. Artykuł~\cite{art3} opisuje rozwiązanie 
hybrydowe, łączące regulator LQR z~modelem rozmytym Takagi--Sugeno 
(z~kompensacją PDC) oraz obserwatorem stanu. Podejście to pozwala 
na~przyspieszenie zbieżności błędu regulacji do~zera oraz poprawę 
jakości estymacji zmiennych w~obecności szumów pomiarowych 
i~niepewności parametrycznej modelu.

Istotnym uzupełnieniem badań symulacyjnych są weryfikacje eksperymentalne, 
szeroko reprezentowane w~krajowej literaturze naukowej. 
Zespół Jezierski i~in.~\cite{jezierski2017} przeprowadził porównanie 
algorytmów LQR i~MPC na~rzeczywistym stanowisku laboratoryjnym. 
Wykazano, że o~ile regulator LQR skutecznie utrzymuje punkt pracy 
i~tłumi zakłócenia, to sterowanie predykcyjne zapewnia łagodniejsze 
sterowanie i~lepsze właściwości śledzenia trajektorii, co ma kluczowe 
znaczenie w~aplikacjach robotycznych.

Z~punktu widzenia podstaw teoretycznych, fundamentem dla implementacji 
sterowania predykcyjnego są prace monograficzne Camacho i~Bordonsa~\cite{camacho2007} 
oraz Tatjewskiego~\cite{tatjewski_pl}. Omawiają one szczegółowo zagadnienia 
doboru funkcji kosztu, horyzontów predykcji, a~także stabilności układu 
zamkniętego. Aspekty wdrożeniowe, w~tym efektywność numeryczna 
algorytmów optymalizacji na~platformach wbudowanych, poruszane są 
w~nowszych publikacjach~\cite{mills2009, diwan2022}. Natomiast w~obszarze 
sterowania rozmytego cennym źródłem wiedzy metodycznej są opracowania 
dotyczące modeli Takagi--Sugeno i~ich porównań z~podejściami 
klasycznymi~\cite{roose2017}.

Przeprowadzona analiza literatury wskazuje na~ewolucję podejść sterowania: 
od~klasycznych paradygmatów PID i~LQR~\cite{art1}, przez ujęcia 
predykcyjne~\cite{art2, tatjewski_pl}, aż po~zaawansowane metody 
hybrydowe i~inteligentne~\cite{art3}.

\subsection{Układ pracy}

Układ pracy został podzielony na~rozdziały, których treść odpowiada 
kolejnym etapom realizacji projektu.

Rozdział drugi przedstawia szczegółowe wyprowadzenie modelu matematycznego 
odwróconego wahadła na~wózku. Zawiera on opis fizyczny obiektu, 
równania dynamiki sformułowane w~oparciu o~prawa mechaniki, a~także 
linearyzację modelu niezbędną do~syntezy wybranych regulatorów.

Rozdział trzeci poświęcony jest opisowi zrealizowanego środowiska
symulacyjnego. Przedstawiono w~nim narzędzia programistyczne, metody 
numeryczne wykorzystane do~rozwiązywania równań różniczkowych 
oraz sposób modelowania zakłóceń zewnętrznych.

Rozdział czwarty zawiera charakterystykę zaimplementowanych algorytmów 
sterowania. Omówiono w~nim podstawy teoretyczne oraz szczegóły 
implementacyjne regulatorów: klasycznego PID, optymalnego LQR, 
predykcyjnego MPC oraz rozmytego Takagi--Sugeno.

Rozdział piąty opisuje metodykę badań symulacyjnych. Zdefiniowano w~nim 
scenariusze testowe, przyjęte wskaźniki jakości oraz procedurę strojenia 
regulatorów, ze~szczególnym uwzględnieniem doboru wag macierzy LQR 
oraz nastaw członu PID.

Rozdział szósty prezentuje wyniki przeprowadzonych eksperymentów. 
Zawiera on szczegółową analizę przebiegów czasowych, zestawienie tabelaryczne 
błędów regulacji w~warunkach nominalnych i~zakłóconych, oraz dyskusję 
porównawczą skuteczności badanych metod.

Pracę kończy podsumowanie, zawierające wnioski końcowe oraz 
kierunki dalszego rozwoju projektu.