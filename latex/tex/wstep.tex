% !TEX encoding = utf8
\section{Wstęp}
Odwrócone wahadło na wózku jest klasycznym przykładem nieliniowego, niestabilnego układu, wykorzystywanym zarówno w dydaktyce, jak i w badaniach nad zaawansowanymi technikami sterowania. Jego atrakcyjność wynika z tego, że choć geometria i parametry są stosunkowo proste, to układ ten jest trudny do stabilizacji i wymaga precyzyjnej regulacji w czasie rzeczywistym.

\subsection{Cel pracy}
Celem niniejszego sprawozdania jest przedstawienie projektu symulacyjnego układu stabilizacji odwróconego wahadła na wózku, w którym porównano klasyczne regulatory:
\begin{enumerate}
    \item Regulator PD lub PID do regulacji położenia wózka \(x\) oraz regulator LQR do regulacji kąta wahadła \(\varphi\).
    \item Regulator PID do regulacji całego systemu.
    \item Regulator LQR do regulacji całego systemu.
\end{enumerate}
Dodatkowo zestawiono wskaźniki jakości regulacji: średni błąd kwadratowy (MSE) oraz średni błąd bezwzględny (MAE) dla trajektorii \(\varphi(t)\) i \(x(t)\), a także przeanalizowano wpływ zakłóceń na jakość regulacji.

\subsection{Omówienie literatury}
W literaturze problem sterowania odwróconego wahadła na wózku jest traktowany jako wzorcowy układ niestabilny i nieliniowy, służący do oceny efektywności różnych algorytmów sterowania. W~\cite{art1} autorzy prezentują:
\begin{itemize}
    \item Model nieliniowy systemu z uwzględnieniem zaburzenia od wiatru oraz jego liniaryzację wokół punktu równowagi.
    \item Projekt dwóch pętli PID: jedna dla kąta \(\theta\), druga dla położenia \(x\), oraz algorytm LQR zaprojektowany na zlinearyzowanym modelu.
    \item Analizę porównawczą działania regulatorów PID, PID+LQR (dwie konfiguracje: \emph{2PID+LQR} oraz \emph{1PID+LQR}), zarówno w przypadku bez zakłóceń, jak i z zakłóceniem od wiatru.
\end{itemize}
Wzorując się na ~\cite{art1} zimplementowano regulator PID+LQR.