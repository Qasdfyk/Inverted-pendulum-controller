% !TEX encoding = utf8
\section{Wstęp}
Odwrócone wahadło na~wózku służy jako kanoniczny układ testowy 
dla metod sterowania i~estymacji stanu, ponieważ łączy w~sobie trudności 
typowe dla systemów rzeczywistych: nieliniowość, niestabilność 
w~otwartym układzie sterowania, ograniczenia urządzenia wykonawczego oraz niepewność 
parametrów. Umożliwia to weryfikację algorytmów w~sytuacjach, 
w~których klasyczne założenia teorii liniowej przestają obowiązywać, 
a~układ wymaga adaptacji lub podejścia optymalnego.

Model ten posiada liczne analogie w~praktyce inżynierskiej. Jego dynamika 
odwzorowuje wiele złożonych zjawisk fizycznych i~konstrukcji technicznych, 
w~tym:
\begin{itemize}
  \item stabilizację robotów dwukołowych (np. typu Segway) oraz robotów 
  mobilnych balansujących na~jednej osi~\cite{Segway2018};
  \item sterowanie suwnicami kontenerowymi oraz manipulatorami przemysłowymi, 
  gdzie kluczowe jest tłumienie oscylacji przenoszonego ładunku~\cite{Crane2012};
  \item równoważenie platform i~pojazdów samobalansujących, wymagające 
  ciągłej korekty siły napędowej względem położenia środka masy~\cite{Prasad2014}.
\end{itemize}

Ze~względu na~powyższe zastosowania, problem stabilizacji odwróconego wahadła 
traktowany jest jako uproszczony model systemów rzeczywistych o~zbliżonej 
dynamice. Badania symulacyjne na~tym obiekcie pozwalają na~wstępną walidację 
skuteczności algorytmów sterowania przed ich implementacją w~bardziej 
złożonych lub kosztownych systemach.

W~literaturze odwrócone wahadło na~wózku traktowane jest 
powszechnie jako wzorcowy układ testowy 
dla weryfikacji algorytmów sterowania układami niestabilnymi. 
Kompletny model nieliniowy obiektu przedstawiono w~pracy~\cite{Prasad2014}, 
uwzględniający zakłócen, na~podstawie którego 
przeprowadzono analizę porównawczą regulatorów PID, LQR w 
konfiguracjach hybrydowych.

Rozszerzenie zakresu badań o~sterowanie predykcyjne (MPC) zaprezentowano 
w~pozycji~\cite{Varghese2017}. Autorzy stworzyli jednorodne środowisko symulacyjne, 
zestawiając przebiegi zmiennych stanu dla metod PID, LQR oraz MPC. 
Uzyskane rezultaty potwierdziły przewagę rozwiązań opartych na~modelu 
(LQR, MPC) nad klasycznym PID w~kontekście jakości regulacji, 
podkreślając jednocześnie kluczową zaletę MPC --- możliwość bezpośredniego 
uwzględniania ograniczeń fizycznych nałożonych na~wielkości sterujące.

Współczesne prace badawcze coraz częściej integrują metody optymalne 
z~metodami sztucznej inteligencji. Artykuł~\cite{Nguyen2024} opisuje rozwiązanie 
hybrydowe, łączące regulator LQR z~modelem rozmytym Takagi--Sugeno oraz obserwatorem stanu. Podejście to pozwala 
na~przyspieszenie zbieżności błędu regulacji do~zera oraz poprawę 
jakości estymacji zmiennych w~obecności szumów pomiarowych 
i~niepewności parametrycznej modelu.

Istotnym uzupełnieniem badań symulacyjnych są weryfikacje eksperymentalne, 
szeroko reprezentowane w~krajowej literaturze naukowej. 
W~pracy Jezierskiego i~in.~\cite{Jezierski2017} przeprowadzono porównanie 
algorytmów LQR i~MPC na~rzeczywistym stanowisku laboratoryjnym. 
Wykazano, że o~ile regulator LQR skutecznie utrzymuje punkt pracy 
i~tłumi zakłócenia, to sterowanie predykcyjne zapewnia łagodniejsze 
sterowanie i~lepsze właściwości śledzenia trajektorii, co ma kluczowe 
znaczenie w~aplikacjach robotycznych.

Z~punktu widzenia podstaw teoretycznych, fundamentem dla implementacji 
sterowania predykcyjnego są prace monograficzne Camacho i~Bordonsa~\cite{Camacho2007}. 
Omawiają one szczegółowo zagadnienia 
doboru funkcji kosztu, horyzontów predykcji, a~także stabilności układu 
zamkniętego. Aspekty wdrożeniowe, w~tym efektywność numeryczna 
algorytmów optymalizacji na~platformach wbudowanych, poruszane są 
w~nowszych publikacjach~\cite{Mills2009}. Natomiast w~obszarze 
sterowania rozmytego cennym źródłem wiedzy metodycznej są opracowania 
dotyczące modeli Takagi--Sugeno i~ich porównań z~podejściami 
klasycznymi~\cite{Roose2017}.

Głównym celem niniejszej pracy jest zaprojektowanie oraz weryfikacja efektywności układu stabilizacji odwróconego wahadła na wózku. Obiekt ten, ze względu na swoje właściwości dynamiczne, stanowi klasyczny przykład układu nieliniowego i niestabilnego, co czyni go doskonałą platformą testową do analizy porównawczej różnorodnych strategii sterowania.

Aby zrealizować ten cel, zakres pracy obejmuje opracowanie autorskiego środowiska symulacyjnego, które wiernie odwzorowuje fizykę i dynamikę ruchu wózka z wahadłem. Środowisko to posłuży do implementacji i testowania wybranych algorytmów regulacji, reprezentujących przekrój współczesnej automatyki. W pracy rozważone zostaną trzy zasadnicze grupy metod:
\begin{itemize}
    \item podejście klasyczne, oparte na konwencjonalnych pętlach sprzężenia zwrotnego,
    \item metody sterowania optymalnego i predykcyjnego, uwzględniające model obiektu oraz ograniczenia sterowania,
    \item metody sterowania inteligentnego, wykorzystujące logikę rozmytą.
\end{itemize}

Kluczowym elementem pracy jest przeprowadzenie wielokryterialnej oceny działania zaprojektowanych układów. Analiza porównawcza nie ogranicza się jedynie do sprawdzenia zdolności utrzymania wahadła w pionie. Badania obejmują również weryfikację jakości regulacji w stanach przejściowych, analizę energochłonności poszczególnych rozwiązań oraz sprawdzenie ich odporności na zakłócenia zewnętrzne i niepewność parametrów modelu. Szczegółowe definicje algorytmów oraz matematyczne sformułowanie wskaźników jakości zostały przedstawione w kolejnych rozdziałach pracy.

Układ pracy został podzielony na~rozdziały, których treść odpowiada 
kolejnym etapom realizacji projektu.
Rozdział drugi przedstawia szczegółowe wyprowadzenie modelu matematycznego 
odwróconego wahadła na~wózku. Zawiera on opis fizyczny obiektu, 
równania dynamiki sformułowane w~oparciu o~prawa mechaniki, a~także 
linearyzację modelu niezbędną do~syntezy wybranych regulatorów.
Rozdział trzeci poświęcony jest opisowi zrealizowanego środowiska
symulacyjnego. Przedstawiono w~nim narzędzia programistyczne, metody 
numeryczne wykorzystane do~rozwiązywania równań różniczkowych 
oraz sposób modelowania zakłóceń zewnętrznych.
Rozdział czwarty zawiera charakterystykę zaimplementowanych algorytmów 
sterowania. Omówiono w~nim podstawy teoretyczne oraz szczegóły 
implementacyjne regulatorów: klasycznego PID-PID, optymalnego PID-LQR, 
predykcyjnego MPC/LMPC oraz rozmytego Takagi--Sugeno.
Rozdział piąty opisuje metodykę badań symulacyjnych. Zdefiniowano w~nim 
scenariusze testowe, przyjęte wskaźniki jakości oraz procedurę strojenia 
regulatorów, ze~szczególnym uwzględnieniem doboru wag macierzy LQR 
oraz nastaw regulatora PID.
Rozdział szósty prezentuje wyniki przeprowadzonych eksperymentów. 
Zawiera on szczegółową analizę przebiegów czasowych, zestawienie tabelaryczne 
błędów regulacji w~warunkach nominalnych i~zakłóconych, oraz dyskusję 
porównawczą skuteczności badanych metod.
Pracę kończy podsumowanie, zawierające wnioski końcowe oraz 
kierunki dalszego rozwoju projektu.