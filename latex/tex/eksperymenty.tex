\section{Eksperymenty}

Rozdział ten definiuje scenariusze testowe, przyjęte miary oceny jakości
sterowania oraz procedurę doboru nastaw regulatorów. Precyzyjne określenie
warunków eksperymentu jest kluczowe dla zapewnienia powtarzalności badań oraz
obiektywnego porównania testowanych algorytmów.

\subsection{Plan eksperymentów}

W~celu weryfikacji skuteczności strategii sterowania, przyjęto jednolity zestaw
testów symulacyjnych. Każdy z~zaimplementowanych regulatorów (PID, LQR, MPC, Fuzzy-LQR) poddany został badaniom w~następujących scenariuszach:

\begin{enumerate}
    \item \textbf{Eksperyment 1: Stabilizacja w~punkcie pracy (warunki nominalne).} \\
    Symulacja odpowiedzi układu na~niezerowe warunki początkowe przy braku
    zakłóceń zewnętrznych.
    \begin{itemize}
        \item Początkowy kąt wychylenia wahadła: $\varphi(0) = 0{,}05$ rad ($\approx 2{,}87^\circ$).
        \item Początkowa pozycja wózka: $x(0) = 0$ m.
        \item Zerowe prędkości początkowe: $\dot{\varphi}(0) = 0$, $\dot{x}(0) = 0$.
    \end{itemize}
    Wybór wartości $\varphi(0) = 0{,}05$ rad podyktowany jest dwoma czynnikami:
    jest to wychylenie na~tyle małe, że mieści się w~obszarze stosowalności 
    modelu zlinearyzowanego (istotne dla LQR), a~jednocześnie wystarczająco duże,
    aby wymagać aktywnej interwencji regulatora. Wartość ta jest również powszechnie
    stosowana w~literaturze przedmiotu jako standardowy warunek testowy~\cite{Prasad2014}.
    Celem jest sprawdzenie zdolności regulatora do~sprowadzenia układu do~pionu
    ($\varphi=0, x=0$) oraz ocena czasu regulacji i~przeregulowań.

    \item \textbf{Eksperyment 2: Odporność na~zakłócenia zewnętrzne.} \\
    Symulacja z~tymi samymi warunkami początkowymi, przy czym na~wahadło
    oddziałuje losowa siła zakłócająca $F_{\mathrm{w}}(t)$ (modelująca
    porywisty wiatr) generowana zgodnie z~procedurą opisaną w~Rozdziale 3. 
    Przyjęto moc zakłócenia $P = 5{,}0$ W, co odpowiada wartości skutecznej
    siły wiatru rzędu $\pm 2{,}2$ N --- jest to poziom zakłóceń stanowiący
    istotne wyzwanie dla układu sterowania, lecz nieprzekraczający możliwości
    kompensacyjnych regulatorów. Test ten pozwala ocenić krzepkość 
    (ang. \textit{robustness}) układu zamkniętego.
\end{enumerate}

Wszystkie symulacje przeprowadzono z~krokiem czasowym $\Delta t = 0{,}1$ s
w~czasie $T_{\mathrm{sim}} = 10$ s (dla testów MPC i~pełnego ustalenia).
Ograniczenie sygnału sterującego ustawiono na $|u| \le 100$ N.

\subsection{Badane algorytmy}

W~ramach eksperymentów przetestowano następujące regulatory (w~wersjach po
optymalizacji nastaw):
\begin{enumerate}
    \item \textbf{PD-PD} -- Kaskadowy układ dwóch regulatorów PD.
    \item \textbf{PD-LQR} -- Hybryda: PD dla wózka, LQR dla wahadła.
    \item \textbf{MPC} -- Klasyczny predykcyjny regulator liniowy.
    \item \textbf{MPC-J2} -- MPC z funkcją kosztu $J_2$ (uwzględniającą kwadrat
    sygnału sterującego wprost).
    \item \textbf{Fuzzy-LQR} -- Regulator rozmyty Takagi-Sugeno wspomagany
    lokalnym LQR.
\end{enumerate}

\subsection{Wskaźniki jakości regulacji}

Do~ilościowej oceny jakości sterowania wykorzystano następujące wskaźniki
błędów, obliczane dla zdyskretyzowanych przebiegów kąta $\varphi[k]$
($N$ próbek):

\begin{itemize}
    \item \textbf{MSE (Mean Squared Error)} -- Średni błąd kwadratowy, karający silniej duże odchyłki.
    \begin{equation}
        \mathrm{MSE} = \frac{1}{N}\sum_{k=1}^{N} (y[k] - y_{\mathrm{ref}}[k])^2
    \end{equation}

    \item \textbf{MAE (Mean Absolute Error)} -- Średni błąd bezwzględny, informujący o przeciętnym uchybie.
    \begin{equation}
        \mathrm{MAE} = \frac{1}{N}\sum_{k=1}^{N} |y[k] - y_{\mathrm{ref}}[k]|
    \end{equation}

    \item \textbf{ISE (Integral of Squared Error)} -- Całkowe kryterium kwadratowe, będące miarą energii uchybu w czasie ciągłym.
    \begin{equation}
        \mathrm{ISE} = \int_{0}^{T} (y(t) - y_{\mathrm{ref}}(t))^2 \, dt
    \end{equation}

    \item \textbf{IAE (Integral of Absolute Error)} -- Całkowe kryterium modułu błędu, akumulujące całkowity uchyb w czasie.
    \begin{equation}
        \mathrm{IAE} = \int_{0}^{T} |y(t) - y_{\mathrm{ref}}(t)| \, dt
    \end{equation}

    \item \textbf{Energia sterowania L2 ($E_{u, L2}$)} -- Koszt kwadratowy sterowania, powiązany z energią elektryczną/mechaniczną.
    \begin{equation}
        E_{u, L2} = \int_{0}^{T} u(t)^2 \, dt
    \end{equation}

    \item \textbf{Energia sterowania L1 ($E_{u, L1}$)} -- Koszt absolutny sterowania (zużycie paliwa/zasobów).
    \begin{equation}
        E_{u, L1} = \int_{0}^{T} |u(t)| \, dt
    \end{equation}

    \item \textbf{Czas regulacji $t_s$ (Settling Time)} -- Czas, po którym sygnał wyjściowy trwale wchodzi w~kanał tolerancji i~już go nie opuszcza. W~niniejszej pracy przyjęto tolerancję $\varepsilon = 2\%$ wartości początkowego wychylenia, tj. $|\varphi| < 0{,}001$ rad dla kąta oraz $|x| < 0{,}002$ m dla pozycji.

    \item \textbf{Przeregulowanie $M_p$ (Overshoot)} -- Maksymalne procentowe odchylenie sygnału od wartości zadanej w odniesieniu do wartości skoku.
    \begin{equation}
        M_p = \frac{\max(y) - y_{\mathrm{ref}}}{y_{\mathrm{ref}}} \cdot 100\%
    \end{equation}

    \item \textbf{Uchyb ustalony $e_{ss}$ (Steady-state Error)} -- Średnia wartość uchybu w końcowej fazie symulacji (ostatnie 10\% czasu), określająca dokładność statyczną regulacji.

\end{itemize}

Dodatkowo analizie poddano charakterystyki jakościowe przebiegów czasowych,
takie jak czas regulacji (czas, po którym błąd trwale mieści się w paśmie
$\pm 2\%$) oraz maksymalne przeregulowanie.
