\section{Analiza wyników}

Rozdział ten poświęcony jest szczegółowej analizie wyników badań symulacyjnych, które zostały
przeprowadzone w celu weryfikacji skuteczności i jakości działania zaprojektowanych układów
sterowania. Głównym celem eksperymentów było zbadanie zachowania wahadła odwróconego w dwóch
diametralnie różnych sytuacjach: podczas stabilizacji punktu pracy w idealnych warunkach
nominalnych oraz w trakcie pracy pod wpływem losowych zakłóceń zewnętrznych.

Podczas analizy wyników szczególny nacisk położono na dwa kluczowe, i nierzadko sprzeczne ze
sobą, aspekty sterowania. Pierwszym z nich jest stabilizacja kątowa, czyli zdolność układu do
utrzymania pręta wahadła w~pionie (pozycja równowagi chwiejnej). Jest to zadanie priorytetowe,
gdyż jego niezrealizowanie prowadzi do~upadku wahadła i~niepowodzenia procesu regulacji. Drugim, równie
istotnym aspektem, jest stabilizacja pozycji wózka. Wymogiem jest, aby proces stabilizacji
kąta nie odbywał się kosztem nadmiernego przemieszczenia wózka poza zadany obszar roboczy. W~systemach
rzeczywistych, takich jak suwnice czy roboty balansujące, utrzymanie pozycji jest często równie
krytyczne co sama stabilizacja ładunku.

Dla zachowania przejrzystości wywodu, badane algorytmy pogrupowano w dwie rodziny: regulatory
klasyczne, do których zaliczono równoległy układ PD oraz hybrydowy PID-LQR, oraz regulatory
zaawansowane, obejmujące predykcyjny algorytm MPC (w dwóch wariantach funkcji kosztu) oraz
sterownik rozmyty Fuzzy-LQR.

\subsection{Stabilizacja w warunkach nominalnych}

Pierwszy scenariusz testowy miał na celu weryfikację dynamiki układu w odpowiedzi na niezerowe
warunki początkowe. Symulacja rozpoczynała się od wychylenia wahadła o kąt około 2.8 stopnia
(0.05 radiana). Jest to typowy test odpowiedzi skokowej, pozwalający ocenić szybkość działania
(czas regulacji) oraz tłumienie oscylacji przez poszczególne regulatory.

\subsubsection{Charakterystyka regulatorów klasycznych}

Na Rysunkach \ref{fig:nom_classical}, \ref{fig:nom_pos_classical} oraz
\ref{fig:nom_control_classical} przedstawiono zbiorcze zestawienie przebiegów czasowych dla
grupy regulatorów klasycznych. Analizując wykres kąta wychylenia $\theta$
(Rys. \ref{fig:nom_classical}), można zaobserwować, że oba regulatory radzą sobie ze stabilizacją,
jednak robią to w różnym stylu.

Zoptymalizowany regulator PID-LQR, wykorzystujący duże wzmocnienia dla błędu pozycji ($Q_x=500$),
charakteryzuje się krótkim czasem regulacji ($T_s \approx 0.2$~s). Jest to wynik
o~33\% lepszy od~regulatora PD ($T_s \approx 0.3$~s). PID-LQR działa z~wysoką dynamiką,
co widać na~wykresie pozycji (Rys. \ref{fig:nom_pos_classical}) --- wózek wykonuje szybki ruch
korekcyjny, niemal natychmiast ustalając pozycję w~pobliżu zera. Kosztem tej dynamiki jest jednak
zwiększone zużycie energii. W~przeciwieństwie do~poprzednich iteracji strojenia, obecny
LQR o~wysokich wzmocnieniach zużywa więcej energii ($E_u \approx 1.48$) niż regulator PD 
o~łagodniejszej charakterystyce ($E_u \approx 0.85$), który charakteryzuje się dłuższym 
i~łagodniejszym dochodzeniem do~równowagi. Można zatem zaobserwować zmianę charakterystyki: PD stał się 
rozwiązaniem bardziej ekonomicznym w~stanie przejściowym, podczas gdy PID-LQR charakteryzuje się 
szybszą odpowiedzią dynamiczną.

\begin{figure}[ht!]
    \centering
    \includegraphics[width=0.95\textwidth]{images/experiments/combined_nominal_classical.png}
    \caption{Przebieg kąta $\theta$ dla regulatorów klasycznych (Warunki nominalne).}
    \label{fig:nom_classical}
\end{figure}
\newpage
\begin{figure}[ht!]
    \centering
    \includegraphics[width=0.95\textwidth]{images/experiments/combined_nominal_pos_classical.png}
    \caption{Przebieg pozycji $x$ dla regulatorów klasycznych (Warunki nominalne).}
    \label{fig:nom_pos_classical}
\end{figure}

\begin{figure}[ht!]
    \centering
    \includegraphics[width=0.95\textwidth]{images/experiments/combined_nominal_control_classical.png}
    \caption{Sygnał sterujący $u$ dla regulatorów klasycznych (Warunki nominalne).}
    \label{fig:nom_control_classical}
\end{figure}

\subsubsection{Charakterystyka regulatorów zaawansowanych}

W grupie regulatorów zaawansowanych, których wyniki zaprezentowano na Rysunkach
\ref{fig:nom_advanced}, \ref{fig:nom_pos_advanced} i~\ref{fig:nom_control_advanced},
można zaobserwować porównanie dwóch odmiennych podejść: predykcji opartej na~modelu (MPC) oraz 
sterowania rozmytego (Fuzzy-LQR).

Regulator Fuzzy-LQR wyróżnia się charakterystyką o~wysokiej dynamice. Jego działanie przypomina
strategię typu bang-bang (sterowanie dwustanowe), gdzie w~pierwszej fazie ruchu generowany jest silny
impuls sterujący (widoczny na~Rys.~\ref{fig:nom_control_advanced}), mający na~celu jak
najszybsze zniwelowanie uchybu kątowego. Dzięki temu podejściu, czas regulacji kąta jest
najkrótszy w~porównaniu z~innymi regulatorami i wynosi zaledwie 0.70 s. Ma to jednak swoją cenę w zachowaniu wózka. Aby
wygenerować tak dużą siłę prostującą wahadło, wózek musi wykonać gwałtowny ruch korekcyjny, co widać na
wykresie pozycji (Rys.~\ref{fig:nom_pos_advanced}). Choć wózek szybko wraca do~zera,
początkowe przyspieszenie jest znaczne. Taka charakterystyka wiąże się z~kosztem
energetycznym wyższym o~405\% względem MPC ($E_u \approx 2.84$ vs. $0.56$), co może być 
nieakceptowalne w~aplikacjach zasilanych bateryjnie.

Odmienną charakterystykę wykazuje regulator MPC. Zastosowanie horyzontu predykcji pozwala
na uwzględnienie przyszłych stanów i wyznaczenie optymalnej trajektorii sterowania na
kilkanaście kroków do przodu. Funkcja kosztu wymusza kompromis między szybkością
redukcji błędu a~minimalizacją wydatku energetycznego. W~efekcie przebieg kąta jest nieco
wolniejszy niż w~przypadku Fuzzy-LQR, ale charakteryzuje się wysoką płynnością 
(Rys.~\ref{fig:nom_advanced}).
Co najważniejsze, ruch wózka jest w pełni kontrolowany i pozbawiony gwałtownych przyspieszeń.
MPC jako jedyny regulator pozwala w sposób jawny uwzględnić ograniczenia fizyczne napędu, co
czyni go rozwiązaniem najbezpieczniejszym dla mechaniki układu. Zużycie energii na poziomie
$E_u \approx 0.56$ jest ponad dwukrotnie niższe od~obecnego LQR ($1.48$) oraz niższe od~PD,
co czyni MPC najbardziej ekonomicznym rozwiązaniem w~warunkach nominalnych.

Wariant MPC-J2, mimo zastosowania innej funkcji kosztu, zachowuje się bardzo podobnie do
klasycznego MPC w~warunkach nominalnych. Różnice między nimi ujawnią się dopiero w~teście
odpornościowym.
\newpage
\begin{figure}[h!]
    \centering
    \includegraphics[width=0.95\textwidth]{images/experiments/combined_nominal_advanced.png}
    \caption{Przebieg kąta $\theta$ dla regulatorów zaawansowanych (Warunki nominalne).}
    \label{fig:nom_advanced}
\end{figure}

\begin{figure}[h!]
    \centering
    \includegraphics[width=0.95\textwidth]{images/experiments/combined_nominal_pos_advanced.png}
    \caption{Przebieg pozycji $x$ dla regulatorów zaawansowanych (Warunki nominalne).}
    \label{fig:nom_pos_advanced}
\end{figure}

\begin{figure}[h!]
    \centering
    \includegraphics[width=0.95\textwidth]{images/experiments/combined_nominal_control_advanced.png}
    \caption{Sygnał sterujący $u$ dla regulatorów zaawansowanych (Warunki nominalne).}
    \label{fig:nom_control_advanced}
\end{figure}

\subsection{Analiza odporności na zakłócenia}

Drugi scenariusz badawczy stanowił bardziej wymagający test. Do układu wprowadzono sygnał
zakłócający, modelujący losowe zakłócenia zewnętrzne o zmiennej sile i kierunku. Test ten miał na
celu sprawdzenie odporności regulatorów na zakłócenia zewnętrzne, czyli ich zdolności do
utrzymania stabilności mimo działania nieznanych, zewnętrznych sił.

Podstawowym problemem fizycznym w tym scenariuszu jest zjawisko sprzężenia dryfu. Aby
skompensować siłę zakłócającą pchającą wahadło np. w prawo, wózek musi nieustannie przyspieszać w
prawo, aby przemieścić się pod środek ciężkości wahadła i wytworzyć moment siły bezwładności
przeciwdziałający zakłóceniu. Oznacza to, że skuteczna kompensacja wychylenia kątowego nieuchronnie
prowadzi do przemieszczania się wózka (dryfu). Istotą problemu jest znalezienie kompromisu --- jak
bardzo pozwolić wózkowi uciec, by utrzymać wahadło w pionie.

W grupie klasycznej (Rys. \ref{fig:wind_classical} i \ref{fig:wind_pos_classical}) nastąpiła
istotna zmiana w stosunku do wcześniejszych analiz. Nowe strojenie PID-LQR, nastawione na
karę za zmianę pozycji, przyniosło znaczące efekty. LQR nie tylko lepiej stabilizuje
kąt ($Max |\theta| \approx 0.060$ vs $0.068$~rad dla PD), ale przede wszystkim istotnie
redukuje dryf pozycji ($Max |x| \approx 0.22$ m vs $0.26$ m). Warto zauważyć, że
osiąga to przy niższym zużyciu energii ($E_u \approx 11.7$) niż regulator PD ($15.4$).
Wynika to z faktu, że optymalny regulator tłumi zakłócenie w~zarodku, nie pozwalając na
rozhustanie się układu, podczas gdy PD kompensuje skutki oscylacji, które sam dopuścił.

W grupie zaawansowanej (Rys. \ref{fig:wind_advanced} i \ref{fig:wind_pos_advanced})
regulator Fuzzy-LQR osiąga najlepsze wyniki w~dziedzinie precyzji kątowej
($Max |\theta| \approx 0.05$~rad --- minimalne wychylenie), jednak koszt energetyczny
jest znaczący ($E_u \approx 85.9$). Sterownik ten charakteryzuje się strategią sterowania 
o~wysokiej dynamice --- wykorzystuje maksymalną dostępną moc w~celu minimalizacji błędu regulacji.

Zupełnie inną charakterystykę w~obliczu zakłóceń zewnętrznych wykazuje MPC. Ze względu na postać funkcji kosztu,
która penalizuje duże wartości sterowania, regulator generuje mniejsze sygnały korekcyjne.
Prowadzi to do pewnych, kontrolowanych wychyleń wahadła pod wpływem zakłóceń. Skutkuje to największym 
dryfem wózka w zestawieniu ($Max |x| \approx 0.40$ m), gdyż układ jest spychany przez siłę zakłócającą, 
ale za to zużycie energii jest niskie ($E_u \approx 12.6$). Warto zauważyć, że w~tych konkretnych 
warunkach PID-LQR okazał się lepszy od MPC zarówno pod względem trzymania pozycji, jak i zużycia energii,
co dowodzi, że dobrze nastrojony regulator liniowy może konkurować z predykcyjnym, o ile nie
występują nasycenia sterowania.

Należy odnotować nieskuteczność wariantu MPC-J2 w~tym teście. Jego funkcja kosztu, penalizująca
bezpośrednio wartość sterowania ($u$), a nie jej zmianę ($\Delta u$), okazała się zbyt
restrykcyjna. Ze względu na wysoką karę za sterowanie, wyznaczane wartości sygnału sterującego 
były niewystarczające do skompensowania zakłóceń, co doprowadziło do przekroczenia krytycznego 
kąta wychylenia i~przewrócenia się wahadła. Jest to istotny wniosek projektowy, ilustrujący 
wpływ doboru funkcji celu na~odporność układu.

\begin{figure}[h!]
    \centering
    \includegraphics[width=0.95\textwidth]{images/experiments/combined_wind_classical.png}
    \caption{Przebieg kąta $\theta$ pod wpływem zakłóceń zewnętrznych -- regulatory klasyczne.}
    \label{fig:wind_classical}
\end{figure}

\begin{figure}[h!]
    \centering
    \includegraphics[width=0.95\textwidth]{images/experiments/combined_wind_pos_classical.png}
    \caption{Dryf pozycji $x$ pod wpływem zakłóceń zewnętrznych -- regulatory klasyczne.}
    \label{fig:wind_pos_classical}
\end{figure}

\begin{figure}[h!]
    \centering
    \includegraphics[width=0.95\textwidth]{images/experiments/combined_wind_advanced.png}
    \caption{Przebieg kąta $\theta$ pod wpływem zakłóceń zewnętrznych -- regulatory zaawansowane.}
    \label{fig:wind_advanced}
\end{figure}

\begin{figure}[h!]
    \centering
    \includegraphics[width=0.95\textwidth]{images/experiments/combined_wind_pos_advanced.png}
    \caption{Dryf pozycji $x$ pod wpływem zakłóceń zewnętrznych -- regulatory zaawansowane.}
    \label{fig:wind_pos_advanced}
\end{figure}

\subsection{Analiza odporności na zmianę parametrów modelu}

Trzeci scenariusz badawczy miał na celu ocenę wrażliwości regulatorów na~niepewność
parametryczną modelu. W~praktycznych zastosowaniach przemysłowych dokładne wartości
parametrów fizycznych układu są~rzadko znane z~wysoką precyzją. Mogą one ulegać
zmianom w~czasie (np.~zużycie mechaniczne, zmiana ładunku), dlatego odporność na~takie
perturbacje jest kluczową właściwością regulatora.

W~eksperymencie zwiększono masę wahadła o~10\% względem wartości nominalnej
($m_{\mathrm{nom}} = 0{,}23$~kg $\rightarrow$ $m_{\mathrm{real}} = 0{,}253$~kg),
podczas gdy regulatory pozostały nastrojone dla parametrów nominalnych. Taka zmiana
masy wpływa na~dynamikę układu w~sposób nieliniowy --- modyfikuje zarówno moment
bezwładności wahadła, jak i~położenie jego środka ciężkości względem osi obrotu.

Wyniki eksperymentu zaprezentowano na~Rysunkach \ref{fig:robust_theta}, \ref{fig:robust_x}
oraz \ref{fig:robust_u}. Kluczową obserwacją jest fakt, że wszystkie badane regulatory
zachowały stabilność mimo nieprawidłowego modelu. Świadczy to o~odpowiednim marginesie
stabilności wynikającym z~procesu optymalizacji nastaw.

Analizując przebieg kąta $\theta$ (Rys.~\ref{fig:robust_theta}), można zauważyć, że
regulatory predykcyjne MPC i~MPC-J2 wykazują największą wrażliwość na~zmianę parametrów.
Ich odpowiedź jest wolniejsza niż w~warunkach nominalnych, co wynika z~faktu, że
wewnętrzny model używany do~predykcji nie odpowiada rzeczywistej dynamice obiektu.
Regulator musi korygować błędy predykcji w~sposób reaktywny, tracąc część zalet
podejścia predykcyjnego.

Regulatory klasyczne (PD-PD, PD-LQR) okazały się bardziej odporne na~perturbację
parametryczną. Ich struktura, oparta na~sprzężeniu zwrotnym od~błędu regulacji,
nie wymaga dokładnego modelu obiektu --- korekta następuje na~podstawie obserwowanego
uchybu, a~nie predykowanych stanów. W~szczególności regulator PD-LQR, dzięki
wysokim wzmocnieniom w~macierzy $Q$, skutecznie kompensuje zwiększoną bezwładność
wahadła.

Regulator Fuzzy-LQR wykazuje interesującą charakterystykę. Jego odpowiedź w~fazie
początkowej jest zbliżona do~warunków nominalnych, jednak widoczne jest niewielkie
przeregulowanie pozycji wózka (Rys.~\ref{fig:robust_x}). Wynika to z~faktu, że
logika rozmyta adaptuje się do~obserwowanego stanu układu, ale lokalne regulatory
LQR zostały zaprojektowane dla parametrów nominalnych.

Na~wykresie sterowania (Rys.~\ref{fig:robust_u}) widać, że regulatory muszą generować
większe sygnały sterujące, aby skompensować zwiększoną masę wahadła. Jest to zgodne
z~intuicją fizyczną --- cięższa masa wymaga większej siły do~przyspieszenia.
Największy wzrost amplitudy sterowania obserwuje się dla regulatora Fuzzy-LQR,
co potwierdza jego agresywną strategię minimalizacji uchybu kątowego.

\begin{figure}[h!]
    \centering
    \includegraphics[width=0.95\textwidth]{images_odpornosc/robustness_theta.png}
    \caption{Przebieg kąta $\theta$ przy zmienionych parametrach modelu (+10\% masy wahadła).}
    \label{fig:robust_theta}
\end{figure}
\newpage
\begin{figure}[h!]
    \centering
    \includegraphics[width=0.95\textwidth]{images_odpornosc/robustness_x.png}
    \caption{Przebieg pozycji $x$ przy zmienionych parametrach modelu (+10\% masy wahadła).}
    \label{fig:robust_x}
\end{figure}

\begin{figure}[h!]
    \centering
    \includegraphics[width=0.95\textwidth]{images_odpornosc/robustness_u.png}
    \caption{Sygnał sterujący $u$ przy zmienionych parametrach modelu (+10\% masy wahadła).}
    \label{fig:robust_u}
\end{figure}

\subsubsection{Analiza wrażliwości na zakres perturbacji}

W~celu pełniejszej oceny zapasów odporności poszczególnych regulatorów przeprowadzono
dodatkową analizę wrażliwości. Zbadano zachowanie układów sterowania w~szerokim zakresie
zmian masy wahadła: od~$-20\%$ do~$+50\%$ wartości nominalnej. Dla każdej wartości
perturbacji obliczono wskaźnik całkowy błędu bezwzględnego kąta ($IAE_\theta$),
który jest miarą skumulowanego uchybu w~czasie symulacji.

Wyniki analizy przedstawiono na~Rysunku~\ref{fig:robust_sensitivity}. Można zaobserwować
kilka istotnych prawidłowości:

\begin{itemize}
    \item \textbf{Regulator Fuzzy-LQR} wykazuje najlepszą odporność na~niepewność
    parametryczną, osiągając najniższe wartości $IAE_\theta$ w~całym badanym zakresie.
    Co więcej, jego charakterystyka jest praktycznie płaska --- zmiana masy wahadła
    nie wpływa istotnie na~jakość regulacji. Wynika to z~adaptacyjnej natury logiki
    rozmytej, która dostosowuje wagi reguł do~obserwowanego stanu układu.
    
    \item \textbf{Regulatory klasyczne (PD-PD, PD-LQR)} również charakteryzują się
    płaską charakterystyką w~całym zakresie perturbacji, choć z~nieco wyższymi
    wartościami błędu niż Fuzzy-LQR. Ich jakość regulacji jest mało wrażliwa na~niepewność 
    parametryczną dzięki strukturze opartej na~sprzężeniu zwrotnym od~błędu.
    
    \item \textbf{Regulatory predykcyjne (MPC, MPC-J2)} wykazują najwyższe wartości
    wskaźnika $IAE_\theta$. Jest to spodziewane zachowanie, gdyż algorytm optymalizacji 
    wykorzystuje wewnętrzny model, który odbiega od~rzeczywistej dynamiki obiektu.
    Niemniej jednak, regulatory te zachowują stabilność w~całym badanym zakresie,
    a~wzrost błędu wraz z~perturbacją jest umiarkowany.
\end{itemize}

Analiza ta~pokazuje, że regulatory wykorzystujące mechanizmy adaptacyjne (Fuzzy-LQR)
lub proste sprzężenie zwrotne od~błędu (PD, LQR) mogą oferować lepszą odporność
na~niepewność modelu niż metody predykcyjne, których skuteczność zależy od~dokładności
wewnętrznego modelu obiektu.

\begin{figure}[h!]
    \centering
    \includegraphics[width=0.95\textwidth]{images_odpornosc/robustness_sensitivity.png}
    \caption{Analiza wrażliwości: zależność wskaźnika $IAE_\theta$ od~zmiany masy
    wahadła dla poszczególnych regulatorów. Linia pionowa oznacza warunki nominalne.}
    \label{fig:robust_sensitivity}
\end{figure}

\subsection{Szczegółowe zestawienie ilościowe}

Poniższe tabele stanowią numeryczne podsumowanie omówionych wyżej zjawisk. Dane zostały
zgrupowane w sposób ułatwiający porównanie osiągów w dwóch domenach: stabilizacji wahadła (kąt)
oraz stabilizacji wózka (pozycja).

\begin{table}[h!]
    \centering
    \caption{Wskaźniki jakości (Kąt i Pozycja) - warunki nominalne}
    \label{tab:results_nominal}
    \begin{tabular}{|l|c|c|c|c|c|}
        \hline
        Wskaźnik & PD & PID-LQR & MPC & MPC-J2 & Fuzzy-LQR \\ \hline
        $MSE_\theta$ & 0.00005 & 0.00004 & 0.00006 & 0.00005 & 0.00007 \\ \hline
        $IAE_\theta$ & 0.01684 & 0.01259 & 0.02287 & 0.02010 & 0.02301 \\ \hline
        $T_{s, \theta}$ [s] & 0.30000 & 0.20000 & 1.20000 & 1.10000 & 0.70000 \\ \hline
        \hline
        $MSE_x$ & 0.00040 & 0.00048 & 0.00038 & 0.00046 & 0.00037 \\ \hline
        $T_{s, x}$ [s] & 1.20000 & 2.20000 & 0.90000 & 1.00000 & 1.20000 \\ \hline
        \hline
        $E_{u}$ & 0.84515 & 1.47775 & 0.56337 & 0.62171 & 2.84489 \\ \hline
    \end{tabular}
\end{table}

\begin{table}[h!]
    \centering
    \caption{Wskaźniki jakości (Kąt i Pozycja) - zakłócenia zewnętrzne}
    \label{tab:results_wind}
    \begin{tabular}{|l|c|c|c|c|c|}
        \hline
        Wskaźnik & PD & PID-LQR & MPC & MPC-J2 & Fuzzy-LQR \\ \hline
        $MSE_\theta$ & 0.00060 & 0.00044 & 0.00058 & 0.00067 & 0.00031 \\ \hline
        $IAE_\theta$ & 0.18981 & 0.16304 & 0.18215 & 0.20848 & 0.13488 \\ \hline
        $Max |\theta|$ [rad] & 0.06821 & 0.06017 & 0.06235 & 0.06577 & 0.05000 \\ \hline
        \hline
        $MSE_x$ & 0.00466 & 0.00328 & 0.01772 & 0.00317 & 0.00440 \\ \hline
        $Max |x|$ [m] & 0.26214 & 0.22189 & 0.40389 & 0.23177 & 0.24186 \\ \hline
        \hline
        $E_{u}$ & 15.39160 & 11.73032 & 12.59235 & 14.42652 & 85.88725 \\ \hline
    \end{tabular}
\end{table}

\begin{table}[h!]
    \centering
    \caption{Wskaźniki jakości (Kąt i Pozycja) - odporność na zmianę parametrów modelu (+10\% masy wahadła)}
    \label{tab:results_robustness}
    \begin{tabular}{|l|c|c|c|c|c|}
        \hline
        Wskaźnik & PD & PID-LQR & MPC & MPC-J2 & Fuzzy-LQR \\ \hline
        $MSE_\theta$ & 0.00005 & 0.00004 & 0.00006 & 0.00005 & 0.00008 \\ \hline
        $IAE_\theta$ & 0.01687 & 0.01262 & 0.02299 & 0.02130 & 0.02312 \\ \hline
        $T_{s, \theta}$ [s] & 0.30 & 0.20 & 1.20 & 1.00 & 0.70 \\ \hline
        \hline
        $MSE_x$ & 0.00040 & 0.00048 & 0.00037 & 0.00047 & 0.00037 \\ \hline
        $T_{s, x}$ [s] & 1.20 & 2.20 & 0.90 & 1.00 & 1.20 \\ \hline
        \hline
        $E_{u}$ & 0.85 & 1.47 & 0.57 & 0.61 & 2.88 \\ \hline
    \end{tabular}
\end{table}

\newpage
\subsection{Analiza wpływu kary za sterowanie w MPC z alternatywną funkcją kosztu}

W celu empirycznej weryfikacji wpływu parametru $R_\mathrm{abs}$ (kary za bezwzględną wartość sterowania)
na jakość regulacji, przeprowadzono dodatkową serię eksperymentów dla regulatora MPC-J2.
Porównano dwa warianty: wariant bazowy ($R_\mathrm{abs}=0$) oraz wariant z włączoną karą energetyczną ($R_\mathrm{abs}=10^{-4}$).

Zestawienie wyników (Tabela \ref{tab:mpc_j2_study}) oraz przebiegi sterowania (Rys. \ref{fig:mpc_j2_study_u})
potwierdzają teoretyczne założenia. Włączenie nawet niewielkiej kary $R_\mathrm{abs}$ powoduje
zauważalną redukcję amplitudy sygnału sterującego. W~warunkach nominalnych pozwala to zaoszczędzić
ok. 10\% energii całkowitej ($E_u$ spada z 0.62 do 0.57), przy praktycznie niezauważalnym pogorszeniu
jakości regulacji kąta ($MSE_\theta$ wzrasta pomijalnie).

Jest to istotny wynik praktyczny, pokazujący, że świadome kształtowanie funkcji kosztu pozwala
na dostrojenie charakterystyki regulatora do specyficznych wymagań aplikacji (np. oszczędzanie akumulatora)
bez utraty stabilności.

\begin{figure}[h!]
    \centering
    \includegraphics[width=0.95\textwidth]{images/experiments/combined_nominal_mpc_j2_study.png}
    \caption{Porównanie sygnału sterującego $u$ dla MPC-J2 z różnymi wartościami $R_{abs}$.}
    \label{fig:mpc_j2_study_u}
\end{figure}

\begin{table}[h!]
    \centering
    \caption{Wpływ parametru $R_{abs}$ na wskaźniki jakości (Warunki nominalne)}
    \label{tab:mpc_j2_study}
    \begin{tabular}{|l|c|c|}
        \hline
        Wskaźnik & MPC-J2 ($R_{abs}=0$) & MPC-J2 ($R_{abs}=10^{-4}$) \\ \hline
        $MSE_\theta$ & 0.00005 & 0.00006 \\ \hline
        $T_{s, \theta}$ [s] & 1.10000 & 1.10000 \\ \hline
        $E_{u}$ & 0.62171 & 0.56937 \\ \hline
    \end{tabular}
\end{table}

\subsection{Porównanie złożoności obliczeniowej}

Istotnym kryterium oceny regulatorów, szczególnie w~kontekście implementacji
na~platformach wbudowanych, jest czas obliczeń wymagany do~wyznaczenia sygnału
sterującego. W~Tabeli \ref{tab:computation_time} zestawiono średnie czasy
wykonania jednej iteracji pętli sterowania dla poszczególnych algorytmów,
zmierzone na~komputerze z~procesorem Intel Core i5-8250U (1.6 GHz).

\begin{table}[h!]
    \centering
    \caption{Średni czas obliczeń jednej iteracji pętli sterowania}
    \label{tab:computation_time}
    \begin{tabular}{|l|c|c|}
        \hline
        Regulator & Czas [ms] & Względem PD \\ \hline
        PD & $< 0{,}01$ & $1\times$ \\ \hline
        PID-LQR & $0{,}02$ & $2\times$ \\ \hline
        Fuzzy-LQR & $0{,}05$ & $5\times$ \\ \hline
        MPC & $2{,}5$ & $250\times$ \\ \hline
        MPC-J2 & $2{,}8$ & $280\times$ \\ \hline
    \end{tabular}
\end{table}

Regulatory klasyczne (PD, PID-LQR) oraz rozmyty (Fuzzy-LQR) charakteryzują się
zaniedbywalnym czasem obliczeń, rzędu mikrosekund. Wynika to z~ich struktury
algebraicznej --- wyznaczenie sterowania sprowadza się do~mnożenia macierzy
i~prostych operacji arytmetycznych.

W~przypadku regulatorów MPC czas obliczeń jest o~dwa rzędy wielkości wyższy
(ok. 2{,}5--3 ms), co wynika z~konieczności rozwiązywania w~każdym kroku
zadania optymalizacji nieliniowej. Wartości te pozostają jednak znacznie poniżej
kroku symulacji ($\Delta t = 100$~ms), co potwierdza możliwość pracy MPC
w~czasie rzeczywistym dla rozpatrywanego obiektu. Należy jednak pamiętać,
że przy implementacji na~mikrokontrolerze czasy te mogą wzrosnąć nawet
10--100-krotnie, co może wymagać zastosowania uproszczonych wariantów MPC
lub dedykowanych bibliotek optymalizacji.
