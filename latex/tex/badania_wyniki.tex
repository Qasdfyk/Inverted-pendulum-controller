\section{Analiza wyników eksperymentalnych}

W~niniejszym rozdziale przedstawiono wyniki symulacji przeprowadzonych zgodnie
z~planem opisanym w~Rozdziale 5. Analizie poddano przebiegi zmiennych stanu,
sygnały sterujące oraz wyznaczone wskaźniki jakości dla wszystkich badanych
regulatorów: PD-PD, PD-LQR, MPC, MPC-J2 oraz Fuzzy-LQR.

\subsection{Eksperyment 1: Stabilizacja w~warunkach nominalnych}

Pierwszy test polegał na~stabilizacji wahadła z~początkowego wychylenia
$\varphi(0) = 45^\circ$. Na~Rysunku \ref{fig:combined_nominal} przedstawiono
porównanie przebiegów kąta wychylenia $\theta$ dla wszystkich regulatorów.

\begin{figure}[h!]
    \centering
    \includegraphics[width=0.95\textwidth]{images/experiments/combined_nominal.png}
    \caption{Porównanie odpowiedzi skokowej regulatorów (Warunki nominalne).}
    \label{fig:combined_nominal}
\end{figure}

Z~przebiegów czasowych wynika, że regulator \textbf{PD-LQR} osiągnął stabilizację
zdecydowanie najszybciej, co znajduje odzwierciedlenie w~bardzo niskiej wartości
błędu średniokwadratowego (MSE). Regulatory \textbf{MPC} oraz \textbf{MPC-J2}
również poprawnie ustabilizowały obiekt, jednak z~widocznym czasem regulacji.
Regulator \textbf{PD-PD} wykazał poprawne działanie, choć z~większym uchybem
w~stanie przejściowym. Regulator \textbf{Fuzzy-LQR} w~badanej konfiguracji nie
poradził sobie z~tak dużym wychyleniem początkowym, wpadając w~niestabilność
(znaczące oscylacje).

\subsection{Eksperyment 2: Odporność na~wiatr}

W~drugim scenariuszu dołączono losowe zakłócenie (wiatr). Rysunek
\ref{fig:combined_wind} ukazuje reakcję układów na te trudniejsze warunki.

\begin{figure}[h!]
    \centering
    \includegraphics[width=0.95\textwidth]{images/experiments/combined_wind.png}
    \caption{Porównanie odpowiedzi regulatorów przy zakłóceniu wiatrem.}
    \label{fig:combined_wind}
\end{figure}

Ponownie \textbf{PD-LQR} wykazał się największą sztywnością i~najmniejszą
podatnością na zakłócenia. Regulatory predykcyjne (MPC) również zachowały
stabilność, choć widoczny jest wpływ zakłóceń na ich przebiegi.

\subsection{Analiza ilościowa -- zestawienie wskaźników}

W~Tabeli \ref{tab:wyniki} zestawiono wartości wskaźników jakości MSE (dla kąta)
oraz całkowitej energii sterowania ($E_{L2}$) dla obu eksperymentów.

\begin{table}[h!]
    \centering
    \caption{Zestawienie wskaźników jakości dla badanych regulatorów}
    \label{tab:wyniki}
    \begin{tabular}{|l|c|c|c|c|}
        \hline
        \multirow{2}{*}{\textbf{Regulator}} & \multicolumn{2}{c|}{\textbf{Nominal (Bez wiatru)}} & \multicolumn{2}{c|}{\textbf{Wiatr}} \\ \cline{2-5} 
         & $\mathbf{MSE}_\theta$ & $E_{u, L2}$ & $\mathbf{MSE}_\theta$ & $E_{u, L2}$ \\ \hline
        PD-PD      & 0.0000    & 0.85       & 0.0006      & 15.39      \\ \hline
        PD-LQR     & 0.0001    & \textbf{0.54} & 0.0007      & 13.50      \\ \hline
        MPC        & 0.0001    & 0.56       & \textbf{0.0006} & \textbf{12.56} \\ \hline
        MPC-J2     & 0.0001    & 0.59       & 5.8563      & 15895.41   \\ \hline
        Fuzzy-LQR  & 0.0001    & 1.57       & 2090.0418   & 80542.54   \\ \hline
    \end{tabular}
\end{table}

\subsection{Dyskusja wynikiów}

\begin{itemize}
    \item \textbf{PD-LQR}: Potwierdził swoją dominację w warunkach nominalnych, zużywając
    najmniej energii ($E_{u, L2}=0.54$). Jest to rozwiązanie najbardziej ekonomiczne
    dla małych wymuszeń.
    \item \textbf{MPC}: Wykazał się największą odpornością na wiatr, uzyskując najniższy
    błąd MSE (0.0006) oraz najmniejszą energię sterowania w trudnych warunkach (12.56).
    Pokazuje to wyższość sterowania predykcyjnego nad reaktywnym LQR w obecności
    zakłóceń, które mona kompensować w horyzoncie predykcji.
    \item \textbf{MPC-J2}: O ile w warunkach nominalnych działał poprawnie, to
    przy silnym wietrze uległ destabilizacji (błąd MSE > 5, ogromna energia).
    Sugeruje to, że funkcja kosztu $J_2$ (bezpośrednie karanie kwadratu sterowania)
    może prowadzić do zbyt ,,pasywnego'' zachowania, które nie wystarcza do
    skontrowania silnych podmuchów.
    \item \textbf{Fuzzy-LQR}: Podobnie jak J2, nie poradził sobie z zakłóceniami,
    wpadając w silne oscylacje (MSE > 2000). Hybrydowa struktura wymagałaby
    bardziej agresywnego strojenia reguł dla dużych uchybów.
\end{itemize}

Ostatecznie, \textbf{MPC} (klasyczny) okazał się najlepszym regulatorem pod względem
robustness (odporności), minimalizując wpływ wiatru przy najmniejszym nakładzie
energetycznym. \textbf{PD-LQR} pozostaje świetną, prostszą alternatywą o bardzo
zbliżonych osiągach.
