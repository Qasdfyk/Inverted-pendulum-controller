\section{Analiza wyników}

Rozdział ten poświęcony jest szczegółowej analizie wyników badań symulacyjnych, które zostały
przeprowadzone w celu weryfikacji skuteczności i jakości działania zaprojektowanych układów
sterowania. Głównym celem eksperymentów było zbadanie zachowania wahadła odwróconego w dwóch
diametralnie różnych sytuacjach: podczas stabilizacji punktu pracy w idealnych warunkach
nominalnych oraz w trakcie pracy pod wpływem losowych zakłóceń zewnętrznych.

Podczas analizy wyników szczególny nacisk położono na trzy kluczowe  aspekty sterowania. 
Pierwszym z nich jest stabilizacja kątowa, czyli zdolność układu do
utrzymania pręta wahadła w~pionie (pozycja równowagi chwiejnej). Jest to zadanie priorytetowe,
gdyż jego niezrealizowanie prowadzi do~upadku wahadła i~niepowodzenia procesu regulacji. Drugim, równie
istotnym aspektem, jest stabilizacja pozycji wózka. Wymogiem jest, aby proces stabilizacji
kąta nie odbywał się kosztem nadmiernego przemieszczenia wózka poza zadany obszar roboczy. W~systemach
rzeczywistych, takich jak suwnice czy roboty balansujące, utrzymanie pozycji jest często równie
krytyczne co sama stabilizacja ładunku. Ostatnim jest jakość sygnału sterującego, która ma bardzo duże znaczenie 
jeśli rozpatrujemy rzeczywiste układy napędowe. Wysoka zmienność sygnału sterującego, oscylacje wysokoczęstotliwościowe 
czy gwałtowne skoki amplitudy mogą prowadzić do szybkiego zużycia mechanicznego elementów wykonawczych, a także generować 
niepożądane straty energii.

Dla zachowania przejrzystości analizy, badane algorytmy pogrupowano w dwie rodziny: regulatory
klasyczne, do których zaliczono równoległy układ PID-PID oraz hybrydowy PID-LQR, regulatory
zaawansowane, obejmujące predykcyjny algorytm MPC (w dwóch wariantach funkcji kosztu), liniowy LMPC oraz
sterownik rozmyty Fuzzy-LQR.

\subsection{Stabilizacja w warunkach nominalnych}

Pierwszy scenariusz testowy miał na celu weryfikację dynamiki układu w odpowiedzi na niezerowe
warunki początkowe. Symulacja rozpoczynała się od wychylenia wahadła o kąt około 2.8 stopnia
(0.05 radiana). Jest to typowy test odpowiedzi skokowej, pozwalający ocenić szybkość działania
(czas regulacji) oraz tłumienie oscylacji przez poszczególne regulatory.

\subsubsection{Charakterystyka regulatorów klasycznych}

Na Rysunkach \ref{fig:nom_classical}, \ref{fig:nom_pos_classical} oraz
\ref{fig:nom_control_classical} przedstawiono zbiorcze zestawienie przebiegów czasowych dla
grupy regulatorów klasycznych. Analizując wykres kąta wychylenia $\theta$
(Rys. \ref{fig:nom_classical}), można zaobserwować wyraźną przewagę regulatora hybrydowego.

Zoptymalizowany regulator PID-LQR, wykorzystujący duże wzmocnienia dla błędu pozycji,
charakteryzuje się bardzo krótkim czasem regulacji ($T_s \approx 0.3$~s). Jest to wynik
ponad pięciokrotnie lepszy od regulatora PID-PID ($T_s \approx 1.6$~s), który wykazuje znacznie
dłuższy okres dochodzenia do równowagi.

Co istotne, wyższa dynamika PID-LQR idzie w parze z oszczędnością energetyczną.
Zużywa on o blisko 40\% mniej energii ($E_u \approx 0.59$) niż regulator PID-PID ($E_u \approx 0.95$).
Oznacza to, że precyzyjnie dobrane wzmocnienia LQR pozwalają stłumić wychylenie
szybkim i skutecznym impulsem, podczas gdy PID-PID działa w sposób bardziej zachowawczy,
ale rozciągnięty w czasie, co ostatecznie generuje większy koszt energetyczny.

Warto jednak zauważyć, że PID-LQR utrzymuje wózek w węższym zakresie roboczym 
($Max |x| \approx 0.10$ m) niż PID-PID ($0.15$ m), przy jednocześnie krótszym czasie ustalania 
pozycji ($T_{s,x} \approx 1.2$~s vs $2.0$~s). Pod każdym względem PID-LQR oferuje
lepszą jakość regulacji niż klasyczny układ PID-PID.

\begin{figure}[ht!]
    \centering
    \includegraphics[width=0.95\textwidth]{images/experiments/combined_nominal_classical.png}
    \caption{Przebieg kąta $\theta$ dla regulatorów klasycznych (Warunki nominalne).}
    \label{fig:nom_classical}
\end{figure}
\newpage
\begin{figure}[ht!]
    \centering
    \includegraphics[width=0.95\textwidth]{images/experiments/combined_nominal_pos_classical.png}
    \caption{Przebieg pozycji $x$ dla regulatorów klasycznych (Warunki nominalne).}
    \label{fig:nom_pos_classical}
\end{figure}

\begin{figure}[ht!]
    \centering
    \includegraphics[width=0.95\textwidth]{images/experiments/combined_nominal_control_classical.png}
    \caption{Sygnał sterujący $u$ dla regulatorów klasycznych (Warunki nominalne).}
    \label{fig:nom_control_classical}
\end{figure}

\subsubsection{Charakterystyka regulatorów zaawansowanych}

W grupie regulatorów zaawansowanych, których wyniki zaprezentowano na Rysunkach
\ref{fig:nom_advanced}, \ref{fig:nom_pos_advanced} i~\ref{fig:nom_control_advanced},
można zaobserwować szerokie spektrum zachowań, wynikające z różnic w sformułowaniu zadań sterowania.

Najlepsze wyniki w warunkach nominalnych osiąga regulator MPC-J2. Dzięki funkcji kosztu
karającej bezpośrednio sterowanie, a nie jego zmiany, uzyskuje on czas regulacji ($T_s \approx 0.4$~s)
i zużycie energii ($E_u \approx 0.51$) na poziomie zbliżonym do agresywnego PID-LQR.
Potwierdza to, że odpowiednio nastrojony regulator predykcyjny może łączyć wysoką dynamikę
z precyzją, nie ustępując klasycznym metodom nawet w prostych zadaniach stabilizacji.

Standardowy regulator MPC (nieliniowy) działa w sposób bardziej zachowawczy.
Jego czas regulacji wynosi $T_s \approx 1.2$~s, co wynika z funkcji kosztu promującej gładkość sterowania.
Mimo wolniejszej reakcji, charakteryzuje się bardzo niskim kosztem energetycznym ($E_u \approx 0.56$),
ustępując pod tym względem jedynie wariantowi J2 i PID-LQR.

Liniowy regulator predykcyjny LMPC plasuje się pośrodku stawki.
Osiąga czas regulacji $T_s \approx 1.3$~s przy zużyciu energii $E_u \approx 0.79$.
Gorsze wyniki energetyczne w porównaniu do nieliniowych wariantów MPC ($0.79$ vs $0.51-0.56$)
wynikają z uproszczeń modelu liniowego, który nie odwzorowuje idealnie dynamiki obiektu
nawet w pobliżu punktu pracy, wymuszając częstsze korekty sterowania.

Zdecydowanie odmienną charakterystykę prezentuje regulator Fuzzy-LQR.
W tym zestawieniu okazuje się rozwiązaniem najbardziej kosztownym energetycznie ($E_u \approx 2.86$),
zużywając ponad 5-krotnie więcej energii niż MPC-J2. Co ciekawe, ten duży wydatek energetyczny
nie przekłada się na najkrótszy czas regulacji ($T_s \approx 1.5$~s).
Sugeruje to, że w idealnych warunkach nominalnych, złożona struktura reguł rozmytych może wprowadzać
niepotrzebną nerwowość sygnału sterującego, która nie jest konieczna do stabilizacji małych wychyleń,
a generuje straty energii. Jego zalety mogą ujawnić się dopiero w trudniejszych warunkach pracy.
\newpage
\begin{figure}[h!]
    \centering
    \includegraphics[width=0.95\textwidth]{images/experiments/combined_nominal_advanced.png}
    \caption{Przebieg kąta $\theta$ dla regulatorów zaawansowanych (Warunki nominalne).}
    \label{fig:nom_advanced}
\end{figure}

\begin{figure}[h!]
    \centering
    \includegraphics[width=0.95\textwidth]{images/experiments/combined_nominal_pos_advanced.png}
    \caption{Przebieg pozycji $x$ dla regulatorów zaawansowanych (Warunki nominalne).}
    \label{fig:nom_pos_advanced}
\end{figure}

\begin{figure}[h!]
    \centering
    \includegraphics[width=0.95\textwidth]{images/experiments/combined_nominal_control_advanced.png}
    \caption{Sygnał sterujący $u$ dla regulatorów zaawansowanych (Warunki nominalne).}
    \label{fig:nom_control_advanced}
\end{figure}


\subsection{Analiza odporności na zakłócenia}

Drugi scenariusz badawczy stanowił bardziej wymagający test. Do układu wprowadzono sygnał
zakłócający, modelujący losowe zakłócenia zewnętrzne o zmiennej sile i kierunku. Test ten miał na
celu sprawdzenie odporności regulatorów na zakłócenia zewnętrzne, czyli ich zdolności do
utrzymania stabilności mimo działania nieznanych, zewnętrznych sił.

Podstawowym problemem fizycznym w tym scenariuszu jest zjawisko sprzężenia dryfu. Aby
skompensować siłę zakłócającą pchającą wahadło np. w prawo, wózek musi nieustannie przyspieszać w
prawo, aby przemieścić się pod środek ciężkości wahadła i wytworzyć moment siły bezwładności
przeciwdziałający zakłóceniu. Oznacza to, że skuteczna kompensacja wychylenia kątowego nieuchronnie
prowadzi do przemieszczania się wózka (dryfu). Istotą problemu jest znalezienie kompromisu --- jak
bardzo pozwolić wózkowi uciec, by utrzymać wahadło w pionie.

W grupie klasycznej (Rys. \ref{fig:wind_classical} i \ref{fig:wind_pos_classical}) nastąpiła
istotna zmiana w stosunku do wcześniejszych analiz. Nowe strojenie PID-LQR, nastawione na
karę za zmianę pozycji, przyniosło znakomite rezultaty. Regulator ten znacząco poprawił
stabilizację kąta ($Max |\theta| \approx 0.062$ vs $0.090$~rad dla PID-PID) oraz
ograniczył dryf pozycji ($Max |x| \approx 0.27$ m vs $0.56$ m).
Co kluczowe, osiągnął to przy niemal dwukrotnie niższym zużyciu energii ($E_u \approx 12.56$)
niż regulator PID-PID ($22.78$). Świadczy to o tym, że szybka i zdecydowana reakcja na
pojawiające się zakłócenie jest bardziej ekonomiczna niż długotrwała walka z oscylacjami.

W grupie zaawansowanej (Rys. \ref{fig:wind_advanced} i \ref{fig:wind_pos_advanced})
najlepsze rezultaty osiągnął Fuzzy-LQR.
Uzyskał on najlepsze wyniki we wszystkich rozpatrywanych kategoriach: najmniejsze wychylenie kątowe
($Max |\theta| \approx 0.052$~rad), najmniejszy dryf wózka ($Max |x| \approx 0.34$ m)
oraz najniższe zużycie energii ($E_u \approx 10.1$). Przeczy to wcześniejszym obserwacjom sugerującym
wysoką energochłonność tego rozwiązania. Okazuje się, że inteligentne dostosowywanie wzmocnień pozwala
na precyzyjne interwencje, które tłumią zakłócenia w fazie początkowej, zanim wymuszą one duży wydatek energetyczny.

Regulator MPC wykazał zrównoważoną charakterystykę. Pozwolił na nieco większy dryf ($0.41$ m)
i zużył więcej energii ($12.44$) niż Fuzzy-LQR.
Wariant MPC-J2, w przeciwieństwie do wcześniejszych prób, utrzymał stabilność układu.
Jednakże, wysoka kara za wartość sterowania ograniczyła jego zdolność do szybkiej reakcji,
co skutkowało największym dryfem wózka w grupie ($Max |x| \approx 0.57$ m), porównywalnym z PID-PID.
Mimo to, jego zużycie energii pozostało na umiarkowanym poziomie ($12.3$).

Regulator LMPC dobrze poradził sobie ze stabilizacją. Jednak gorzej poradził sobie jeśli chodzi o 
koszt energetyczny ($E_u \approx 22.7$), zbliżony do wyniku regulatora PID-PID.
Ograniczenia modelu liniowego w obliczu silnych zakłóceń wymusiły mniejszą efektywność sterowania,
co widać również w nieco gorszej stabilizacji kąta ($0.085$ rad) w porównaniu
do nieliniowego MPC.

Porównanie sygnałów sterujących dla obu grup przedstawiono na Rysunkach \ref{fig:wind_control_classical}
i \ref{fig:wind_control_advanced}.
\newpage
\begin{figure}[h!]
    \centering
    \includegraphics[width=0.95\textwidth]{images/experiments/combined_wind_classical.png}
    \caption{Przebieg kąta $\theta$ pod wpływem zakłóceń zewnętrznych -- regulatory klasyczne.}
    \label{fig:wind_classical}
\end{figure}

\begin{figure}[h!]
    \centering
    \includegraphics[width=0.95\textwidth]{images/experiments/combined_wind_pos_classical.png}
    \caption{Dryf pozycji $x$ pod wpływem zakłóceń zewnętrznych -- regulatory klasyczne.}
    \label{fig:wind_pos_classical}
\end{figure}

\begin{figure}[h!]
    \centering
    \includegraphics[width=0.95\textwidth]{images/experiments/combined_wind_advanced.png}
    \caption{Przebieg kąta $\theta$ pod wpływem zakłóceń zewnętrznych -- regulatory zaawansowane.}
    \label{fig:wind_advanced}
\end{figure}

\begin{figure}[h!]
    \centering
    \includegraphics[width=0.95\textwidth]{images/experiments/combined_wind_pos_advanced.png}
    \caption{Dryf pozycji $x$ pod wpływem zakłóceń zewnętrznych -- regulatory zaawansowane.}
    \label{fig:wind_pos_advanced}
\end{figure}

\begin{figure}[h!]
    \centering
    \includegraphics[width=0.95\textwidth]{images/experiments/combined_wind_control_classical.png}
    \caption{Sygnał sterujący $u$ pod wpływem zakłóceń zewnętrznych -- regulatory klasyczne.}
    \label{fig:wind_control_classical}
\end{figure}

\begin{figure}[h!]
    \centering
    \includegraphics[width=0.95\textwidth]{images/experiments/combined_wind_control_advanced.png}
    \caption{Sygnał sterujący $u$ pod wpływem zakłóceń zewnętrznych -- regulatory zaawansowane.}
    \label{fig:wind_control_advanced}
\end{figure}

\subsection{Analiza odporności na zmianę parametrów modelu}

Trzeci scenariusz badawczy miał na celu ocenę wrażliwości regulatorów na~niepewność
parametryczną modelu. W~praktycznych zastosowaniach przemysłowych dokładne wartości
parametrów fizycznych układu są~rzadko znane z~wysoką precyzją. Mogą one ulegać
zmianom w~czasie (np.~zużycie mechaniczne, zmiana ładunku), dlatego odporność na~takie
perturbacje jest kluczową właściwością regulatora.

W~eksperymencie zwiększono masę wahadła o~100\% względem wartości nominalnej
($m_{\mathrm{nom}} = 0{,}23$~kg $\rightarrow$ $m_{\mathrm{real}} = 0{,}46$~kg),
podczas gdy regulatory pozostały nastrojone dla parametrów nominalnych. 

Wyniki eksperymentu zaprezentowano na~Rysunkach \ref{fig:robust_theta}, \ref{fig:robust_x}
oraz \ref{fig:robust_u}. Kluczową obserwacją jest fakt, że wszystkie badane regulatory
zachowały stabilność mimo niedokładnego modelu. Świadczy to o~odpowiednim zapasie
stabilności wynikającym z~procesu optymalizacji nastaw.

Analizując przebieg kąta $\theta$ (Rys.~\ref{fig:robust_theta}), można zauważyć, że
zarówno regulatory klasyczne (w szczególności PID-LQR), jak i predykcyjne (MPC, MPC-J2)
wykazują wysoką odporność na zmianę parametrów. Wbrew obawom o wrażliwość metod opartych na modelu,
algorytmy predykcyjne skutecznie kompensują błąd modelowania. Mechanizm sprzężenia zwrotnego
oraz przesuwny horyzont predykcji pozwalają na bieżącą korektę sterowania,
dzięki czemu spadek jakości regulacji jest minimalny.

Regulator PID-LQR, dzięki wysokim wzmocnieniom, a także regulatory MPC,
utrzymują precyzję stabilizacji zbliżoną do warunków nominalnych.
Wskazuje to, że dla perturbacji parametrów (rzędu 100\%),
dobrze nastrojony regulator liniowy oraz nieliniowy MPC są równie skuteczne.

Zdecydowanie najsłabsze wyniki w tym zestawieniu osiągnął klasyczny układ PID-PID.
Charakteryzuje się on najdłuższym czasem regulacji ($T_s \approx 2.8$~s) oraz największym
uchybem całkowym ($IAE_\theta \approx 0.065$). Brak adaptacji oraz brak modelu predykcyjnego
sprawiają, że regulator ten z trudem kompensuje tak znaczną zmianę dynamiki obiektu,
co prowadzi do powolnego i oscylacyjnego dochodzenia do równowagi.
Regulator LMPC plasuje się pośrodku stawki – radzi sobie lepiej niż PID-PID,
ale ustępuje nieliniowym odpowiednikom MPC, co wynika z ograniczeń modelu liniowego.

Regulator Fuzzy-LQR, mimo że zachowuje stabilność, wyróżnia się
najwyższym wydatkiem energetycznym ($E_u \approx 3.75$).
Złożona struktura sterownika w obliczu stałej zmiany parametrów prowadzi do agresywnych reakcji,
co generuje duży koszt sterowania, choć pozwala na szybszą stabilizację niż w przypadku PID-PID.

Na~wykresie sterowania (Rys.~\ref{fig:robust_u}) widać wzrost amplitudy sygnałów sterujących
dla wszystkich regulatorów, co jest fizyczną koniecznością przy sterowaniu obiektem o większej bezwładności.
Największą aktywność wykazuje regulator Fuzzy-LQR, co potwierdza jego agresywną charakterystykę działania.

\begin{figure}[h!]
    \centering
    \includegraphics[width=0.95\textwidth]{images_odpornosc/robustness_theta.png}
    \caption{Przebieg kąta $\theta$ przy zmienionych parametrach modelu (+100\% masy wahadła).}
    \label{fig:robust_theta}
\end{figure}
\newpage
\begin{figure}[h!]
    \centering
    \includegraphics[width=0.95\textwidth]{images_odpornosc/robustness_x.png}
    \caption{Przebieg pozycji $x$ przy zmienionych parametrach modelu (+100\% masy wahadła).}
    \label{fig:robust_x}
\end{figure}

\begin{figure}[h!]
    \centering
    \includegraphics[width=0.95\textwidth]{images_odpornosc/robustness_u.png}
    \caption{Sygnał sterujący $u$ przy zmienionych parametrach modelu (+100\% masy wahadła).}
    \label{fig:robust_u}
\end{figure}

\subsubsection{Analiza wrażliwości na zakres zmian}

W~celu pełniejszej oceny zapasów odporności poszczególnych regulatorów przeprowadzono
dodatkową analizę wrażliwości. Zbadano zachowanie układów sterowania w~szerokim zakresie
zmian masy wahadła: od~$-75\%$ do~$+200\%$ wartości nominalnej. Dla każdej wartości
perturbacji obliczono wskaźnik całkowy błędu bezwzględnego kąta ($IAE_\theta$),
który jest miarą skumulowanego uchybu w~czasie symulacji.

Wyniki analizy przedstawiono na~Rysunku~\ref{fig:robust_sensitivity}. Można zaobserwować
kilka istotnych prawidłowości:

\begin{itemize}
    \item Regulator Fuzzy-LQR wykazuje najlepszą odporność na~niepewność
    parametryczną, osiągając najniższe wartości $IAE_\theta$ w~całym badanym zakresie.
    Co więcej, jego charakterystyka jest praktycznie płaska --- zmiana masy wahadła
    nie wpływa istotnie na~jakość regulacji. Wynika to z~adaptacyjnej natury logiki
    rozmytej, która dostosowuje wagi reguł do~obserwowanego stanu układu.
    
    \item Regulatory klasyczne (PID-PID, PID-LQR) również charakteryzują się
    płaską charakterystyką w~całym zakresie perturbacji, choć z~nieco wyższymi
    wartościami błędu niż Fuzzy-LQR. Ich jakość regulacji jest mało wrażliwa na~niepewność 
    parametryczną dzięki strukturze opartej na~sprzężeniu zwrotnym od~błędu.
    
    \item Regulatory predykcyjne (MPC, MPC-J2) wykazują najwyższe wartości
    wskaźnika $IAE_\theta$. Jest to spodziewane zachowanie, gdyż algorytm optymalizacji 
    wykorzystuje wewnętrzny model, który odbiega od~rzeczywistej dynamiki obiektu.
    Niemniej jednak, regulatory te zachowują stabilność w~całym badanym zakresie,
    a~wzrost błędu wraz z~perturbacją jest umiarkowany.
\end{itemize}

Analiza ta~pokazuje, że regulatory wykorzystujące mechanizmy adaptacyjne (Fuzzy-LQR)
lub proste sprzężenie zwrotne od~błędu (PID, LQR) mogą oferować lepszą odporność
na~niepewność modelu niż metody predykcyjne, których skuteczność zależy od~dokładności
wewnętrznego modelu obiektu.

\begin{figure}[h!]
    \centering
    \includegraphics[width=0.95\textwidth]{images_odpornosc/robustness_sensitivity.png}
    \caption{Analiza wrażliwości: zależność wskaźnika $IAE_\theta$ od~zmiany masy
    wahadła dla poszczególnych regulatorów. Linia pionowa oznacza warunki nominalne.}
    \label{fig:robust_sensitivity}
\end{figure}

\subsection{Szczegółowe zestawienie ilościowe}

Poniższe tabele stanowią numeryczne podsumowanie omówionych wyżej zjawisk. Dane zostały
zgrupowane w sposób ułatwiający porównanie osiągów w dwóch domenach: stabilizacji wahadła (kąt)
oraz stabilizacji wózka (pozycja).

\begin{table}[h!]
    \centering
    \caption{Wskaźniki jakości (Kąt i Pozycja) - warunki nominalne}
    \label{tab:results_nominal}
    \begin{tabular}{|l|c|c|c|c|c|c|}
        \hline
        Wskaźnik & PID-PID & PID-LQR & MPC & MPC-J2 & Fuzzy-LQR & LMPC \\ \hline
        $MSE_\theta$ & 0.00011 & 0.00005 & 0.00006 & 0.00007 & 0.00017 & 0.00008 \\ \hline
        $IAE_\theta$ & 0.04567 & 0.01805 & 0.02323 & 0.02848 & 0.04935 & 0.02992 \\ \hline
        $T_{s, \theta}$ [s] & 1.60000 & 0.30000 & 1.20000 & 0.40000 & 1.50000 & 1.30000 \\ \hline
        \hline
        $MSE_x$ & 0.00051 & 0.00038 & 0.00038 & 0.00073 & 0.00093 & 0.00039 \\ \hline
        $T_{s, x}$ [s] & 2.00000 & 1.20000 & 2.20000 & 4.30000 & 3.30000 & 2.10000 \\ \hline
        \hline
        $E_{u}$ & 0.95046 & 0.59229 & 0.55762 & 0.51499 & 2.85957 & 0.78994 \\ \hline
    \end{tabular}
\end{table}

\begin{table}[h!]
    \centering
    \caption{Wskaźniki jakości (Kąt i Pozycja) - zakłócenia zewnętrzne}
    \label{tab:results_wind}
    \begin{tabular}{|l|c|c|c|c|c|c|}
        \hline
        Wskaźnik & PID-PID & PID-LQR & MPC & MPC-J2 & Fuzzy-LQR & LMPC \\ \hline
        $MSE_\theta$ & 0.00146 & 0.00057 & 0.00058 & 0.00063 & 0.00037 & 0.00118 \\ \hline
        $IAE_\theta$ & 0.30988 & 0.18541 & 0.18151 & 0.18607 & 0.14246 & 0.27259 \\ \hline
        $Max |\theta|$ [rad] & 0.09006 & 0.06187 & 0.06246 & 0.06715 & 0.05189 & 0.08539 \\ \hline
        \hline
        $MSE_x$ & 0.04382 & 0.00522 & 0.01843 & 0.04383 & 0.01059 & 0.01551 \\ \hline
        $Max |x|$ [m] & 0.55997 & 0.26817 & 0.40938 & 0.56993 & 0.33723 & 0.39981 \\ \hline
        \hline
        $E_{u}$ & 22.77583 & 12.56274 & 12.43920 & 12.26560 & 10.06415 & 22.74390 \\ \hline
    \end{tabular}
\end{table}

\begin{table}[h!]
    \centering
    \caption{Wskaźniki jakości (Kąt i Pozycja) - odporność na zmianę parametrów modelu (+100\% masy wahadła)}
    \label{tab:results_robustness}
    \begin{tabular}{|l|c|c|c|c|c|c|}
        \hline
        Wskaźnik & PID-PID & PID-LQR & MPC & MPC-J2 & Fuzzy-LQR & LMPC \\ \hline
        $MSE_\theta$ & 0.00015 & 0.00005 & 0.00007 & 0.00007 & 0.00020 & 0.00009 \\ \hline
        $IAE_\theta$ & 0.06524 & 0.01854 & 0.02517 & 0.02530 & 0.05480 & 0.03379 \\ \hline
        $T_{s, \theta}$ [s] & 2.80000 & 0.90000 & 1.30000 & 1.30000 & 1.60000 & 1.40000 \\ \hline
        \hline
        $MSE_x$ & 0.00065 & 0.00038 & 0.00038 & 0.00038 & 0.00111 & 0.00040 \\ \hline
        $T_{s, x}$ [s] & 3.10000 & 1.10000 & 2.00000 & 2.00000 & 3.30000 & 1.90000 \\ \hline
        \hline
        $E_{u}$ & 1.49205 & 0.62520 & 0.64753 & 0.64818 & 3.74712 & 1.01039 \\ \hline
    \end{tabular}
\end{table}

\newpage
\subsection{Porównanie złożoności obliczeniowej}

Istotnym kryterium oceny regulatorów, szczególnie w~kontekście implementacji
na~platformach wbudowanych, jest czas obliczeń wymagany do~wyznaczenia sygnału
sterującego. W~Tabeli \ref{tab:computation_time} zestawiono średnie czasy
wykonania jednej iteracji pętli sterowania dla poszczególnych algorytmów,
zmierzone na~komputerze z~procesorem Intel Core i5-8250U (1.6 GHz).

\begin{table}[h!]
    \centering
    \caption{Średni czas obliczeń jednej iteracji pętli sterowania}
    \label{tab:computation_time}
    \begin{tabular}{|l|c|c|}
        \hline
        Regulator & Czas [ms] & Względem PID \\ \hline
        PID-PID & $< 0{,}02$ & $1\times$ \\ \hline
        PID-LQR & $< 0{,}02$ & $1\times$ \\ \hline
        Fuzzy-LQR & $0{,}04$ & $2\times$ \\ \hline
        LMPC & $2{,}8$ & $140\times$ \\ \hline
        MPC & $11{,}3$ & $565\times$ \\ \hline
        MPC-J2 & $14{,}7$ & $735\times$ \\ \hline
    \end{tabular}
\end{table}

Regulatory klasyczne (PID-PID, PID-LQR) oraz rozmyty (Fuzzy-LQR) charakteryzują się
zaniedbywalnym czasem obliczeń, rzędu mikrosekund. Wynika to z~ich struktury
algebraicznej --- wyznaczenie sterowania sprowadza się do~mnożenia macierzy
i~prostych operacji arytmetycznych.

W~przypadku regulatorów predykcyjnych czas obliczeń jest o~blisko trzy rzędy wielkości wyższy
(ok. 3--15 ms), co wynika z~konieczności rozwiązywania w~każdym kroku
zadania optymalizacji nieliniowej (lub kwadratowej dla LMPC). Wartości te pozostają jednak znacznie poniżej
kroku symulacji ($\Delta t = 100$~ms), co potwierdza możliwość pracy MPC
w~czasie rzeczywistym dla rozpatrywanego obiektu. Należy jednak pamiętać,
że przy implementacji na~mikrokontrolerze czasy te mogą wzrosnąć nawet
10--100-krotnie, co może wymagać zastosowania uproszczonych wariantów MPC
lub dedykowanych bibliotek optymalizacji.
