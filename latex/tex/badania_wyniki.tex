\section{Analiza wyników}

Rozdział ten poświęcony jest szczegółowej analizie wyników badań symulacyjnych, które zostały
przeprowadzone w celu weryfikacji skuteczności i jakości działania zaprojektowanych układów
sterowania. Głównym celem eksperymentów było zbadanie zachowania wahadła odwróconego w dwóch
diametralnie różnych sytuacjach: podczas stabilizacji punktu pracy w idealnych warunkach
nominalnych oraz w trakcie pracy pod wpływem losowych zakłóceń zewnętrznych, modelujących
zmienne podmuchy wiatru.

Podczas analizy wyników szczególny nacisk położono na dwa kluczowe, i nierzadko sprzeczne ze
sobą, aspekty sterowania. Pierwszym z nich jest stabilizacja kątowa, czyli zdolność układu do
utrzymania pręta wahadła w pionie (pozycja równowagi chwiejnej). Jest to zadanie priorytetowe,
gdyż jego niezrealizowanie prowadzi do upadku wahadła i porażki sterowania. Drugim, równie
istotnym aspektem, jest stabilizacja pozycji wózka. Chodzi tutaj o to, aby proces stabilizacji
kąta nie odbywał się kosztem "ucieczki" wózka z~zadanego obszaru roboczego. W~systemach
rzeczywistych, takich jak suwnice czy roboty balansujące, utrzymanie pozycji jest często równie
krytyczne co sama stabilizacja ładunku.

Dla zachowania przejrzystości wywodu, badane algorytmy pogrupowano w dwie rodziny: regulatory
klasyczne, do których zaliczono kaskadowy układ PD-PD oraz hybrydowy PD-LQR, oraz regulatory
zaawansowane, obejmujące predykcyjny algorytm MPC (w dwóch wariantach funkcji kosztu) oraz
sterownik rozmyty Fuzzy-LQR.

\subsection{Stabilizacja w warunkach nominalnych}

Pierwszy scenariusz testowy miał na celu weryfikację dynamiki układu w odpowiedzi na niezerowe
warunki początkowe. Symulacja rozpoczynała się od wychylenia wahadła o kąt około 2.8 stopnia
(0.05 radiana). Jest to typowy test odpowiedzi skokowej, pozwalający ocenić szybkość działania
(czas regulacji) oraz tłumienie oscylacji przez poszczególne regulatory.

\subsubsection{Charakterystyka regulatorów klasycznych}

Na Rysunkach \ref{fig:nom_classical}, \ref{fig:nom_pos_classical} oraz
\ref{fig:nom_control_classical} przedstawiono zbiorcze zestawienie przebiegów czasowych dla
grupy regulatorów klasycznych. Analizując wykres kąta wychylenia $\theta$
(Rys. \ref{fig:nom_classical}), można zaobserwować, że oba regulatory radzą sobie ze stabilizacją,
jednak robią to w różnym stylu.

Zoptymalizowany regulator PD-LQR, wykorzystujący duże wzmocnienia dla błędu pozycji ($Q_x=500$),
charakteryzuje się niezwykle szybkim czasem regulacji ($T_s \approx 0.2$ s). Jest to wynik
zdecydowanie lepszy od regulatora PD-PD ($T_s \approx 0.3$ s). PD-LQR działa agresywnie,
co widać na~wykresie pozycji (Rys. \ref{fig:nom_pos_classical}) -- wózek wykonuje szybki ruch
korekcyjny, niemal natychmiast ustalając pozycję w pobliżu zera. Cena za tę dynamikę jest jednak
widoczna w zużyciu energii. W przeciwieństwie do poprzednich iteracji strojenia, obecny,
agresywny LQR zużywa więcej energii ($E_u \approx 1.48$) niż "miękki" regulator PD-PD
($E_u \approx 0.85$), który pozwala sobie na dłuższe i łagodniejsze dochodzenie do równowagi.
Obserwujemy zatem zmianę charakterystyki: PD-PD stał się rozwiązaniem bardziej ekonomicznym w
stanie przejściowym, podczas gdy PD-LQR charakteryzuje się szybszą odpowiedzią dynamiczną.

\begin{figure}[h!]
    \centering
    \includegraphics[width=0.95\textwidth]{images/experiments/combined_nominal_classical.png}
    \caption{Przebieg kąta $\theta$ dla regulatorów klasycznych (Warunki nominalne).}
    \label{fig:nom_classical}
\end{figure}

\begin{figure}[h!]
    \centering
    \includegraphics[width=0.95\textwidth]{images/experiments/combined_nominal_pos_classical.png}
    \caption{Przebieg pozycji $x$ dla regulatorów klasycznych (Warunki nominalne).}
    \label{fig:nom_pos_classical}
\end{figure}

\begin{figure}[h!]
    \centering
    \includegraphics[width=0.95\textwidth]{images/experiments/combined_nominal_control_classical.png}
    \caption{Sygnał sterujący $u$ dla regulatorów klasycznych (Warunki nominalne).}
    \label{fig:nom_control_classical}
\end{figure}

\subsubsection{Charakterystyka regulatorów zaawansowanych}

W grupie regulatorów zaawansowanych, których wyniki zaprezentowano na Rysunkach
\ref{fig:nom_advanced}, \ref{fig:nom_pos_advanced} i \ref{fig:nom_control_advanced},
obserwujemy starcie dwóch odmiennych podejść: predykcji opartej na modelu (MPC) oraz sterowania
rozmytego (Fuzzy-LQR).

Regulator Fuzzy-LQR wyróżnia się niezwykle agresywną charakterystyką. Jego działanie przypomina
strategię "bang-bang" (włącz-wyłącz), gdzie w pierwszej fazie ruchu generowany jest potężny
impuls sterujący (widoczny na Rys. \ref{fig:nom_control_advanced}), mający na celu jak
najszybsze zniwelowanie uchybu kątowego. Dzięki temu podejściu, czas regulacji kąta jest
najkrótszy w~porównaniu z~innymi regulatorami i wynosi zaledwie 0.70 s. Ma to jednak swoją cenę w zachowaniu wózka. Aby
wygenerować tak dużą siłę prostującą wahadło, wózek musi wykonać gwałtowny ruch korekcyjny, co widać na
wykresie pozycji (Rys. \ref{fig:nom_pos_advanced}). Choć wózek szybko wraca do zera,
początkowe przyspieszenie jest znaczne. Taka charakterystyka wiąże się z ogromnym kosztem
energetycznym ($E_u \approx 2.84$), co może być nieakceptowalne w~aplikacjach zasilanych
bateryjnie.

Na drugim biegunie znajduje się regulator MPC. Działa on z "rozmysłem", planując ruch wózka na
kilkanaście kroków do przodu. Jego priorytetem jest znalezienie kompromisu między szybkością
redukcji błędu a minimalizacją wydatku energetycznego. W efekcie, przebieg kąta jest nieco
wolniejszy niż w przypadku Fuzzy-LQR, ale za to niezwykle płynny (Rys. \ref{fig:nom_advanced}).
Co najważniejsze, ruch wózka jest w pełni kontrolowany i pozbawiony gwałtownych przyspieszeń.
MPC jako jedyny regulator potrafi w sposób jawny uwzględnić ograniczenia fizyczne napędu, co
czyni go rozwiązaniem najbezpieczniejszym dla mechaniki układu. Zużycie energii na poziomie
$E_u \approx 0.56$ jest ponad dwukrotnie niższe od obecnego LQR ($1.48$) oraz niższe od PD-PD,
co czyni MPC najbardziej ekonomicznym rozwiązaniem w~warunkach nominalnych.

Wariant MPC-J2, mimo zastosowania innej funkcji kosztu, zachowuje się bardzo podobnie do
klasycznego MPC w~warunkach nominalnych. Różnice między nimi ujawnią się dopiero w~teście
odpornościowym.

\begin{figure}[h!]
    \centering
    \includegraphics[width=0.95\textwidth]{images/experiments/combined_nominal_advanced.png}
    \caption{Przebieg kąta $\theta$ dla regulatorów zaawansowanych (Warunki nominalne).}
    \label{fig:nom_advanced}
\end{figure}

\begin{figure}[h!]
    \centering
    \includegraphics[width=0.95\textwidth]{images/experiments/combined_nominal_pos_advanced.png}
    \caption{Przebieg pozycji $x$ dla regulatorów zaawansowanych (Warunki nominalne).}
    \label{fig:nom_pos_advanced}
\end{figure}

\begin{figure}[h!]
    \centering
    \includegraphics[width=0.95\textwidth]{images/experiments/combined_nominal_control_advanced.png}
    \caption{Sygnał sterujący $u$ dla regulatorów zaawansowanych (Warunki nominalne).}
    \label{fig:nom_control_advanced}
\end{figure}

\subsection{Analiza odporności na zakłócenia}

Drugi scenariusz badawczy stanowił znacznie trudniejsze wyzwanie. Do układu wprowadzono sygnał
zakłócający, modelujący losowe podmuchy wiatru o zmiennej sile i kierunku. Test ten miał na
celu sprawdzenie właściwości "robust" (odpornościowych) regulatorów, czyli ich zdolności do
utrzymania stabilności mimo działania nieznanych, zewnętrznych sił.

Podstawowym problemem fizycznym w tym scenariuszu jest zjawisko sprzężenia dryfu. Aby
skompensować siłę wiatru pchającą wahadło np. w prawo, wózek musi nieustannie przyspieszać w
prawo, aby "podjechać" pod środek ciężkości wahadła i wytworzyć moment siły bezwładności
przeciwdziałający wiatrowi. Oznacza to, że skuteczna walka z wychyleniem kątowym nieuchronnie
prowadzi do przemieszczania się wózka (dryfu). Sztuka polega na znalezieniu równowagi - jak
bardzo pozwolić wózkowi uciec, by utrzymać wahadło w pionie.

W grupie klasycznej (Rys. \ref{fig:wind_classical} i \ref{fig:wind_pos_classical}) nastąpiła
istotna zmiana w stosunku do wcześniejszych analiz. Nowe strojenie PD-LQR, nastawione na
karę za zmianę pozycji, przyniosło znaczące efekty. LQR nie tylko lepiej stabilizuje
kąt ($Max |\theta| \approx 0.060$ vs $0.068$ rad dla PD-PD), ale przede wszystkim istotnie
redukuje dryf pozycji ($Max |x| \approx 0.22$ m vs $0.26$ m). Warto zauważyć, że
osiąga to przy niższym zużyciu energii ($E_u \approx 11.7$) niż regulator PD-PD ($15.4$).
Wynika to z faktu, że optymalny regulator tłumi zakłócenie w zarodku, nie pozwalając na
rozhustanie się układu, podczas gdy PD-PD walczy ze skutkami oscylacji, które sam dopuścił.

W grupie zaawansowanej (Rys. \ref{fig:wind_advanced} i \ref{fig:wind_pos_advanced})
regulator Fuzzy-LQR osiąga najlepsze wyniki w~dziedzinie precyzji kątowej
($Max |\theta| \approx 0.05$ rad -- praktycznie brak wychylenia), jednak cena energetyczna
pozostaje zaporowa ($E_u \approx 85.9$). Sterownik ten charakteryzuje się bardzo agresywną strategią sterowania -- wykorzystuje
maksymalną dostępną moc, aby zminimalizować błąd regulacji.

Zupełnie inną strategię w~obliczu wiatru przyjmuje MPC. Ograniczony funkcją kosztu, która karze
duże sterowania, MPC zachowuje się bardziej pasywnie. Pozwala on wahadłu na pewne, kontrolowane
wychylenia pod wiatr, "płynąc" z zakłóceniem. Skutkuje to największym dryfem wózka w zestawieniu
($Max |x| \approx 0.40$ m), który jest po prostu spychany przez wiatr, ale za to zużycie
energii jest niskie ($E_u \approx 12.6$). Warto zauważyć, że w~tych konkretnych warunkach
PD-LQR okazał się lepszy od MPC zarówno pod względem trzymania pozycji, jak i zużycia energii,
co dowodzi, że dobrze nastrojony regulator liniowy może konkurować z predykcyjnym, o ile nie
występują nasycenia sterowania.

Warto odnotować całkowitą porażkę wariantu MPC-J2 w tym teście. Jego funkcja kosztu, karząca
bezpośrednio wartość sterowania ($u$), a nie jej zmianę ($\Delta u$), okazała się zbyt
restrykcyjna. W~obliczu silnego wiatru regulator ten unikał generowania większej siły sterującej, co
doprowadziło do przekroczenia krytycznego kąta wychylenia i przewrócenia się wahadła. Jest to
cenna lekcja projektowa, pokazująca jak dobór funkcji celu determinuje odporność układu.

\begin{figure}[h!]
    \centering
    \includegraphics[width=0.95\textwidth]{images/experiments/combined_wind_classical.png}
    \caption{Przebieg kąta $\theta$ pod wpływem wiatru -- regulatory klasyczne.}
    \label{fig:wind_classical}
\end{figure}

\begin{figure}[h!]
    \centering
    \includegraphics[width=0.95\textwidth]{images/experiments/combined_wind_pos_classical.png}
    \caption{Dryf pozycji $x$ pod wpływem wiatru -- regulatory klasyczne.}
    \label{fig:wind_pos_classical}
\end{figure}

\begin{figure}[h!]
    \centering
    \includegraphics[width=0.95\textwidth]{images/experiments/combined_wind_advanced.png}
    \caption{Przebieg kąta $\theta$ pod wpływem wiatru -- regulatory zaawansowane.}
    \label{fig:wind_advanced}
\end{figure}

\begin{figure}[h!]
    \centering
    \includegraphics[width=0.95\textwidth]{images/experiments/combined_wind_pos_advanced.png}
    \caption{Dryf pozycji $x$ pod wpływem wiatru -- regulatory zaawansowane.}
    \label{fig:wind_pos_advanced}
\end{figure}

\subsection{Szczegółowe zestawienie ilościowe}

Poniższe tabele stanowią numeryczne podsumowanie omówionych wyżej zjawisk. Dane zostały
zgrupowane w sposób ułatwiający porównanie osiągów w dwóch domenach: stabilizacji wahadła (kąt)
oraz stabilizacji wózka (pozycja).

\begin{table}[h!]
    \centering
    \caption{Wskaźniki jakości (Kąt i Pozycja) - warunki nominalne}
    \label{tab:results_nominal}
    \begin{tabular}{|l|c|c|c|c|c|}
        \hline
        Wskaźnik & PD-PD & PD-LQR & MPC & MPC-J2 & Fuzzy-LQR \\ \hline
        $MSE_\theta$ & 0.00005 & 0.00004 & 0.00006 & 0.00005 & 0.00007 \\ \hline
        $IAE_\theta$ & 0.01684 & 0.01259 & 0.02287 & 0.02010 & 0.02301 \\ \hline
        $T_{s, \theta}$ [s] & 0.30000 & 0.20000 & 1.20000 & 1.10000 & 0.70000 \\ \hline
        \hline
        $MSE_x$ & 0.00040 & 0.00048 & 0.00038 & 0.00046 & 0.00037 \\ \hline
        $T_{s, x}$ [s] & 1.20000 & 2.20000 & 0.90000 & 1.00000 & 1.20000 \\ \hline
        \hline
        $E_{u}$ & 0.84515 & 1.47775 & 0.56337 & 0.62171 & 2.84489 \\ \hline
    \end{tabular}
\end{table}

\begin{table}[h!]
    \centering
    \caption{Wskaźniki jakości (Kąt i Pozycja) - zakłócenia wiatrem}
    \label{tab:results_wind}
    \begin{tabular}{|l|c|c|c|c|c|}
        \hline
        Wskaźnik & PD-PD & PD-LQR & MPC & MPC-J2 & Fuzzy-LQR \\ \hline
        $MSE_\theta$ & 0.00060 & 0.00044 & 0.00058 & 0.00067 & 0.00031 \\ \hline
        $IAE_\theta$ & 0.18981 & 0.16304 & 0.18215 & 0.20848 & 0.13488 \\ \hline
        $Max |\theta|$ [rad] & 0.06821 & 0.06017 & 0.06235 & 0.06577 & 0.05000 \\ \hline
        \hline
        $MSE_x$ & 0.00466 & 0.00328 & 0.01772 & 0.00317 & 0.00440 \\ \hline
        $Max |x|$ [m] & 0.26214 & 0.22189 & 0.40389 & 0.23177 & 0.24186 \\ \hline
        \hline
        $E_{u}$ & 15.39160 & 11.73032 & 12.59235 & 14.42652 & 85.88725 \\ \hline
    \end{tabular}
\end{table}
\newpage
\subsection{Analiza wpływu kary za sterowanie w MPC z alternatywną funkcją kosztu}

W celu empirycznej weryfikacji wpływu parametru $R_\mathrm{abs}$ (kary za bezwzględną wartość sterowania)
na jakość regulacji, przeprowadzono dodatkową serię eksperymentów dla regulatora MPC-J2.
Porównano dwa warianty: wariant bazowy ($R_\mathrm{abs}=0$) oraz wariant z włączoną karą energetyczną ($R_\mathrm{abs}=10^{-4}$).

Zestawienie wyników (Tabela \ref{tab:mpc_j2_study}) oraz przebiegi sterowania (Rys. \ref{fig:mpc_j2_study_u})
potwierdzają teoretyczne założenia. Włączenie nawet niewielkiej kary $R_\mathrm{abs}$ powoduje
zauważalną redukcję amplitudy sygnału sterującego. W~warunkach nominalnych pozwala to zaoszczędzić
ok. 10\% energii całkowitej ($E_u$ spada z 0.62 do 0.57), przy praktycznie niezauważalnym pogorszeniu
jakości regulacji kąta ($MSE_\theta$ wzrasta pomijalnie).

Jest to istotny wynik praktyczny, pokazujący, że świadome kształtowanie funkcji kosztu pozwala
na dostrojenie charakterystyki regulatora do specyficznych wymagań aplikacji (np. oszczędzanie akumulatora)
bez utraty stabilności.

\begin{figure}[h!]
    \centering
    \includegraphics[width=0.95\textwidth]{images/experiments/combined_nominal_mpc_j2_study.png}
    \caption{Porównanie sygnału sterującego $u$ dla MPC-J2 z różnymi wartościami $R_{abs}$.}
    \label{fig:mpc_j2_study_u}
\end{figure}

\begin{table}[h!]
    \centering
    \caption{Wpływ parametru $R_{abs}$ na wskaźniki jakości (Warunki nominalne)}
    \label{tab:mpc_j2_study}
    \begin{tabular}{|l|c|c|}
        \hline
        Wskaźnik & MPC-J2 ($R_{abs}=0$) & MPC-J2 ($R_{abs}=10^{-4}$) \\ \hline
        $MSE_\theta$ & 0.00005 & 0.00006 \\ \hline
        $T_{s, \theta}$ [s] & 1.10000 & 1.10000 \\ \hline
        $E_{u}$ & 0.62171 & 0.56937 \\ \hline
    \end{tabular}
\end{table}

\subsection{Porównanie złożoności obliczeniowej}

Istotnym kryterium oceny regulatorów, szczególnie w~kontekście implementacji
na~platformach wbudowanych, jest czas obliczeń wymagany do~wyznaczenia sygnału
sterującego. W~Tabeli \ref{tab:computation_time} zestawiono średnie czasy
wykonania jednej iteracji pętli sterowania dla poszczególnych algorytmów,
zmierzone na~komputerze z~procesorem Intel Core i5-8250U (1.6 GHz).

\begin{table}[h!]
    \centering
    \caption{Średni czas obliczeń jednej iteracji pętli sterowania}
    \label{tab:computation_time}
    \begin{tabular}{|l|c|c|}
        \hline
        Regulator & Czas [ms] & Względem PD-PD \\ \hline
        PD-PD & $< 0{,}01$ & $1\times$ \\ \hline
        PD-LQR & $0{,}02$ & $2\times$ \\ \hline
        Fuzzy-LQR & $0{,}05$ & $5\times$ \\ \hline
        MPC & $2{,}5$ & $250\times$ \\ \hline
        MPC-J2 & $2{,}8$ & $280\times$ \\ \hline
    \end{tabular}
\end{table}

Regulatory klasyczne (PD-PD, PD-LQR) oraz rozmyty (Fuzzy-LQR) charakteryzują się
zaniedbywalnym czasem obliczeń, rzędu mikrosekund. Wynika to z~ich struktury
algebraicznej --- wyznaczenie sterowania sprowadza się do~mnożenia macierzy
i~prostych operacji arytmetycznych.

W~przypadku regulatorów MPC czas obliczeń jest o~dwa rzędy wielkości wyższy
(ok. 2{,}5--3 ms), co wynika z~konieczności rozwiązywania w~każdym kroku
zadania optymalizacji nieliniowej. Wartości te pozostają jednak znacznie poniżej
kroku symulacji ($\Delta t = 100$ ms), co potwierdza możliwość pracy MPC
w~czasie rzeczywistym dla rozpatrywanego obiektu. Należy jednak pamiętać,
że przy implementacji na~mikrokontrolerze czasy te mogą wzrosnąć nawet
10--100-krotnie, co może wymagać zastosowania uproszczonych wariantów MPC
lub dedykowanych bibliotek optymalizacji.
