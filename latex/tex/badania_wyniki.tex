\section{Analiza wyników}

Rozdział ten poświęcony jest szczegółowej analizie wyników badań symulacyjnych, które zostały
przeprowadzone w celu weryfikacji skuteczności i jakości działania zaprojektowanych układów
sterowania. Głównym celem eksperymentów było zbadanie zachowania wahadła odwróconego w dwóch
diametralnie różnych sytuacjach: podczas stabilizacji punktu pracy w idealnych warunkach
nominalnych oraz w trakcie pracy pod wpływem losowych zakłóceń zewnętrznych.
Dodatkowo zakres badań obejmował weryfikację odporności układów na zmiany parametrów modelu
oraz analizę złożoności obliczeniowej algorytmów.

Podczas analizy wyników szczególny nacisk położono na trzy kluczowe  aspekty sterowania. 
Pierwszym z nich jest stabilizacja kątowa, czyli zdolność układu do
utrzymania pręta wahadła w~pionie (pozycja równowagi chwiejnej). Jest to zadanie priorytetowe,
gdyż jego niezrealizowanie prowadzi do~upadku wahadła i~niepowodzenia procesu regulacji. Drugim, równie
istotnym aspektem, jest stabilizacja pozycji wózka. Wymogiem jest, aby proces stabilizacji
kąta nie odbywał się kosztem nadmiernego przemieszczenia wózka poza zadany obszar roboczy. W~systemach
rzeczywistych, takich jak suwnice czy roboty balansujące, utrzymanie pozycji jest często równie
krytyczne co sama stabilizacja ładunku. Ostatnim jest jakość sygnału sterującego, która ma bardzo duże znaczenie 
jeśli rozpatrujemy rzeczywiste układy napędowe. Wysoka zmienność sygnału sterującego, oscylacje wysokoczęstotliwościowe 
czy gwałtowne skoki amplitudy mogą prowadzić do szybkiego zużycia mechanicznego elementów wykonawczych, a także generować 
niepożądane straty energii.

Dla zachowania przejrzystości analizy, badane algorytmy pogrupowano w dwie rodziny: regulatory
klasyczne, do których zaliczono równoległy układ PID-PID oraz hybrydowy PID-LQR, regulatory
zaawansowane, obejmujące predykcyjny algorytm MPC (w dwóch wariantach funkcji kosztu), liniowy LMPC oraz
sterownik rozmyty Fuzzy-LQR.

\subsection{Stabilizacja w warunkach nominalnych}

Pierwszy scenariusz testowy miał na celu weryfikację dynamiki układu w odpowiedzi na niezerowe
warunki początkowe. Symulacja rozpoczynała się od wychylenia wahadła o kąt około 2.8 stopnia
(0.05 radiana). Jest to typowy test odpowiedzi skokowej, pozwalający ocenić szybkość działania
(czas regulacji) oraz tłumienie oscylacji przez poszczególne regulatory.

\subsubsection{Charakterystyka regulatorów klasycznych}

Na Rysunkach \ref{fig:nom_classical}, \ref{fig:nom_pos_classical} oraz
\ref{fig:nom_control_classical} przedstawiono zbiorcze zestawienie przebiegów czasowych dla
grupy regulatorów klasycznych. Analizując wykres kąta wychylenia $\theta$
(Rys. \ref{fig:nom_classical}), można zaobserwować wyraźną przewagę regulatora hybrydowego.

Zoptymalizowany regulator PID-LQR, wykorzystujący duże wzmocnienia dla błędu pozycji,
charakteryzuje się znacznie krótszym czasem regulacji ($T_s \approx 2{,}9$~s) niż regulator PID-PID ($T_s \approx 4{,}2$~s).
Mimo to, jego zużycie energii ($E_u \approx 0{,}92$) jest zbliżone do klasycznego układu PID-PID ($E_u \approx 0{,}95$).
Oznacza to, że poprawa dynamiki stabilizacji została osiągnięta głównie poprzez bardziej agresywne sterowanie,
co nie przełożyło się na oszczędność energetyczną w tym przypadku.

Dla PID-LQR kluczową zaletą pozostaje znacznie lepsze utrzymanie pozycji wózka.
Maksymalne wychylenie wynosi jedynie ($Max |x| \approx 0.11$ m) wobec ($0.15$ m) dla PID-PID,
przy ponad dwukrotnie szybszym ustaleniu pozycji ($T_{s,x} \approx 2{,}3$~s w porównaniu do $4{,}6$~s).
PID-LQR oferuje zatem lepszą, ,,sztywniejszą'' regulację pozycji kosztem podobnego wydatku energetycznego.

\begin{figure}[ht!]
    \centering
    \includegraphics[width=0.95\textwidth]{images/experiments/combined_nominal_classical.png}
    \caption{Przebieg kąta $\theta$ dla regulatorów klasycznych (Warunki nominalne).}
    \label{fig:nom_classical}
\end{figure}
\newpage
\begin{figure}[ht!]
    \centering
    \includegraphics[width=0.95\textwidth]{images/experiments/combined_nominal_pos_classical.png}
    \caption{Przebieg pozycji $x$ dla regulatorów klasycznych (Warunki nominalne).}
    \label{fig:nom_pos_classical}
\end{figure}

\begin{figure}[ht!]
    \centering
    \includegraphics[width=0.95\textwidth]{images/experiments/combined_nominal_control_classical.png}
    \caption{Sygnał sterujący $u$ dla regulatorów klasycznych (Warunki nominalne).}
    \label{fig:nom_control_classical}
\end{figure}

\subsubsection{Charakterystyka regulatorów zaawansowanych}

W grupie regulatorów zaawansowanych, których wyniki zaprezentowano na Rysunkach
\ref{fig:nom_advanced}, \ref{fig:nom_pos_advanced} i~\ref{fig:nom_control_advanced},
można zaobserwować szerokie spektrum zachowań, wynikające z różnic w sformułowaniu zadań sterowania.

Najlepszy wynik w warunkach nominalnych osiąga regulator MPC (nieliniowy).
MPC-alt (J2), mimo funkcji kosztu karającej bezpośrednio sterowanie, uzyskuje czas regulacji ($T_s \approx 2{,}7$~s).
Lepszą dynamikę wykazuje standardowy MPC ($T_s \approx 2{,}1$~s), który deklasuje konkurencję.
Potwierdza to, że odpowiednio nastrojony regulator predykcyjny może łączyć wysoką dynamikę
z oszczędnością energii ($E_u \approx 0{,}51$ dla MPC-alt).

Standardowy regulator MPC (nieliniowy) działa znacznie szybciej niż PID-LQR czy PID-PID.
Jego czas regulacji wynosi $T_s \approx 2{,}1$~s.
Mimo świetnej reakcji, charakteryzuje się bardzo niskim kosztem energetycznym ($E_u \approx 0{,}56$),
będąc jednym z najbardziej efektywnych rozwiązań.

Liniowy regulator predykcyjny LMPC plasuje się pośrodku stawki.
Osiąga czas regulacji $T_s \approx 3{,}1$~s przy zużyciu energii $E_u \approx 0{,}79$.
Gorsze wyniki energetyczne w porównaniu do nieliniowych wariantów MPC ($0{,}51$ i $0{,}56$)
wynikają z uproszczeń modelu liniowego, który nie odwzorowuje idealnie dynamiki obiektu
nawet w pobliżu punktu pracy, wymuszając częstsze korekty sterowania.

Zdecydowanie odmienną charakterystykę prezentuje regulator Fuzzy-LQR.
W tym zestawieniu uzyskuje on umiarkowany wynik energetyczny ($E_u \approx 0{,}78$),
zbliżony do regulatora LMPC.
Wyróżnia się dobrym czasem regulacji ($T_s \approx 2{,}3$~s), ustępując nieznacznie PID-LQR i MPC.
Sugeruje to, że w warunkach nominalnych, hybrydowa struktura regulatora pozwala na dynamiczną
reakcję, zachowując rozsądny balans między szybkością a kosztem sterowania.
\newpage
\begin{figure}[h!]
    \centering
    \includegraphics[width=0.95\textwidth]{images/experiments/combined_nominal_advanced.png}
    \caption{Przebieg kąta $\theta$ dla regulatorów zaawansowanych (Warunki nominalne).}
    \label{fig:nom_advanced}
\end{figure}

\begin{figure}[h!]
    \centering
    \includegraphics[width=0.95\textwidth]{images/experiments/combined_nominal_pos_advanced.png}
    \caption{Przebieg pozycji $x$ dla regulatorów zaawansowanych (Warunki nominalne).}
    \label{fig:nom_pos_advanced}
\end{figure}

\begin{figure}[h!]
    \centering
    \includegraphics[width=0.95\textwidth]{images/experiments/combined_nominal_control_advanced.png}
    \caption{Sygnał sterujący $u$ dla regulatorów zaawansowanych (Warunki nominalne).}
    \label{fig:nom_control_advanced}
\end{figure}

\newpage
\subsubsection{Zestawienie wyników}

W~celu bezpośredniego porównania wszystkich zaimplementowanych strategii sterowania, 
na~Rysunkach \ref{fig:nom_advanced_all}--\ref{fig:nom_control_advanced_all} zestawiono 
przebiegi czasowe dla wszystkich regulatorów. Wykresy te potwierdzają, że w~idealnych 
warunkach nominalnych najlepszą dynamikę oferuje regulator MPC, 
który najszybciej sprowadza wahadło do~pionu przy umiarkowanym przemieszczeniu wózka. 
Widać wyraźny kontrast między metodami szybkimi a~zachowawczym regulatorem PID-PID, 
który charakteryzuje się znacznie dłuższym czasem regulacji. PID-LQR plasuje się pośrodku,
zapewniając dobrą stabilizację pozycji, ale ustępując MPC w kwestii szybkości i oszczędności energii.
Zestawienie to uwidacznia 
również różnicę w~charakterystyce sygnałów sterujących -- od~gładkich przebiegów dla MPC, 
po~bardziej agresywne działania Fuzzy-LQR, co przekłada się na~omówione wcześniej różnice 
w~kosztach energetycznych.

\newpage
\begin{figure}[h!]
    \centering
    \includegraphics[width=0.95\textwidth]{images/experiments/combined_nominal_all_lmpc_theta.png}
    \caption{Przebieg kąta $\theta$ dla regulatorów zaawansowanych (Warunki nominalne).}
    \label{fig:nom_advanced_all}
\end{figure}

\begin{figure}[h!]
    \centering
    \includegraphics[width=0.95\textwidth]{images/experiments/combined_nominal_all_lmpc_pos.png}
    \caption{Przebieg pozycji $x$ dla regulatorów zaawansowanych (Warunki nominalne).}
    \label{fig:nom_pos_advanced_all}
\end{figure}

\begin{figure}[h!]
    \centering
    \includegraphics[width=0.95\textwidth]{images/experiments/combined_nominal_all_lmpc_u.png}
    \caption{Sygnał sterujący $u$ dla regulatorów zaawansowanych (Warunki nominalne).}
    \label{fig:nom_control_advanced_all}
\end{figure}

\subsection{Analiza odporności na zakłócenia}

Drugi scenariusz badawczy polegał na wprowadzeniu do układu sygnału
zakłócającego, modelującego losowe zakłócenia zewnętrzne o zmiennej sile i kierunku. Test ten miał na
celu sprawdzenie odporności regulatorów na zakłócenia zewnętrzne, czyli ich zdolności do
utrzymania stabilności mimo działania nieznanych, zewnętrznych sił.

Podstawowym problemem fizycznym w tym scenariuszu jest zjawisko sprzężenia dryfu. Aby
skompensować siłę zakłócającą pchającą wahadło np. w prawo, wózek musi nieustannie przyspieszać w
prawo, aby przemieścić się pod środek ciężkości wahadła i wytworzyć moment siły bezwładności
przeciwdziałający zakłóceniu. Oznacza to, że skuteczna kompensacja wychylenia kątowego nieuchronnie
prowadzi do przemieszczania się wózka (dryfu). Istotą problemu jest znalezienie kompromisu --- jak
bardzo pozwolić wózkowi uciec, by utrzymać wahadło w pionie.

\subsubsection{Charakterystyka regulatorów klasycznych}

W grupie klasycznej (Rys. \ref{fig:wind_classical}--\ref{fig:wind_control_classical}) nastąpiła
istotna zmiana w porównaniu do warunków nominalnych. Strojenie PID-LQR, nastawione na
utrzymanie wahadła w pionie, skutkuje mieszanymi rezultatami. Z jednej strony regulator ten 
skutecznie ograniczył dryf pozycji ($Max |x| \approx 0.30$ m).
Z drugiej jednak, odbyło się to kosztem pogorszenia jakości stabilizacji kąta 
($Max |\theta| \approx 0.096$~rad) oraz drastycznym wzrostem zużycia energii 
($E_u \approx 33.37$). W warunkach występowania zakłóceń zewnętrznych, charakterystyka 
regulatora PID-LQR wykazuje mniejszy dryf pozycji wózka w porównaniu do układu klasycznego, 
co wiąże się z generowaniem sygnałów sterujących o większej energii całkowitej.

\begin{figure}[h!]
    \centering
    \includegraphics[width=0.95\textwidth]{images/experiments/combined_wind_classical.png}
    \caption{Przebieg kąta $\theta$ pod wpływem zakłóceń zewnętrznych -- regulatory klasyczne.}
    \label{fig:wind_classical}
\end{figure}

\begin{figure}[h!]
    \centering
    \includegraphics[width=0.95\textwidth]{images/experiments/combined_wind_pos_classical.png}
    \caption{Dryf pozycji $x$ pod wpływem zakłóceń zewnętrznych -- regulatory klasyczne.}
    \label{fig:wind_pos_classical}
\end{figure}

\begin{figure}[h!]
    \centering
    \includegraphics[width=0.95\textwidth]{images/experiments/combined_wind_control_classical.png}
    \caption{Sygnał sterujący $u$ pod wpływem zakłóceń zewnętrznych -- regulatory klasyczne.}
    \label{fig:wind_control_classical}
\end{figure}

\subsubsection{Charakterystyka regulatorów zaawansowanych}

W grupie zaawansowanej (Rys. \ref{fig:wind_advanced}--\ref{fig:wind_control_advanced})
Fuzzy-LQR potwierdza swoją skuteczność.
Uzyskał on bardzo dobre wyniki stabilizacji: małe wychylenie kątowe ($Max |\theta| \approx 0.063$~rad)
oraz niski dryf wózka ($Max |x| \approx 0.31$ m).
Pod względem zużycia energii ($E_u \approx 13.97$) ustępuje on nieco regulatorom MPC, ale jest znacznie
bardziej oszczędny niż klasyczne PID-PID czy PID-LQR.
Mimo to, niskie wartości wskaźników całkowych błędu ($IAE_\theta$) świadczą o wysokiej jakości regulacji
i zdolności do szybkiego tłumienia zakłóceń.

Regulator MPC wykazał zrównoważoną charakterystykę. Pozwolił na nieco większy dryf ($0.41$ m)
i zużył więcej energii ($12.44$) niż Fuzzy-LQR.
Wariant MPC-alt, w przeciwieństwie do wcześniejszych prób, utrzymał stabilność układu.
Jednakże, wysoka kara za wartość sterowania ograniczyła jego zdolność do szybkiej reakcji,
co skutkowało największym dryfem wózka w grupie ($Max |x| \approx 0.57$ m), porównywalnym z PID-PID.
Mimo to, jego zużycie energii pozostało na umiarkowanym poziomie ($12.3$).

Regulator LMPC dobrze poradził sobie ze stabilizacją, utrzymując kąt w ryzach lepiej niż PID-PID czy PID-LQR.
Jednak gorzej poradził sobie jeśli chodzi o 
koszt energetyczny ($E_u \approx 22.7$), zbliżony do wyniku regulatora PID-PID.
Ograniczenia modelu liniowego w obliczu silnych zakłóceń wymusiły mniejszą efektywność sterowania,
co widać również w nieco gorszej stabilizacji kąta ($0.085$ rad) w porównaniu
do nieliniowego MPC.

\begin{figure}[h!]
    \centering
    \includegraphics[width=0.95\textwidth]{images/experiments/combined_wind_advanced.png}
    \caption{Przebieg kąta $\theta$ pod wpływem zakłóceń zewnętrznych -- regulatory zaawansowane.}
    \label{fig:wind_advanced}
\end{figure}

\begin{figure}[h!]
    \centering
    \includegraphics[width=0.95\textwidth]{images/experiments/combined_wind_pos_advanced.png}
    \caption{Dryf pozycji $x$ pod wpływem zakłóceń zewnętrznych -- regulatory zaawansowane.}
    \label{fig:wind_pos_advanced}
\end{figure}

\begin{figure}[h!]
    \centering
    \includegraphics[width=0.95\textwidth]{images/experiments/combined_wind_control_advanced.png}
    \caption{Sygnał sterujący $u$ pod wpływem zakłóceń zewnętrznych -- regulatory zaawansowane.}
    \label{fig:wind_control_advanced}
\end{figure}

\newpage
\subsubsection{Zestawienie wyników}

Analogicznie do warunków nominalnych, przeprowadzono zbiorcze 
zestawienie wyników dla wszystkich badanych regulatorów w obecności 
zakłóceń (Rys. \ref{fig:wind_all_theta}--\ref{fig:wind_all_u}).
Wykresy te dobitnie pokazują przewagę regulatora Fuzzy-LQR oraz MPC
w tłumieniu zakłóceń.
Widać wyraźnie, że metody te utrzymują oscylacje wahadła w najwęższym paśmie, 
podczas gdy klasyczny PID-PID oraz PID-LQR pozwalają na znacznie większe wychylenia.
Zestawienie sygnałów sterujących ujawnia koszt tej precyzji - Fuzzy-LQR charakteryzuje 
się najbardziej aktywnym sterowaniem, co jednak w ogólnym rozrachunku (dzięki szybkiej stabilizacji) 
nie prowadzi do najgorszego zużycia energii.
\newpage
\begin{figure}[h!]
    \centering
    \includegraphics[width=0.95\textwidth]{images/experiments/combined_wind_all_lmpc_theta.png}
    \caption{Przebieg kąta $\theta$ pod wpływem zakłóceń zewnętrznych -- wszystkie regulatory.}
    \label{fig:wind_all_theta}
\end{figure}

\begin{figure}[h!]
    \centering
    \includegraphics[width=0.95\textwidth]{images/experiments/combined_wind_all_lmpc_pos.png}
    \caption{Dryf pozycji $x$ pod wpływem zakłóceń zewnętrznych -- wszystkie regulatory.}
    \label{fig:wind_all_pos}
\end{figure}

\begin{figure}[h!]
    \centering
    \includegraphics[width=0.95\textwidth]{images/experiments/combined_wind_all_lmpc_u.png}
    \caption{Sygnał sterujący $u$ pod wpływem zakłóceń zewnętrznych -- wszystkie regulatory.}
    \label{fig:wind_all_u}
\end{figure}

\subsection{Analiza odporności na zmianę parametrów modelu}

Trzeci scenariusz badawczy miał na celu ocenę wrażliwości regulatorów na~niepewność
parametryczną modelu. W~praktycznych zastosowaniach przemysłowych dokładne wartości
parametrów fizycznych układu są~rzadko znane z~wysoką precyzją. Mogą one ulegać
zmianom w~czasie (np.~zużycie mechaniczne, zmiana ładunku), dlatego odporność na~takie
perturbacje jest kluczową właściwością regulatora.

W~eksperymencie zwiększono masę wahadła o~100\% względem wartości nominalnej
($m_{\mathrm{nom}} = 0{,}23$~kg $\rightarrow$ $m_{\mathrm{real}} = 0{,}46$~kg),
podczas gdy regulatory pozostały nastrojone dla parametrów nominalnych. 

Wyniki eksperymentu zaprezentowano na~Rysunkach \ref{fig:robust_theta}, \ref{fig:robust_x}
oraz \ref{fig:robust_u}. Kluczową obserwacją jest fakt, że wszystkie badane regulatory
zachowały stabilność mimo niedokładnego modelu. Świadczy to o~odpowiednim zapasie
stabilności wynikającym z~procesu optymalizacji nastaw.

Analizując przebieg kąta $\theta$ (Rys.~\ref{fig:robust_theta}), można zauważyć, że
zarówno regulatory klasyczne (w szczególności PID-LQR), jak i predykcyjne (MPC, MPC-alt)
wykazują wysoką odporność na zmianę parametrów. Wbrew obawom o wrażliwość metod opartych na modelu,
algorytmy predykcyjne skutecznie kompensują błąd modelowania. Mechanizm sprzężenia zwrotnego
oraz przesuwny horyzont predykcji pozwalają na bieżącą korektę sterowania,
dzięki czemu spadek jakości regulacji jest minimalny.

Regulator PID-LQR, dzięki wysokim wzmocnieniom, a także regulatory MPC,
utrzymują precyzję stabilizacji zbliżoną do warunków nominalnych, choć w przypadku PID-LQR
widoczny jest wzrost czasu regulacji.
Wskazuje to, że dla perturbacji parametrów (rzędu 100\%),
dobrze nastrojony regulator liniowy oraz nieliniowy MPC zachowują poprawność działania.

Zdecydowanie najsłabsze wyniki w tym zestawieniu osiągnął klasyczny układ PID-PID.
Charakteryzuje się on najdłuższym czasem regulacji ($T_s \approx 6{,}8$~s) oraz największym
uchybem całkowym ($IAE_\theta \approx 0{,}065$). Brak adaptacji oraz brak modelu predykcyjnego
sprawiają, że regulator ten z trudem kompensuje tak znaczną zmianę dynamiki obiektu,
co prowadzi do powolnego i oscylacyjnego dochodzenia do równowagi.
Regulator LMPC plasuje się pośrodku stawki – radzi sobie lepiej niż PID-PID,
ale ustępuje nieliniowym odpowiednikom MPC, co wynika z ograniczeń modelu liniowego.

Regulator Fuzzy-LQR w tym scenariuszu prezentuje się bardzo dobrze.
Osiąga czas regulacji $T_s \approx 2{,}3$~s oraz bardzo niski uchyb całkowy ($IAE_\theta \approx 0{,}016$),
znacząco przewyższając pod tym względem PID-LQR.
Co ważne, jego zużycie energii ($E_u \approx 0.80$) jest niskie i porównywalne z innymi regulatorami,
co stanowi znaczącą poprawę w stosunku do historycznych wyników.
Elastyczność logiki rozmytej pozwala na skuteczną adaptację do zmienionej dynamiki obiektu
bez ponoszenia nadmiernych kosztów sterowania.

Na~wykresie sterowania (Rys.~\ref{fig:robust_u}) widać wzrost amplitudy sygnałów sterujących
dla wszystkich regulatorów, co jest fizyczną koniecznością przy sterowaniu obiektem o większej bezwładności.
Największą aktywność wykazuje regulator Fuzzy-LQR, co potwierdza jego agresywną charakterystykę działania.

\begin{figure}[h!]
    \centering
    \includegraphics[width=0.95\textwidth]{images_odpornosc/robustness_theta.png}
    \caption{Przebieg kąta $\theta$ przy zmienionych parametrach modelu (+100\% masy wahadła).}
    \label{fig:robust_theta}
\end{figure}
\newpage
\begin{figure}[h!]
    \centering
    \includegraphics[width=0.95\textwidth]{images_odpornosc/robustness_x.png}
    \caption{Przebieg pozycji $x$ przy zmienionych parametrach modelu (+100\% masy wahadła).}
    \label{fig:robust_x}
\end{figure}

\begin{figure}[h!]
    \centering
    \includegraphics[width=0.95\textwidth]{images_odpornosc/robustness_u.png}
    \caption{Sygnał sterujący $u$ przy zmienionych parametrach modelu (+100\% masy wahadła).}
    \label{fig:robust_u}
\end{figure}

\subsubsection{Analiza wrażliwości na zakres zmian}

W~celu pełniejszej oceny zapasów odporności poszczególnych regulatorów przeprowadzono
dodatkową analizę wrażliwości. Zbadano zachowanie układów sterowania w~szerokim zakresie
zmian masy wahadła: od~$-75\%$ do~$+200\%$ wartości nominalnej. Dla każdej wartości
perturbacji obliczono wskaźnik całkowy błędu bezwzględnego kąta ($IAE_\theta$),
który jest miarą skumulowanego uchybu w~czasie symulacji.

Wyniki analizy przedstawiono na~Rysunku~\ref{fig:robust_sensitivity}. Można zaobserwować
kilka istotnych prawidłowości:

\begin{itemize}
    \item Regulator Fuzzy-LQR wykazuje najlepszą odporność na~niepewność
    parametryczną, osiągając najniższe wartości $IAE_\theta$ w~całym badanym zakresie.
    Co więcej, jego charakterystyka jest praktycznie płaska --- zmiana masy wahadła
    nie wpływa istotnie na~jakość regulacji. Wynika to z~adaptacyjnej natury logiki
    rozmytej, która dostosowuje wagi reguł do~obserwowanego stanu układu.
    
    \item Regulatory klasyczne (PID-PID, PID-LQR) również charakteryzują się
    płaską charakterystyką w~całym zakresie perturbacji, choć z~nieco wyższymi
    wartościami błędu niż Fuzzy-LQR. Ich jakość regulacji jest mało wrażliwa na~niepewność 
    parametryczną dzięki strukturze opartej na~sprzężeniu zwrotnym od~błędu.
    
    \item Regulatory predykcyjne (MPC, MPC-alt) wykazują najwyższe wartości
    wskaźnika $IAE_\theta$. Jest to spodziewane zachowanie, gdyż algorytm optymalizacji 
    wykorzystuje wewnętrzny model, który odbiega od~rzeczywistej dynamiki obiektu.
    Niemniej jednak, regulatory te zachowują stabilność w~całym badanym zakresie,
    a~wzrost błędu wraz z~perturbacją jest umiarkowany.
\end{itemize}

Analiza ta~pokazuje, że regulatory wykorzystujące mechanizmy adaptacyjne Fuzzy-LQR
lub proste sprzężenie zwrotne od~błędu (PID, LQR) mogą oferować lepszą odporność
na~niepewność modelu niż metody predykcyjne, których skuteczność zależy od~dokładności
wewnętrznego modelu obiektu.

\begin{figure}[h!]
    \centering
    \includegraphics[width=0.95\textwidth]{images_odpornosc/robustness_sensitivity.png}
    \caption{Analiza wrażliwości: zależność wskaźnika $IAE_\theta$ od~zmiany masy
    wahadła dla poszczególnych regulatorów. Linia pionowa oznacza warunki nominalne.}
    \label{fig:robust_sensitivity}
\end{figure}

\subsection{Szczegółowe zestawienie ilościowe}

Poniższe tabele stanowią numeryczne podsumowanie omówionych wyżej zjawisk. Dane zostały
zgrupowane w sposób ułatwiający porównanie osiągów w dwóch domenach: stabilizacji wahadła (kąt)
oraz stabilizacji wózka (pozycja).

\begin{table}[h!]
    \centering
    \caption{Wskaźniki jakości (Kąt i Pozycja) - warunki nominalne}
    \label{tab:results_nominal}
    \begin{tabular}{|l|c|c|c|c|c|c|}
        \hline
        Wskaźnik & PID-PID & PID-LQR & MPC & MPC-alt & Fuzzy-LQR & LMPC \\ \hline
        $MSE_\theta$ & \cellcolor{red!25}0,00011 & 0,00009 & 0,00006 & 0,00007 & \cellcolor{green!25}0,00005 & 0,00008 \\ \hline
        $IAE_\theta$ & \cellcolor{red!25}0,04567 & 0,03169 & 0,02323 & 0,02848 & \cellcolor{green!25}0,01578 & 0,02992 \\ \hline
        $T_{s, \theta}$ & \cellcolor{red!25}4,20000 & 2,90000 & \cellcolor{green!25}2,10000 & 2,70000 & 2,30000 & 3,10000 \\ \hline
        \hline
        $MSE_x$ & 0,00051 & \cellcolor{green!25}0,00036 & 0,00038 & \cellcolor{red!25}0,00073 & 0,00043 & 0,00039 \\ \hline
        $T_{s, x}$ & 4,60000 & \cellcolor{green!25}2,30000 & 3,60000 & \cellcolor{red!25}5,20000 & 3,00000 & 2,70000 \\ \hline
        \hline
        $E_{u}$ & \cellcolor{red!25}0,95046 & 0,92481 & 0,55762 & \cellcolor{green!25}0,51499 & 0,77627 & 0,78994 \\ \hline
    \end{tabular}
\end{table}

\begin{table}[h!]
    \centering
    \caption{Wskaźniki jakości (Kąt i Pozycja) - zakłócenia zewnętrzne}
    \label{tab:results_wind}
    \begin{tabular}{|l|c|c|c|c|c|c|}
        \hline
        Wskaźnik & PID-PID & PID-LQR & MPC & MPC-alt & Fuzzy-LQR & LMPC \\ \hline
        $MSE_\theta$ & \cellcolor{red!25}\phantom{0}0,00146 & \phantom{0}0,00178 & \phantom{0}0,00058 & \phantom{0}0,00063 & \cellcolor{green!25}\phantom{0}0,00054 & \phantom{0}0,00118 \\ \hline
        $IAE_\theta$ & \cellcolor{red!25}\phantom{0}0,30988 & \phantom{0}0,33268 & \phantom{0}0,18151 & \phantom{0}0,18607 & \cellcolor{green!25}\phantom{0}0,17797 & \phantom{0}0,27259 \\ \hline
        $Max |\theta|$ & \phantom{0}0,09006 & \cellcolor{red!25}\phantom{0}0,09650 & \phantom{0}0,06246 & \phantom{0}0,06715 & \cellcolor{green!25}\phantom{0}0,06344 & \phantom{0}0,08539 \\ \hline
        \hline
        $MSE_x$ & \phantom{0}0,04382 & \cellcolor{green!25}\phantom{0}0,00653 & \phantom{0}0,01843 & \cellcolor{red!25}\phantom{0}0,04383 & \phantom{0}0,00859 & \phantom{0}0,01551 \\ \hline
        $Max |x|$ & \phantom{0}0,55997 & \cellcolor{green!25}\phantom{0}0,29510 & \phantom{0}0,40938 & \cellcolor{red!25}\phantom{0}0,56993 & \phantom{0}0,31153 & \phantom{0}0,39981 \\ \hline
        \hline
        $E_{u}$ & 22,77583 & \cellcolor{red!25}33,36971 & 12,43920 & \cellcolor{green!25}12,26560 & 13,96835 & 22,74390 \\ \hline
    \end{tabular}
\end{table}

\begin{table}[h!]
    \centering
    \caption{Wskaźniki jakości (Kąt i Pozycja) - odporność na zmianę parametrów modelu (+100\% masy wahadła)}
    \label{tab:results_robustness}
    \begin{tabular}{|l|c|c|c|c|c|c|}
        \hline
        Wskaźnik & PID-PID & PID-LQR & MPC & MPC-alt & Fuzzy-LQR & LMPC \\ \hline
        $MSE_\theta$ & \cellcolor{red!25}0,00015 & 0,00011 & 0,00007 & 0,00007 & \cellcolor{green!25}0,00005 & 0,00009 \\ \hline
        $IAE_\theta$ & \cellcolor{red!25}0,06524 & 0,03900 & 0,02517 & 0,02530 & \cellcolor{green!25}0,01620 & 0,03379 \\ \hline
        $T_{s, \theta}$ & \cellcolor{red!25}6,80000 & 3,70000 & \cellcolor{green!25}2,00000 & 2,00000 & 2,30000 & 2,90000 \\ \hline
        \hline
        $MSE_x$ & \cellcolor{red!25}0,00065 & \cellcolor{green!25}0,00037 & 0,00038 & 0,00038 & 0,00043 & 0,00040 \\ \hline
        $T_{s, x}$& \cellcolor{red!25}8,30000 & 2,90000 & 3,00000 & 3,00000 & 3,10000 & \cellcolor{green!25}2,20000 \\ \hline
        \hline
        $E_{u}$ & \cellcolor{red!25}1,49205 & 1,28519 & 0,64753 & \cellcolor{green!25}0,64818 & 0,79775 & 1,01039 \\ \hline
    \end{tabular}
\end{table}

\newpage
\subsection{Porównanie złożoności obliczeniowej}

Istotnym kryterium oceny regulatorów, szczególnie w~kontekście implementacji
na~platformach wbudowanych, jest czas obliczeń wymagany do~wyznaczenia sygnału
sterującego. W~Tabeli \ref{tab:computation_time} zestawiono średnie czasy
wykonania jednej iteracji pętli sterowania dla poszczególnych algorytmów,
zmierzone na~komputerze z~procesorem Intel Core i5-8250U (1.6 GHz).

\begin{table}[h!]
    \centering
    \caption{Średni czas obliczeń jednej iteracji pętli sterowania}
    \label{tab:computation_time}
    \begin{tabular}{|l|c|c|}
        \hline
        Regulator & Czas [ms] & Względem PID \\ \hline
        PID-PID & $< 0{,}02$ & $1\times$ \\ \hline
        PID-LQR & $< 0{,}02$ & $1\times$ \\ \hline
        Fuzzy-LQR & $0{,}04$ & $2\times$ \\ \hline
        LMPC & $2{,}8$ & $140\times$ \\ \hline
        MPC & $11{,}3$ & $565\times$ \\ \hline
        MPC-alt & $14{,}7$ & $735\times$ \\ \hline
    \end{tabular}
\end{table}

Regulatory klasyczne (PID-PID, PID-LQR) oraz rozmyty (Fuzzy-LQR) charakteryzują się
zaniedbywalnym czasem obliczeń, rzędu mikrosekund. Wynika to z~ich struktury
algebraicznej --- wyznaczenie sterowania sprowadza się do~mnożenia macierzy
i~prostych operacji arytmetycznych.

W~przypadku regulatorów predykcyjnych czas obliczeń jest o~blisko trzy rzędy wielkości wyższy
(ok. 3--15 ms), co wynika z~konieczności rozwiązywania w~każdym kroku
zadania optymalizacji nieliniowej (lub kwadratowej dla LMPC). Wartości te pozostają jednak znacznie poniżej
kroku symulacji, co potwierdza możliwość pracy MPC
w~czasie rzeczywistym dla rozpatrywanego obiektu. Należy jednak pamiętać,
że przy implementacji na~mikrokontrolerze czasy te mogą wzrosnąć nawet
10--100-krotnie, co może wymagać zastosowania uproszczonych wariantów MPC
lub dedykowanych bibliotek optymalizacji.
