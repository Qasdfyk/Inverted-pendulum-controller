\section{Analiza wyników eksperymentalnych}

Rozdział ten poświęcony jest szczegółowej analizie wyników badań symulacyjnych, które zostały
przeprowadzone w celu weryfikacji skuteczności i jakości działania zaprojektowanych układów
sterowania. Głównym celem eksperymentów było zbadanie zachowania wahadła odwróconego w dwóch
diametralnie różnych sytuacjach: podczas stabilizacji punktu pracy w idealnych warunkach
nominalnych oraz w trakcie pracy pod wpływem losowych zakłóceń zewnętrznych, modelujących
zmienne podmuchy wiatru.

Podczas analizy wyników szczególny nacisk położono na dwa kluczowe, i nierzadko sprzeczne ze
sobą, aspekty sterowania. Pierwszym z nich jest stabilizacja kątowa, czyli zdolność układu do
utrzymania pręta wahadła w pionie (pozycja równowagi chwiejnej). Jest to zadanie priorytetowe,
gdyż jego niezrealizowanie prowadzi do upadku wahadła i porażki sterowania. Drugim, równie
istotnym aspektem, jest stabilizacja pozycji wózka. Chodzi tutaj o to, aby proces stabilizacji
kąta nie odbywał się kosztem "ucieczki" wózka z zadanego obszaru roboczego. W systemach
rzeczywistych, takich jak suwnice czy roboty balansujące, utrzymanie pozycji jest często równie
krytyczne co sama stabilizacja ładunku.

Dla zachowania przejrzystości wywodu, badane algorytmy pogrupowano w dwie rodziny: regulatory
klasyczne, do których zaliczono kaskadowy układ PD-PD oraz hybrydowy PD-LQR, oraz regulatory
zaawansowane, obejmujące predykcyjny algorytm MPC (w dwóch wariantach funkcji kosztu) oraz
sterownik rozmyty Fuzzy-LQR.

\subsection{Eksperyment 1: Stabilizacja w warunkach nominalnych}

Pierwszy scenariusz testowy miał na celu weryfikację dynamiki układu w odpowiedzi na niezerowe
warunki początkowe. Symulacja rozpoczynała się od wychylenia wahadła o kąt około 2.8 stopnia
(0.05 radiana). Jest to typowy test odpowiedź skokowej, pozwalający ocenić szybkość działania
(czas regulacji) oraz tłumienie oscylacji przez poszczególne regulatory.

\subsubsection{Charakterystyka regulatorów klasycznych}

Na Rysunkach \ref{fig:nom_classical}, \ref{fig:nom_pos_classical} oraz
\ref{fig:nom_control_classical} przedstawiono zbiorcze zestawienie przebiegów czasowych dla
grupy regulatorów klasycznych. Analizując wykres kąta wychylenia $\theta$
(Rys. \ref{fig:nom_classical}), można zaobserwować wyraźne różnice w filozofii działania obu
układów. Regulator PD-LQR, korzystający z pełnego sprzężenia od wektora stanu i macierzy
wzmocnień wyznaczonej metodą optymalizacji LQR, charakteryzuje się bardzo pożądanym,
aperiodycznym przebiegiem dochodzenia do równowagi. Krzywa ma charakter łagodny, pozbawiony
zbędnych oscylacji, co świadczy o wysokim zapasie stabilności i doskonałym tłumieniu.

Zupełnie inaczej zachowuje się regulator kaskadowy PD-PD. Jego odpowiedź, choć stabilna,
obarczona jest znacznie dłuższym czasem ustalania się. Widoczne są charakterystyczne oscylacje
resztkowe, wynikające z braku bezpośredniego uwzględnienia sprzężeń między dynamiką kątową a
liniową w procesie doboru nastaw. Te oscylacje kątowe mają swoje bezpośrednie przełożenie na
pozycję wózka, co doskonale widać na Rysunku \ref{fig:nom_pos_classical}. W przypadku
regulatora PD-LQR, wózek wykonuje jeden zdecydowany ruch kompensacyjny, po czym płynnie
i monotonicznie wraca w okolice punktu zerowego. Cały proces stabilizacji pozycji zamyka się
w czasie około 3 sekund. Tymczasem wózek sterowany regulatorem PD-PD "błądzi" wokół zera
znacznie dłużej, wykonując szereg korekt przód-tył, będących odpowiedzią na oscylacje wahadła.

Potwierdzeniem wyższej efektywności podejścia LQR jest analiza sygnału sterującego
(Rys. \ref{fig:nom_control_classical}). Regulator hybrydowy generuje sygnał o mniejszej
amplitudzie szczytowej i znacznie łagodniejszym przebiegu. Wymaga on zatem mniejszej siły
(a co za tym idzie momentu napędowego silnika) do osiągnięcia lepszego efektu końcowego. Jest to
dowód na to, że regulator. LQR, minimalizując wskaźnik kwadratowy, w sposób naturalny dąży do
oszczędzania energii sterowania ($E_u \approx 0.54$), podczas gdy "sztywny" regulator PD-PD
zużywa jej niemal dwukrotnie więcej ($E_u \approx 0.85$) na walkę z własnymi oscylacjami.

\begin{figure}[h!]
    \centering
    \includegraphics[width=0.95\textwidth]{images/experiments/combined_nominal_classical.png}
    \caption{Przebieg kąta $\theta$ dla regulatorów klasycznych (Warunki nominalne).}
    \label{fig:nom_classical}
\end{figure}

\begin{figure}[h!]
    \centering
    \includegraphics[width=0.95\textwidth]{images/experiments/combined_nominal_pos_classical.png}
    \caption{Przebieg pozycji $x$ dla regulatorów klasycznych (Warunki nominalne).}
    \label{fig:nom_pos_classical}
\end{figure}

\begin{figure}[h!]
    \centering
    \includegraphics[width=0.95\textwidth]{images/experiments/combined_nominal_control_classical.png}
    \caption{Sygnał sterujący $u$ dla regulatorów klasycznych (Warunki nominalne).}
    \label{fig:nom_control_classical}
\end{figure}

\subsubsection{Charakterystyka regulatorów zaawansowanych}

W grupie regulatorów zaawansowanych, których wyniki zaprezentowano na Rysunkach
\ref{fig:nom_advanced}, \ref{fig:nom_pos_advanced} i \ref{fig:nom_control_advanced},
obserwujemy starcie dwóch odmiennych podejść: predykcji opartej na modelu (MPC) oraz sterowania
rozmytego (Fuzzy-LQR).

Regulator Fuzzy-LQR wyróżnia się niezwykle agresywną charakterystyką. Jego działanie przypomina
strategię "bang-bang" (włącz-wyłącz), gdzie w pierwszej fazie ruchu generowany jest potężny
impuls sterujący (widoczny na Rys. \ref{fig:nom_control_advanced}), mający na celu jak
najszybsze zniwelowanie uchybu kątowego. Dzięki temu podejściu, czas regulacji kąta jest
bezkonkurencyjny i wynosi zaledwie 0.70 s. Ma to jednak swoją cenę w zachowaniu wózka. Aby
wygenerować tak dużą siłę prostującą wahadło, wózek musi wykonać gwałtowny "zryw", co widać na
wykresie pozycji (Rys. \ref{fig:nom_pos_advanced}). Choć wózek szybko wraca do zera,
początkowe szarpnięcie jest znaczne. Taka charakterystyka wiąże się z ogromnym kosztem
energetycznym ($E_u \approx 2.84$), co może być nieakceptowalne w aplikacjach zasilanych
bateryjnie.

Na drugim biegunie znajduje się regulator MPC. Działa on z "rozmysłem", planując ruch wózka na
kilkanaście kroków do przodu. Jego priorytetem jest znalezienie kompromisu między szybkością
redukcji błędu a minimalizacją wydatku energetycznego. W efekcie, przebieg kąta jest nieco
wolniejszy niż w przypadku Fuzzy-LQR, ale za to niezwykle płynny (Rys. \ref{fig:nom_advanced}).
Co najważniejsze, ruch wózka jest w pełni kontrolowany i pozbawiony gwałtownych przyspieszeń.
MPC jako jedyny regulator potrafi w sposób jawny uwzględnić ograniczenia fizyczne napędu, co
czyni go rozwiązaniem najbezpieczniejszym dla mechaniki układu. Zużycie energii na poziomie
$E_u \approx 0.56$ jest zbliżone do optymalnego LQR, co potwierdza wysoką efektywność
algorytmów predykcyjnych.

Wariant MPC-J2, mimo zastosowania innej funkcji kosztu, zachowuje się bardzo podobnie do
klasycznego MPC w warunkach nominalnych. Różnice między nimi ujawnią się dopiero w teście
odpornościowym.

\begin{figure}[h!]
    \centering
    \includegraphics[width=0.95\textwidth]{images/experiments/combined_nominal_advanced.png}
    \caption{Przebieg kąta $\theta$ dla regulatorów zaawansowanych (Warunki nominalne).}
    \label{fig:nom_advanced}
\end{figure}

\begin{figure}[h!]
    \centering
    \includegraphics[width=0.95\textwidth]{images/experiments/combined_nominal_pos_advanced.png}
    \caption{Przebieg pozycji $x$ dla regulatorów zaawansowanych (Warunki nominalne).}
    \label{fig:nom_pos_advanced}
\end{figure}

\begin{figure}[h!]
    \centering
    \includegraphics[width=0.95\textwidth]{images/experiments/combined_nominal_control_advanced.png}
    \caption{Sygnał sterujący $u$ dla regulatorów zaawansowanych (Warunki nominalne).}
    \label{fig:nom_control_advanced}
\end{figure}

\subsection{Eksperyment 2: Analiza odporności na zakłócenia}

Drugi scenariusz badawczy stanowił znacznie trudniejsze wyzwanie. Do układu wprowadzono sygnał
zakłócający, modelujący losowe podmuchy wiatru o zmiennej sile i kierunku. Test ten miał na
celu sprawdzenie właściwości "robust" (odpornościowych) regulatorów, czyli ich zdolności do
utrzymania stabilności mimo działania nieznanych, zewnętrznych sił.

Podstawowym problemem fizycznym w tym scenariuszu jest zjawisko sprzężenia dryfu. Aby
skompensować siłę wiatru pchającą wahadło np. w prawo, wózek musi nieustannie przyspieszać w
prawo, aby "podjechać" pod środek ciężkości wahadła i wytworzyć moment siły bezwładności
przeciwdziałający wiatrowi. Oznacza to, że skuteczna walka z wychyleniem kątowym nieuchronnie
prowadzi do przemieszczania się wózka (dryfu). Sztuka polega na znalezieniu równowagi - jak
bardzo pozwolić wózkowi uciec, by utrzymać wahadło w pionie.

W grupie klasycznej (Rys. \ref{fig:wind_classical} i \ref{fig:wind_pos_classical}) oba
regulatory radzą sobie poprawnie, choć widoczne są ograniczenia ich struktur. PD-PD dopuszcza
do powstawania większych oscylacji kątowych ($MSE_\theta \approx 6.0 \cdot 10^{-4}$), co jest
efektem słumiennego reagowania członu różniczkującego na szum. PD-LQR, dzięki lepszemu
tłumieniu, utrzymuje wahadło stabilniej, jednak odbywa się to kosztem większego dryfu pozycji
($Max |x| \approx 0.35$ m wobec $0.26$ m dla PD-PD).

Prawdziwa przepaść jakościowa widoczna jest jednak w grupie zaawansowanej
(Rys. \ref{fig:wind_advanced} i \ref{fig:wind_pos_advanced}). Tutaj bezdyskusyjnym zwycięzcą
w kategorii precyzji stabilizacji okazał się regulator Fuzzy-LQR. Dzięki nieliniowej strukturze
bazy reguł, regulator ten potrafi dynamicznie zmieniać swoje wzmocnienia. Dla małych wychyleń
działa łagodnie, ale każda nagła zmiana kąta czy prędkości kątowej wywołana podmuchem wiatru
spotyka się z natychmiastową, silną kontrreakcją. Efekt jest spektakularny: wahadło jest
trzymane niemal "na sztywno" w pionie, co skutkuje najmniejszym błędem średniokwadratowym
($MSE_\theta \approx 3.0 \cdot 10^{-4}$) oraz, co ciekawe, najmniejszym dryfem pozycji
($Max |x| \approx 0.24$ m). Ponieważ wahadło nie wychyla się mocno, wózek nie musi wykonywać
długich rajdów ratunkowych. Ceną za ten wyczyn jest jednak gigantyczne zużycie energii
($E_u \approx 85.9$), wielokrotnie wyższe od konkurencji. Regulator ciągle "szarpie" wózkiem,
co w rzeczywistym układzie mogłoby prowadzić do przegrzania napędu.

Zupełnie inną strategię w obliczu wiatru przyjmuje MPC. Ograniczony funkcją kosztu, która karze
duże sterowania, MPC zachowuje się bardziej pasywnie. Pozwala on wahadłu na pewne, kontrolowane
wychylenia pod wiatr, "płynąc" z zakłóceniem. Skutkuje to większym dryfem wózka
($Max |x| \approx 0.40$ m), który jest po prostu spychany przez wiatr, ale za to zużycie
energii jest absolutnie minimalne ($E_u \approx 12.6$). Jest to strategia "ekonomiczna" -
przetrwać zakłócenie przy minimalnym wysiłku.

Warto odnotować całkowitą porażkę wariantu MPC-J2 w tym teście. Jego funkcja kosztu, karząca
bezpośrednio wartość sterowania ($u$), a nie jej zmianę ($\Delta u$), okazała się zbyt
restrykcyjna. W obliczu silnego wiatru regulator ten "bał się" użyć większej siły, co
doprowadziło do przekroczenia krytycznego kąta wychylenia i przewrócenia się wahadła. Jest to
cenna lekcja projektowa, pokazująca jak dobór funkcji celu determinuje odporność układu.

\begin{figure}[h!]
    \centering
    \includegraphics[width=0.95\textwidth]{images/experiments/combined_wind_pos_classical.png}
    \caption{Dryf pozycji $x$ pod wpływem wiatru -- regulatory klasyczne.}
    \label{fig:wind_pos_classical}
\end{figure}

\begin{figure}[h!]
    \centering
    \includegraphics[width=0.95\textwidth]{images/experiments/combined_wind_pos_advanced.png}
    \caption{Dryf pozycji $x$ pod wpływem wiatru -- regulatory zaawansowane.}
    \label{fig:wind_pos_advanced}
\end{figure}

\subsection{Szczegółowe zestawienie ilościowe}

Poniższe tabele stanowią numeryczne podsumowanie omówionych wyżej zjawisk. Dane zostały
zgrupowane w sposób ułatwiający porównanie osiągów w dwóch domenach: stabilizacji pręta (kąt)
oraz stabilizacji wózka (pozycja).

\begin{table}[h!]
    \centering
    \caption{Wskaźniki jakości (Kąt i Pozycja) - warunki nominalne}
    \label{tab:results_nominal}
    \begin{tabular}{|l|c|c|c|c|c|}
        \hline
        Wskaźnik & PD-PD & PD-LQR & MPC & MPC-J2 & Fuzzy-LQR \\ \hline
        $MSE_\theta$ & 0.00005 & 0.00006 & 0.00006 & 0.00006 & 0.00007 \\ \hline
        $IAE_\theta$ & 0.0168 & 0.0244 & 0.0229 & 0.0205 & 0.0230 \\ \hline
        $T_{s, \theta}$ [s] & 0.30 & 1.30 & 1.20 & 1.10 & 0.70 \\ \hline
        \hline
        $MSE_x$ & 0.0004 & 0.0004 & 0.0004 & 0.0004 & 0.0004 \\ \hline
        $T_{s, x}$ [s] & 1.20 & 2.20 & 0.90 & 1.00 & 1.20 \\ \hline
        \hline
        $E_{u}$ & 0.85 & 0.54 & 0.56 & 0.59 & 2.84 \\ \hline
    \end{tabular}
\end{table}

\begin{table}[h!]
    \centering
    \caption{Wskaźniki jakości (Kąt i Pozycja) - zakłócenia wiatrem}
    \label{tab:results_wind}
    \begin{tabular}{|l|c|c|c|c|c|}
        \hline
        Wskaźnik & PD-PD & PD-LQR & MPC & MPC-J2 & Fuzzy-LQR \\ \hline
        $MSE_\theta$ & 0.00060 & 0.00067 & 0.00058 & 5.85635 & 0.00031 \\ \hline
        $IAE_\theta$ & 0.1898 & 0.2010 & 0.1816 & 21.0429 & 0.1349 \\ \hline
        $Max |\theta|$ [rad] & 0.0682 & 0.0631 & 0.0623 & 3.3814 & 0.0500 \\ \hline
        \hline
        $MSE_x$ & 0.0047 & 0.0115 & 0.0178 & 1.0051 & 0.0044 \\ \hline
        $Max |x|$ [m] & 0.2621 & 0.3534 & 0.4037 & 2.0309 & 0.2419 \\ \hline
        \hline
        $E_{u}$ & 15.39 & 13.50 & 12.56 & 15895.41 & 85.89 \\ \hline
    \end{tabular}
\end{table}

\subsection{Dyskusja i wnioski końcowe}

Przeprowadzone badania symulacyjne, w zestawieniu z literaturą przedmiotu, pozwalają na
sformułowanie szeregu istotnych wniosków końcowych. Wykazano, że wybór odpowiedniego algorytmu
sterowania dla wahadła odwróconego nie jest kwestią trywialną i zależy ściśle od przyjętych
kryteriów projektowych oraz ograniczeń systemowych.

Po pierwsze, wyniki jednoznacznie potwierdzają tezę, że nie istnieje jeden, uniwersalny
"super-regulator", który dominowałby we wszystkich aspektach sterowania. Mamy do czynienia
z fundamentalnym kompromisem inżynierskim między jakością regulacji a kosztami eksploatacyjnymi.

\paragraph{Precyzja geometryczna a sterowanie rozmyte}
Jeżeli priorytetem jest bezwzględne utrzymanie punktu pracy, np. w robotyce precyzyjnej lub
układach stabilizacji broni, bezkonkurencyjny okazał się system rozmyty (Fuzzy-LQR). Jak
zauważają w swojej pracy Roose i in. \cite{Roose2017}, sterowniki oparte na logice rozmytej
wykazują naturalną odporność na nieliniowości i zakłócenia, co znalazło potwierdzenie w
eksperymencie z wiatrem (Tabela \ref{tab:results_wind}). Fuzzy-LQR potrafił niemal całkowicie
zniwelować wpływ losowych podmuchów, utrzymując wahadło w pionie "na sztywno". Jest to zbieżne
z obserwacjami Nguyen i Tran \cite{Nguyen2024}, którzy wskazują na wyższość hybrydowych układów
rozmytych nad klasycznymi metodami liniowymi w trudnych warunkach pracy. Należy jednak pamiętać,
że ta precyzja jest okupiona ogromnym wydatkiem energetycznym, co może dyskwalifikować to
rozwiązanie w systemach autonomicznych o ograniczonym zasilaniu.

\paragraph{Ekonomia i bezpieczeństwo ruchu (MPC)}
Z drugiej strony, jeżeli zależy nam na oszczędności energii, płynności ruchu i ochronie mechaniki
-- co jest kluczowe w pojazdach elektrycznych czy dronach -- najlepszym wyborem staje się
sterowanie predykcyjne (MPC). Zgodnie z teorią przedstawioną przez Camacho i Bordonsa
\cite{Camacho2007}, główną przewagą MPC jest jawne uwzględnianie ograniczeń (w tym przypadku
nasycenia sygnału sterującego) w procesie optymalizacji. W naszych badaniach MPC wykazał się
"inteligentnym" zarządzaniem zasobami -- w obliczu wiatru pozwolił na niewielkie, tymczasowe
odchyłki pozycji (dryf), aby uniknąć gwałtownych szarpnięć silnikiem. Potwierdza to wyniki
uzyskane przez Jezierskiego i in. \cite{Jezierski2017}, którzy również wskazali na wyższą
efektywność energetyczną MPC w porównaniu do klasycznego LQR.

\paragraph{Rola rozwiązań klasycznych}
Nie można również pominąć prostego regulatora PD-LQR. Mimo że ustępuje on rozwiązaniom
zaawansowanym w skrajnych sytuacjach, oferuje on bardzo korzystny stosunek jakości do złożoności
obliczeniowej. Jak wykazali Varghese i in. \cite{Varghese2017}, LQR pozostaje solidnym standardem
przemysłowym, zapewniającym wystarczającą stabilność dla szerokiej klasy obiektów przy
minimalnym narzucie implementacyjnym. W naszych testach PD-LQR stanowił doskonały
"złoty środek", będąc znacznie lepszym od prostego PD-PD (zgodnie z wynikami Prasad et al.
\cite{Prasad2014}), a jednocześnie tańszym obliczeniowo od MPC.

Podsumowując, projektant systemu sterowania musi świadomie balansować między sztywnością
regulacji (Fuzzy) a jej kosztem (MPC). Niniejsza praca dowodzi, że w zależności od definicji
funkcji celu -- czy jest nią minimalizacja błędu, czy minimalizacja energii -- optymalny wybór
algorytmu ulega diametralnej zmianie.



