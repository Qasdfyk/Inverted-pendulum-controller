\section{Podsumowanie}

Zrealizowano założone cele badawcze, implementując i poddając optymalizacji sześć strategii sterowania, od klasycznych PID, przez regulatory hybrydowe, aż po zaawansowane metody predykcyjne i rozmyte. Przeprowadzone badania symulacyjne pozwoliły na sformułowanie istotnych wniosków dotyczących doboru metody sterowania do konkretnych zastosowań. Kompletna implementacja środowiska symulacyjnego oraz kody źródłowe badanych regulatorów zostały udostępnione w repozytorium publicznym \href{https://github.com/Qasdfyk/Inverted-pendulum-controller}{Github}.

Wyniki wskazują jednoznacznie, że nie istnieje uniwersalny regulator dominujący we wszystkich aspektach. Wybór odpowiedniego algorytmu zawsze wiąże się z fundamentalnym kompromisem inżynierskim między jakością regulacji, definiowaną jako precyzja utrzymania punktu pracy, a kosztami eksploatacyjnymi i obciążeniem układu wykonawczego.

Wśród badanych metod szczególną skutecznością w zadaniach wymagających najwyższej precyzji wyróżnił się system rozmyty Fuzzy-LQR, który najefektywniej niwelował wpływ zakłóceń zewnętrznych, choć kosztem większego zużycia energii. Z kolei w warunkach nominalnych najbardziej ekonomicznym i bezpiecznym dla mechaniki układu rozwiązaniem okazało się sterowanie predykcyjne MPC, które w sposób jawny uwzględnia ograniczenia fizyczne napędu. Jako rozwiązanie alternatywne można wskazać regulator hybrydowy PID-LQR, który w warunkach nominalnych zapewnia szybką stabilizację pozycji. Jednak w obecności silnych zakłóceń, jego gwałtowne działanie prowadzi do znacznego wzrostu zużycia energii oraz większych oscylacji wahadła w porównaniu do metod predykcyjnych. Badania potwierdziły również, że kluczowym czynnikiem wpływającym na stabilność i odporność układu jest właściwy dobór funkcji kosztu i parametrów regulatora.

Możliwe kierunki dalszego rozwoju projektu obejmują weryfikację eksperymentalną algorytmów na rzeczywistym obiekcie fizycznym, co pozwoliłoby ocenić wpływ realnych szumów pomiarowych i opóźnień transmisji. Istotnym rozszerzeniem prac byłaby również optymalizacja kodu regulatora MPC pod kątem implementacji na platformach wbudowanych o ograniczonych zasobach obliczeniowych.


