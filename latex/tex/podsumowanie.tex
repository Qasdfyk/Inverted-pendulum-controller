\section{Podsumowanie}

Zrealizowano założone cele badawcze, implementując i poddając optymalizacji
sześć strategii sterowania, od klasycznych PID, przez regulatory
hybrydowe, aż po zaawansowane metody predykcyjne i rozmyte.

Przeprowadzone badania symulacyjne pozwoliły na sformułowanie istotnych
wniosków dotyczących doboru metody sterowania do konkretnych zastosowań.
Wyniki wskazują jednoznacznie, że nie istnieje uniwersalny regulator
dominujący we wszystkich aspektach. Wybór odpowiedniego algorytmu zawsze
wiąże się z fundamentalnym kompromisem inżynierskim między jakością
regulacji, definiowaną jako precyzja utrzymania punktu pracy, a kosztami
eksploatacyjnymi i obciążeniem układu wykonawczego.

Wśród badanych metod szczególną skutecznością w zadaniach wymagających
najwyższej precyzji wyróżnił się system rozmyty Fuzzy-LQR, który
najefektywniej niwelował wpływ zakłóceń zewnętrznych, choć kosztem
większego zużycia energii. Z kolei w warunkach nominalnych najbardziej
ekonomicznym i bezpiecznym dla mechaniki układu rozwiązaniem okazało się
sterowanie predykcyjne MPC, które w sposób jawny uwzględnia ograniczenia
fizyczne napędu. Jako rozwiązanie uniwersalne, łączące zalety obu podejść,
wskazać można regulator hybrydowy PID-LQR, oferujący dobry balans między
wydajnością a energochłonnością. Badania potwierdziły również, że kluczowym
czynnikiem wpływającym na stabilność i odporność układu jest właściwy
dobór funkcji kosztu i parametrów regulatora.

