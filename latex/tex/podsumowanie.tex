\section{Podsumowanie}

Niniejsza praca miała na~celu opracowanie zestawu skryptów symulacyjnych
oraz przeprowadzenie wielokryterialnej analizy porównawczej algorytmów sterowania
dla nieliniowego układu odwróconego wahadła na~wózku. Zrealizowano wszystkie
założone cele badawcze: zaimplementowano pięć różnych strategii sterowania
(PD, PID-LQR, MPC, MPC-J2, Fuzzy-LQR), przeprowadzono ich optymalizację
parametryczną oraz zweryfikowano skuteczność w~warunkach nominalnych
i~przy obecności zakłóceń zewnętrznych.

\subsection{Wnioski końcowe}

Przeprowadzone badania symulacyjne, w~zestawieniu z~literaturą przedmiotu, pozwalają na
sformułowanie szeregu istotnych wniosków:

\begin{enumerate}
    \item \textbf{Brak uniwersalnego regulatora.} Wyniki jednoznacznie potwierdzają,
    że nie istnieje jeden regulator dominujący we~wszystkich aspektach sterowania.
    Mamy do~czynienia z~fundamentalnym kompromisem inżynierskim między jakością
    regulacji a~kosztami eksploatacyjnymi.
    
    \item \textbf{Najwyższa precyzja: Fuzzy-LQR.} Jeżeli priorytetem jest bezwzględne
    utrzymanie punktu pracy (np. w~robotyce precyzyjnej), najlepsze wyniki osiągnął
    system rozmyty Fuzzy-LQR. Sterownik ten potrafił niemal całkowicie zniwelować
    wpływ losowych zakłóceń, utrzymując wahadło w~pionie z~maksymalnym wychyleniem
    zaledwie $0{,}05$ rad. Jest to jednak rozwiązanie bardzo kosztowne energetycznie
    ($E_u \approx 86$ --- ponad 7-krotnie więcej niż MPC).
    
    \item \textbf{Najlepsza ekonomia: MPC.} Sterowanie predykcyjne okazało się
    najbardziej ekonomicznym rozwiązaniem w~warunkach nominalnych ($E_u \approx 0{,}56$).
    MPC charakteryzuje się płynnym, przewidywalnym sterowaniem oraz jawnym
    uwzględnianiem ograniczeń fizycznych napędu, co czyni go rozwiązaniem
    najbezpieczniejszym dla mechaniki układu.
    
    \item \textbf{Uniwersalność PID-LQR.} Regulator hybrydowy PID-LQR, po~odpowiednim
    doborze wag ($Q_x = 500$), okazał się rozwiązaniem bardzo uniwersalnym.
    W~warunkach wietrznych osiągnął lepsze wyniki niż MPC pod~względem trzymania
    pozycji i~zużycia energii, przy znacznie niższej złożoności obliczeniowej.
    
    \item \textbf{Wrażliwość na~funkcję kosztu.} Analiza wariantu MPC-J2 wykazała,
    że dobór funkcji kosztu ma krytyczny wpływ na~odporność układu. Zbyt restrykcyjna
    kara za~energię sterowania może prowadzić do~utraty stabilności w~obecności
    silnych zakłóceń.
\end{enumerate}

\subsection{Ograniczenia pracy}

Przeprowadzone badania mają charakter symulacyjny i~wiążą się z~pewnymi
ograniczeniami, które należy uwzględnić przy interpretacji wyników:

\begin{itemize}
    \item \textbf{Idealne warunki pomiarowe.} W~symulacjach założono, że pełny
    wektor stanu jest dostępny bezpośrednio, bez szumów pomiarowych i~opóźnień.
    W~rzeczywistych układach konieczne byłoby zastosowanie obserwatora stanu
    (np.~filtra Kalmana), co mogłoby wpłynąć na~jakość regulacji.
    
    \item \textbf{Brak dynamiki aktuatora.} Model nie uwzględnia bezwładności
    i~ograniczeń dynamicznych silnika napędzającego wózek. W~systemach rzeczywistych
    mogłyby wystąpić dodatkowe opóźnienia i~ograniczenia szybkości narastania siły.
    
    \item \textbf{Uproszczony model zakłóceń.} Przyjęty model wiatru (filtrowany
    szum gaussowski) jest uproszczeniem rzeczywistych warunków środowiskowych,
    które mogą charakteryzować się bardziej złożoną strukturą czasowo-przestrzenną.
    
    \item \textbf{Brak weryfikacji eksperymentalnej.} Wyniki nie zostały
    zweryfikowane na~rzeczywistym stanowisku laboratoryjnym, co uniemożliwia
    ocenę wpływu niedokładności modelu i~nieuwzględnionych zjawisk fizycznych.
\end{itemize}

\subsection{Kierunki dalszych badań}

Na~podstawie przeprowadzonych analiz można wskazać następujące kierunki
rozwoju projektu:

\begin{enumerate}
    \item \textbf{Implementacja algorytmu swing-up.} Rozszerzenie funkcjonalności
    o~fazę wprowadzania wahadła z~pozycji dolnej do~górnej, co pozwoliłoby
    na~pełną automatyzację procesu stabilizacji.
    
    \item \textbf{Adaptacyjne sterowanie MPC.} Implementacja mechanizmów adaptacji
    online, pozwalających na~automatyczne dostrajanie wag funkcji kosztu
    w~zależności od~aktualnych warunków pracy.
    
    \item \textbf{Uwzględnienie szumów pomiarowych.} Rozbudowa modelu o~realistyczne
    szumy czujników oraz implementacja estymatora stanu (filtr Kalmana lub EKF),
    co przybliżyłoby symulacje do~warunków rzeczywistych.
    
    \item \textbf{Implementacja sprzętowa.} Weryfikacja algorytmów na~rzeczywistym
    stanowisku laboratoryjnym z~wykorzystaniem platformy mikroprocesorowej
    (np.~STM32, Raspberry Pi), co pozwoliłoby na~ocenę wydajności obliczeniowej
    i~praktycznej stosowalności poszczególnych metod.
    
    \item \textbf{Porównanie z~metodami uczenia maszynowego.} Zestawienie
    klasycznych metod sterowania z~podejściami opartymi na~uczeniu ze~wzmocnieniem
    (Reinforcement Learning), które zyskują coraz większą popularność
    w~sterowaniu systemami nieliniowymi.
\end{enumerate}

Opracowane środowisko symulacyjne stanowi solidną bazę do~dalszych badań
nad sterowaniem nieliniowym i~może być wykorzystane zarówno w~celach
dydaktycznych, jak i~badawczych.

