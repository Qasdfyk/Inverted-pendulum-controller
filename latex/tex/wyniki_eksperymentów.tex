\section{Eksperymenty}

\subsection{Opis przeprowadzonych eksperymentów}
W celu porównania skuteczności różnych strategii sterowania dla odwróconego wahadła na wózku, przeprowadzono szereg symulacji numerycznych. Każdy eksperyment polegał na zasymulowaniu zachowania układu w odpowiedzi na tę samą konfigurację początkową:
\[
    x(0) = 0,\quad \dot{x}(0) = 0,\quad \varphi(0) = 45^\circ,\quad \dot{\varphi}(0) = 0.
\]
Symulacje trwały 5 sekund, z krokiem czasowym \(\Delta t = 0{,}01\,\text{s}\). Rozważono trzy warianty regulatorów:
\begin{itemize}
    \item \textbf{LQR} - regulator liniowo-kwadratowy oparty na pełnym stanie,
    \item \textbf{PID} - regulator PID zaprojektowany dla modelu przenoszenia kąta \(\varphi\),
    \item \textbf{Composite} - układ złożony: regulator PID dla pozycji wózka \(x\) i LQR dla kąta \(\varphi\).
\end{itemize}

Każdy z powyższych regulatorów został przetestowany w dwóch wariantach:
\begin{enumerate}
    \item bez zakłóceń,
    \item z zakłóceniem w postaci dwóch podmuchów.
\end{enumerate}

W trakcie symulacji zarejestrowano:
\begin{itemize}
    \item trajektorię kąta wahadła \(\varphi(t)\),
    \item trajektorię położenia wózka \(x(t)\),
    \item sygnał sterujący \(u(t)\).
\end{itemize}

Na podstawie zebranych danych obliczono również dwie metryki jakości:
\begin{itemize}
    \item \textbf{MSE} - średni błąd kwadratowy: \(\displaystyle \mathrm{MSE}_\varphi = \frac{1}{N}\sum_{k=1}^{N} \varphi[k]^2\),
    \item \textbf{MAE} - średni błąd bezwzględny: \(\displaystyle \mathrm{MAE}_\varphi = \frac{1}{N}\sum_{k=1}^{N} |\varphi[k]|\),
\end{itemize}

\newpage
\subsection{Dobór parametrów regulatorów}
\begin{figure}[H]
    \centering
    \includegraphics[width=0.8\textwidth]{img/pid_zly.png}
    \caption{Źle dostrojony regulator PID.}
\end{figure}
\begin{figure}[H]
    \centering
    \includegraphics[width=0.8\textwidth]{img/pid_git.png}
    \caption{Dostrojony regulator PID przy użyciu pidtune.}
\end{figure}

Regulator PID dostrojono przy użyciu gotowej funkcji pidtune, natomiast regulator LQR dostrojono ręcznie.
\begin{figure}[H]
    \centering
    \includegraphics[width=0.95\textwidth]{img/przebiegi_phi.png}
    \caption{Sygnały wyjściowe.}
\end{figure}

\begin{figure}[H]
    \centering
    \includegraphics[width=0.95\textwidth]{img/zaklocenia.png}
    \caption{Sygnały wyjściowe z zakłóceniami.}
\end{figure}
\newpage
\subsection{Porównanie jakości regulacji - wskaźniki błędów}

\begin{table}[H]
    \centering
    \caption{Wartości wskaźników MSE i MAE dla kąta \(\varphi(t)\) - bez zakłóceń}
    \begin{tabular}{l|c|c}
        \textbf{Regulator} & \textbf{MSE\(_\varphi\)} & \textbf{MAE\(_\varphi\)} \\
        \hline
        LQR & \texttt{0.0161} & \texttt{0.0585} \\
        PID & \texttt{0.0042} & \texttt{0.0126} \\
        Composite & \texttt{0.0209} & \texttt{0.0677} \\
    \end{tabular}
\end{table}

\vspace{1em}

\begin{table}[H]
    \centering
    \caption{Wartości wskaźników MSE i MAE dla kąta \(\varphi(t)\) - z zakłóceniem}
    \begin{tabular}{l|c|c}
        \textbf{Regulator} & \textbf{MSE\(_\varphi\)} & \textbf{MAE\(_\varphi\)} \\
        \hline
        LQR & \texttt{0.0171} & \texttt{0.0687} \\
        PID & \texttt{0.0067} & \texttt{0.0402} \\
        Composite & \texttt{0.0322} & \texttt{0.1226} \\
    \end{tabular}
\end{table}


\subsection{Podsumowanie eksperymentów}

Przeprowadzone eksperymenty pozwoliły na porównanie trzech różnych strategii sterowania odwróconym wahadłem na wózku w identycznych warunkach początkowych oraz w obecności zakłóceń. Na podstawie analizy sygnałów oraz wartości wskaźników błędów (MSE i MAE) można sformułować następujące wnioski:

\begin{itemize}
    \item Regulator \textbf{PID} wykazał najlepszą skuteczność w warunkach bez zakłóceń, osiągając najniższe wartości MSE i MAE, co sugeruje dobrą zdolność do szybkiego i precyzyjnego tłumienia wychylenia wahadła.
    \item W obecności zakłóceń regulator PID nadal utrzymywał dobrą jakość regulacji, jednak zauważalnie większy błąd bezwzględny (MAE) wskazuje na ograniczoną odporność na zaburzenia w porównaniu do LQR.
    \item Regulator \textbf{LQR}, choć nieco mniej precyzyjny w warunkach nominalnych, wykazał większą stabilność i odporność na zakłócenia, co czyni go solidnym wyborem w sytuacjach, gdzie przewiduje się obecność niestandardowych sił.
    \item Układ \textbf{Composite}, łączący zalety obu strategii, nie przyniósł oczekiwanej poprawy jakości regulacji. W szczególności zauważono pogorszenie wskaźników błędów, co może być związane z nieoptymalnym podziałem zadań między PID i LQR.
\end{itemize}

Podsumowując, dobór odpowiedniego regulatora powinien być uzależniony od warunków pracy układu - w sytuacjach nominalnych lepiej sprawdza się PID, natomiast w obecności zakłóceń bardziej odpowiedni może okazać się LQR.
