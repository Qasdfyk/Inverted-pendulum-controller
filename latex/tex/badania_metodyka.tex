\section{Metodyka badań i plan eksperymentów}

Rozdział ten definiuje scenariusze testowe, przyjęte miary oceny jakości sterowania oraz procedurę doboru nastaw regulatorów. Precyzyjne określenie warunków eksperymentu jest kluczowe dla zapewnienia powtarzalności badań oraz obiektywnego porównania testowanych algorytmów.

\subsection{Plan eksperymentów}

W~celu weryfikacji skuteczności strategii sterowania, przyjęto jednolity zestaw testów symulacyjnych. Każdy z~zaimplementowanych regulatorów (PID, LQR, Composite) poddany został badaniom w~następujących scenariuszach:

\begin{enumerate}
    \item \textbf{Eksperyment 1: Stabilizacja w~punkcie pracy (warunki nominalne).} \\
    Symulacja odpowiedzi układu na~niezerowe warunki początkowe przy braku zakłóceń zewnętrznych.
    \begin{itemize}
        \item Początkowy kąt wychylenia wahadła: $\varphi(0) = 45^\circ$ ($\approx 0{,}785$ rad).
        \item Początkowa pozycja wózka: $x(0) = 0$ m.
        \item Zerowe prędkości początkowe: $\dot{\varphi}(0) = 0$, $\dot{x}(0) = 0$.
    \end{itemize}
    Celem jest sprawdzenie zdolności regulatora do sprowadzenia układu do pionu ($\varphi=0, x=0$) oraz ocena czasu regulacji i przeregulowań.

    \item \textbf{Eksperyment 2: Odporność na~zakłócenia zewnętrzne.} \\
    Symulacja z~tymi samymi warunkami początkowymi, przy czym na~wahadło oddziałuje losowa siła zakłócająca $F_w(t)$ (modelująca porywisty wiatr) generowana zgodnie z~procedurą opisaną w~Rozdziale 3.  Test ten pozwala ocenić krzepkość (ang. \textit{robustness}) układu zamkniętego.
\end{enumerate}

Wszystkie symulacje przeprowadzono z~krokiem czasowym $\Delta t = 0{,}01$ s w~czasie $T_{sim} = 5$ s. Ograniczenie sygnału sterującego ustawiono na $|u| \le 100$ N.

\subsection{Wskaźniki jakości regulacji}

Do~ilościowej oceny jakości sterowania wykorzystano następujące wskaźniki błędów, obliczane dla zdyskretyzowanych przebiegów kąta $\varphi[k]$ ($N$ próbek):

\begin{itemize}
    \item \textbf{MSE (Mean Squared Error)} -- Średni błąd kwadratowy, karający silniej duże odchyłki od wartości zadanej. Jest to miara powiązana z energią błędu regulacji.
    \begin{equation}
        \mathrm{MSE}_\varphi = \frac{1}{N}\sum_{k=1}^{N} \varphi[k]^2
    \end{equation}
    
    \item \textbf{MAE (Mean Absolute Error)} -- Średni błąd bezwzględny, dający bardziej intuicyjną informację o przeciętnym odchyleniu wahadła od pionu w trakcie trwania symulacji.
    \begin{equation}
        \mathrm{MAE}_\varphi = \frac{1}{N}\sum_{k=1}^{N} |\varphi[k]|
    \end{equation}
\end{itemize}

Dodatkowo analizie poddano charakterystyki jakościowe przebiegów czasowych, takie jak czas regulacji (czas, po którym błąd trwale mieści się w paśmie $\pm 2\%$) oraz maksymalne przeregulowanie.

\subsection{Procedura doboru nastaw regulatorów}

Dobór parametrów regulatorów jest krytycznym etapem projektowania, determinującym ostateczne osiągi układu. W~pracy zastosowano podejście hybrydowe, łączące metody automatyczne z~korektą ekspercką.

\subsubsection{Strojenie regulatora PID}

Dla klasycznego regulatora PID i~PD wykorzystano narzędzie \texttt{pidtune} dostępne w~środowisku MATLAB/Simulink. Procedura wyglądała następująco:
\begin{enumerate}
    \item Zlinearyzowany model obiektu podzielono na dwa tory: tor stabilizacji wahadła (szybka dynamika) i tor pozycji wózka (wolna dynamika).
    \item Dla pętli stabilizacji kąta zdefiniowano wymagania dotyczące szybkości reakcji (szerokość pasma) oraz zapasu fazy (dla zapewnienia stabilności). Narzędzie \texttt{pidtune} wygenerowało wstępne nastawy $K_p, K_i, K_d$.
    \item Uzyskane nastawy zostały zweryfikowane w modelu nieliniowym. W przypadku wystąpienia nasycenia sterowania lub zbyt agresywnych oscylacji, dokonano ręcznej korekty (zmniejszenie wzmocnienia $K_p$, zwiększenie stałej różniczkowania $K_d$ dla poprawy tłumienia).
\end{enumerate}


\subsubsection{Dobór wag regulatora LQR}

W~przypadku regulatora LQR, problem projektowy sprowadza się do doboru macierzy wag $Q$ (koszt stanu) i $R$ (koszt sterowania) w funkcjanale jakości:
\begin{equation}
 J = \int_{0}^{\infty} (\mathbf{x}^T Q \mathbf{x} + u^T R u) dt
\end{equation}

Zastosowano metodę prób i błędów wg strategii Brysona, przyjmując początkowo macierze diagonalne, gdzie elementy na przekątnej odpowiadają odwrotnościom kwadratów dopuszczalnych maksymalnych wychyleń. 
\begin{itemize}
    \item Macierz $Q$ została dobrana tak, aby priorytetem była stabilizacja kąta $\varphi$ i prędkości kątowej $\dot{\varphi}$ (duże wagi dla $x_1, x_2$).
    \item Waga przy pozycji wózka $x$ została dobrana mniejsza, aby pozwolić na swobodniejszy ruch wózka niezbędny do balansowania wahadłem.
    \item Waga $R$ została dobrana eksperymentalnie tak, aby sygnał sterujący mieścił się w~realnym zakresie pracy silnika ($|u| < 100$), unikając ciągłego nasycenia siłownika (tzw. \textit{bang-bang}).
\end{itemize}

Ostatecznie przyjęte macierze wag zapewniły kompromis między szybkością zanikania błędów a~agresywnością sygnału sterującego.
