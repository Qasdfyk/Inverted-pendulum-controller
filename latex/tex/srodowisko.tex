\section{Środowisko}
\begin{itemize}
    \item \textbf{Oprogramowanie: } MATLAB R2024a \& Simulink.
    \item \textbf{Biblioteki: } Control System Toolbox.
    \item \textbf{Parametry fizyczne układu:}
    \[
        M = 0{,}5\text{ kg}, \quad
        m = 0{,}2\text{ kg}, \quad
        l = 0{,}3\text{ m}, \quad
        I = 0{,}006\text{ kg·m}^2, \quad
        b = 0{,}1\text{ N·s/m}, \quad
        g = 9{,}81\text{ m/s}^2.
    \]
    \item \textbf{Warunki początkowe:} 
    \[
        \varphi(0) = 45^\circ,\quad \dot\varphi(0)=0,\quad x(0)=0,\quad \dot x(0)=0.
    \]
    \item \textbf{Czas symulacji: } \(t \in [0,\,5]\text{ s}\) z krokiem \(\Delta t = 0{,}01\,\text{s}\).
    \item \textbf{Zaburzenie od wiatru:} Wersja z zakłóceniem realizuje się jako dodanie do równania ruchu siły \(F_w(t)\). Zby łatwiej było zaobserwować wpływ wiatru na działanie układu, jako zakłócenie przyjęto dwa podmuchy w 1 i 3 sekundzie symulacji
    \item \textbf{Animacja wahadła: } 
    W celu lepszego zobrazowania dynamiki układu, projekt zawiera funkcję \texttt{animate\_pendulum}, która rysuje ruch wózka i wahadła w czasie. 
    Wózek wizualizowany jest jako prostokąt, drążek wahadła jako linia, a jego koniec jako czerwona kropka.
    \begin{figure}[H]
    \centering
    \includegraphics[width=0.7\textwidth]{img/animacja.png}
    \caption{Animacja wahadła na wózku}
    \label{fig:schemat_ukladu}
\end{figure}
\end{itemize}
