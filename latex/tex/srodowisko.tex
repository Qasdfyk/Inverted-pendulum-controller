\section{Środowisko symulacyjne i implementacja}

W~celu przeprowadzenia badań i~weryfikacji działania algorytmów sterowania,
przygotowano autorskie środowisko symulacyjne zrealizowane w~języku
\textbf{Python 3}. Wybór tego języka podyktowany był jego powszechnością
w~zastosowaniach naukowych, dostępnością bibliotek do~obliczeń numerycznych
i~optymalizacji, a~także łatwością prototypowania złożonych struktur sterowania.

\subsection{Narzędzia programistyczne}

W~projekcie wykorzystano następujące biblioteki i narzędzia:
\begin{itemize}
    \item \textbf{NumPy} -- podstawowa biblioteka do~obliczeń macierzowych
    i~operacji na~wielowymiarowych tablicach danych, wykorzystywana
    do~implementacji równań stanu oraz przechowywania przebiegów symulacji.
    \item \textbf{SciPy} -- pakiet naukowy dostarczający zaawansowanych
    algorytmów numerycznych. W pracy użyto modułów:
    \begin{itemize}
        \item \texttt{scipy.linalg} -- do~rozwiązywania algebraicznego równania
        Riccatiego (ARE) w algorytmie LQR.
        \item \texttt{scipy.optimize} -- zawierającego solwer \texttt{minimize}
        (metoda SLSQP), wykorzystywany do~rozwiązywania zadań optymalizacji
        nieliniowej z~ograniczeniami w~regulatorze MPC.
    \end{itemize}
    \item \textbf{Matplotlib} -- biblioteka służąca do~wizualizacji wyników
    w postaci wykresów przebiegów czasowych oraz do generowania animacji ruchu
    wahadła.
\end{itemize}

\subsection{Konfiguracja symulacji}

Symulator opiera~się na~numerycznym całkowaniu wyprowadzonych wcześniej
nieliniowych równań dynamiki. Zaimplementowano procedurę całkowania metodą
\textbf{Rungego-Kutty czwartego rzędu (RK4)}, co zapewnia wysoki kompromis
pomiędzy dokładnością a szybkością obliczeń.

Przyjęto stały krok symulacji oraz sterowania wynoszący
\(\Delta t = 0{,}1\,\text{s}\). Jest to wartość, przy której dynamika wahadła
jest odwzorowana z~wystarczającą precyzją, a~jednocześnie pozwala na~efektywne
działanie numerycznych algorytmów optymalizacji w~czasie rzeczywistym.

\begin{table}[H]
	\centering
	\caption{Parametry fizyczne modelu przyjęte w~symulacji}
	\label{tab:parametry_modelu}
	\renewcommand{\arraystretch}{1.2}
	\begin{tabular}{|l|c|c|c|}
		\hline
		\textbf{Parametr} & \textbf{Symbol} & \textbf{Wartość} & \textbf{Jednostka} \\ \hline
		Masa wózka & $M$ & $2{,}40$ & $\text{kg}$ \\ \hline
		Masa wahadła & $m$ & $0{,}23$ & $\text{kg}$ \\ \hline
		Długość wahadła & $l$ & $0{,}36$ & $\text{m}$ \\ \hline
		Przyspieszenie ziemskie & $g$ & $9{,}81$ & $\text{m/s}^2$ \\ \hline
        Ograniczenie sterowania & $F_\mathrm{max}$ & $100{,}00$ & $\text{N}$ \\ \hline
	\end{tabular}
\end{table}

Symulacje przeprowadzane~są dla zadania stabilizacji układu w~pionie (tzw. punkt
pracy), startując z niezerowych warunków początkowych lub wymuszając zmianę
pozycji wózka.

\textbf{Warunki początkowe (domyślne):}
\[
\mathbf{x}_0 = [\theta, \dot{\theta}, x, \dot{x}]^T = [0{,}05, 0, 0, 0]^T
\]
Oznacza to niewielkie (ok. $2{,}86^\circ$) początkowe wychylenie wahadła, które
regulator musi zniwelować.

\textbf{Wartości zadane:}
Celem układu jest osiągnięcie stanu
$\mathbf{x}_\mathrm{ref} = [0, 0, x_\mathrm{ref}, 0]^T$, gdzie
$x_\mathrm{ref}$ (np. $0{,}10$ m) jest zadaną nową pozycją wózka, przy
jednoczesnym utrzymaniu pionowej pozycji wahadła ($\theta = 0$).

\subsection{Modelowanie zakłóceń}

Aby zweryfikować odporność układów sterowania, zaimplementowano generator
zakłóceń symulujący podmuchy wiatru działające na~wahadło. Generator ten działa
w~sposób dyskretny, realizując w~każdym kroku symulacji $k$ następujące
operacje:

\textbf{1. Dyskretne próbkowanie szumu:}
\begin{equation}
    w_k \sim \mathcal{N}\!\left(0,\;\sigma^2\right),\qquad \sigma^2=\frac{P}{T_{\mathrm{s}}},
\end{equation}
gdzie $P$ oznacza moc zakłócenia (parametr konfigurowalny), a $T_{\mathrm{s}} = \Delta t$
to krok symulacji.

\textbf{2. Wygładzanie (ruchoma średnia):}
\begin{equation}
    F_{\mathrm{w},k}=\frac{1}{N_{\mathrm{s}}}\sum_{i=0}^{N_{\mathrm{s}}-1} w_{k-i},
\end{equation}
gdzie $F_{\mathrm{w},k}$ to wypadkowa siła wiatru w~chwili $k$, a~$N_{\mathrm{s}}$ to długość okna
uśredniającego (przyjęto $N_{\mathrm{s}}=10$). Takie podejście pozwala na~uzyskanie
ciągłego, wolnozmiennego sygnału lepiej odwzorowującego rzeczywiste podmuchy.
Przykładowy przebieg wygenerowanego sygnału przedstawiono na~Rys.
\ref{fig:wind_signal}.

\begin{figure}[H]
    \centering
    \includegraphics[width=0.95\textwidth]{img/wind_signal.png}
    \caption{Przykładowa realizacja stochastycznego procesu zakłócenia (wiatru)
    działającego na wahadło w czasie symulacji.}
    \label{fig:wind_signal}
\end{figure}

\subsection{Wizualizacja i animacja}

Oprócz standardowych wykresów zmiennych stanu i sterowania, środowisko wyposażono
w moduł wizualizacji dynamicznej (Rys. \ref{fig:animacja_screenshot}).
Implementacja animacji oparta jest na bibliotece \texttt{Matplotlib} i klasie
\texttt{FuncAnimation}, która pozwala na cykliczne odświeżanie obiektów
graficznych zgodnie z taktowaniem symulacji.

Graficzna reprezentacja obiektu (robot) zbudowana jest z~prostych prymitywów
geometrycznych:
\begin{itemize}
    \item \textbf{Wózek}: obiekt typu \texttt{Rectangle}, którego pozycja
    pozioma aktualizowana jest w każdej klatce na podstawie zmiennej stanu
    $x(t)$.
    \item \textbf{Koła}: obiekty \texttt{Circle}, poruszające się wraz z
    wózkiem.
    \item \textbf{Wahadło}: obiekt liniowy, którego współrzędne końcowe
    wyznaczane są trigonometrycznie na podstawie kąta $\theta(t)$.
\end{itemize}

Kluczowym elementem implementacji jest funkcja aktualizująca \texttt{update},
wywoływana dla każdego kroku czasowego. Odpowiada ona za~przeliczenie
współrzędnych kinematycznych oraz przesunięcie okna widoku kamery tak, aby wózek
znajdował się zawsze w~centrum, co pozwala na~obserwację ruchu na~długim
dystansie. Dodatkowo rysowany jest „ślad” (ang. \textit{trail}) przebytej
drogi przez oś wózka, co ułatwia wizualną ocenę stabilności pozycji.

Wykorzystanie animacji pozwala na szybką, intuicyjną weryfikację poprawności
modelu fizycznego oraz ocenę jakości regulacji w sposób trudny do uchwycenia na
statycznych wykresach (np. nienaturalne drgania czy gwałtowne reakcje
„szarpnięcia”).

\begin{figure}[H]
    \centering
    \includegraphics[width=0.95\textwidth]{img/animation.png}
    \caption{Zrzut ekranu z animacji realizowanej w środowisku Python
    (biblioteka Matplotlib). Widoczny wózek, wahadło oraz zakres ruchu.}
    \label{fig:animacja_screenshot}
\end{figure}
