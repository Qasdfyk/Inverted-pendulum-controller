%%%%%%%%%%%%%%%%%%%%%%%%%%%%%%%%%%%%%%%%%%%%%%%%%%%%%%%
%% Bachelor's & Master's Thesis Template             %%
%% Copyleft by Artur M. Brodzki & Piotr Woźniak      %%
%% Faculty of Electronics and Information Technology %%
%% Warsaw University of Technology, 2019-2020        %%
%%%%%%%%%%%%%%%%%%%%%%%%%%%%%%%%%%%%%%%%%%%%%%%%%%%%%%%

\documentclass[
    left=2.5cm,         % Sadly, generic margin parameter
    right=2.5cm,        % doesnt't work, as it is
    top=2.5cm,          % superseded by more specific
    bottom=3cm,         % left...bottom parameters.
    bindingoffset=6mm,  % Optional binding offset.
    nohyphenation=false % You may turn off hyphenation, if don't like.
]{eiti/eiti-thesis}

\langpol % Dla języka angielskiego mamy \langeng
\graphicspath{{img/}}             % Katalog z obrazkami.
\addbibresource{bibliografia.bib} % Plik .bib z bibliografią

\begin{document}

%--------------------------------------
% Strona tytułowa
%--------------------------------------
\EngineerThesis % Dla pracy inżynierskiej mamy \EngineerThesis
\instytut{Instytut Automatyki i Informatyki Stosowanej}
\kierunek{Automatyka i Robotyka}
% \specjalnosc{XXXXXX}
\title{
    Efektywny układ stabilizacji odwróconego wahadła na wózku
}
\engtitle{ % Tytuł po angielsku do angielskiego streszczenia
    Effective stabilisation system of the inverted pendulum on the cart
}
\author{Adam Sokołowski}
\album{324892}
\promotor{Robert Nebeluk}
\date{\the\year}
\maketitle

%--------------------------------------
% Streszczenie po polsku
%--------------------------------------
\cleardoublepage % Zaczynamy od nieparzystej strony
\streszczenie
W pracy porównano wybrane metody sterowania odwróconym wahadłem na wózku. Przeprowadzono symulacje numeryczne z wykorzystaniem regulatorów PID, LQR oraz układu złożonego. Analizie poddano odpowiedzi układu w warunkach nominalnych i przy zakłóceniach. Wyniki przedstawiono w formie wykresów oraz wskaźników błędów, co umożliwiło ocenę skuteczności poszczególnych metod.

\slowakluczowe odwrócone wahadło, regulator PID, regulator LQR, sterowanie optymalne, zakłócenia, symulacja, układ nieliniowy

%--------------------------------------
% Streszczenie po angielsku
%--------------------------------------
\newpage
\abstract
The paper compares selected control methods for an inverted pendulum on a cart. Numerical simulations were performed using PID, LQR controllers, and a complex system. The system responses were analyzed under nominal conditions and under disturbances. The results were presented in the form of graphs and error indices, which enabled the evaluation of the effectiveness of individual methods.

\keywords inverted pendulum, PID controller, LQR controller, optimal control, disturbances, simulation, nonlinear system


%--------------------------------------
% Oświadczenie o autorstwie
%--------------------------------------
\cleardoublepage  % Zaczynamy od nieparzystej strony
%\pagestyle{plain}
%\makeauthorship

%--------------------------------------
% Spis treści
%--------------------------------------
\cleardoublepage % Zaczynamy od nieparzystej strony
\tableofcontents

%--------------------------------------
% Rozdziały
%--------------------------------------
\cleardoublepage % Zaczynamy od nieparzystej strony
\pagestyle{headings}

%\input{tex/1-wstep}         % Wygodnie jest trzymać każdy rozdział w osobnym pliku.
%\input{tex/2-de-finibus}    % Umożliwia to również łatwą migrację do nowej wersji szablonu:
%\input{tex/3-code-listings} % wystarczy podmienić swoje pliki main.tex i eiti-thesis.cls
                            % na nowe wersje, a cały tekst pracy pozostaje nienaruszony.

% !TEX encoding = utf8
\section{Wstęp}
Odwrócone wahadło na wózku jest klasycznym przykładem nieliniowego, niestabilnego układu, wykorzystywanym zarówno w dydaktyce, jak i w badaniach nad zaawansowanymi technikami sterowania. Jego atrakcyjność wynika z tego, że choć geometria i parametry są stosunkowo proste, to układ ten jest trudny do stabilizacji i wymaga precyzyjnej regulacji w czasie rzeczywistym.

\subsection{Cel pracy}
Celem niniejszego sprawozdania jest przedstawienie projektu symulacyjnego układu stabilizacji odwróconego wahadła na wózku, w którym porównano klasyczne regulatory:
\begin{enumerate}
    \item Regulator PD lub PID do regulacji położenia wózka \(x\) oraz regulator LQR do regulacji kąta wahadła \(\varphi\).
    \item Regulator PID do regulacji całego systemu.
    \item Regulator LQR do regulacji całego systemu.
\end{enumerate}
Dodatkowo zestawiono wskaźniki jakości regulacji: średni błąd kwadratowy (MSE) oraz średni błąd bezwzględny (MAE) dla trajektorii \(\varphi(t)\) i \(x(t)\), a także przeanalizowano wpływ zakłóceń na jakość regulacji.

\subsection{Omówienie literatury}
W literaturze problem sterowania odwróconego wahadła na wózku jest traktowany jako wzorcowy układ niestabilny i nieliniowy, służący do oceny efektywności różnych algorytmów sterowania. W~\cite{art1} autorzy prezentują:
\begin{itemize}
    \item Model nieliniowy systemu z uwzględnieniem zaburzenia od wiatru oraz jego liniaryzację wokół punktu równowagi.
    \item Projekt dwóch pętli PID: jedna dla kąta \(\theta\), druga dla położenia \(x\), oraz algorytm LQR zaprojektowany na zlinearyzowanym modelu.
    \item Analizę porównawczą działania regulatorów PID, PID+LQR (dwie konfiguracje: \emph{2PID+LQR} oraz \emph{1PID+LQR}), zarówno w przypadku bez zakłóceń, jak i z zakłóceniem od wiatru.
\end{itemize}
Wzorując się na ~\cite{art1} zimplementowano regulator PID+LQR.
\newpage 
\section{Model matematyczny układu}
\subsection{Opis fizyczny układu}
Rozważany układ składa się z wózka o masie \(M\), poruszającego się poziomo, oraz przegubowo zamocowanego wahadła o masie \(m\) i długości \(l\), wychylającego się od pionu o kąt \(\varphi(t)\). Na wózek działa sterowanie w postaci siły \(u(t)\). Wahadło jest niestateczne w pozycji pionowej i wymaga stabilizacji aktywnej.

Parametry fizyczne:
\begin{itemize}
    \item \(M\) – masa wózka [kg],
    \item \(m\) – masa wahadła [kg],
    \item \(l\) – długość wahadła [m],
    \item \(I\) – moment bezwładności wahadła względem osi [kg·m²],
    \item \(b\) – współczynnik tłumienia wózka [N·s/m],
    \item \(g\) – przyspieszenie ziemskie [m/s²].
\end{itemize}

\subsection{Model liniowy w przestrzeni stanów}
Model został uzyskany przez uproszczenie dynamiczne i liniaryzację wokół punktu równowagi (\(\varphi = 0\)). Stan układu opisuje wektor:
\[
    \mathbf{x} = \begin{bmatrix}x \\ \dot{x} \\ \varphi \\ \dot{\varphi}\end{bmatrix},
\]
gdzie \(x\) – pozycja wózka, \(\varphi\) – kąt odchylenia wahadła. Model przyjmuje postać:
\[
    \dot{\mathbf{x}}(t) = A\,\mathbf{x}(t) + B\,u(t), 
    \quad
    \mathbf{y}(t) = C\,\mathbf{x}(t),
\]
przy czym:
\[
    p = I\,(M + m) + M m l^2,
\]
\[
    A =
    \begin{bmatrix}
        0 & 1 & 0 & 0 \\
        0 & -\frac{(I + m l^2)\,b}{p} & \frac{m^2 g l^2}{p} & 0 \\
        0 & 0 & 0 & 1 \\
        0 & -\frac{m l\,b}{p} & \frac{m g l\,(M + m)}{p} & 0
    \end{bmatrix}, 
    \quad
    B =
    \begin{bmatrix}
        0 \\[6pt]
        \frac{I + m l^2}{p} \\[6pt]
        0 \\[6pt]
        \frac{m l}{p}
    \end{bmatrix},
\]
\[
    C = 
    \begin{bmatrix}
        1 & 0 & 0 & 0 \\ 
        0 & 0 & 1 & 0
    \end{bmatrix}, 
    \quad
    D = \begin{bmatrix}0 \\ 0\end{bmatrix}.
\]

Model został zaimplementowany w MATLAB-ie jako funkcja \texttt{state\_space\_model.m}, która zwraca obiekt przestrzeni stanów \texttt{ss(A, B, C, D)}.

\newpage
\section{Środowisko symulacyjne i implementacja}

W celu przeprowadzenia badań i weryfikacji działania algorytmów sterowania, przygotowano autorskie środowisko symulacyjne zrealizowane w języku \textbf{Python 3}. Wybór tego języka podyktowany był jego powszechnością w zastosowaniach naukowych, dostępnością bibliotek do obliczeń numerycznych i optymalizacji, a także łatwością prototypowania złożonych struktur sterowania.

\subsection{Narzędzia programistyczne}

W projekcie wykorzystano następujące biblioteki i narzędzia:
\begin{itemize}
    \item \textbf{NumPy} -- podstawowa biblioteka do obliczeń macierzowych i operacji na wielowymiarowych tablicach danych, wykorzystywana do implementacji równań stanu oraz przechowywania przebiegów symulacji.
    \item \textbf{SciPy} -- pakiet naukowy dostarczający zaawansowanych algorytmów numerycznych. W pracy użyto modułów:
    \begin{itemize}
        \item \texttt{scipy.linalg} -- do rozwiązywania algebraicznego równania Riccatiego (ARE) w algorytmie LQR.
        \item \texttt{scipy.optimize} -- zawierającego solwer \texttt{minimize} (metoda SLSQP), wykorzystywany do rozwiązywania zadań optymalizacji nieliniowej z ograniczeniami w regulatorze MPC.
    \end{itemize}
    \item \textbf{Matplotlib} -- biblioteka służąca do wizualizacji wyników w postaci wykresów przebiegów czasowych oraz do generowania animacji ruchu wahadła.
\end{itemize}

\subsection{Konfiguracja symulacji}

Symulator opiera się na numerycznym całkowaniu wyprowadzonych wcześniej nieliniowych równań dynamiki. Zaimplementowano procedurę całkowania metodą \textbf{Rungego-Kutty czwartego rzędu (RK4)}, co zapewnia wysoki kompromis pomiędzy dokładnością a szybkością obliczeń. Kluczowy fragment implementacji algorytmu przedstawiono na Listingu \ref{lst:rk4}.

\begin{lstlisting}[language=Python, caption={Implementacja metody Rungego-Kutty 4. rzędu}, label={lst:rk4}, basicstyle=\ttfamily\footnotesize, breaklines=true]
def rk4_step(f, x, u, pars, dt):
    k1 = f(x, u, pars)
    k2 = f(x + 0.5 * dt * k1, u, pars)
    k3 = f(x + 0.5 * dt * k2, u, pars)
    k4 = f(x + dt * k3, u, pars)
    return x + (dt / 6.0) * (k1 + 2*k2 + 2*k3 + k4)
\end{lstlisting}

Przyjęto stały krok symulacji oraz sterowania wynoszący \(\Delta t = 0{,}1\,\text{s}\). Jest to wartość, przy której dynamika wahadła jest odwzorowana z wystarczającą precyzją, a jednocześnie pozwala na efektywne działanie numerycznych algorytmów optymalizacji w czasie rzeczywistym.

\begin{table}[H]
	\centering
	\caption{Parametry fizyczne modelu przyjęte w symulacji}
	\label{tab:parametry_modelu}
	\renewcommand{\arraystretch}{1.2}
	\begin{tabular}{|l|c|c|c|}
		\hline
		\textbf{Parametr} & \textbf{Symbol} & \textbf{Wartość} & \textbf{Jednostka} \\ \hline
		Masa wózka & $M$ & $2{,}4$ & $\text{kg}$ \\ \hline
		Masa wahadła & $m$ & $0{,}23$ & $\text{kg}$ \\ \hline
		Długość wahadła & $l$ & $0{,}36$ & $\text{m}$ \\ \hline
		Przyspieszenie ziemskie & $g$ & $9{,}81$ & $\text{m/s}^2$ \\ \hline
        Ograniczenie sterowania & $u_{max}$ & $100{,}0$ & $\text{N}$ \\ \hline
	\end{tabular}
\end{table}

Symulacje przeprowadzane są dla zadania stabilizacji układu w pionie (tzw. punkt pracy), startując z niezerowych warunków początkowych lub wymuszając zmianę pozycji wózka.

\textbf{Warunki początkowe (domyślne):}
\[
\mathbf{x}_0 = [\theta, \dot{\theta}, x, \dot{x}]^T = [0{,}05\,\text{rad}, 0, 0, 0]^T
\]
Oznacza to niewielkie (ok. $2{,}86^\circ$) początkowe wychylenie wahadła, które regulator musi zniwelować.

\textbf{Wartości zadane:}
Celem układu jest osiągnięcie stanu $\mathbf{x}_{ref} = [0, 0, x_{ref}, 0]^T$, gdzie $x_{ref}$ (np. $0{,}1$ m) jest zadaną nową pozycją wózka, przy jednoczesnym utrzymaniu pionowej pozycji wahadła ($\theta = 0$).

\subsection{Modelowanie zakłóceń}

Aby zweryfikować odporność układów sterowania, zaimplementowano generator zakłóceń symulujący podmuchy wiatru działające na wahadło. Model zakłócenia $F_w(t)$ oparty jest na procesie stochastycznym:
\begin{enumerate}
    \item Generowany jest biały szum gaussowski o zadanej mocy.
    \item Sygnał jest wygładzany filtrem uśredniającym (splot z oknem prostokątnym), co pozwala uzyskać bardziej realistyczne, ciągłe w czasie przebiegi siły wiatru, zamiast nieskorelowanego szumu.
\end{enumerate}
W eksperymentach z zakłóceniami, siła $F_w$ jest dodawana bezpośrednio do równań dynamiki w każdym kroku całkowania. Przykładowy przebieg wygenerowanego sygnału zakłócającego przedstawiono na Rys. \ref{fig:wind_signal}, a sposób jego generacji w kodzie źródłowym na Listingu \ref{lst:wind}.

\begin{lstlisting}[language=Python, caption={Klasa generatora zakłóceń wiatru}, label={lst:wind}, basicstyle=\ttfamily\footnotesize, breaklines=true]
class Wind:
    def __init__(self, t_end: float, seed=23341, Ts=0.1, power=1e-3, smooth=5):
        rng = np.random.default_rng(seed)
        self.tgrid = np.arange(0.0, t_end + Ts, Ts)
        sigma = np.sqrt(power / Ts)
        w = rng.normal(0.0, sigma, size=self.tgrid.shape)
        if smooth and smooth > 1:
            kernel = np.ones(smooth) / smooth
            self.Fw = np.convolve(w, kernel, mode='same')
        else:
            self.Fw = w

    def __call__(self, t: float) -> float:
        return float(np.interp(t, self.tgrid, self.Fw))
\end{lstlisting}

\begin{figure}[H]
    \centering
    \includegraphics[width=0.8\textwidth]{img/wind_signal.png}
    \caption{Przykładowa realizacja stochastycznego procesu zakłócenia (wiatru) działającego na wahadło w czasie symulacji.}
    \label{fig:wind_signal}
\end{figure}

\subsection{Wizualizacja i animacja}

Oprócz standardowych wykresów zmiennych stanu i sterowania, środowisko wyposażono w moduł wizualizacji dynamicznej (Rys. \ref{fig:animacja_screenshot}). Implementacja animacji oparta jest na bibliotece \texttt{Matplotlib} i klasie \texttt{FuncAnimation}, która pozwala na cykliczne odświeżanie obiektów graficznych zgodnie z taktowaniem symulacji.

Graficzna reprezentacja obiektu (robot) zbudowana jest z prostych prymitywów geometrycznych:
\begin{itemize}
    \item \textbf{Wózek}: obiekt typu \texttt{Rectangle}, którego pozycja pozioma aktualizowana jest w każdej klatce na podstawie zmiennej stanu $x(t)$.
    \item \textbf{Koła}: obiekty \texttt{Circle}, poruszające się wraz z wózkiem.
    \item \textbf{Wahadło}: obiekt liniowy, którego współrzędne końcowe wyznaczane są trigonometrycznie na podstawie kąta $\theta(t)$.
\end{itemize}

Kluczowym elementem implementacji jest funkcja aktualizująca \texttt{update}, wywoływana dla każdego kroku czasowego. Odpowiada ona za przeliczenie współrzędnych kinematycznych (Listing \ref{lst:animation_logic}) oraz przesunięcie okna widoku kamery tak, aby wózek znajdował się zawsze w centrum, co pozwala na obserwację ruchu na długim dystansie. Dodatkowo rysowany jest „ślad” (ang. \textit{trail}) przebytej drogi przez oś wózka, co ułatwia wizualną ocenę stabilności pozycji.

\begin{lstlisting}[language=Python, caption={Logika aktualizacji klatki animacji}, label={lst:animation_logic}, basicstyle=\ttfamily\footnotesize, breaklines=true]
    def pole_end(i):
        cx = x[i]; cy = wheel_r + cart_h
        # Kinematyka prosta wahadla
        px = cx + pole_len * np.sin(th[i]) 
        py = cy + pole_len * np.cos(th[i])
        return cx, cy, px, py

    def update(i):
        cx, cy, px, py = pole_end(i)
        # Aktualizacja pozycji obiektow graficznych
        cart.set_x(cx - cart_w/2)
        wheel1.center = (cx - cart_w/3, wheel_r)
        pole_line.set_data([cx, px], [cy, py])
        # Centrowanie kamery na wozku
        ax.set_xlim(cx - pad, cx + pad)
        return cart, wheel1, pole_line
\end{lstlisting}

Wykorzystanie animacji pozwala na szybką, intuicyjną weryfikację poprawności modelu fizycznego oraz ocenę jakości regulacji w sposób trudny do uchwycenia na statycznych wykresach (np. nienaturalne drgania czy gwałtowne reakcje „szarpnięcia”).

\begin{figure}[H]
    \centering
    \includegraphics[width=0.7\textwidth]{img/animation.png} 
    \caption{Zrzut ekranu z animacji realizowanej w środowisku Python (biblioteka Matplotlib). Widoczny wózek, wahadło oraz zakres ruchu.}
    \label{fig:animacja_screenshot}
\end{figure}

\newpage
\section{Algorytmy sterowania}

W~niniejszym rozdziale przedstawiono szczegółowy opis algorytmów sterowania
zaimplementowanych i~przeanalizowanych w~ramach pracy. Kod sterowników został
zrealizowany w~języku Python w~postaci klas dziedziczących wspólną strukturę,
co zapewnia modularność i~łatwą wymienność w~pętli symulacyjnej. Każdy
regulator wyznacza sygnał sterujący $u(t)$ (siłę przyłożoną do wózka)
na~podstawie aktualnego wektora stanu
$x(t) = [\theta, \dot{\theta}, x, \dot{x}]^T$ oraz wartości zadanych
$x_{\mathrm{ref}}$.

W~literaturze problem sterowania wahadłem odwróconym jest szeroko omawiany
jako klasyczny problem testowy dla metod sterowania liniowego i~nieliniowego
\cite{Prasad2014, Nguyen2024}. Poniżej opisano teoretyczne podstawy oraz szczegóły
implementacyjne zbadanych struktur sterowania.

\subsection{Równoległy regulator PD-PD}

Pierwszym zaimplementowanym układem jest regulator o~strukturze kaskadowej lub
równoległej, wykorzystujący klasyczne sprzężenie zwrotne typu PD
(Proporcjonalno-Różniczkujące). W~literaturze podejście to jest często
stosowane jako punkt odniesienia dla bardziej zaawansowanych metod
\cite{Moreno2023, Prasad2014}.

W~klasie \texttt{PDPDController} zastosowano strukturę równoległą, w~której
całkowity sygnał sterujący jest sumą reakcji na~błąd kąta oraz błąd pozycji.
Jest to podejście intuicyjne, dekomponujące problem na~dwa podzadania:
stabilizację wahadła w~pozycji pionowej oraz doprowadzenie wózka do~zadanej
pozycji.

Prawo sterowania wyraża się wzorem:
\begin{equation}
    u(t) = \mathrm{sat}_{u_{\mathrm{max}}} \left( u_{\theta}(t) + u_{x}(t) \right),
\end{equation}
gdzie funkcja nasycenia $\mathrm{sat}(\cdot)$ wynika z~ograniczeń fizycznych
Definiując uchyby regulacji jako $e_{\theta}(t) = \theta_{\mathrm{ref}} - \theta(t)$ oraz $e_x(t) = x_{\mathrm{ref}} - x(t)$, prawo sterowania dla poszczególnych pętli można zapisać w~ogólnej postaci regulatora PID:
\begin{align}
    u_{\theta}(t) &= K_{p,\theta} e_{\theta}(t) + K_{i,\theta} \int_0^t e_{\theta}(\tau)\,d\tau + K_{d,\theta} \frac{d e_{\theta}(t)}{dt}, \label{eq:pd_theta} \\
    u_{x}(t) &= K_{p,x} e_x(t) + K_{i,x} \int_0^t e_x(\tau)\,d\tau + K_{d,x} \frac{d e_x(t)}{dt}. \label{eq:pd_x}
\end{align}
W~powyższych równaniach przyjęto upraszczające założenie, że docelowe
prędkości ($\dot{\theta}_{\mathrm{ref}}, \dot{x}_{\mathrm{ref}}$) wynoszą zero.

W~implementacji programowej (plik \texttt{pd\_pd.py}) przyjęto następujące
nastawy dobrane eksperymentalnie:
\begin{itemize}
    \item Tor stabilizacji kąta: $K_{p,\theta} = -95.0$, $K_{d,\theta} = -14.0$.
    Ujemne znaki wynikają z~przyjętej konwencji układu współrzędnych i~zwrotu siły.
    \item Tor pozycji: $K_{p,x} = -16.0$, $K_{d,x} = -14.0$.
\end{itemize}
Mimo iż klasa umożliwia włączenie członu całkującego (PID), w~badaniach
\cite{Varghese2017} często wskazuje się, że dla obiektów tej klasy człon różniczkujący
(PD) jest kluczowy dla tłumienia oscylacji, a~całkowanie może wprowadzać
niestabilność w~stanach nieustalonych bez odpowiednich mechanizmów
$\mathrm{anti\text{-}windup}$.

\subsubsection{Proces doboru nastaw oraz analiza PID}
Dobór nastaw dla regulatora PD-PD został zrealizowany wieloetapowo, ewoluując
od metod heurystycznych do pełnej optymalizacji numerycznej. Wstępne próby
doboru metodą ,,prób i~błędów'', oparte na~dekompozycji problemu (najpierw
stabilizacja wahadła, potem pozycja wózka), pozwoliły uzyskać stabilność,
jednak jakość regulacji była niezadowalająca. Układ charakteryzował się
powolnym dochodzeniem do~punktu pracy i~znacznymi oscylacjami
(\ref{fig:pdpd_manual}).

Aby wyeliminować subiektywność strojenia ręcznego, zastosowano algorytm
Ewolucji Różnicowej (Differential Evolution), zaimplementowany w~module
\texttt{scipy.optimize}. Zdefiniowano globalną funkcję kosztu $J$, która
gwarantuje ,,uczciwe'' porównanie wszystkich badanych regulatorów:
\begin{equation}
    J = w_{\theta} \cdot \text{MSE}(\theta) + w_{x} \cdot \text{MSE}(x) + w_{u} \cdot \text{RMS}(u),
\end{equation}
gdzie przyjęto wagi $w_{\theta}=4.0$ (priorytet stabilizacji), $w_{x}=1.0$
(dokładność pozycjonowania) oraz $w_{u}=0.01$ (koszt energii). Algorytm operował
na populacji 10 osobników przez 20 generacji, co pozwoliło uniknąć minimów
lokalnych i~znaleźć optymalny zestaw wzmocnień (\ref{fig:pdpd_opt}).

\paragraph{Analiza porównawcza struktur PD i PID}
W~literaturze przedmiotu \cite{Varghese2017, Nguyen2024} często podkreśla się, że dla
obiektów niestabilnych statycznie, takich jak wahadło odwrócone, kluczowa jest
szybka reakcja na~zmiany kąta, którą zapewnia człon różniczkujący (D).
Włączenie członu całkującego (I), tworzącego regulator PID, wprowadza dodatkowe
przesunięcie fazowe (opóźnienie), co w~układzie o~dynamice astatycznej
i~nieliniowej prowadzi do znacznego pogorszenia zapasu stabilności.

Przeprowadzone badania symulacyjne potwierdziły te wnioski. Próba dodania akcji
całkującej ($K_i \neq 0$) skutkowała zjawiskiem \textit{integral wind-up} --
akumulacją błędu w~fazie rozruchu, co powodowało przeregulowania wykraczające
poza obszar przyciągania stabilnego punktu równowagi. Jak widać na~rysunku
\ref{fig:pid_bad}, regulator PID wpada w~niegasnące oscylacje lub doprowadza
do przewrócenia wahadła, podczas gdy ,,czysty'' regulator PD zapewnia sztywne
i~szybkie sterowanie.

\begin{figure}[H]
    \centering
    \includegraphics[width=1.0\textwidth]{images/tuning/pid_1_integral_bad.png}
    \caption{Destabilizujący wpływ członu całkującego (PID) - widoczne narastające oscylacje i utrata stabilności.}
    \label{fig:pid_bad}
\end{figure}

\begin{figure}[H]
    \centering
    \includegraphics[width=1.0\textwidth]{images/tuning/pdpd_2_manual.png}
    \caption{Stabilna, lecz oscylacyjna praca regulatora PD-PD przy strojeniu ręcznym.}
    \label{fig:pdpd_manual}
\end{figure}

\begin{figure}[H]
    \centering
    \includegraphics[width=1.0\textwidth]{images/tuning/pdpd_3_opt.png}
    \caption{Przebiegi czasowe dla zoptymalizowanych nastaw regulatora PD-PD (algorytm Differential Evolution).}
    \label{fig:pdpd_opt}
\end{figure}

\subsection{Układ hybrydowy PD-LQR}

Kolejnym analizowanym algorytmem jest regulator liniowo-kwadratowy (LQR),
będący standardem w~sterowaniu optymalnym systemów wielowymiarowych MIMO
\cite{Jezierski2017}. Klasa \texttt{PDLQRController} implementuje sterowanie
oparte na~pełnym wektorze stanu, wspomagane dodatkowym członem PD dla uchybu
pozycji, co tworzy strukturę hybrydową opisaną m.in. w~\cite{Prasad2014} oraz
\cite{Nguyen2024} (w~kontekście porównawczym).

Problem LQR polega na~znalezieniu prawa sterowania
$u(t) = -K x(t)$, które minimalizuje wskaźnik jakości:
\begin{equation}
    J = \int_{0}^{\infty} \left( x(t)^T Q x(t) + u(t)^T R u(t) \right) dt,
\end{equation}
gdzie $Q \succeq 0$ jest macierzą wag stanu, a~$R > 0$ wagą
sterowania. Optymalna macierz wzmocnień $K$ wyznaczana jest poprzez
rozwiązanie algebraicznego równania Riccatiego (CARE):
\begin{equation}
    A^T P + P A - P B R^{-1} B^T P + Q = 0,
\end{equation}
skąd $K = R^{-1} B^T P$. Macierze $A$
i~$B$ pochodzą z~linearyzacji modelu wahadła wokół punktu równowagi
górnej ($\theta = 0$).

W~zaimplementowanym rozwiązaniu (plik \texttt{pd\_lqr.py}), sygnał sterujący
składa się z~dwóch komponentów:
\begin{equation}
    u(t) = u_{\mathrm{LQR}}(t) + u_{\mathrm{PD,pos}}(t).
\end{equation}
Składnik LQR realizuje stabilizację wokół punktu pracy:
\begin{equation}
    u_{\mathrm{LQR}}(t) = -K \cdot (x(t) - x_{\mathrm{ref}}).
\end{equation}
Zastosowane wagi optymalne to:
\begin{equation}
    Q = \text{diag}([1.0,\; 1.0,\; 500.0,\; 250.0]), \quad R = 1.0.
\end{equation}
Dodatkowy człon PD na~pętli pozycji (zrealizowany analogicznie do wzoru
\ref{eq:pd_x}) ma na~celu poprawę śledzenia skokowych zmian wartości zadanej
$x_{\mathrm{ref}}$, co jest częstą praktyką w~aplikacjach praktycznych, gdzie LQR
zapewnia stabilność, a~regulator zewnętrzny dba o~uchyb w~stanie ustalonym
\cite{Varghese2017}.

\subsubsection{Dobór wag macierzy Q i R}
Dobór wag dla regulatora LQR również charakteryzował się ewolucyjnym podejściem
do~problemu optymalizacji wskaźnika jakości.

W~pierwszej fazie przyjęcie jednostkowej macierzy diagonalnej $Q=I$ oraz $R=1$
okazało się niewystarczające. Mimo teoretycznej stabilności wynikającej z~rozwiązania
równania CARE, wahadło wykonywało bardzo duże wychylenia, a~wózek wielokrotnie
wyjeżdżał poza dopuszczalny zakres roboczy toru. Świadczyło to o~zbyt małej karze
nałożonej na~uchyb kątowy.

\begin{figure}[H]
    \centering
    \includegraphics[width=1.0\textwidth]{images/tuning/pdlqr_1_bad.png}
    \caption{Próba sterowania LQR z wagami jednostkowymi ($Q=I, R=1$). Widoczna duża bezwładność układu.}
    \label{fig:lqr_bad}
\end{figure}

Następnie przeprowadzono strojenie ręczne metodą prób i~błędów (zgodnie z~regułą
Brysona). Ręczne zwiększanie kar za~wychylenie kąta ($Q_{\theta}$) poprawiło
sztywność wahadła. Udało się ustalić zestaw wag zapewniający stabilną pracę, choć
czas regulacji był wciąż niezadowalający, a~reakcja na~zakłócenia powolna.

\begin{figure}[H]
    \centering
    \includegraphics[width=1.0\textwidth]{images/tuning/pdlqr_2_manual.png}
    \caption{Wyniki strojenia ręcznego LQR metodą Brysona.}
    \label{fig:lqr_manual}
\end{figure}

W~ostatnim etapie zastosowano optymalizację numeryczną. Algorytm genetyczny
poszukiwał optymalnych elementów diagonalnych macierzy $Q$ oraz skalara $R$,
    minimalizując czas regulacji. Zoptymalizowane wagi (w szczególności wysoka kara
    $Q_{x} = 500$) sprawiają, że regulator bardzo agresywnie pilnuje pozycji wózka,
    co pośrednio wymusza stabilne trzymanie wahadła (wymagane do kontroli pozycji).

\begin{figure}[H]
    \centering
    \includegraphics[width=1.0\textwidth]{images/tuning/pdlqr_3_opt.png}
    \caption{Optymalne przebiegi regulatora PD-LQR po zastosowaniu algorytmu genetycznego.}
    \label{fig:lqr_opt}
\end{figure}

\subsection{Nieliniowe sterowanie predykcyjne (MPC)}

Algorytm MPC (Model Predictive Control) stanowi zaawansowaną metodę sterowania,
która w~odróżnieniu od~LQR, uwzględnia wprost ograniczenia sygnału sterującego
oraz nieliniową dynamikę obiektu \cite{Camacho2007, Rawlings2017}.
Zaimplementowany w~klasie \texttt{MPCController} (plik \texttt{mpc.py})
algorytm rozwiązuje w~każdym kroku symulacji problem optymalizacji dynamicznej
nieliniowej (NMPC).

Zadanie optymalizacji, rozwiązywane numerycznie metodą SQP (Sequential
Quadratic Programming) przy użyciu solwera \texttt{SLSQP}, zdefiniowane jest
następująco:
\begin{equation}
    \min_{\Delta U} J = \sum_{k=1}^{N_{\mathrm{p}}} (\hat{x}_k - x_{\mathrm{ref}})^T Q (\hat{x}_k - x_{\mathrm{ref}}) + R \sum_{k=0}^{N_{\mathrm{c}}-1} (\Delta u_k)^2,
\end{equation}
przy ograniczeniach:
\begin{align}
    \hat{x}_{k+1} &= f(\hat{x}_k, u_k), \quad k=0,\dots,N_{\mathrm{p}}-1 \\
    u_{\mathrm{min}} &\le u_k \le u_{\mathrm{max}}, \\
    u_k &= u_{k-1} + \Delta u_k.
\end{align}
Gdzie:
\begin{itemize}
    \item $N_{\mathrm{p}} = 12$ -- horyzont predykcji,
    \item $N_{\mathrm{c}} = 4$ -- horyzont sterowania (blokowanie sterowania dla $k \ge N_{\mathrm{c}}$),
    \item $f(\cdot)$ -- nieliniowy model dyskretny obiektu (całkowanie metodą
    Rungego-Kutt 4. rzędu),
    \item $Q = \text{diag}([158.4,\; 36.8,\; 43.4,\; 19.7])$ -- macierz kar stanu,
    \item $R = 0.086$ -- współczynnik kary za~zmianę sterowania ($\Delta u$).
\end{itemize}
Kluczową zaletę MPC, podkreślaną w~pracach \cite{Mills2009} oraz
\cite{Jezierski2017}, jest możliwość bezpośredniego uwzględnienia ograniczeń
(saturacji) już na~etapie wyliczania sterowania, co zapobiega zjawisku
nasycenia elementu wykonawczego, które mogłoby mieć miejsce w~przypadku LQR.

Analiza wykazała, że bezpośrednie przeniesienie macierzy wag $Q$ i $R$
z~regulatora LQR do~sterownika MPC prowadziło do~znaczącego pogorszenia jakości
sterowania (wydłużenie czasu regulacji z~ok. 3s do~ponad 9s). Wynika to z~faktu,
że model MPC, dzięki jawnemu uwzględnieniu ograniczeń sygnału sterującego, pozwala
na~zastosowanie znacznie bardziej agresywnych nastaw (większych kar za~błędy stanu),
które w~liniowym regulatorze LQR powodowałyby nasycenie i~potencjalną niestabilność.
Dlatego zdecydowano się na~niezależną optymalizację parametrów obu regulatorów,
aby porównywać ich najlepsze możliwe konfiguracje, a~nie identyczne, ale
nieoptymalne nastawy.

\subsubsection{Dobór horyzontu i wag funkcji celu}
Dla regulatora MPC kluczowym wyzwaniem był dobór horyzontu predykcji oraz
macierzy wag, determinujących zachowanie układu w~stanie nieustalonym.

Początkowe ustawienie zbyt krótkiego horyzontu predykcji ($N_{\mathrm{p}} < 5$)
prowadziło do~niestabilności układu zamkniętego. Regulator ,,nie widział'', że
rozpędzając wózek w~celu korekcji kąta, nie zdąży wyhamować przed upadkiem
wahadła lub osiągnięciem końca toru. Zwiększenie horyzontu do $N_{\mathrm{p}}=10$
w~ramach korekty ręcznej ustabilizowało proces. Dodatkowa manipulacja wagami $Q$
pozwoliła na~uzyskanie poprawnego sterowania, jednak koszt obliczeniowy był
wysoki, a~przebiegi wciąż wykazywały niepożądane przeregulowania.

Automatyzacja procesu strojenia przy użyciu skryptu \texttt{tune\_mpc.py} pozwoliła
na~znalezienie kompromisu między długością horyzontu a~wagami. Algorytm
optymalizacyjny wskazał $N_{\mathrm{p}}=12$ jako optimum dla tego modelu
dyskretnego, zapewniając stabilność przy akceptowalnym czasie obliczeń.

\begin{figure}[H]
    \centering
    \includegraphics[width=1.0\textwidth]{images/tuning/mpc_3_opt.png}
    \caption{Zoptymalizowany regulator MPC ($N_p=12$) - szybka i gładka stabilizacja.}
    \label{fig:mpc_opt}
\end{figure}

\subsection{MPC z~rozszerzonym wskaźnikiem jakości (MPC-J2)}

Zaimplementowano sterownik \texttt{MPCControllerJ2} jako wariant badawczy algorytmu predykcyjnego
(plik \texttt{mpc\_J2.py}). Jego struktura jest
zbliżona do~podstawowego MPC, jednak funkcja kosztu została rozbudowana
o~dodatkowy składnik karzący bezwzględną wartość sygnału sterującego
(energię), a~nie tylko jego przyrosty.

Zmodyfikowana funkcja celu przyjmuje postać:
\begin{equation}
    J = \sum_{k=1}^{N_{\mathrm{p}}} (x_k - x_{\mathrm{ref}})^T Q (x_k - x_{\mathrm{ref}}) \;+\; R_{\Delta} \sum_{k=0}^{N_{\mathrm{c}}-1} (\Delta u_k)^2 \;+\; R_{\mathrm{abs}} \sum_{k=0}^{N_{\mathrm{c}}-1} (u_k)^2.
\end{equation}
Wprowadzenie parametru $R_{\mathrm{abs}}$ pozwala na~bezpośrednie minimalizowanie
zużycia energii sterowania, co jest podejściem powszechnie stosowanym
w~praktycznych implementacjach algorytmów predykcyjnych \cite{Camacho2007, Rawlings2017}.
Ograniczenie amplitudy sygnału sterującego nie tylko redukuje wydatek energetyczny
(istotny w~aplikacjach mobilnych), ale także zmniejsza obciążenie mechaniczne
elementów wykonawczych, co wpływa na~żywotność napędu.

W~badaniach przyjęto wagi: $q_{\theta}=80.0$,
$q_x=120.0$ (elementy macierzy diagonalnej $Q$), kładąc większy
nacisk na~precyzję pozycjonowania wózka w~porównaniu do~standardowego MPC.

\subsubsection{Analiza wpływu kary za energię}
W~przypadku wariantu MPC-J2 analizowano nieliniowy wpływ parametru $R_{\mathrm{abs}}$
na~zachowanie układu.

Przyjęcie zbyt dużej wartości kary za~sterowanie bezwzględne ($R_{\mathrm{abs}}$)
spowodowało, że regulator wykazywał tendencję do~pasywności. Wahadło przewracało się,
ponieważ koszt energetyczny utrzymania go w~pionie przewyższał zysk wynikający
z~małego błędu kąta w~funkcji celu.

Stopniowe, ręczne zmniejszanie parametru $R_{\mathrm{abs}}$ pozwoliło znaleźć punkt
pracy, w~którym układ odzyskał stabilność przy zachowaniu relatywnej oszczędności
energetycznej. Odpowiedź dynamiczna była jednak powolna i~zbyt asekuracyjna dla
większych zakłóceń.

Algorytm optymalizacyjny precyzyjnie dostroił $R_{\mathrm{abs}}$, minimalizując złożony
wskaźnik kosztu (błąd + energia). Znaleziono ,,złoty środek'', w~którym układ
stabilizuje się szybko, ale sterowanie pozbawione jest zbędnych oscylacji
wysokoczęstotliwościowych, co przekłada się na~oszczędność energii.

\begin{figure}[H]
    \centering
    \includegraphics[width=1.0\textwidth]{images/tuning/mpcJ2_3_opt.png}
    \caption{Optymalny kompromis między jakością regulacji a energią w MPC-J2.}
    \label{fig:mpcj2_opt}
\end{figure}

\subsection{Regulator rozmyty wspomagany LQR (Fuzzy-LQR)}

Ostatnim zbadanym układem jest sterownik hybrydowy \texttt{TSFuzzyController}
(plik \texttt{fuzzy\_lqr.py}), łączący liniowy regulator LQR z~systemem
wnioskowania rozmytego typu Takagi-Sugeno (T-S). Koncepcja ta, opisana szerzej
w~\cite{Nguyen2024} oraz \cite{Roose2017}, ma na~celu adaptację wzmocnień regulatora
w~zależności od~punktu pracy, co pozwala na~agresywniejszą reakcję w~przypadku
dużych odchyleń od~pionu.

Sygnał sterujący jest sumą:
\begin{equation}
    u(t) = u_{\mathrm{LQR}}(t) + u_{\mathrm{Fuzzy}}(t).
\end{equation}
Część rozmyta $u_{\mathrm{Fuzzy}}(t)$ wykorzystuje bazę reguł postaci:
\begin{quote}
    JEŚLI $e_\theta$ jest $A_i$ ORAZ $\dot{\theta}$ jest $B_i$ ... TO $u_i = f_i(x)$,
\end{quote}
gdzie $f_i(x)$ jest liniową funkcją stanu (lokalny regulator
liniowy). Zastosowano funkcje przynależności trójkątne dla zmiennych stanu,
dzieląc przestrzeń na~obszary ,,Mały błąd'' i~,,Duży błąd''.
Baza wiedzy składa się z~16 reguł ($2^4$ kombinacji dla 4 zmiennych stanu).
Wyjście sterownika obliczane jest jako średnia ważona:
\begin{equation}
    u_{\mathrm{Fuzzy}} = G \cdot \frac{\sum_{i=1}^{16} w_i(x) \cdot u_i}{\sum_{i=1}^{16} w_i(x)},
\end{equation}
gdzie $w_i$ to stopień aktywacji $i$-tej reguły, a~$G = 0.9$ to globalne
wzmocnienie skalujące.

Zastosowany mechanizm ,,Gain Scheduling'' pozwala na:
\begin{enumerate}
    \item Zachowanie łagodnej charakterystyki LQR w~pobliżu punktu równowagi
    (małe wzmocnienia w~regułach dla ,,Małych błędów'').
    \item Zwiększenie sztywności układu w~sytuacjach krytycznych (duże
    wzmocnienia zdefiniowane w~zmiennej \texttt{F\_rules} dla ,,Dużych
    błędów'').
\end{enumerate} 
Takie podejście pozwala na~rozszerzenie obszaru stabilności regulatora
w~porównaniu do~klasycznego LQR, co potwierdzają wyniki badań w~pracy
\cite{Nguyen2024}.

\subsubsection{Dobór reguł i funkcji przynależności}
Strojenie rozmytego regulatora Fuzzy-LQR jest zadaniem złożonym ze względu na~dużą
liczbę parametrów definiujących bazę reguł i~funkcje przynależności.

Błędne zdefiniowanie zbyt wąskich funkcji przynależności dla strefy ,,małego błędu''
skutkowało gwałtownym przełączaniem się regulatora na~agresywne reguły (chatter).
Prowadziło to do~silnych drgań wokół punktu równowagi, co jest zjawiskiem
niepożądanym w~rzeczywistych układach napędowych.
Opierając się na~literaturze \cite{Nguyen2024}, dobrano ręcznie szerokości trójkątnych
funkcji przynależności tak, aby przejście między strefami było płynne. Układ uzyskał
stabilność asymptotyczną, jednak nie wykorzystywał w~pełni potencjału szybkiej
reakcji na~duże zakłócenia, działając zachowawczo.

Ostatecznie, dedykowany skrypt \texttt{tune\_fuzzy\_lqr.py} posłużył do~optymalizacji
wag pojedynczych reguł oraz parametrów kształtu funkcji przynależności. Uzyskano
nieliniową powierzchnię sterowania, która łączy zalety miękkiego sterowania LQR
w~pobliżu zera z~maksymalną siłą reakcji przy dużych wychyleniach.

\begin{figure}[H]
    \centering
    \includegraphics[width=1.0\textwidth]{images/tuning/fuzzy_3_opt.png}
    \caption{Efektywne sterowanie Fuzzy-LQR po optymalizacji bazy reguł.}
    \label{fig:fuzzy_opt}
\end{figure}

\newpage
\section{Eksperymenty}

\subsection{Opis przeprowadzonych eksperymentów}
W celu porównania skuteczności różnych strategii sterowania dla odwróconego wahadła na wózku, przeprowadzono szereg symulacji numerycznych. Każdy eksperyment polegał na zasymulowaniu zachowania układu w odpowiedzi na tę samą konfigurację początkową:
\[
    x(0) = 0,\quad \dot{x}(0) = 0,\quad \varphi(0) = 45^\circ,\quad \dot{\varphi}(0) = 0.
\]
Symulacje trwały 5 sekund, z krokiem czasowym \(\Delta t = 0{,}01\,\text{s}\). Rozważono trzy warianty regulatorów:
\begin{itemize}
    \item \textbf{LQR} - regulator liniowo-kwadratowy oparty na pełnym stanie,
    \item \textbf{PID} - regulator PID zaprojektowany dla modelu przenoszenia kąta \(\varphi\),
    \item \textbf{Composite} - układ złożony: regulator PID dla pozycji wózka \(x\) i LQR dla kąta \(\varphi\).
\end{itemize}

Każdy z powyższych regulatorów został przetestowany w dwóch wariantach:
\begin{enumerate}
    \item bez zakłóceń,
    \item z zakłóceniem w postaci dwóch podmuchów.
\end{enumerate}

W trakcie symulacji zarejestrowano:
\begin{itemize}
    \item trajektorię kąta wahadła \(\varphi(t)\),
    \item trajektorię położenia wózka \(x(t)\),
    \item sygnał sterujący \(u(t)\).
\end{itemize}

Na podstawie zebranych danych obliczono również dwie metryki jakości:
\begin{itemize}
    \item \textbf{MSE} - średni błąd kwadratowy: \(\displaystyle \mathrm{MSE}_\varphi = \frac{1}{N}\sum_{k=1}^{N} \varphi[k]^2\),
    \item \textbf{MAE} - średni błąd bezwzględny: \(\displaystyle \mathrm{MAE}_\varphi = \frac{1}{N}\sum_{k=1}^{N} |\varphi[k]|\),
\end{itemize}

\newpage
\subsection{Dobór parametrów regulatorów}
\begin{figure}[H]
    \centering
    \includegraphics[width=0.8\textwidth]{img/pid_zly.png}
    \caption{Źle dostrojony regulator PID.}
\end{figure}
\begin{figure}[H]
    \centering
    \includegraphics[width=0.8\textwidth]{img/pid_git.png}
    \caption{Dostrojony regulator PID przy użyciu pidtune.}
\end{figure}

Regulator PID dostrojono przy użyciu gotowej funkcji pidtune, natomiast regulator LQR dostrojono ręcznie.
\begin{figure}[H]
    \centering
    \includegraphics[width=0.95\textwidth]{img/przebiegi_phi.png}
    \caption{Sygnały wyjściowe.}
\end{figure}

\begin{figure}[H]
    \centering
    \includegraphics[width=0.95\textwidth]{img/zaklocenia.png}
    \caption{Sygnały wyjściowe z zakłóceniami.}
\end{figure}
\newpage
\subsection{Porównanie jakości regulacji - wskaźniki błędów}

\begin{table}[H]
    \centering
    \caption{Wartości wskaźników MSE i MAE dla kąta \(\varphi(t)\) - bez zakłóceń}
    \begin{tabular}{l|c|c}
        \textbf{Regulator} & \textbf{MSE\(_\varphi\)} & \textbf{MAE\(_\varphi\)} \\
        \hline
        LQR & \texttt{0.0161} & \texttt{0.0585} \\
        PID & \texttt{0.0042} & \texttt{0.0126} \\
        Composite & \texttt{0.0209} & \texttt{0.0677} \\
    \end{tabular}
\end{table}

\vspace{1em}

\begin{table}[H]
    \centering
    \caption{Wartości wskaźników MSE i MAE dla kąta \(\varphi(t)\) - z zakłóceniem}
    \begin{tabular}{l|c|c}
        \textbf{Regulator} & \textbf{MSE\(_\varphi\)} & \textbf{MAE\(_\varphi\)} \\
        \hline
        LQR & \texttt{0.0171} & \texttt{0.0687} \\
        PID & \texttt{0.0067} & \texttt{0.0402} \\
        Composite & \texttt{0.0322} & \texttt{0.1226} \\
    \end{tabular}
\end{table}


\subsection{Podsumowanie eksperymentów}

Przeprowadzone eksperymenty pozwoliły na porównanie trzech różnych strategii sterowania odwróconym wahadłem na wózku w identycznych warunkach początkowych oraz w obecności zakłóceń. Na podstawie analizy sygnałów oraz wartości wskaźników błędów (MSE i MAE) można sformułować następujące wnioski:

\begin{itemize}
    \item Regulator \textbf{PID} wykazał najlepszą skuteczność w warunkach bez zakłóceń, osiągając najniższe wartości MSE i MAE, co sugeruje dobrą zdolność do szybkiego i precyzyjnego tłumienia wychylenia wahadła.
    \item W obecności zakłóceń regulator PID nadal utrzymywał dobrą jakość regulacji, jednak zauważalnie większy błąd bezwzględny (MAE) wskazuje na ograniczoną odporność na zaburzenia w porównaniu do LQR.
    \item Regulator \textbf{LQR}, choć nieco mniej precyzyjny w warunkach nominalnych, wykazał większą stabilność i odporność na zakłócenia, co czyni go solidnym wyborem w sytuacjach, gdzie przewiduje się obecność niestandardowych sił.
    \item Układ \textbf{Composite}, łączący zalety obu strategii, nie przyniósł oczekiwanej poprawy jakości regulacji. W szczególności zauważono pogorszenie wskaźników błędów, co może być związane z nieoptymalnym podziałem zadań między PID i LQR.
\end{itemize}

Podsumowując, dobór odpowiedniego regulatora powinien być uzależniony od warunków pracy układu - w sytuacjach nominalnych lepiej sprawdza się PID, natomiast w obecności zakłóceń bardziej odpowiedni może okazać się LQR.


%--------------------------------------------
% Literatura
%--------------------------------------------
\cleardoublepage % Zaczynamy od nieparzystej strony
\printbibliography

%--------------------------------------------
% Spisy (opcjonalne)
%--------------------------------------------
\newpage
\pagestyle{plain}

% Wykaz symboli i skrótów.
% Pamiętaj, żeby posortować symbole alfabetycznie
% we własnym zakresie. Ponieważ mało kto używa takiego wykazu,
% uznałem, że robienie automatycznie sortowanej listy
% na poziomie LaTeXa to za duży overkill.
% Makro \acronymlist generuje właściwy tytuł sekcji,
% w zależności od języka.
% Makro \acronym dodaje skrót/symbol do listy,
% zapewniając podstawowe formatowanie.
% //AB
\vspace{0.8cm}
\acronymlist
\acronym{$x$}{Położenie wózka [m]}
\acronym{$\dot{x}$}{Prędkość wózka [m/s]}
\acronym{$\varphi$}{Kąt odchylenia wahadła od pionu [rad]}
\acronym{$\dot{\varphi}$}{Prędkość kątowa wahadła [rad/s]}
\acronym{$u$}{Sygnał sterujący (siła działająca na wózek) [N]}
\acronym{LQR}{Linear-Quadratic Regulator – regulator liniowo-kwadratowy}
\acronym{PID}{Proporcjonalno-Całkująco-Różniczkujący regulator}
\acronym{MSE}{Mean Squared Error – średni błąd kwadratowy}
\acronym{MAE}{Mean Absolute Error – średni błąd bezwzględny}
\acronym{$\Delta t$}{Krok czasowy symulacji [s]}
\acronym{$N$}{Liczba próbek w sygnale dyskretnym}


\listoffigurestoc     % Spis rysunków.
\vspace{1cm}          % vertical space
\listoftablestoc      % Spis tabel.
\vspace{1cm}          % vertical space
% \listofappendicestoc  % Spis załączników

% Załączniki
% \newpage
% \appendix{Nazwa załącznika 1}
% \lipsum[1-8]

% \newpage
% \appendix{Nazwa załącznika 2}
% \lipsum[1-4]

% Używając powyższych spisów jako szablonu,
% możesz tu dodać swój własny wykaz bądź listę,
% np. spis algorytmów.

\end{document} % Dobranoc.
