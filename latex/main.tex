%%%%%%%%%%%%%%%%%%%%%%%%%%%%%%%%%%%%%%%%%%%%%%%%%%%%%%%
%% Bachelor's & Master's Thesis Template             %%
%% Copyleft by Artur M. Brodzki & Piotr Woźniak      %%
%% Faculty of Electronics and Information Technology %%
%% Warsaw University of Technology, 2019-2020        %%
%%%%%%%%%%%%%%%%%%%%%%%%%%%%%%%%%%%%%%%%%%%%%%%%%%%%%%%

\documentclass[
    left=2.5cm,         % Sadly, generic margin parameter
    right=2.5cm,        % doesnt't work, as it is
    top=2.5cm,          % superseded by more specific
    bottom=3cm,         % left...bottom parameters.
    bindingoffset=6mm,  % Optional binding offset.
    nohyphenation=false % You may turn off hyphenation, if don't like.
]{eiti/eiti-thesis}

\langpol % Dla języka angielskiego mamy \langeng
\graphicspath{{img/}}             % Katalog z obrazkami.
\addbibresource{bibliografia.bib} % Plik .bib z bibliografią

% Pakiety do diagramów blokowych
\usepackage{tikz}
\usetikzlibrary{shapes, arrows.meta, positioning, calc, fit, backgrounds}

% Style dla diagramów blokowych
\tikzstyle{block} = [draw, fill=blue!10, rectangle, minimum height=2.5em, minimum width=4em, rounded corners=2pt]
\tikzstyle{bigblock} = [draw, fill=blue!10, rectangle, minimum height=3em, minimum width=6em, rounded corners=2pt]
\tikzstyle{sum} = [draw, fill=white, circle, minimum size=1.5em, inner sep=0pt]
\tikzstyle{input} = [coordinate]
\tikzstyle{output} = [coordinate]
\tikzstyle{arrow} = [-{Stealth[length=2mm]}, thick]
\tikzstyle{line} = [thick]

\begin{document}

%--------------------------------------
% Strona tytułowa
%--------------------------------------
\EngineerThesis % Dla pracy inżynierskiej mamy \EngineerThesis
\instytut{Automatyki i Informatyki Stosowanej}
\kierunek{Automatyka i Robotyka}
% \specjalnosc{XXXXXX}
\title{
    Efektywny układ stabilizacji odwróconego wahadła na wózku
}
\engtitle{ % Tytuł po angielsku do angielskiego streszczenia
    Effective stabilisation system of the inverted pendulum on the cart
}
\author{Adam Sokołowski}
\album{324892}
\promotor{mgr inż. Robert Nebeluk}
\date{\the\year}
\maketitle

%--------------------------------------
% Streszczenie po polsku
%--------------------------------------
\cleardoublepage
\streszczenie
% W pracy przeprowadzono analizę i porównanie wybranych metod sterowania odwróconym wahadłem na wózku — klasycznym przykładem nieliniowego i niestabilnego układu dynamicznego. Celem badań było opracowanie środowiska symulacyjnego umożliwiającego ocenę skuteczności różnych algorytmów regulacji oraz ich odporności na zakłócenia zewnętrzne. W ramach pracy zaimplementowano i przetestowano następujące podejścia: klasyczny regulator kaskadowy PID--PID, układ złożony PID--LQR, regulator LQR dla całego systemu, sterowanie predykcyjne modelowe MPC bez ograniczeń (MPC\_NO) w kilku wariantach funkcji kosztu, oraz regulator rozmyty Takagi--Sugeno.

% Dla każdego z rozważanych regulatorów wykonano symulacje numeryczne w warunkach nominalnych oraz w obecności zakłóceń poziomych działających na wózek. Analizie poddano przebiegi kątów i położeń, a także obliczono wskaźniki jakości regulacji: średni błąd kwadratowy (MSE), średni błąd bezwzględny (MAE), czas regulacji, przeregulowanie, energię sygnału sterującego oraz średni czas obliczeń. Wyniki przedstawiono w postaci wykresów i tabel umożliwiających obiektywne porównanie metod.

% Otrzymane rezultaty wskazują, że włączenie metod optymalnych i predykcyjnych pozwala znacząco poprawić stabilność oraz odporność układu w porównaniu z klasycznym sterowaniem PID. Regulator MPC zapewniał najkrótszy czas stabilizacji przy zachowaniu umiarkowanego kosztu sterowania, natomiast regulator rozmyty Takagi--Sugeno wykazał dobrą adaptacyjność przy zmiennych warunkach pracy. Opracowane środowisko może stanowić podstawę do dalszych badań nad sterowaniem nieliniowym, w tym nad integracją metod rozmytych, optymalnych i predykcyjnych.
Celem pracy było zaprojektowanie autorskiego środowiska symulacyjnego oraz przeprowadzenie 
wielokryterialnej analizy porównawczej algorytmów sterowania dla nieliniowego układu odwróconego
wahadła na wózku. W ramach badań zaimplementowano i przetestowano cztery strategie sterowania: 
klasyczny regulator kaskadowy PD-PD, układ hybrydowy PD-LQR, nieliniowe sterowanie predykcyjne (MPC) 
w dwóch wariantach funkcji kosztu oraz sterownik rozmyty typu Takagi-Sugeno wspomagany regulatorem LQR 
(Fuzzy-LQR). Badania przeprowadzono w środowisku języka Python, wykorzystując pełny model nieliniowy obiektu. 
Skuteczność algorytmów zweryfikowano w dwóch scenariuszach: stabilizacji w punkcie pracy oraz pracy w warunkach 
zakłóceń zewnętrznych (symulacja wiatru). Analiza wyników wykazała, że nie istnieje jeden uniwersalny regulator
dla wszystkich zastosowań. Sterownik Fuzzy-LQR zapewnił najwyższą precyzję stabilizacji i odporność na 
zakłócenia, jednak odbyło się to kosztem bardzo wysokiego zużycia energii sterowania. Z kolei regulator 
MPC okazał się rozwiązaniem najbardziej ekonomicznym, zapewniając płynność ruchu i uwzględnienie ograniczeń
fizycznych napędu, co jest kluczowe w systemach rzeczywistych. Układ PD-LQR stanowił kompromis między 
złożonością obliczeniową a jakością regulacji, przewyższając klasyczne podejście PID.
\slowakluczowe odwrócone wahadło,\allowbreak regulator PID,\allowbreak regulator LQR,\allowbreak regulator MPC,\allowbreak Takagi\allowbreak--\allowbreak Sugeno,\allowbreak sterowanie predykcyjne,\allowbreak sterowanie optymalne,\allowbreak zakłócenia,\allowbreak stabilizacja,\allowbreak symulacja,\allowbreak układ nieliniowy

%--------------------------------------
% Streszczenie po angielsku
%--------------------------------------
\newpage
\abstract
% This thesis presents the analysis and comparison of several control methods for an inverted pendulum on a cart — a classical benchmark of nonlinear and unstable dynamics. The aim of the study was to develop a simulation framework enabling an objective evaluation of different control algorithms and their robustness to external disturbances. The implemented control strategies included a classical cascade PID--PID controller, a combined PID--LQR scheme, a full-state LQR controller, a model predictive controller without constraints (MPC\_NO) with alternative cost functions, and a fuzzy Takagi--Sugeno controller.

% For each controller, numerical simulations were performed under nominal conditions and under horizontal disturbance forces acting on the cart. The responses of the pendulum angle and cart position were analyzed, and several performance indices were calculated: mean squared error (MSE), mean absolute error (MAE), settling time, overshoot, control energy, and average computation time. The results were visualized in plots and summarized in tables, allowing a comprehensive comparison of control quality and computational efficiency.

% The obtained results show that introducing optimal and predictive control significantly improves system stability and robustness compared to classical PID regulation. The MPC controller provided the fastest stabilization with a moderate control effort, while the Takagi--Sugeno fuzzy controller demonstrated good adaptability under varying operating conditions. The developed simulation environment constitutes a solid foundation for further research on nonlinear control, particularly on hybrid combinations of fuzzy, optimal, and predictive methods.
The aim of this thesis was to design a custom simulation environment and 
conduct a multi-criteria comparative analysis of control algorithms for a 
nonlinear inverted pendulum system on a cart. As part of the research, four 
control strategies were implemented and tested: a classical cascade PD-PD 
controller, a hybrid PD-LQR system, nonlinear Model Predictive Control (MPC) 
with two cost function variants, and a Takagi-Sugeno fuzzy controller supported 
by LQR (Fuzzy-LQR). The study was conducted in a Python environment using a full 
nonlinear model of the object. The effectiveness of the algorithms was verified 
in two scenarios: stabilization at the operating point and operation under external 
disturbances (simulated wind). The analysis of the results showed that there is 
no single universal controller for all applications. The Fuzzy-LQR controller 
provided the highest stabilization precision and robustness to disturbances, but 
at the cost of very high control energy consumption. On the other hand, the MPC 
controller proved to be the most economical solution, ensuring smooth motion and 
compliance with physical drive constraints, which is crucial in real-world systems. 
The PD-LQR system represented a compromise between computational complexity and control 
quality, outperforming the classical PID approach.
\keywords inverted pendulum, PID controller, LQR controller, MPC controller, Takagi\allowbreak--\allowbreak Sugeno, predictive control, optimal control, disturbances, stabilization, nonlinear system, simulation

%--------------------------------------
% Oświadczenie o autorstwie
%--------------------------------------
\cleardoublepage  % Zaczynamy od nieparzystej strony
%\pagestyle{plain}
%\makeauthorship

%--------------------------------------
% Spis treści
%--------------------------------------
\cleardoublepage % Zaczynamy od nieparzystej strony
\tableofcontents

%--------------------------------------
% Rozdziały
%--------------------------------------
\cleardoublepage % Zaczynamy od nieparzystej strony
\pagestyle{headings}

%\input{tex/1-wstep}         % Wygodnie jest trzymać każdy rozdział w osobnym pliku.
%\input{tex/2-de-finibus}    % Umożliwia to również łatwą migrację do nowej wersji szablonu:
%\input{tex/3-code-listings} % wystarczy podmienić swoje pliki main.tex i eiti-thesis.cls
                            % na nowe wersje, a cały tekst pracy pozostaje nienaruszony.

% !TEX encoding = utf8
\section{Wstęp}
Odwrócone wahadło na wózku jest klasycznym przykładem nieliniowego, niestabilnego układu, wykorzystywanym zarówno w dydaktyce, jak i w badaniach nad zaawansowanymi technikami sterowania. Jego atrakcyjność wynika z tego, że choć geometria i parametry są stosunkowo proste, to układ ten jest trudny do stabilizacji i wymaga precyzyjnej regulacji w czasie rzeczywistym.

\subsection{Cel pracy}
Celem niniejszego sprawozdania jest przedstawienie projektu symulacyjnego układu stabilizacji odwróconego wahadła na wózku, w którym porównano klasyczne regulatory:
\begin{enumerate}
    \item Regulator PD lub PID do regulacji położenia wózka \(x\) oraz regulator LQR do regulacji kąta wahadła \(\varphi\).
    \item Regulator PID do regulacji całego systemu.
    \item Regulator LQR do regulacji całego systemu.
\end{enumerate}
Dodatkowo zestawiono wskaźniki jakości regulacji: średni błąd kwadratowy (MSE) oraz średni błąd bezwzględny (MAE) dla trajektorii \(\varphi(t)\) i \(x(t)\), a także przeanalizowano wpływ zakłóceń na jakość regulacji.

\subsection{Omówienie literatury}
W literaturze problem sterowania odwróconego wahadła na wózku jest traktowany jako wzorcowy układ niestabilny i nieliniowy, służący do oceny efektywności różnych algorytmów sterowania. W~\cite{art1} autorzy prezentują:
\begin{itemize}
    \item Model nieliniowy systemu z uwzględnieniem zaburzenia od wiatru oraz jego liniaryzację wokół punktu równowagi.
    \item Projekt dwóch pętli PID: jedna dla kąta \(\theta\), druga dla położenia \(x\), oraz algorytm LQR zaprojektowany na zlinearyzowanym modelu.
    \item Analizę porównawczą działania regulatorów PID, PID+LQR (dwie konfiguracje: \emph{2PID+LQR} oraz \emph{1PID+LQR}), zarówno w przypadku bez zakłóceń, jak i z zakłóceniem od wiatru.
\end{itemize}
Wzorując się na ~\cite{art1} zimplementowano regulator PID+LQR.
\newpage 
\section{Model matematyczny układu}
\subsection{Opis fizyczny układu}
Rozważany układ składa się z wózka o masie \(M\), poruszającego się poziomo, oraz przegubowo zamocowanego wahadła o masie \(m\) i długości \(l\), wychylającego się od pionu o kąt \(\varphi(t)\). Na wózek działa sterowanie w postaci siły \(u(t)\). Wahadło jest niestateczne w pozycji pionowej i wymaga stabilizacji aktywnej.

Parametry fizyczne:
\begin{itemize}
    \item \(M\) – masa wózka [kg],
    \item \(m\) – masa wahadła [kg],
    \item \(l\) – długość wahadła [m],
    \item \(I\) – moment bezwładności wahadła względem osi [kg·m²],
    \item \(b\) – współczynnik tłumienia wózka [N·s/m],
    \item \(g\) – przyspieszenie ziemskie [m/s²].
\end{itemize}

\subsection{Model liniowy w przestrzeni stanów}
Model został uzyskany przez uproszczenie dynamiczne i liniaryzację wokół punktu równowagi (\(\varphi = 0\)). Stan układu opisuje wektor:
\[
    \mathbf{x} = \begin{bmatrix}x \\ \dot{x} \\ \varphi \\ \dot{\varphi}\end{bmatrix},
\]
gdzie \(x\) – pozycja wózka, \(\varphi\) – kąt odchylenia wahadła. Model przyjmuje postać:
\[
    \dot{\mathbf{x}}(t) = A\,\mathbf{x}(t) + B\,u(t), 
    \quad
    \mathbf{y}(t) = C\,\mathbf{x}(t),
\]
przy czym:
\[
    p = I\,(M + m) + M m l^2,
\]
\[
    A =
    \begin{bmatrix}
        0 & 1 & 0 & 0 \\
        0 & -\frac{(I + m l^2)\,b}{p} & \frac{m^2 g l^2}{p} & 0 \\
        0 & 0 & 0 & 1 \\
        0 & -\frac{m l\,b}{p} & \frac{m g l\,(M + m)}{p} & 0
    \end{bmatrix}, 
    \quad
    B =
    \begin{bmatrix}
        0 \\[6pt]
        \frac{I + m l^2}{p} \\[6pt]
        0 \\[6pt]
        \frac{m l}{p}
    \end{bmatrix},
\]
\[
    C = 
    \begin{bmatrix}
        1 & 0 & 0 & 0 \\ 
        0 & 0 & 1 & 0
    \end{bmatrix}, 
    \quad
    D = \begin{bmatrix}0 \\ 0\end{bmatrix}.
\]

Model został zaimplementowany w MATLAB-ie jako funkcja \texttt{state\_space\_model.m}, która zwraca obiekt przestrzeni stanów \texttt{ss(A, B, C, D)}.

\newpage
\section{Środowisko symulacyjne i implementacja}

W celu przeprowadzenia badań i weryfikacji działania algorytmów sterowania, przygotowano autorskie środowisko symulacyjne zrealizowane w języku \textbf{Python 3}. Wybór tego języka podyktowany był jego powszechnością w zastosowaniach naukowych, dostępnością bibliotek do obliczeń numerycznych i optymalizacji, a także łatwością prototypowania złożonych struktur sterowania.

\subsection{Narzędzia programistyczne}

W projekcie wykorzystano następujące biblioteki i narzędzia:
\begin{itemize}
    \item \textbf{NumPy} -- podstawowa biblioteka do obliczeń macierzowych i operacji na wielowymiarowych tablicach danych, wykorzystywana do implementacji równań stanu oraz przechowywania przebiegów symulacji.
    \item \textbf{SciPy} -- pakiet naukowy dostarczający zaawansowanych algorytmów numerycznych. W pracy użyto modułów:
    \begin{itemize}
        \item \texttt{scipy.linalg} -- do rozwiązywania algebraicznego równania Riccatiego (ARE) w algorytmie LQR.
        \item \texttt{scipy.optimize} -- zawierającego solwer \texttt{minimize} (metoda SLSQP), wykorzystywany do rozwiązywania zadań optymalizacji nieliniowej z ograniczeniami w regulatorze MPC.
    \end{itemize}
    \item \textbf{Matplotlib} -- biblioteka służąca do wizualizacji wyników w postaci wykresów przebiegów czasowych oraz do generowania animacji ruchu wahadła.
\end{itemize}

\subsection{Konfiguracja symulacji}

Symulator opiera się na numerycznym całkowaniu wyprowadzonych wcześniej nieliniowych równań dynamiki. Zaimplementowano procedurę całkowania metodą \textbf{Rungego-Kutty czwartego rzędu (RK4)}, co zapewnia wysoki kompromis pomiędzy dokładnością a szybkością obliczeń. Kluczowy fragment implementacji algorytmu przedstawiono na Listingu \ref{lst:rk4}.

\begin{lstlisting}[language=Python, caption={Implementacja metody Rungego-Kutty 4. rzędu}, label={lst:rk4}, basicstyle=\ttfamily\footnotesize, breaklines=true]
def rk4_step(f, x, u, pars, dt):
    k1 = f(x, u, pars)
    k2 = f(x + 0.5 * dt * k1, u, pars)
    k3 = f(x + 0.5 * dt * k2, u, pars)
    k4 = f(x + dt * k3, u, pars)
    return x + (dt / 6.0) * (k1 + 2*k2 + 2*k3 + k4)
\end{lstlisting}

Przyjęto stały krok symulacji oraz sterowania wynoszący \(\Delta t = 0{,}1\,\text{s}\). Jest to wartość, przy której dynamika wahadła jest odwzorowana z wystarczającą precyzją, a jednocześnie pozwala na efektywne działanie numerycznych algorytmów optymalizacji w czasie rzeczywistym.

\begin{table}[H]
	\centering
	\caption{Parametry fizyczne modelu przyjęte w symulacji}
	\label{tab:parametry_modelu}
	\renewcommand{\arraystretch}{1.2}
	\begin{tabular}{|l|c|c|c|}
		\hline
		\textbf{Parametr} & \textbf{Symbol} & \textbf{Wartość} & \textbf{Jednostka} \\ \hline
		Masa wózka & $M$ & $2{,}4$ & $\text{kg}$ \\ \hline
		Masa wahadła & $m$ & $0{,}23$ & $\text{kg}$ \\ \hline
		Długość wahadła & $l$ & $0{,}36$ & $\text{m}$ \\ \hline
		Przyspieszenie ziemskie & $g$ & $9{,}81$ & $\text{m/s}^2$ \\ \hline
        Ograniczenie sterowania & $u_{max}$ & $100{,}0$ & $\text{N}$ \\ \hline
	\end{tabular}
\end{table}

Symulacje przeprowadzane są dla zadania stabilizacji układu w pionie (tzw. punkt pracy), startując z niezerowych warunków początkowych lub wymuszając zmianę pozycji wózka.

\textbf{Warunki początkowe (domyślne):}
\[
\mathbf{x}_0 = [\theta, \dot{\theta}, x, \dot{x}]^T = [0{,}05\,\text{rad}, 0, 0, 0]^T
\]
Oznacza to niewielkie (ok. $2{,}86^\circ$) początkowe wychylenie wahadła, które regulator musi zniwelować.

\textbf{Wartości zadane:}
Celem układu jest osiągnięcie stanu $\mathbf{x}_{ref} = [0, 0, x_{ref}, 0]^T$, gdzie $x_{ref}$ (np. $0{,}1$ m) jest zadaną nową pozycją wózka, przy jednoczesnym utrzymaniu pionowej pozycji wahadła ($\theta = 0$).

\subsection{Modelowanie zakłóceń}

Aby zweryfikować odporność układów sterowania, zaimplementowano generator zakłóceń symulujący podmuchy wiatru działające na wahadło. Model zakłócenia $F_w(t)$ oparty jest na procesie stochastycznym:
\begin{enumerate}
    \item Generowany jest biały szum gaussowski o zadanej mocy.
    \item Sygnał jest wygładzany filtrem uśredniającym (splot z oknem prostokątnym), co pozwala uzyskać bardziej realistyczne, ciągłe w czasie przebiegi siły wiatru, zamiast nieskorelowanego szumu.
\end{enumerate}
W eksperymentach z zakłóceniami, siła $F_w$ jest dodawana bezpośrednio do równań dynamiki w każdym kroku całkowania. Przykładowy przebieg wygenerowanego sygnału zakłócającego przedstawiono na Rys. \ref{fig:wind_signal}, a sposób jego generacji w kodzie źródłowym na Listingu \ref{lst:wind}.

\begin{lstlisting}[language=Python, caption={Klasa generatora zakłóceń wiatru}, label={lst:wind}, basicstyle=\ttfamily\footnotesize, breaklines=true]
class Wind:
    def __init__(self, t_end: float, seed=23341, Ts=0.1, power=1e-3, smooth=5):
        rng = np.random.default_rng(seed)
        self.tgrid = np.arange(0.0, t_end + Ts, Ts)
        sigma = np.sqrt(power / Ts)
        w = rng.normal(0.0, sigma, size=self.tgrid.shape)
        if smooth and smooth > 1:
            kernel = np.ones(smooth) / smooth
            self.Fw = np.convolve(w, kernel, mode='same')
        else:
            self.Fw = w

    def __call__(self, t: float) -> float:
        return float(np.interp(t, self.tgrid, self.Fw))
\end{lstlisting}

\begin{figure}[H]
    \centering
    \includegraphics[width=0.8\textwidth]{img/wind_signal.png}
    \caption{Przykładowa realizacja stochastycznego procesu zakłócenia (wiatru) działającego na wahadło w czasie symulacji.}
    \label{fig:wind_signal}
\end{figure}

\subsection{Wizualizacja i animacja}

Oprócz standardowych wykresów zmiennych stanu i sterowania, środowisko wyposażono w moduł wizualizacji dynamicznej (Rys. \ref{fig:animacja_screenshot}). Implementacja animacji oparta jest na bibliotece \texttt{Matplotlib} i klasie \texttt{FuncAnimation}, która pozwala na cykliczne odświeżanie obiektów graficznych zgodnie z taktowaniem symulacji.

Graficzna reprezentacja obiektu (robot) zbudowana jest z prostych prymitywów geometrycznych:
\begin{itemize}
    \item \textbf{Wózek}: obiekt typu \texttt{Rectangle}, którego pozycja pozioma aktualizowana jest w każdej klatce na podstawie zmiennej stanu $x(t)$.
    \item \textbf{Koła}: obiekty \texttt{Circle}, poruszające się wraz z wózkiem.
    \item \textbf{Wahadło}: obiekt liniowy, którego współrzędne końcowe wyznaczane są trigonometrycznie na podstawie kąta $\theta(t)$.
\end{itemize}

Kluczowym elementem implementacji jest funkcja aktualizująca \texttt{update}, wywoływana dla każdego kroku czasowego. Odpowiada ona za przeliczenie współrzędnych kinematycznych (Listing \ref{lst:animation_logic}) oraz przesunięcie okna widoku kamery tak, aby wózek znajdował się zawsze w centrum, co pozwala na obserwację ruchu na długim dystansie. Dodatkowo rysowany jest „ślad” (ang. \textit{trail}) przebytej drogi przez oś wózka, co ułatwia wizualną ocenę stabilności pozycji.

\begin{lstlisting}[language=Python, caption={Logika aktualizacji klatki animacji}, label={lst:animation_logic}, basicstyle=\ttfamily\footnotesize, breaklines=true]
    def pole_end(i):
        cx = x[i]; cy = wheel_r + cart_h
        # Kinematyka prosta wahadla
        px = cx + pole_len * np.sin(th[i]) 
        py = cy + pole_len * np.cos(th[i])
        return cx, cy, px, py

    def update(i):
        cx, cy, px, py = pole_end(i)
        # Aktualizacja pozycji obiektow graficznych
        cart.set_x(cx - cart_w/2)
        wheel1.center = (cx - cart_w/3, wheel_r)
        pole_line.set_data([cx, px], [cy, py])
        # Centrowanie kamery na wozku
        ax.set_xlim(cx - pad, cx + pad)
        return cart, wheel1, pole_line
\end{lstlisting}

Wykorzystanie animacji pozwala na szybką, intuicyjną weryfikację poprawności modelu fizycznego oraz ocenę jakości regulacji w sposób trudny do uchwycenia na statycznych wykresach (np. nienaturalne drgania czy gwałtowne reakcje „szarpnięcia”).

\begin{figure}[H]
    \centering
    \includegraphics[width=0.7\textwidth]{img/animation.png} 
    \caption{Zrzut ekranu z animacji realizowanej w środowisku Python (biblioteka Matplotlib). Widoczny wózek, wahadło oraz zakres ruchu.}
    \label{fig:animacja_screenshot}
\end{figure}

\newpage
\section{Algorytmy sterowania}

W~niniejszym rozdziale przedstawiono szczegółowy opis algorytmów sterowania
zaimplementowanych i~przeanalizowanych w~ramach pracy. Kod sterowników został
zrealizowany w~języku Python w~postaci klas dziedziczących wspólną strukturę,
co zapewnia modularność i~łatwą wymienność w~pętli symulacyjnej. Każdy
regulator wyznacza sygnał sterujący $u(t)$ (siłę przyłożoną do wózka)
na~podstawie aktualnego wektora stanu
$x(t) = [\theta, \dot{\theta}, x, \dot{x}]^T$ oraz wartości zadanych
$x_{\mathrm{ref}}$.

W~literaturze problem sterowania wahadłem odwróconym jest szeroko omawiany
jako klasyczny problem testowy dla metod sterowania liniowego i~nieliniowego
\cite{Prasad2014, Nguyen2024}. Poniżej opisano teoretyczne podstawy oraz szczegóły
implementacyjne zbadanych struktur sterowania.

\subsection{Równoległy regulator PD-PD}

Pierwszym zaimplementowanym układem jest regulator o~strukturze kaskadowej lub
równoległej, wykorzystujący klasyczne sprzężenie zwrotne typu PD
(Proporcjonalno-Różniczkujące). W~literaturze podejście to jest często
stosowane jako punkt odniesienia dla bardziej zaawansowanych metod
\cite{Moreno2023, Prasad2014}.

W~klasie \texttt{PDPDController} zastosowano strukturę równoległą, w~której
całkowity sygnał sterujący jest sumą reakcji na~błąd kąta oraz błąd pozycji.
Jest to podejście intuicyjne, dekomponujące problem na~dwa podzadania:
stabilizację wahadła w~pozycji pionowej oraz doprowadzenie wózka do~zadanej
pozycji.

Prawo sterowania wyraża się wzorem:
\begin{equation}
    u(t) = \mathrm{sat}_{u_{\mathrm{max}}} \left( u_{\theta}(t) + u_{x}(t) \right),
\end{equation}
gdzie funkcja nasycenia $\mathrm{sat}(\cdot)$ wynika z~ograniczeń fizycznych
Definiując uchyby regulacji jako $e_{\theta}(t) = \theta_{\mathrm{ref}} - \theta(t)$ oraz $e_x(t) = x_{\mathrm{ref}} - x(t)$, prawo sterowania dla poszczególnych pętli można zapisać w~ogólnej postaci regulatora PID:
\begin{align}
    u_{\theta}(t) &= K_{p,\theta} e_{\theta}(t) + K_{i,\theta} \int_0^t e_{\theta}(\tau)\,d\tau + K_{d,\theta} \frac{d e_{\theta}(t)}{dt}, \label{eq:pd_theta} \\
    u_{x}(t) &= K_{p,x} e_x(t) + K_{i,x} \int_0^t e_x(\tau)\,d\tau + K_{d,x} \frac{d e_x(t)}{dt}. \label{eq:pd_x}
\end{align}
W~powyższych równaniach przyjęto upraszczające założenie, że docelowe
prędkości ($\dot{\theta}_{\mathrm{ref}}, \dot{x}_{\mathrm{ref}}$) wynoszą zero.

W~implementacji programowej (plik \texttt{pd\_pd.py}) przyjęto następujące
nastawy dobrane eksperymentalnie:
\begin{itemize}
    \item Tor stabilizacji kąta: $K_{p,\theta} = -95.0$, $K_{d,\theta} = -14.0$.
    Ujemne znaki wynikają z~przyjętej konwencji układu współrzędnych i~zwrotu siły.
    \item Tor pozycji: $K_{p,x} = -16.0$, $K_{d,x} = -14.0$.
\end{itemize}
Mimo iż klasa umożliwia włączenie członu całkującego (PID), w~badaniach
\cite{Varghese2017} często wskazuje się, że dla obiektów tej klasy człon różniczkujący
(PD) jest kluczowy dla tłumienia oscylacji, a~całkowanie może wprowadzać
niestabilność w~stanach nieustalonych bez odpowiednich mechanizmów
$\mathrm{anti\text{-}windup}$.

\subsubsection{Proces doboru nastaw oraz analiza PID}
Dobór nastaw dla regulatora PD-PD został zrealizowany wieloetapowo, ewoluując
od metod heurystycznych do pełnej optymalizacji numerycznej. Wstępne próby
doboru metodą ,,prób i~błędów'', oparte na~dekompozycji problemu (najpierw
stabilizacja wahadła, potem pozycja wózka), pozwoliły uzyskać stabilność,
jednak jakość regulacji była niezadowalająca. Układ charakteryzował się
powolnym dochodzeniem do~punktu pracy i~znacznymi oscylacjami
(\ref{fig:pdpd_manual}).

Aby wyeliminować subiektywność strojenia ręcznego, zastosowano algorytm
Ewolucji Różnicowej (Differential Evolution), zaimplementowany w~module
\texttt{scipy.optimize}. Zdefiniowano globalną funkcję kosztu $J$, która
gwarantuje ,,uczciwe'' porównanie wszystkich badanych regulatorów:
\begin{equation}
    J = w_{\theta} \cdot \text{MSE}(\theta) + w_{x} \cdot \text{MSE}(x) + w_{u} \cdot \text{RMS}(u),
\end{equation}
gdzie przyjęto wagi $w_{\theta}=4.0$ (priorytet stabilizacji), $w_{x}=1.0$
(dokładność pozycjonowania) oraz $w_{u}=0.01$ (koszt energii). Algorytm operował
na populacji 10 osobników przez 20 generacji, co pozwoliło uniknąć minimów
lokalnych i~znaleźć optymalny zestaw wzmocnień (\ref{fig:pdpd_opt}).

\paragraph{Analiza porównawcza struktur PD i PID}
W~literaturze przedmiotu \cite{Varghese2017, Nguyen2024} często podkreśla się, że dla
obiektów niestabilnych statycznie, takich jak wahadło odwrócone, kluczowa jest
szybka reakcja na~zmiany kąta, którą zapewnia człon różniczkujący (D).
Włączenie członu całkującego (I), tworzącego regulator PID, wprowadza dodatkowe
przesunięcie fazowe (opóźnienie), co w~układzie o~dynamice astatycznej
i~nieliniowej prowadzi do znacznego pogorszenia zapasu stabilności.

Przeprowadzone badania symulacyjne potwierdziły te wnioski. Próba dodania akcji
całkującej ($K_i \neq 0$) skutkowała zjawiskiem \textit{integral wind-up} --
akumulacją błędu w~fazie rozruchu, co powodowało przeregulowania wykraczające
poza obszar przyciągania stabilnego punktu równowagi. Jak widać na~rysunku
\ref{fig:pid_bad}, regulator PID wpada w~niegasnące oscylacje lub doprowadza
do przewrócenia wahadła, podczas gdy ,,czysty'' regulator PD zapewnia sztywne
i~szybkie sterowanie.

\begin{figure}[H]
    \centering
    \includegraphics[width=1.0\textwidth]{images/tuning/pid_1_integral_bad.png}
    \caption{Destabilizujący wpływ członu całkującego (PID) - widoczne narastające oscylacje i utrata stabilności.}
    \label{fig:pid_bad}
\end{figure}

\begin{figure}[H]
    \centering
    \includegraphics[width=1.0\textwidth]{images/tuning/pdpd_2_manual.png}
    \caption{Stabilna, lecz oscylacyjna praca regulatora PD-PD przy strojeniu ręcznym.}
    \label{fig:pdpd_manual}
\end{figure}

\begin{figure}[H]
    \centering
    \includegraphics[width=1.0\textwidth]{images/tuning/pdpd_3_opt.png}
    \caption{Przebiegi czasowe dla zoptymalizowanych nastaw regulatora PD-PD (algorytm Differential Evolution).}
    \label{fig:pdpd_opt}
\end{figure}

\subsection{Układ hybrydowy PD-LQR}

Kolejnym analizowanym algorytmem jest regulator liniowo-kwadratowy (LQR),
będący standardem w~sterowaniu optymalnym systemów wielowymiarowych MIMO
\cite{Jezierski2017}. Klasa \texttt{PDLQRController} implementuje sterowanie
oparte na~pełnym wektorze stanu, wspomagane dodatkowym członem PD dla uchybu
pozycji, co tworzy strukturę hybrydową opisaną m.in. w~\cite{Prasad2014} oraz
\cite{Nguyen2024} (w~kontekście porównawczym).

Problem LQR polega na~znalezieniu prawa sterowania
$u(t) = -K x(t)$, które minimalizuje wskaźnik jakości:
\begin{equation}
    J = \int_{0}^{\infty} \left( x(t)^T Q x(t) + u(t)^T R u(t) \right) dt,
\end{equation}
gdzie $Q \succeq 0$ jest macierzą wag stanu, a~$R > 0$ wagą
sterowania. Optymalna macierz wzmocnień $K$ wyznaczana jest poprzez
rozwiązanie algebraicznego równania Riccatiego (CARE):
\begin{equation}
    A^T P + P A - P B R^{-1} B^T P + Q = 0,
\end{equation}
skąd $K = R^{-1} B^T P$. Macierze $A$
i~$B$ pochodzą z~linearyzacji modelu wahadła wokół punktu równowagi
górnej ($\theta = 0$).

W~zaimplementowanym rozwiązaniu (plik \texttt{pd\_lqr.py}), sygnał sterujący
składa się z~dwóch komponentów:
\begin{equation}
    u(t) = u_{\mathrm{LQR}}(t) + u_{\mathrm{PD,pos}}(t).
\end{equation}
Składnik LQR realizuje stabilizację wokół punktu pracy:
\begin{equation}
    u_{\mathrm{LQR}}(t) = -K \cdot (x(t) - x_{\mathrm{ref}}).
\end{equation}
Zastosowane wagi optymalne to:
\begin{equation}
    Q = \text{diag}([1.0,\; 1.0,\; 500.0,\; 250.0]), \quad R = 1.0.
\end{equation}
Dodatkowy człon PD na~pętli pozycji (zrealizowany analogicznie do wzoru
\ref{eq:pd_x}) ma na~celu poprawę śledzenia skokowych zmian wartości zadanej
$x_{\mathrm{ref}}$, co jest częstą praktyką w~aplikacjach praktycznych, gdzie LQR
zapewnia stabilność, a~regulator zewnętrzny dba o~uchyb w~stanie ustalonym
\cite{Varghese2017}.

\subsubsection{Dobór wag macierzy Q i R}
Dobór wag dla regulatora LQR również charakteryzował się ewolucyjnym podejściem
do~problemu optymalizacji wskaźnika jakości.

W~pierwszej fazie przyjęcie jednostkowej macierzy diagonalnej $Q=I$ oraz $R=1$
okazało się niewystarczające. Mimo teoretycznej stabilności wynikającej z~rozwiązania
równania CARE, wahadło wykonywało bardzo duże wychylenia, a~wózek wielokrotnie
wyjeżdżał poza dopuszczalny zakres roboczy toru. Świadczyło to o~zbyt małej karze
nałożonej na~uchyb kątowy.

\begin{figure}[H]
    \centering
    \includegraphics[width=1.0\textwidth]{images/tuning/pdlqr_1_bad.png}
    \caption{Próba sterowania LQR z wagami jednostkowymi ($Q=I, R=1$). Widoczna duża bezwładność układu.}
    \label{fig:lqr_bad}
\end{figure}

Następnie przeprowadzono strojenie ręczne metodą prób i~błędów (zgodnie z~regułą
Brysona). Ręczne zwiększanie kar za~wychylenie kąta ($Q_{\theta}$) poprawiło
sztywność wahadła. Udało się ustalić zestaw wag zapewniający stabilną pracę, choć
czas regulacji był wciąż niezadowalający, a~reakcja na~zakłócenia powolna.

\begin{figure}[H]
    \centering
    \includegraphics[width=1.0\textwidth]{images/tuning/pdlqr_2_manual.png}
    \caption{Wyniki strojenia ręcznego LQR metodą Brysona.}
    \label{fig:lqr_manual}
\end{figure}

W~ostatnim etapie zastosowano optymalizację numeryczną. Algorytm genetyczny
poszukiwał optymalnych elementów diagonalnych macierzy $Q$ oraz skalara $R$,
    minimalizując czas regulacji. Zoptymalizowane wagi (w szczególności wysoka kara
    $Q_{x} = 500$) sprawiają, że regulator bardzo agresywnie pilnuje pozycji wózka,
    co pośrednio wymusza stabilne trzymanie wahadła (wymagane do kontroli pozycji).

\begin{figure}[H]
    \centering
    \includegraphics[width=1.0\textwidth]{images/tuning/pdlqr_3_opt.png}
    \caption{Optymalne przebiegi regulatora PD-LQR po zastosowaniu algorytmu genetycznego.}
    \label{fig:lqr_opt}
\end{figure}

\subsection{Nieliniowe sterowanie predykcyjne (MPC)}

Algorytm MPC (Model Predictive Control) stanowi zaawansowaną metodę sterowania,
która w~odróżnieniu od~LQR, uwzględnia wprost ograniczenia sygnału sterującego
oraz nieliniową dynamikę obiektu \cite{Camacho2007, Rawlings2017}.
Zaimplementowany w~klasie \texttt{MPCController} (plik \texttt{mpc.py})
algorytm rozwiązuje w~każdym kroku symulacji problem optymalizacji dynamicznej
nieliniowej (NMPC).

Zadanie optymalizacji, rozwiązywane numerycznie metodą SQP (Sequential
Quadratic Programming) przy użyciu solwera \texttt{SLSQP}, zdefiniowane jest
następująco:
\begin{equation}
    \min_{\Delta U} J = \sum_{k=1}^{N_{\mathrm{p}}} (\hat{x}_k - x_{\mathrm{ref}})^T Q (\hat{x}_k - x_{\mathrm{ref}}) + R \sum_{k=0}^{N_{\mathrm{c}}-1} (\Delta u_k)^2,
\end{equation}
przy ograniczeniach:
\begin{align}
    \hat{x}_{k+1} &= f(\hat{x}_k, u_k), \quad k=0,\dots,N_{\mathrm{p}}-1 \\
    u_{\mathrm{min}} &\le u_k \le u_{\mathrm{max}}, \\
    u_k &= u_{k-1} + \Delta u_k.
\end{align}
Gdzie:
\begin{itemize}
    \item $N_{\mathrm{p}} = 12$ -- horyzont predykcji,
    \item $N_{\mathrm{c}} = 4$ -- horyzont sterowania (blokowanie sterowania dla $k \ge N_{\mathrm{c}}$),
    \item $f(\cdot)$ -- nieliniowy model dyskretny obiektu (całkowanie metodą
    Rungego-Kutt 4. rzędu),
    \item $Q = \text{diag}([158.4,\; 36.8,\; 43.4,\; 19.7])$ -- macierz kar stanu,
    \item $R = 0.086$ -- współczynnik kary za~zmianę sterowania ($\Delta u$).
\end{itemize}
Kluczową zaletę MPC, podkreślaną w~pracach \cite{Mills2009} oraz
\cite{Jezierski2017}, jest możliwość bezpośredniego uwzględnienia ograniczeń
(saturacji) już na~etapie wyliczania sterowania, co zapobiega zjawisku
nasycenia elementu wykonawczego, które mogłoby mieć miejsce w~przypadku LQR.

Analiza wykazała, że bezpośrednie przeniesienie macierzy wag $Q$ i $R$
z~regulatora LQR do~sterownika MPC prowadziło do~znaczącego pogorszenia jakości
sterowania (wydłużenie czasu regulacji z~ok. 3s do~ponad 9s). Wynika to z~faktu,
że model MPC, dzięki jawnemu uwzględnieniu ograniczeń sygnału sterującego, pozwala
na~zastosowanie znacznie bardziej agresywnych nastaw (większych kar za~błędy stanu),
które w~liniowym regulatorze LQR powodowałyby nasycenie i~potencjalną niestabilność.
Dlatego zdecydowano się na~niezależną optymalizację parametrów obu regulatorów,
aby porównywać ich najlepsze możliwe konfiguracje, a~nie identyczne, ale
nieoptymalne nastawy.

\subsubsection{Dobór horyzontu i wag funkcji celu}
Dla regulatora MPC kluczowym wyzwaniem był dobór horyzontu predykcji oraz
macierzy wag, determinujących zachowanie układu w~stanie nieustalonym.

Początkowe ustawienie zbyt krótkiego horyzontu predykcji ($N_{\mathrm{p}} < 5$)
prowadziło do~niestabilności układu zamkniętego. Regulator ,,nie widział'', że
rozpędzając wózek w~celu korekcji kąta, nie zdąży wyhamować przed upadkiem
wahadła lub osiągnięciem końca toru. Zwiększenie horyzontu do $N_{\mathrm{p}}=10$
w~ramach korekty ręcznej ustabilizowało proces. Dodatkowa manipulacja wagami $Q$
pozwoliła na~uzyskanie poprawnego sterowania, jednak koszt obliczeniowy był
wysoki, a~przebiegi wciąż wykazywały niepożądane przeregulowania.

Automatyzacja procesu strojenia przy użyciu skryptu \texttt{tune\_mpc.py} pozwoliła
na~znalezienie kompromisu między długością horyzontu a~wagami. Algorytm
optymalizacyjny wskazał $N_{\mathrm{p}}=12$ jako optimum dla tego modelu
dyskretnego, zapewniając stabilność przy akceptowalnym czasie obliczeń.

\begin{figure}[H]
    \centering
    \includegraphics[width=1.0\textwidth]{images/tuning/mpc_3_opt.png}
    \caption{Zoptymalizowany regulator MPC ($N_p=12$) - szybka i gładka stabilizacja.}
    \label{fig:mpc_opt}
\end{figure}

\subsection{MPC z~rozszerzonym wskaźnikiem jakości (MPC-J2)}

Zaimplementowano sterownik \texttt{MPCControllerJ2} jako wariant badawczy algorytmu predykcyjnego
(plik \texttt{mpc\_J2.py}). Jego struktura jest
zbliżona do~podstawowego MPC, jednak funkcja kosztu została rozbudowana
o~dodatkowy składnik karzący bezwzględną wartość sygnału sterującego
(energię), a~nie tylko jego przyrosty.

Zmodyfikowana funkcja celu przyjmuje postać:
\begin{equation}
    J = \sum_{k=1}^{N_{\mathrm{p}}} (x_k - x_{\mathrm{ref}})^T Q (x_k - x_{\mathrm{ref}}) \;+\; R_{\Delta} \sum_{k=0}^{N_{\mathrm{c}}-1} (\Delta u_k)^2 \;+\; R_{\mathrm{abs}} \sum_{k=0}^{N_{\mathrm{c}}-1} (u_k)^2.
\end{equation}
Wprowadzenie parametru $R_{\mathrm{abs}}$ pozwala na~bezpośrednie minimalizowanie
zużycia energii sterowania, co jest podejściem powszechnie stosowanym
w~praktycznych implementacjach algorytmów predykcyjnych \cite{Camacho2007, Rawlings2017}.
Ograniczenie amplitudy sygnału sterującego nie tylko redukuje wydatek energetyczny
(istotny w~aplikacjach mobilnych), ale także zmniejsza obciążenie mechaniczne
elementów wykonawczych, co wpływa na~żywotność napędu.

W~badaniach przyjęto wagi: $q_{\theta}=80.0$,
$q_x=120.0$ (elementy macierzy diagonalnej $Q$), kładąc większy
nacisk na~precyzję pozycjonowania wózka w~porównaniu do~standardowego MPC.

\subsubsection{Analiza wpływu kary za energię}
W~przypadku wariantu MPC-J2 analizowano nieliniowy wpływ parametru $R_{\mathrm{abs}}$
na~zachowanie układu.

Przyjęcie zbyt dużej wartości kary za~sterowanie bezwzględne ($R_{\mathrm{abs}}$)
spowodowało, że regulator wykazywał tendencję do~pasywności. Wahadło przewracało się,
ponieważ koszt energetyczny utrzymania go w~pionie przewyższał zysk wynikający
z~małego błędu kąta w~funkcji celu.

Stopniowe, ręczne zmniejszanie parametru $R_{\mathrm{abs}}$ pozwoliło znaleźć punkt
pracy, w~którym układ odzyskał stabilność przy zachowaniu relatywnej oszczędności
energetycznej. Odpowiedź dynamiczna była jednak powolna i~zbyt asekuracyjna dla
większych zakłóceń.

Algorytm optymalizacyjny precyzyjnie dostroił $R_{\mathrm{abs}}$, minimalizując złożony
wskaźnik kosztu (błąd + energia). Znaleziono ,,złoty środek'', w~którym układ
stabilizuje się szybko, ale sterowanie pozbawione jest zbędnych oscylacji
wysokoczęstotliwościowych, co przekłada się na~oszczędność energii.

\begin{figure}[H]
    \centering
    \includegraphics[width=1.0\textwidth]{images/tuning/mpcJ2_3_opt.png}
    \caption{Optymalny kompromis między jakością regulacji a energią w MPC-J2.}
    \label{fig:mpcj2_opt}
\end{figure}

\subsection{Regulator rozmyty wspomagany LQR (Fuzzy-LQR)}

Ostatnim zbadanym układem jest sterownik hybrydowy \texttt{TSFuzzyController}
(plik \texttt{fuzzy\_lqr.py}), łączący liniowy regulator LQR z~systemem
wnioskowania rozmytego typu Takagi-Sugeno (T-S). Koncepcja ta, opisana szerzej
w~\cite{Nguyen2024} oraz \cite{Roose2017}, ma na~celu adaptację wzmocnień regulatora
w~zależności od~punktu pracy, co pozwala na~agresywniejszą reakcję w~przypadku
dużych odchyleń od~pionu.

Sygnał sterujący jest sumą:
\begin{equation}
    u(t) = u_{\mathrm{LQR}}(t) + u_{\mathrm{Fuzzy}}(t).
\end{equation}
Część rozmyta $u_{\mathrm{Fuzzy}}(t)$ wykorzystuje bazę reguł postaci:
\begin{quote}
    JEŚLI $e_\theta$ jest $A_i$ ORAZ $\dot{\theta}$ jest $B_i$ ... TO $u_i = f_i(x)$,
\end{quote}
gdzie $f_i(x)$ jest liniową funkcją stanu (lokalny regulator
liniowy). Zastosowano funkcje przynależności trójkątne dla zmiennych stanu,
dzieląc przestrzeń na~obszary ,,Mały błąd'' i~,,Duży błąd''.
Baza wiedzy składa się z~16 reguł ($2^4$ kombinacji dla 4 zmiennych stanu).
Wyjście sterownika obliczane jest jako średnia ważona:
\begin{equation}
    u_{\mathrm{Fuzzy}} = G \cdot \frac{\sum_{i=1}^{16} w_i(x) \cdot u_i}{\sum_{i=1}^{16} w_i(x)},
\end{equation}
gdzie $w_i$ to stopień aktywacji $i$-tej reguły, a~$G = 0.9$ to globalne
wzmocnienie skalujące.

Zastosowany mechanizm ,,Gain Scheduling'' pozwala na:
\begin{enumerate}
    \item Zachowanie łagodnej charakterystyki LQR w~pobliżu punktu równowagi
    (małe wzmocnienia w~regułach dla ,,Małych błędów'').
    \item Zwiększenie sztywności układu w~sytuacjach krytycznych (duże
    wzmocnienia zdefiniowane w~zmiennej \texttt{F\_rules} dla ,,Dużych
    błędów'').
\end{enumerate} 
Takie podejście pozwala na~rozszerzenie obszaru stabilności regulatora
w~porównaniu do~klasycznego LQR, co potwierdzają wyniki badań w~pracy
\cite{Nguyen2024}.

\subsubsection{Dobór reguł i funkcji przynależności}
Strojenie rozmytego regulatora Fuzzy-LQR jest zadaniem złożonym ze względu na~dużą
liczbę parametrów definiujących bazę reguł i~funkcje przynależności.

Błędne zdefiniowanie zbyt wąskich funkcji przynależności dla strefy ,,małego błędu''
skutkowało gwałtownym przełączaniem się regulatora na~agresywne reguły (chatter).
Prowadziło to do~silnych drgań wokół punktu równowagi, co jest zjawiskiem
niepożądanym w~rzeczywistych układach napędowych.
Opierając się na~literaturze \cite{Nguyen2024}, dobrano ręcznie szerokości trójkątnych
funkcji przynależności tak, aby przejście między strefami było płynne. Układ uzyskał
stabilność asymptotyczną, jednak nie wykorzystywał w~pełni potencjału szybkiej
reakcji na~duże zakłócenia, działając zachowawczo.

Ostatecznie, dedykowany skrypt \texttt{tune\_fuzzy\_lqr.py} posłużył do~optymalizacji
wag pojedynczych reguł oraz parametrów kształtu funkcji przynależności. Uzyskano
nieliniową powierzchnię sterowania, która łączy zalety miękkiego sterowania LQR
w~pobliżu zera z~maksymalną siłą reakcji przy dużych wychyleniach.

\begin{figure}[H]
    \centering
    \includegraphics[width=1.0\textwidth]{images/tuning/fuzzy_3_opt.png}
    \caption{Efektywne sterowanie Fuzzy-LQR po optymalizacji bazy reguł.}
    \label{fig:fuzzy_opt}
\end{figure}

\newpage
\section{Eksperymenty}

Rozdział ten definiuje scenariusze testowe, przyjęte miary oceny jakości
sterowania oraz procedurę doboru nastaw regulatorów. Precyzyjne określenie
warunków eksperymentu jest kluczowe dla zapewnienia powtarzalności badań oraz
obiektywnego porównania testowanych algorytmów.

\subsection{Plan eksperymentów}

W~celu weryfikacji skuteczności strategii sterowania, przyjęto jednolity zestaw
testów symulacyjnych. Każdy z~zaimplementowanych regulatorów (PD, PID-LQR, MPC, Fuzzy-LQR) poddany został badaniom w~następujących scenariuszach:

\begin{enumerate}
    \item \textbf{Eksperyment 1: Stabilizacja w~punkcie pracy (warunki nominalne).} \\
    Symulacja odpowiedzi układu na~niezerowe warunki początkowe przy braku
    zakłóceń zewnętrznych.
    \begin{itemize}
        \item Początkowy kąt wychylenia wahadła: $\theta(0) = 0{,}05$ rad ($\approx 2{,}87^\circ$).
        \item Początkowa pozycja wózka: $x(0) = 0$ m.
        \item Zerowe prędkości początkowe: $\dot{\theta}(0) = 0$, $\dot{x}(0) = 0$.
    \end{itemize}
    Wybór wartości $\theta(0) = 0{,}05$ rad podyktowany jest dwoma czynnikami:
    jest to wychylenie na~tyle małe, że mieści się w~obszarze stosowalności 
    modelu zlinearyzowanego (istotne dla LQR), a~jednocześnie wystarczająco duże,
    aby wymagać aktywnej interwencji regulatora. Wartość ta jest również powszechnie
    stosowana w~literaturze przedmiotu jako standardowy warunek testowy~\cite{Prasad2014}.
    Celem jest sprawdzenie zdolności regulatora do~sprowadzenia układu do~pionu
    ($\theta=0, x=0$) oraz ocena czasu regulacji i~przeregulowań.

    \item \textbf{Eksperyment 2: Odporność na~zakłócenia zewnętrzne.} \\
    Symulacja z~tymi samymi warunkami początkowymi, przy czym na~wahadło
    oddziałuje losowa siła zakłócająca $F_{\mathrm{w}}(t)$ (modelująca
    porywisty wiatr) generowana zgodnie z~procedurą opisaną w~Rozdziale 3. 
    Przyjęto odchylenie standardowe siły zakłócającej $\sigma = 2{,}2$~N,
    co odpowiada wartości skutecznej (RMS) siły wiatru rzędu $2{,}2$~N
    i~chwilowym wartościom szczytowym do~$\pm 6{,}6$~N. Jest to poziom 
    zakłóceń stanowiący znaczące obciążenie dla układu sterowania 
    (kilka procent $F_{\max}$), lecz nieprzekraczający możliwości
    kompensacyjnych regulatorów. Test ten pozwala ocenić krzepkość 
    (ang. \textit{robustness}) układu zamkniętego.
\end{enumerate}

Wszystkie symulacje przeprowadzono z~krokiem czasowym $\Delta t = 0{,}01$~s
w~czasie $T_{\mathrm{sim}} = 10$~s (dla testów MPC i~pełnego ustalenia).
Ograniczenie sygnału sterującego ustawiono na $|u| \le 100$ N.

\subsection{Badane algorytmy}

W~ramach eksperymentów przetestowano następujące regulatory (w~wersjach po
optymalizacji nastaw):
\begin{enumerate}
    \item \textbf{PD} -- Równoległy układ dwóch regulatorów PD ($K_i=0$).
    \item \textbf{PID-LQR} -- Hybryda: PID dla wózka, LQR dla wahadła.
    \item \textbf{MPC} -- Nieliniowe sterowanie predykcyjne (NMPC) z~pełnym 
    modelem dynamiki i~całkowaniem metodą Rungego-Kutty 4.~rzędu.
    \item \textbf{MPC-J2} -- Wariant NMPC z~rozszerzoną funkcją kosztu $J_2$ 
    (uwzględniającą dodatkowo kwadrat bezwzględnej wartości sterowania).
    \item \textbf{Fuzzy-LQR} -- Regulator rozmyty Takagi-Sugeno wspomagany
    lokalnym LQR.
\end{enumerate}

\subsection{Wskaźniki jakości regulacji}

Do~ilościowej oceny jakości sterowania wykorzystano następujące wskaźniki
błędów, obliczane dla zdyskretyzowanych przebiegów kąta $\theta[k]$
($N$ próbek):

\begin{itemize}
    \item \textbf{MSE (Mean Squared Error)} -- Średni błąd kwadratowy, karający silniej duże odchyłki.
    \begin{equation}
        \mathrm{MSE} = \frac{1}{N}\sum_{k=1}^{N} (y[k] - y_{\mathrm{ref}}[k])^2
    \end{equation}

    \item \textbf{MAE (Mean Absolute Error)} -- Średni błąd bezwzględny, informujący o przeciętnym uchybie.
    \begin{equation}
        \mathrm{MAE} = \frac{1}{N}\sum_{k=1}^{N} |y[k] - y_{\mathrm{ref}}[k]|
    \end{equation}

    \item \textbf{ISE (Integral of Squared Error)} -- Całkowe kryterium kwadratowe, będące miarą energii uchybu w czasie ciągłym.
    \begin{equation}
        \mathrm{ISE} = \int_{0}^{T} (y(t) - y_{\mathrm{ref}}(t))^2 \, dt
    \end{equation}

    \item \textbf{IAE (Integral of Absolute Error)} -- Całkowe kryterium modułu błędu, akumulujące całkowity uchyb w czasie.
    \begin{equation}
        \mathrm{IAE} = \int_{0}^{T} |y(t) - y_{\mathrm{ref}}(t)| \, dt
    \end{equation}

    \item \textbf{Energia sterowania L2 ($E_{u, L2}$)} -- Koszt kwadratowy sterowania, powiązany z energią elektryczną/mechaniczną.
    \begin{equation}
        E_{u, L2} = \int_{0}^{T} u(t)^2 \, dt
    \end{equation}

    \item \textbf{Energia sterowania L1 ($E_{u, L1}$)} -- Koszt absolutny sterowania (zużycie paliwa/zasobów).
    \begin{equation}
        E_{u, L1} = \int_{0}^{T} |u(t)| \, dt
    \end{equation}

    \item \textbf{Czas regulacji $t_s$ (Settling Time)} -- Czas, po którym sygnał wyjściowy trwale wchodzi w~kanał tolerancji i~już go nie opuszcza. W~niniejszej pracy przyjęto tolerancję $\varepsilon = 2\%$ wartości początkowego wychylenia, tj. $|\theta| < 0{,}001$ rad dla kąta oraz $|x| < 0{,}002$ m dla pozycji.

    \item \textbf{Przeregulowanie $M_p$ (Overshoot)} -- Maksymalne procentowe odchylenie sygnału od wartości zadanej w odniesieniu do wartości skoku.
    \begin{equation}
        M_p = \frac{\max(y) - y_{\mathrm{ref}}}{y_{\mathrm{ref}}} \cdot 100\%
    \end{equation}

    \item \textbf{Uchyb ustalony $e_{ss}$ (Steady-state Error)} -- Średnia wartość uchybu w końcowej fazie symulacji (ostatnie 10\% czasu), określająca dokładność statyczną regulacji.

\end{itemize}

Dodatkowo analizie poddano charakterystyki jakościowe przebiegów czasowych,
takie jak czas regulacji (czas, po którym błąd trwale mieści się w paśmie
$\pm 2\%$) oraz maksymalne przeregulowanie.

\newpage
\section{Analiza wyników}

Rozdział ten poświęcony jest szczegółowej analizie wyników badań symulacyjnych, które zostały
przeprowadzone w celu weryfikacji skuteczności i jakości działania zaprojektowanych układów
sterowania. Głównym celem eksperymentów było zbadanie zachowania wahadła odwróconego w dwóch
diametralnie różnych sytuacjach: podczas stabilizacji punktu pracy w idealnych warunkach
nominalnych oraz w trakcie pracy pod wpływem losowych zakłóceń zewnętrznych.
Dodatkowo zakres badań obejmował weryfikację odporności układów na zmiany parametrów modelu
oraz analizę złożoności obliczeniowej algorytmów.

Podczas analizy wyników szczególny nacisk położono na trzy kluczowe  aspekty sterowania. 
Pierwszym z nich jest stabilizacja kątowa, czyli zdolność układu do
utrzymania pręta wahadła w~pionie (pozycja równowagi chwiejnej). Jest to zadanie priorytetowe,
gdyż jego niezrealizowanie prowadzi do~upadku wahadła i~niepowodzenia procesu regulacji. Drugim, równie
istotnym aspektem, jest stabilizacja pozycji wózka. Wymogiem jest, aby proces stabilizacji
kąta nie odbywał się kosztem nadmiernego przemieszczenia wózka poza zadany obszar roboczy. W~systemach
rzeczywistych, takich jak suwnice czy roboty balansujące, utrzymanie pozycji jest często równie
krytyczne co sama stabilizacja ładunku. Ostatnim jest jakość sygnału sterującego, która ma bardzo duże znaczenie 
jeśli rozpatrujemy rzeczywiste układy napędowe. Wysoka zmienność sygnału sterującego, oscylacje wysokoczęstotliwościowe 
czy gwałtowne skoki amplitudy mogą prowadzić do szybkiego zużycia mechanicznego elementów wykonawczych, a także generować 
niepożądane straty energii.

Dla zachowania przejrzystości analizy, badane algorytmy pogrupowano w dwie rodziny: regulatory
klasyczne, do których zaliczono równoległy układ PID-PID oraz hybrydowy PID-LQR, regulatory
zaawansowane, obejmujące predykcyjny algorytm MPC (w dwóch wariantach funkcji kosztu), liniowy LMPC oraz
sterownik rozmyty Fuzzy-LQR.

\subsection{Stabilizacja w warunkach nominalnych}

Pierwszy scenariusz testowy miał na celu weryfikację dynamiki układu w odpowiedzi na niezerowe
warunki początkowe. Symulacja rozpoczynała się od wychylenia wahadła o kąt około 2.8 stopnia
(0.05 radiana). Jest to typowy test odpowiedzi skokowej, pozwalający ocenić szybkość działania
(czas regulacji) oraz tłumienie oscylacji przez poszczególne regulatory.

\subsubsection{Charakterystyka regulatorów klasycznych}

Na Rysunkach \ref{fig:nom_classical}, \ref{fig:nom_pos_classical} oraz
\ref{fig:nom_control_classical} przedstawiono zbiorcze zestawienie przebiegów czasowych dla
grupy regulatorów klasycznych. Analizując wykres kąta wychylenia $\theta$
(Rys. \ref{fig:nom_classical}), można zaobserwować wyraźną przewagę regulatora hybrydowego.

Zoptymalizowany regulator PID-LQR, wykorzystujący duże wzmocnienia dla błędu pozycji,
charakteryzuje się znacznie krótszym czasem regulacji ($T_s \approx 2{,}9$~s) niż regulator PID-PID ($T_s \approx 4{,}2$~s).
Mimo to, jego zużycie energii ($E_u \approx 0{,}92$) jest zbliżone do klasycznego układu PID-PID ($E_u \approx 0{,}95$).
Oznacza to, że poprawa dynamiki stabilizacji została osiągnięta głównie poprzez bardziej agresywne sterowanie,
co nie przełożyło się na oszczędność energetyczną w tym przypadku.

Dla PID-LQR kluczową zaletą pozostaje znacznie lepsze utrzymanie pozycji wózka.
Maksymalne wychylenie wynosi jedynie ($Max |x| \approx 0.11$ m) wobec ($0.15$ m) dla PID-PID,
przy ponad dwukrotnie szybszym ustaleniu pozycji ($T_{s,x} \approx 2{,}3$~s w porównaniu do $4{,}6$~s).
PID-LQR oferuje zatem lepszą, ,,sztywniejszą'' regulację pozycji kosztem podobnego wydatku energetycznego.

\begin{figure}[ht!]
    \centering
    \includegraphics[width=0.95\textwidth]{images/experiments/combined_nominal_classical.png}
    \caption{Przebieg kąta $\theta$ dla regulatorów klasycznych (Warunki nominalne).}
    \label{fig:nom_classical}
\end{figure}
\newpage
\begin{figure}[ht!]
    \centering
    \includegraphics[width=0.95\textwidth]{images/experiments/combined_nominal_pos_classical.png}
    \caption{Przebieg pozycji $x$ dla regulatorów klasycznych (Warunki nominalne).}
    \label{fig:nom_pos_classical}
\end{figure}

\begin{figure}[ht!]
    \centering
    \includegraphics[width=0.95\textwidth]{images/experiments/combined_nominal_control_classical.png}
    \caption{Sygnał sterujący $u$ dla regulatorów klasycznych (Warunki nominalne).}
    \label{fig:nom_control_classical}
\end{figure}

\subsubsection{Charakterystyka regulatorów zaawansowanych}

W grupie regulatorów zaawansowanych, których wyniki zaprezentowano na Rysunkach
\ref{fig:nom_advanced}, \ref{fig:nom_pos_advanced} i~\ref{fig:nom_control_advanced},
można zaobserwować szerokie spektrum zachowań, wynikające z różnic w sformułowaniu zadań sterowania.

Najlepszy wynik w warunkach nominalnych osiąga regulator MPC (nieliniowy).
MPC-alt (J2), mimo funkcji kosztu karającej bezpośrednio sterowanie, uzyskuje czas regulacji ($T_s \approx 2{,}7$~s).
Lepszą dynamikę wykazuje standardowy MPC ($T_s \approx 2{,}1$~s), który deklasuje konkurencję.
Potwierdza to, że odpowiednio nastrojony regulator predykcyjny może łączyć wysoką dynamikę
z oszczędnością energii ($E_u \approx 0{,}51$ dla MPC-alt).

Standardowy regulator MPC (nieliniowy) działa znacznie szybciej niż PID-LQR czy PID-PID.
Jego czas regulacji wynosi $T_s \approx 2{,}1$~s.
Mimo świetnej reakcji, charakteryzuje się bardzo niskim kosztem energetycznym ($E_u \approx 0{,}56$),
będąc jednym z najbardziej efektywnych rozwiązań.

Liniowy regulator predykcyjny LMPC plasuje się pośrodku stawki.
Osiąga czas regulacji $T_s \approx 3{,}1$~s przy zużyciu energii $E_u \approx 0{,}79$.
Gorsze wyniki energetyczne w porównaniu do nieliniowych wariantów MPC ($0{,}51$ i $0{,}56$)
wynikają z uproszczeń modelu liniowego, który nie odwzorowuje idealnie dynamiki obiektu
nawet w pobliżu punktu pracy, wymuszając częstsze korekty sterowania.

Zdecydowanie odmienną charakterystykę prezentuje regulator Fuzzy-LQR.
W tym zestawieniu uzyskuje on umiarkowany wynik energetyczny ($E_u \approx 0{,}78$),
zbliżony do regulatora LMPC.
Wyróżnia się dobrym czasem regulacji ($T_s \approx 2{,}3$~s), ustępując nieznacznie PID-LQR i MPC.
Sugeruje to, że w warunkach nominalnych, hybrydowa struktura regulatora pozwala na dynamiczną
reakcję, zachowując rozsądny balans między szybkością a kosztem sterowania.
\newpage
\begin{figure}[h!]
    \centering
    \includegraphics[width=0.95\textwidth]{images/experiments/combined_nominal_advanced.png}
    \caption{Przebieg kąta $\theta$ dla regulatorów zaawansowanych (Warunki nominalne).}
    \label{fig:nom_advanced}
\end{figure}

\begin{figure}[h!]
    \centering
    \includegraphics[width=0.95\textwidth]{images/experiments/combined_nominal_pos_advanced.png}
    \caption{Przebieg pozycji $x$ dla regulatorów zaawansowanych (Warunki nominalne).}
    \label{fig:nom_pos_advanced}
\end{figure}

\begin{figure}[h!]
    \centering
    \includegraphics[width=0.95\textwidth]{images/experiments/combined_nominal_control_advanced.png}
    \caption{Sygnał sterujący $u$ dla regulatorów zaawansowanych (Warunki nominalne).}
    \label{fig:nom_control_advanced}
\end{figure}

\newpage
\subsubsection{Zestawienie wyników}

W~celu bezpośredniego porównania wszystkich zaimplementowanych strategii sterowania, 
na~Rysunkach \ref{fig:nom_advanced_all}--\ref{fig:nom_control_advanced_all} zestawiono 
przebiegi czasowe dla wszystkich regulatorów. Wykresy te potwierdzają, że w~idealnych 
warunkach nominalnych najlepszą dynamikę oferuje regulator MPC, 
który najszybciej sprowadza wahadło do~pionu przy umiarkowanym przemieszczeniu wózka. 
Widać wyraźny kontrast między metodami szybkimi a~zachowawczym regulatorem PID-PID, 
który charakteryzuje się znacznie dłuższym czasem regulacji. PID-LQR plasuje się pośrodku,
zapewniając dobrą stabilizację pozycji, ale ustępując MPC w kwestii szybkości i oszczędności energii.
Zestawienie to uwidacznia 
również różnicę w~charakterystyce sygnałów sterujących -- od~gładkich przebiegów dla MPC, 
po~bardziej agresywne działania Fuzzy-LQR, co przekłada się na~omówione wcześniej różnice 
w~kosztach energetycznych.

\newpage
\begin{figure}[h!]
    \centering
    \includegraphics[width=0.95\textwidth]{images/experiments/combined_nominal_all_lmpc_theta.png}
    \caption{Przebieg kąta $\theta$ dla regulatorów zaawansowanych (Warunki nominalne).}
    \label{fig:nom_advanced_all}
\end{figure}

\begin{figure}[h!]
    \centering
    \includegraphics[width=0.95\textwidth]{images/experiments/combined_nominal_all_lmpc_pos.png}
    \caption{Przebieg pozycji $x$ dla regulatorów zaawansowanych (Warunki nominalne).}
    \label{fig:nom_pos_advanced_all}
\end{figure}

\begin{figure}[h!]
    \centering
    \includegraphics[width=0.95\textwidth]{images/experiments/combined_nominal_all_lmpc_u.png}
    \caption{Sygnał sterujący $u$ dla regulatorów zaawansowanych (Warunki nominalne).}
    \label{fig:nom_control_advanced_all}
\end{figure}

\subsection{Analiza odporności na zakłócenia}

Drugi scenariusz badawczy polegał na wprowadzeniu do układu sygnału
zakłócającego, modelującego losowe zakłócenia zewnętrzne o zmiennej sile i kierunku. Test ten miał na
celu sprawdzenie odporności regulatorów na zakłócenia zewnętrzne, czyli ich zdolności do
utrzymania stabilności mimo działania nieznanych, zewnętrznych sił.

Podstawowym problemem fizycznym w tym scenariuszu jest zjawisko sprzężenia dryfu. Aby
skompensować siłę zakłócającą pchającą wahadło np. w prawo, wózek musi nieustannie przyspieszać w
prawo, aby przemieścić się pod środek ciężkości wahadła i wytworzyć moment siły bezwładności
przeciwdziałający zakłóceniu. Oznacza to, że skuteczna kompensacja wychylenia kątowego nieuchronnie
prowadzi do przemieszczania się wózka (dryfu). Istotą problemu jest znalezienie kompromisu --- jak
bardzo pozwolić wózkowi uciec, by utrzymać wahadło w pionie.

\subsubsection{Charakterystyka regulatorów klasycznych}

W grupie klasycznej (Rys. \ref{fig:wind_classical}--\ref{fig:wind_control_classical}) nastąpiła
istotna zmiana w porównaniu do warunków nominalnych. Strojenie PID-LQR, nastawione na
utrzymanie wahadła w pionie, skutkuje mieszanymi rezultatami. Z jednej strony regulator ten 
skutecznie ograniczył dryf pozycji ($Max |x| \approx 0.30$ m).
Z drugiej jednak, odbyło się to kosztem pogorszenia jakości stabilizacji kąta 
($Max |\theta| \approx 0.096$~rad) oraz drastycznym wzrostem zużycia energii 
($E_u \approx 33.37$). W warunkach występowania zakłóceń zewnętrznych, charakterystyka 
regulatora PID-LQR wykazuje mniejszy dryf pozycji wózka w porównaniu do układu klasycznego, 
co wiąże się z generowaniem sygnałów sterujących o większej energii całkowitej.

\begin{figure}[h!]
    \centering
    \includegraphics[width=0.95\textwidth]{images/experiments/combined_wind_classical.png}
    \caption{Przebieg kąta $\theta$ pod wpływem zakłóceń zewnętrznych -- regulatory klasyczne.}
    \label{fig:wind_classical}
\end{figure}

\begin{figure}[h!]
    \centering
    \includegraphics[width=0.95\textwidth]{images/experiments/combined_wind_pos_classical.png}
    \caption{Dryf pozycji $x$ pod wpływem zakłóceń zewnętrznych -- regulatory klasyczne.}
    \label{fig:wind_pos_classical}
\end{figure}

\begin{figure}[h!]
    \centering
    \includegraphics[width=0.95\textwidth]{images/experiments/combined_wind_control_classical.png}
    \caption{Sygnał sterujący $u$ pod wpływem zakłóceń zewnętrznych -- regulatory klasyczne.}
    \label{fig:wind_control_classical}
\end{figure}

\subsubsection{Charakterystyka regulatorów zaawansowanych}

W grupie zaawansowanej (Rys. \ref{fig:wind_advanced}--\ref{fig:wind_control_advanced})
Fuzzy-LQR potwierdza swoją skuteczność.
Uzyskał on bardzo dobre wyniki stabilizacji: małe wychylenie kątowe ($Max |\theta| \approx 0.063$~rad)
oraz niski dryf wózka ($Max |x| \approx 0.31$ m).
Pod względem zużycia energii ($E_u \approx 13.97$) ustępuje on nieco regulatorom MPC, ale jest znacznie
bardziej oszczędny niż klasyczne PID-PID czy PID-LQR.
Mimo to, niskie wartości wskaźników całkowych błędu ($IAE_\theta$) świadczą o wysokiej jakości regulacji
i zdolności do szybkiego tłumienia zakłóceń.

Regulator MPC wykazał zrównoważoną charakterystykę. Pozwolił na nieco większy dryf ($0.41$ m)
i zużył więcej energii ($12.44$) niż Fuzzy-LQR.
Wariant MPC-alt, w przeciwieństwie do wcześniejszych prób, utrzymał stabilność układu.
Jednakże, wysoka kara za wartość sterowania ograniczyła jego zdolność do szybkiej reakcji,
co skutkowało największym dryfem wózka w grupie ($Max |x| \approx 0.57$ m), porównywalnym z PID-PID.
Mimo to, jego zużycie energii pozostało na umiarkowanym poziomie ($12.3$).

Regulator LMPC dobrze poradził sobie ze stabilizacją, utrzymując kąt w ryzach lepiej niż PID-PID czy PID-LQR.
Jednak gorzej poradził sobie jeśli chodzi o 
koszt energetyczny ($E_u \approx 22.7$), zbliżony do wyniku regulatora PID-PID.
Ograniczenia modelu liniowego w obliczu silnych zakłóceń wymusiły mniejszą efektywność sterowania,
co widać również w nieco gorszej stabilizacji kąta ($0.085$ rad) w porównaniu
do nieliniowego MPC.

\begin{figure}[h!]
    \centering
    \includegraphics[width=0.95\textwidth]{images/experiments/combined_wind_advanced.png}
    \caption{Przebieg kąta $\theta$ pod wpływem zakłóceń zewnętrznych -- regulatory zaawansowane.}
    \label{fig:wind_advanced}
\end{figure}

\begin{figure}[h!]
    \centering
    \includegraphics[width=0.95\textwidth]{images/experiments/combined_wind_pos_advanced.png}
    \caption{Dryf pozycji $x$ pod wpływem zakłóceń zewnętrznych -- regulatory zaawansowane.}
    \label{fig:wind_pos_advanced}
\end{figure}

\begin{figure}[h!]
    \centering
    \includegraphics[width=0.95\textwidth]{images/experiments/combined_wind_control_advanced.png}
    \caption{Sygnał sterujący $u$ pod wpływem zakłóceń zewnętrznych -- regulatory zaawansowane.}
    \label{fig:wind_control_advanced}
\end{figure}

\newpage
\subsubsection{Zestawienie wyników}

Analogicznie do warunków nominalnych, przeprowadzono zbiorcze 
zestawienie wyników dla wszystkich badanych regulatorów w obecności 
zakłóceń (Rys. \ref{fig:wind_all_theta}--\ref{fig:wind_all_u}).
Wykresy te dobitnie pokazują przewagę regulatora Fuzzy-LQR oraz MPC
w tłumieniu zakłóceń.
Widać wyraźnie, że metody te utrzymują oscylacje wahadła w najwęższym paśmie, 
podczas gdy klasyczny PID-PID oraz PID-LQR pozwalają na znacznie większe wychylenia.
Zestawienie sygnałów sterujących ujawnia koszt tej precyzji - Fuzzy-LQR charakteryzuje 
się najbardziej aktywnym sterowaniem, co jednak w ogólnym rozrachunku (dzięki szybkiej stabilizacji) 
nie prowadzi do najgorszego zużycia energii.
\newpage
\begin{figure}[h!]
    \centering
    \includegraphics[width=0.95\textwidth]{images/experiments/combined_wind_all_lmpc_theta.png}
    \caption{Przebieg kąta $\theta$ pod wpływem zakłóceń zewnętrznych -- wszystkie regulatory.}
    \label{fig:wind_all_theta}
\end{figure}

\begin{figure}[h!]
    \centering
    \includegraphics[width=0.95\textwidth]{images/experiments/combined_wind_all_lmpc_pos.png}
    \caption{Dryf pozycji $x$ pod wpływem zakłóceń zewnętrznych -- wszystkie regulatory.}
    \label{fig:wind_all_pos}
\end{figure}

\begin{figure}[h!]
    \centering
    \includegraphics[width=0.95\textwidth]{images/experiments/combined_wind_all_lmpc_u.png}
    \caption{Sygnał sterujący $u$ pod wpływem zakłóceń zewnętrznych -- wszystkie regulatory.}
    \label{fig:wind_all_u}
\end{figure}

\subsection{Analiza odporności na zmianę parametrów modelu}

Trzeci scenariusz badawczy miał na celu ocenę wrażliwości regulatorów na~niepewność
parametryczną modelu. W~praktycznych zastosowaniach przemysłowych dokładne wartości
parametrów fizycznych układu są~rzadko znane z~wysoką precyzją. Mogą one ulegać
zmianom w~czasie (np.~zużycie mechaniczne, zmiana ładunku), dlatego odporność na~takie
perturbacje jest kluczową właściwością regulatora.

W~eksperymencie zwiększono masę wahadła o~100\% względem wartości nominalnej
($m_{\mathrm{nom}} = 0{,}23$~kg $\rightarrow$ $m_{\mathrm{real}} = 0{,}46$~kg),
podczas gdy regulatory pozostały nastrojone dla parametrów nominalnych. 

Wyniki eksperymentu zaprezentowano na~Rysunkach \ref{fig:robust_theta}, \ref{fig:robust_x}
oraz \ref{fig:robust_u}. Kluczową obserwacją jest fakt, że wszystkie badane regulatory
zachowały stabilność mimo niedokładnego modelu. Świadczy to o~odpowiednim zapasie
stabilności wynikającym z~procesu optymalizacji nastaw.

Analizując przebieg kąta $\theta$ (Rys.~\ref{fig:robust_theta}), można zauważyć, że
zarówno regulatory klasyczne (w szczególności PID-LQR), jak i predykcyjne (MPC, MPC-alt)
wykazują wysoką odporność na zmianę parametrów. Wbrew obawom o wrażliwość metod opartych na modelu,
algorytmy predykcyjne skutecznie kompensują błąd modelowania. Mechanizm sprzężenia zwrotnego
oraz przesuwny horyzont predykcji pozwalają na bieżącą korektę sterowania,
dzięki czemu spadek jakości regulacji jest minimalny.

Regulator PID-LQR, dzięki wysokim wzmocnieniom, a także regulatory MPC,
utrzymują precyzję stabilizacji zbliżoną do warunków nominalnych, choć w przypadku PID-LQR
widoczny jest wzrost czasu regulacji.
Wskazuje to, że dla perturbacji parametrów (rzędu 100\%),
dobrze nastrojony regulator liniowy oraz nieliniowy MPC zachowują poprawność działania.

Zdecydowanie najsłabsze wyniki w tym zestawieniu osiągnął klasyczny układ PID-PID.
Charakteryzuje się on najdłuższym czasem regulacji ($T_s \approx 6{,}8$~s) oraz największym
uchybem całkowym ($IAE_\theta \approx 0{,}065$). Brak adaptacji oraz brak modelu predykcyjnego
sprawiają, że regulator ten z trudem kompensuje tak znaczną zmianę dynamiki obiektu,
co prowadzi do powolnego i oscylacyjnego dochodzenia do równowagi.
Regulator LMPC plasuje się pośrodku stawki – radzi sobie lepiej niż PID-PID,
ale ustępuje nieliniowym odpowiednikom MPC, co wynika z ograniczeń modelu liniowego.

Regulator Fuzzy-LQR w tym scenariuszu prezentuje się bardzo dobrze.
Osiąga czas regulacji $T_s \approx 2{,}3$~s oraz bardzo niski uchyb całkowy ($IAE_\theta \approx 0{,}016$),
znacząco przewyższając pod tym względem PID-LQR.
Co ważne, jego zużycie energii ($E_u \approx 0.80$) jest niskie i porównywalne z innymi regulatorami,
co stanowi znaczącą poprawę w stosunku do historycznych wyników.
Elastyczność logiki rozmytej pozwala na skuteczną adaptację do zmienionej dynamiki obiektu
bez ponoszenia nadmiernych kosztów sterowania.

Na~wykresie sterowania (Rys.~\ref{fig:robust_u}) widać wzrost amplitudy sygnałów sterujących
dla wszystkich regulatorów, co jest fizyczną koniecznością przy sterowaniu obiektem o większej bezwładności.
Największą aktywność wykazuje regulator Fuzzy-LQR, co potwierdza jego agresywną charakterystykę działania.

\begin{figure}[h!]
    \centering
    \includegraphics[width=0.95\textwidth]{images_odpornosc/robustness_theta.png}
    \caption{Przebieg kąta $\theta$ przy zmienionych parametrach modelu (+100\% masy wahadła).}
    \label{fig:robust_theta}
\end{figure}
\newpage
\begin{figure}[h!]
    \centering
    \includegraphics[width=0.95\textwidth]{images_odpornosc/robustness_x.png}
    \caption{Przebieg pozycji $x$ przy zmienionych parametrach modelu (+100\% masy wahadła).}
    \label{fig:robust_x}
\end{figure}

\begin{figure}[h!]
    \centering
    \includegraphics[width=0.95\textwidth]{images_odpornosc/robustness_u.png}
    \caption{Sygnał sterujący $u$ przy zmienionych parametrach modelu (+100\% masy wahadła).}
    \label{fig:robust_u}
\end{figure}

\subsubsection{Analiza wrażliwości na zakres zmian}

W~celu pełniejszej oceny zapasów odporności poszczególnych regulatorów przeprowadzono
dodatkową analizę wrażliwości. Zbadano zachowanie układów sterowania w~szerokim zakresie
zmian masy wahadła: od~$-75\%$ do~$+200\%$ wartości nominalnej. Dla każdej wartości
perturbacji obliczono wskaźnik całkowy błędu bezwzględnego kąta ($IAE_\theta$),
który jest miarą skumulowanego uchybu w~czasie symulacji.

Wyniki analizy przedstawiono na~Rysunku~\ref{fig:robust_sensitivity}. Można zaobserwować
kilka istotnych prawidłowości:

\begin{itemize}
    \item Regulator Fuzzy-LQR wykazuje najlepszą odporność na~niepewność
    parametryczną, osiągając najniższe wartości $IAE_\theta$ w~całym badanym zakresie.
    Co więcej, jego charakterystyka jest praktycznie płaska --- zmiana masy wahadła
    nie wpływa istotnie na~jakość regulacji. Wynika to z~adaptacyjnej natury logiki
    rozmytej, która dostosowuje wagi reguł do~obserwowanego stanu układu.
    
    \item Regulatory klasyczne (PID-PID, PID-LQR) również charakteryzują się
    płaską charakterystyką w~całym zakresie perturbacji, choć z~nieco wyższymi
    wartościami błędu niż Fuzzy-LQR. Ich jakość regulacji jest mało wrażliwa na~niepewność 
    parametryczną dzięki strukturze opartej na~sprzężeniu zwrotnym od~błędu.
    
    \item Regulatory predykcyjne (MPC, MPC-alt) wykazują najwyższe wartości
    wskaźnika $IAE_\theta$. Jest to spodziewane zachowanie, gdyż algorytm optymalizacji 
    wykorzystuje wewnętrzny model, który odbiega od~rzeczywistej dynamiki obiektu.
    Niemniej jednak, regulatory te zachowują stabilność w~całym badanym zakresie,
    a~wzrost błędu wraz z~perturbacją jest umiarkowany.
\end{itemize}

Analiza ta~pokazuje, że regulatory wykorzystujące mechanizmy adaptacyjne Fuzzy-LQR
lub proste sprzężenie zwrotne od~błędu (PID, LQR) mogą oferować lepszą odporność
na~niepewność modelu niż metody predykcyjne, których skuteczność zależy od~dokładności
wewnętrznego modelu obiektu.

\begin{figure}[h!]
    \centering
    \includegraphics[width=0.95\textwidth]{images_odpornosc/robustness_sensitivity.png}
    \caption{Analiza wrażliwości: zależność wskaźnika $IAE_\theta$ od~zmiany masy
    wahadła dla poszczególnych regulatorów. Linia pionowa oznacza warunki nominalne.}
    \label{fig:robust_sensitivity}
\end{figure}

\subsection{Szczegółowe zestawienie ilościowe}

Poniższe tabele stanowią numeryczne podsumowanie omówionych wyżej zjawisk. Dane zostały
zgrupowane w sposób ułatwiający porównanie osiągów w dwóch domenach: stabilizacji wahadła (kąt)
oraz stabilizacji wózka (pozycja).

\begin{table}[h!]
    \centering
    \caption{Wskaźniki jakości (Kąt i Pozycja) - warunki nominalne}
    \label{tab:results_nominal}
    \begin{tabular}{|l|c|c|c|c|c|c|}
        \hline
        Wskaźnik & PID-PID & PID-LQR & MPC & MPC-alt & Fuzzy-LQR & LMPC \\ \hline
        $MSE_\theta$ & \cellcolor{red!25}0,00011 & 0,00009 & 0,00006 & 0,00007 & \cellcolor{green!25}0,00005 & 0,00008 \\ \hline
        $IAE_\theta$ & \cellcolor{red!25}0,04567 & 0,03169 & 0,02323 & 0,02848 & \cellcolor{green!25}0,01578 & 0,02992 \\ \hline
        $T_{s, \theta}$ & \cellcolor{red!25}4,20000 & 2,90000 & \cellcolor{green!25}2,10000 & 2,70000 & 2,30000 & 3,10000 \\ \hline
        \hline
        $MSE_x$ & 0,00051 & \cellcolor{green!25}0,00036 & 0,00038 & \cellcolor{red!25}0,00073 & 0,00043 & 0,00039 \\ \hline
        $T_{s, x}$ & 4,60000 & \cellcolor{green!25}2,30000 & 3,60000 & \cellcolor{red!25}5,20000 & 3,00000 & 2,70000 \\ \hline
        \hline
        $E_{u}$ & \cellcolor{red!25}0,95046 & 0,92481 & 0,55762 & \cellcolor{green!25}0,51499 & 0,77627 & 0,78994 \\ \hline
    \end{tabular}
\end{table}

\begin{table}[h!]
    \centering
    \caption{Wskaźniki jakości (Kąt i Pozycja) - zakłócenia zewnętrzne}
    \label{tab:results_wind}
    \begin{tabular}{|l|c|c|c|c|c|c|}
        \hline
        Wskaźnik & PID-PID & PID-LQR & MPC & MPC-alt & Fuzzy-LQR & LMPC \\ \hline
        $MSE_\theta$ & \cellcolor{red!25}\phantom{0}0,00146 & \phantom{0}0,00178 & \phantom{0}0,00058 & \phantom{0}0,00063 & \cellcolor{green!25}\phantom{0}0,00054 & \phantom{0}0,00118 \\ \hline
        $IAE_\theta$ & \cellcolor{red!25}\phantom{0}0,30988 & \phantom{0}0,33268 & \phantom{0}0,18151 & \phantom{0}0,18607 & \cellcolor{green!25}\phantom{0}0,17797 & \phantom{0}0,27259 \\ \hline
        $Max |\theta|$ & \phantom{0}0,09006 & \cellcolor{red!25}\phantom{0}0,09650 & \phantom{0}0,06246 & \phantom{0}0,06715 & \cellcolor{green!25}\phantom{0}0,06344 & \phantom{0}0,08539 \\ \hline
        \hline
        $MSE_x$ & \phantom{0}0,04382 & \cellcolor{green!25}\phantom{0}0,00653 & \phantom{0}0,01843 & \cellcolor{red!25}\phantom{0}0,04383 & \phantom{0}0,00859 & \phantom{0}0,01551 \\ \hline
        $Max |x|$ & \phantom{0}0,55997 & \cellcolor{green!25}\phantom{0}0,29510 & \phantom{0}0,40938 & \cellcolor{red!25}\phantom{0}0,56993 & \phantom{0}0,31153 & \phantom{0}0,39981 \\ \hline
        \hline
        $E_{u}$ & 22,77583 & \cellcolor{red!25}33,36971 & 12,43920 & \cellcolor{green!25}12,26560 & 13,96835 & 22,74390 \\ \hline
    \end{tabular}
\end{table}

\begin{table}[h!]
    \centering
    \caption{Wskaźniki jakości (Kąt i Pozycja) - odporność na zmianę parametrów modelu (+100\% masy wahadła)}
    \label{tab:results_robustness}
    \begin{tabular}{|l|c|c|c|c|c|c|}
        \hline
        Wskaźnik & PID-PID & PID-LQR & MPC & MPC-alt & Fuzzy-LQR & LMPC \\ \hline
        $MSE_\theta$ & \cellcolor{red!25}0,00015 & 0,00011 & 0,00007 & 0,00007 & \cellcolor{green!25}0,00005 & 0,00009 \\ \hline
        $IAE_\theta$ & \cellcolor{red!25}0,06524 & 0,03900 & 0,02517 & 0,02530 & \cellcolor{green!25}0,01620 & 0,03379 \\ \hline
        $T_{s, \theta}$ & \cellcolor{red!25}6,80000 & 3,70000 & \cellcolor{green!25}2,00000 & 2,00000 & 2,30000 & 2,90000 \\ \hline
        \hline
        $MSE_x$ & \cellcolor{red!25}0,00065 & \cellcolor{green!25}0,00037 & 0,00038 & 0,00038 & 0,00043 & 0,00040 \\ \hline
        $T_{s, x}$& \cellcolor{red!25}8,30000 & 2,90000 & 3,00000 & 3,00000 & 3,10000 & \cellcolor{green!25}2,20000 \\ \hline
        \hline
        $E_{u}$ & \cellcolor{red!25}1,49205 & 1,28519 & 0,64753 & \cellcolor{green!25}0,64818 & 0,79775 & 1,01039 \\ \hline
    \end{tabular}
\end{table}

\newpage
\subsection{Porównanie złożoności obliczeniowej}

Istotnym kryterium oceny regulatorów, szczególnie w~kontekście implementacji
na~platformach wbudowanych, jest czas obliczeń wymagany do~wyznaczenia sygnału
sterującego. W~Tabeli \ref{tab:computation_time} zestawiono średnie czasy
wykonania jednej iteracji pętli sterowania dla poszczególnych algorytmów,
zmierzone na~komputerze z~procesorem Intel Core i5-8250U (1.6 GHz).

\begin{table}[h!]
    \centering
    \caption{Średni czas obliczeń jednej iteracji pętli sterowania}
    \label{tab:computation_time}
    \begin{tabular}{|l|c|c|}
        \hline
        Regulator & Czas [ms] & Względem PID \\ \hline
        PID-PID & $< 0{,}02$ & $1\times$ \\ \hline
        PID-LQR & $< 0{,}02$ & $1\times$ \\ \hline
        Fuzzy-LQR & $0{,}04$ & $2\times$ \\ \hline
        LMPC & $2{,}8$ & $140\times$ \\ \hline
        MPC & $11{,}3$ & $565\times$ \\ \hline
        MPC-alt & $14{,}7$ & $735\times$ \\ \hline
    \end{tabular}
\end{table}

Regulatory klasyczne (PID-PID, PID-LQR) oraz rozmyty (Fuzzy-LQR) charakteryzują się
zaniedbywalnym czasem obliczeń, rzędu mikrosekund. Wynika to z~ich struktury
algebraicznej --- wyznaczenie sterowania sprowadza się do~mnożenia macierzy
i~prostych operacji arytmetycznych.

W~przypadku regulatorów predykcyjnych czas obliczeń jest o~blisko trzy rzędy wielkości wyższy
(ok. 3--15 ms), co wynika z~konieczności rozwiązywania w~każdym kroku
zadania optymalizacji nieliniowej (lub kwadratowej dla LMPC). Wartości te pozostają jednak znacznie poniżej
kroku symulacji, co potwierdza możliwość pracy MPC
w~czasie rzeczywistym dla rozpatrywanego obiektu. Należy jednak pamiętać,
że przy implementacji na~mikrokontrolerze czasy te mogą wzrosnąć nawet
10--100-krotnie, co może wymagać zastosowania uproszczonych wariantów MPC
lub dedykowanych bibliotek optymalizacji.

\newpage
\section{Podsumowanie}

Niniejsza praca miała na~celu opracowanie zestawu skryptów symulacyjnych
oraz przeprowadzenie wielokryterialnej analizy porównawczej algorytmów sterowania
dla nieliniowego układu odwróconego wahadła na~wózku. Zrealizowano wszystkie
założone cele badawcze: zaimplementowano pięć różnych strategii sterowania
(PD, PID-LQR, MPC, MPC-J2, Fuzzy-LQR), przeprowadzono ich optymalizację
parametryczną oraz zweryfikowano skuteczność w~warunkach nominalnych
i~przy obecności zakłóceń zewnętrznych.

\subsection{Wnioski końcowe}

Przeprowadzone badania symulacyjne, w~zestawieniu z~literaturą przedmiotu, pozwalają na
sformułowanie szeregu istotnych wniosków:

\begin{enumerate}
    \item \textbf{Brak uniwersalnego regulatora.} Wyniki jednoznacznie potwierdzają,
    że nie istnieje jeden regulator dominujący we~wszystkich aspektach sterowania.
    Mamy do~czynienia z~fundamentalnym kompromisem inżynierskim między jakością
    regulacji a~kosztami eksploatacyjnymi.
    
    \item \textbf{Najwyższa precyzja: Fuzzy-LQR.} Jeżeli priorytetem jest bezwzględne
    utrzymanie punktu pracy (np. w~robotyce precyzyjnej), najlepsze wyniki osiągnął
    system rozmyty Fuzzy-LQR. Sterownik ten potrafił niemal całkowicie zniwelować
    wpływ losowych zakłóceń, utrzymując wahadło w~pionie z~maksymalnym wychyleniem
    zaledwie $0{,}05$ rad. Jest to jednak rozwiązanie bardzo kosztowne energetycznie
    ($E_u \approx 86$ --- ponad 7-krotnie więcej niż MPC).
    
    \item \textbf{Najlepsza ekonomia: MPC.} Sterowanie predykcyjne okazało się
    najbardziej ekonomicznym rozwiązaniem w~warunkach nominalnych ($E_u \approx 0{,}56$).
    MPC charakteryzuje się płynnym, przewidywalnym sterowaniem oraz jawnym
    uwzględnianiem ograniczeń fizycznych napędu, co czyni go rozwiązaniem
    najbezpieczniejszym dla mechaniki układu.
    
    \item \textbf{Uniwersalność PID-LQR.} Regulator hybrydowy PID-LQR, po~odpowiednim
    doborze wag ($Q_x = 500$), okazał się rozwiązaniem bardzo uniwersalnym.
    W~warunkach wietrznych osiągnął lepsze wyniki niż MPC pod~względem trzymania
    pozycji i~zużycia energii, przy znacznie niższej złożoności obliczeniowej.
    
    \item \textbf{Wrażliwość na~funkcję kosztu.} Analiza wariantu MPC-J2 wykazała,
    że dobór funkcji kosztu ma krytyczny wpływ na~odporność układu. Zbyt restrykcyjna
    kara za~energię sterowania może prowadzić do~utraty stabilności w~obecności
    silnych zakłóceń.
\end{enumerate}

\subsection{Ograniczenia pracy}

Przeprowadzone badania mają charakter symulacyjny i~wiążą się z~pewnymi
ograniczeniami, które należy uwzględnić przy interpretacji wyników:

\begin{itemize}
    \item \textbf{Idealne warunki pomiarowe.} W~symulacjach założono, że pełny
    wektor stanu jest dostępny bezpośrednio, bez szumów pomiarowych i~opóźnień.
    W~rzeczywistych układach konieczne byłoby zastosowanie obserwatora stanu
    (np.~filtra Kalmana), co mogłoby wpłynąć na~jakość regulacji.
    
    \item \textbf{Brak dynamiki aktuatora.} Model nie uwzględnia bezwładności
    i~ograniczeń dynamicznych silnika napędzającego wózek. W~systemach rzeczywistych
    mogłyby wystąpić dodatkowe opóźnienia i~ograniczenia szybkości narastania siły.
    
    \item \textbf{Uproszczony model zakłóceń.} Przyjęty model wiatru (filtrowany
    szum gaussowski) jest uproszczeniem rzeczywistych warunków środowiskowych,
    które mogą charakteryzować się bardziej złożoną strukturą czasowo-przestrzenną.
    
    \item \textbf{Brak weryfikacji eksperymentalnej.} Wyniki nie zostały
    zweryfikowane na~rzeczywistym stanowisku laboratoryjnym, co uniemożliwia
    ocenę wpływu niedokładności modelu i~nieuwzględnionych zjawisk fizycznych.
\end{itemize}

\subsection{Kierunki dalszych badań}

Na~podstawie przeprowadzonych analiz można wskazać następujące kierunki
rozwoju projektu:

\begin{enumerate}
    \item \textbf{Implementacja algorytmu swing-up.} Rozszerzenie funkcjonalności
    o~fazę wprowadzania wahadła z~pozycji dolnej do~górnej, co pozwoliłoby
    na~pełną automatyzację procesu stabilizacji.
    
    \item \textbf{Adaptacyjne sterowanie MPC.} Implementacja mechanizmów adaptacji
    online, pozwalających na~automatyczne dostrajanie wag funkcji kosztu
    w~zależności od~aktualnych warunków pracy.
    
    \item \textbf{Uwzględnienie szumów pomiarowych.} Rozbudowa modelu o~realistyczne
    szumy czujników oraz implementacja estymatora stanu (filtr Kalmana lub EKF),
    co przybliżyłoby symulacje do~warunków rzeczywistych.
    
    \item \textbf{Implementacja sprzętowa.} Weryfikacja algorytmów na~rzeczywistym
    stanowisku laboratoryjnym z~wykorzystaniem platformy mikroprocesorowej
    (np.~STM32, Raspberry Pi), co pozwoliłoby na~ocenę wydajności obliczeniowej
    i~praktycznej stosowalności poszczególnych metod.
    
    \item \textbf{Porównanie z~metodami uczenia maszynowego.} Zestawienie
    klasycznych metod sterowania z~podejściami opartymi na~uczeniu ze~wzmocnieniem
    (Reinforcement Learning), które zyskują coraz większą popularność
    w~sterowaniu systemami nieliniowymi.
\end{enumerate}

Opracowane środowisko symulacyjne stanowi solidną bazę do~dalszych badań
nad sterowaniem nieliniowym i~może być wykorzystane zarówno w~celach
dydaktycznych, jak i~badawczych.



%--------------------------------------------
% Literatura
%--------------------------------------------
\cleardoublepage % Zaczynamy od nieparzystej strony
\printbibliography

%--------------------------------------------
% Spisy (opcjonalne)
%--------------------------------------------
\newpage
\pagestyle{plain}

% Wykaz symboli i skrótów.
% Pamiętaj, żeby posortować symbole alfabetycznie
% we własnym zakresie. Ponieważ mało kto używa takiego wykazu,
% uznałem, że robienie automatycznie sortowanej listy
% na poziomie LaTeXa to za duży overkill.
% Makro \acronymlist generuje właściwy tytuł sekcji,
% w zależności od języka.
% Makro \acronym dodaje skrót/symbol do listy,
% zapewniając podstawowe formatowanie.
% //AB
\vspace{0.8cm}
\acronymlist
\acronym{$x$}{Położenie wózka [m]}
\acronym{$\dot{x}$}{Prędkość wózka [m/s]}
\acronym{$\varphi$}{Kąt odchylenia wahadła od pionu [rad]}
\acronym{$\dot{\varphi}$}{Prędkość kątowa wahadła [rad/s]}
\acronym{$u$}{Sygnał sterujący (siła działająca na wózek) [N]}
\acronym{LQR}{Linear-Quadratic Regulator - regulator liniowo-kwadratowy}
\acronym{PID}{Proporcjonalno-Całkująco-Różniczkujący regulator}
\acronym{MSE}{Mean Squared Error - średni błąd kwadratowy}
\acronym{MAE}{Mean Absolute Error - średni błąd bezwzględny}
\acronym{$\Delta t$}{Krok czasowy symulacji [s]}
\acronym{$N$}{Liczba próbek w sygnale dyskretnym}


\listoffigurestoc     % Spis rysunków.
\vspace{1cm}          % vertical space
\listoftablestoc      % Spis tabel.
\vspace{1cm}          % vertical space
% \listofappendicestoc  % Spis załączników

% Załączniki
% \newpage
% \appendix{Nazwa załącznika 1}
% \lipsum[1-8]

% \newpage
% \appendix{Nazwa załącznika 2}
% \lipsum[1-4]

% Używając powyższych spisów jako szablonu,
% możesz tu dodać swój własny wykaz bądź listę,
% np. spis algorytmów.

\end{document} % Dobranoc.
