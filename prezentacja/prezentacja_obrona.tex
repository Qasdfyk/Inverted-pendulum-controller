%======================
% Prezentacja na obronę pracy inżynierskiej - wersja finalna
%======================
\documentclass[aspectratio=169]{beamer}

%--------- Fonts & language
\usepackage[T1]{fontenc}
\usepackage[utf8]{inputenc}
\usepackage{lmodern}
\usepackage[polish]{babel}
\usepackage{graphicx}
\usepackage{xcolor}
\usepackage{tikz}
\usepackage{appendixnumberbeamer}
\usepackage{booktabs}

%--------- Paths (relative to prezentacja folder)
\graphicspath{{../latex/images/experiments/}{../latex/images/diagrams/}{../latex/img/}{../latex/images_odpornosc/}}

%--------- Colors
\definecolor{Primary}{HTML}{1E3A8A}
\definecolor{Accent}{HTML}{3B82C4}
\definecolor{Dark}{HTML}{162E6A}
\definecolor{Text}{HTML}{1F2937}
\definecolor{BG}{HTML}{FFFFFF}
\definecolor{Success}{HTML}{059669}
\definecolor{Warning}{HTML}{D97706}

%--------- Beamer color config
\setbeamercolor{normal text}{fg=Text,bg=BG}
\setbeamercolor{structure}{fg=Primary}
\setbeamercolor{frametitle}{fg=Text,bg=white}
\setbeamercolor{title}{fg=white,bg=Primary}
\setbeamercolor{author in head/foot}{fg=white,bg=Primary}
\setbeamercolor{date in head/foot}{fg=white,bg=Primary}
\setbeamercolor{section in head/foot}{fg=white,bg=Primary}
\setbeamercolor{page number in head/foot}{fg=white,bg=Primary}
\setbeamercolor{block title}{fg=white,bg=Primary}
\setbeamercolor{block body}{fg=Text,bg=Primary!10}

%--------- Fonts sizing
\setbeamerfont{title}{series=\bfseries,size=\Large}
\setbeamerfont{frametitle}{series=\bfseries,size=\large}
\setbeamerfont{footline}{size=\scriptsize}
\setbeamerfont{headline}{size=\scriptsize}

%--------- Remove default nav symbols
\setbeamertemplate{navigation symbols}{}

%--------- Top bar (headline) with section title
\setbeamertemplate{headline}{%
  \leavevmode%
  \begin{beamercolorbox}[wd=\paperwidth,ht=2.6ex,dp=1.2ex,left]{section in head/foot}
    \hspace{1.2ex}%
    \strut%
    \ifx\insertsectionhead\empty
      \insertshorttitle
    \else
      \insertsectionhead
    \fi
  \end{beamercolorbox}%
  \begingroup
    \color{Accent}\rule{\paperwidth}{0.6pt}%
  \endgroup
}

%--------- Bottom bar
\setbeamertemplate{footline}{%
  \begin{tikzpicture}[remember picture,overlay]
    \fill[Primary] (current page.south west) rectangle ([yshift=4.6ex]current page.south east);
    \draw[Accent, line width=0.6pt] ([yshift=4.6ex]current page.south west) -- ([yshift=4.6ex]current page.south east);
    \node[anchor=south west, text=white, font=\scriptsize, inner sep=1.2ex]
      at (current page.south west) {\insertshorttitle};
    \node[anchor=south east, text=white, font=\scriptsize, inner sep=1.2ex]
      at (current page.south east) {\insertframenumber{} / \inserttotalframenumber};
  \end{tikzpicture}%
  \vspace{4.8ex}%
}

%--------- Frame title styling
\setbeamertemplate{frametitle}{%
  \vspace{0.2em}
  \nointerlineskip%
  \begin{beamercolorbox}[wd=\paperwidth,dp=0.6ex,sep=1.2ex,left]{frametitle}
    \insertframetitle
  \end{beamercolorbox}%
}

%--------- TOC at section start
\AtBeginSection[]{
  {%
    \setbeamertemplate{footline}{}
    \begin{frame}[noframenumbering]{Plan prezentacji}
      \tableofcontents[currentsection]
    \end{frame}
  }%
}

%--------- Title info
\title[Stabilizacja wahadła na wózku]{Efektywny układ stabilizacji\\odwróconego wahadła na wózku}
\author[Adam Sokołowski]{Autor: Adam Sokołowski \\ Opiekun: mgr inż. Robert Nebeluk}
\institute{Wydział Elektroniki i Technik Informacyjnych PW\\Automatyka i Robotyka}
\date{08.01.2026}

\newcommand{\TitleSlide}{%
  {%
  \setbeamertemplate{headline}{}%
  \setbeamertemplate{footline}{}%
  \begin{frame}[plain]
    \begin{beamercolorbox}[wd=\paperwidth,ht=0.5\paperheight,center,dp=0ex]{title}
      \vspace*{1.5em}
      {\usebeamerfont{title}\inserttitle\par}
      \vspace{0.5em}
      {\large \insertsubtitle\par}
      \vfill
      {\normalsize \insertauthor\par}
      {\footnotesize \insertinstitute\par}
      {\footnotesize \insertdate\par}
      \vspace*{1.5em}
    \end{beamercolorbox}
  \end{frame}
  }%
}

%--------- Document
\begin{document}

%----- Title Slide
\TitleSlide

%----- Plan of the presentation
\begin{frame}{Plan prezentacji}
  \tableofcontents
\end{frame}

%======================
% Section 1: Cel pracy
%======================
\section{Cel pracy}

\begin{frame}{Cel pracy}
  \begin{itemize}
    \item Stworzenie środowiska symulacyjnego do testowania regulatorów
    \item Implementacja i optymalizacja pięciu różnych strategii sterowania
    \item Wielokryterialne porównanie algorytmów regulacji
    \item Ocena odporności układów na zakłócenia i zmienność parametrów
  \end{itemize}
  
  \vspace{1em}
  \textbf{Zaimplementowane regulatory:}
  \begin{columns}
  \begin{column}{0.5\textwidth}
    \begin{itemize}
      \item PD–PD (kaskadowy)
      \item PD–LQR (hybrydowy)
      \item MPC (predykcyjny)
    \end{itemize}
  \end{column}
  \begin{column}{0.5\textwidth}
    \begin{itemize}
      \item MPC-J2 (alternatywna funkcja kosztu)
      \item Fuzzy-LQR (rozmyty Takagi–Sugeno)
    \end{itemize}
  \end{column}
  \end{columns}
\end{frame}

\begin{frame}{Zastosowanie}
\begin{columns}
\begin{column}{0.55\textwidth}
  Odwrócone wahadło jest klasycznym benchmarkiem w teorii sterowania:
  \begin{itemize}
    \item Robotyka mobilna (roboty balansujące)
    \item Sterowanie rakiet i pojazdów kosmicznych
    \item Stabilizacja pojazdów dwukołowych
    \item Układy suwnicowe
  \end{itemize}
\end{column}

\begin{column}{0.45\textwidth}
\centering
\includegraphics[width=0.85\linewidth]{model.png}
\end{column}
\end{columns}
\end{frame}

%======================
% Section 2: Model obiektu
%======================
\section{Model obiektu}

\begin{frame}{Charakterystyka układu}
    \begin{columns}[T,onlytextwidth]
    \column{0.52\textwidth}
      \textbf{Cechy układu:}
      \begin{itemize}
        \item Wysoka nieliniowość
        \item Niestabilny w punkcie pracy
        \item System niedoaktuowany (1 wejście, 2 wyjścia)
        \item Duża wrażliwość na zakłócenia
        \item Pośrednie sterowanie wahadłem przez wózek
      \end{itemize}

    \column{0.48\textwidth}
      \centering
      \includegraphics[width=0.85\linewidth]{model.png}
  \end{columns}
\end{frame}

\begin{frame}{Równania stanu układu}
\small
\[
\frac{d}{dt}
\begin{bmatrix}
\theta \\ \dot\theta \\ x \\ \dot x
\end{bmatrix}
=
\begin{bmatrix}
\dot\theta \\
\frac{u \cos \theta - (M + m) g \sin \theta + m l (\cos \theta \sin \theta) \dot\theta^2}
{m l \cos^2 \theta - (M + m) l} \\
\dot x \\
\frac{u + m l (\sin \theta) \dot\theta^2 - m g \cos \theta \sin \theta}
{M + m - m \cos^2 \theta}
\end{bmatrix}
\]

\vspace{0.5em}
\textbf{Parametry modelu:}
\begin{itemize}
    \item $M = 1.0$ kg -- masa wózka
    \item $m = 0.23$ kg -- masa wahadła
    \item $l = 0.5$ m -- długość wahadła
    \item $g = 9.81$ m/s$^2$ -- przyspieszenie ziemskie
\end{itemize}
\end{frame}

\begin{frame}{Liniaryzacja w punkcie równowagi}
Model liniowy dla małych wychyleń wahadła ($\theta \approx 0$):
\[
\frac{d \delta x}{dt} =
\begin{bmatrix}
0 & 1 & 0 & 0 \\
\frac{(M + m) g}{M l} & 0 & 0 & 0 \\
0 & 0 & 0 & 1 \\
-\frac{m g}{M} & 0 & 0 & 0
\end{bmatrix}
\delta x
+
\begin{bmatrix}
0 \\[4pt]
-\frac{1}{M l} \\[4pt]
0 \\[4pt]
\frac{1}{M}
\end{bmatrix}
\delta u
\]

Model liniowy wykorzystywany jest przez:
\begin{itemize}
    \item Regulator LQR (wyznaczanie optymalnego wzmocnienia)
    \item Regulator MPC (predykcja zachowania układu)
\end{itemize}
\end{frame}

%======================
% Section 3: Regulatory
%======================
\section{Regulatory}

\begin{frame}{Regulator PD–PD (kaskadowy)}
\begin{columns}
\begin{column}{0.55\textwidth}
\small
Układ kaskadowy z dwóch regulatorów PD:

\textbf{Pętla wewnętrzna (kąt):}
\[
u_\theta = K_{p\theta} e_{\theta} + K_{d\theta} \dot{e}_{\theta}
\]

\textbf{Pętla zewnętrzna (pozycja):}
\[
\theta_{ref} = K_{px} e_{x} + K_{dx} \dot{e}_{x}
\]

\vspace{0.5em}
\textcolor{Success}{\textbf{+}} Prosta implementacja, niski koszt obliczeniowy\\
\textcolor{Warning}{\textbf{--}} Wymaga starannego strojenia obu pętli
\end{column}

\begin{column}{0.45\textwidth}
\centering
\includegraphics[width=\linewidth]{pd_pd.png}
\end{column}
\end{columns}
\end{frame}

\begin{frame}{Regulator PD–LQR (hybrydowy)}
\begin{columns}
\begin{column}{0.55\textwidth}
\small
Połączenie LQR ze sprzężeniem od pozycji (PD):

\textbf{Optymalne wzmocnienie LQR:}
\[
K = R^{-1} B^\top P
\]
gdzie $P$ z równania Riccatiego:
\[
A^\top P + P A - P B R^{-1} B^\top P + Q = 0
\]

\textbf{Sterowanie hybrydowe:}
\[
u = -K x + K_{px} e_x + K_{dx} \dot{e}_x
\]

\textcolor{Success}{\textbf{+}} Optymalna stabilizacja kąta\\
\textcolor{Success}{\textbf{+}} Dobra kontrola pozycji
\end{column}

\begin{column}{0.45\textwidth}
\centering
\includegraphics[width=\linewidth]{pid_lqr.png}
\end{column}
\end{columns}
\end{frame}

\begin{frame}{Regulator MPC (predykcyjny)}
\begin{columns}
\begin{column}{0.6\textwidth}
\small
Minimalizacja funkcji kosztu na horyzoncie predykcji:
\[
J = \sum_{k=1}^{N} (x_{k} - x_{ref})^\top Q (x_{k} - x_{ref}) + R \sum_{k=1}^{N_u} (\Delta u_k)^2
\]

\textbf{Parametry:}
\begin{itemize}
    \item $N = 20$ -- horyzont predykcji
    \item $N_u = 10$ -- horyzont sterowania
    \item $Q$ -- macierz wag błędów stanu
    \item $R$ -- waga kary za zmiany sterowania
\end{itemize}

\textcolor{Success}{\textbf{+}} Jawne uwzględnienie ograniczeń\\
\textcolor{Warning}{\textbf{--}} Wysoki koszt obliczeniowy
\end{column}

\begin{column}{0.4\textwidth}
\centering
\includegraphics[width=0.9\linewidth]{animation.png}
\end{column}
\end{columns}
\end{frame}

\begin{frame}{Regulator MPC-J2 (alternatywna funkcja kosztu)}
\begin{columns}
\begin{column}{0.6\textwidth}
\small
Zmodyfikowana funkcja kosztu:
\[
J = \sum_{k=1}^{N} (x_k - x_{ref})^T Q (x_k - x_{ref}) + r \sum (\Delta u_k)^2 + r_{abs} \sum u_k^2
\]

\textbf{Różnice względem MPC:}
\begin{itemize}
    \item Jawne wagi dla prędkości $\dot\theta$, $\dot x$ w macierzy $Q$
    \item Parametr $r_{abs}$ -- kara za bezwzględną wartość $u$
    \item Bezpośrednia minimalizacja energii sterowania
\end{itemize}

\vspace{0.3em}
\textcolor{Success}{\textbf{+}} Płynniejsze sterowanie\\
\textcolor{Warning}{\textbf{--}} Wrażliwość na dobór $r_{abs}$ przy zakłóceniach
\end{column}

\begin{column}{0.4\textwidth}
\centering
\includegraphics[width=\linewidth]{combined_nominal_mpc_j2_study.png}
\small\textit{Wpływ $r_{abs}$ na sterowanie}
\end{column}
\end{columns}
\end{frame}

\begin{frame}{Regulator Fuzzy-LQR (hybrydowy)}
\begin{columns}
\begin{column}{0.6\textwidth}
\small
\textbf{Prawo sterowania (hybrydowe):}
\[
u(t) = u_{\mathrm{LQR}}(t) + u_{\mathrm{Fuzzy}}(t)
\]

\textbf{Część rozmyta (Takagi-Sugeno):}
\[
u_{\mathrm{Fuzzy}} = \frac{\sum_{i=1}^{M} w_i(\theta) \cdot (-K_i x)}{\sum_{i=1}^{M} w_i(\theta)}
\]
gdzie $w_i(\theta)$ to stopień aktywacji reguły (z funkcji przynależności).

\vspace{0.4em}
\textbf{Koncepcja ,,Gain Scheduling'':}
\begin{itemize}
    \item \textbf{Mały błąd:} $w_i$ aktywuje łagodne reguły (jak LQR).
    \item \textbf{Duży błąd:} $w_i$ aktywuje agresywne reguły (szybki powrót).
\end{itemize}
\end{column}

\begin{column}{0.4\textwidth}
\centering
\includegraphics[width=\linewidth]{fuzzy_membership.png}
\small\textit{Funkcje przynależności $w_i(\theta)$}
\end{column}
\end{columns}
\end{frame}



%======================
% Section 4: Wyniki
%======================
\section{Wyniki eksperymentów}

\begin{frame}{Metodyka badań}
\textbf{Scenariusze testowe:}
\begin{enumerate}
    \item \textbf{Warunki nominalne} -- stabilizacja z wychylenia początkowego $\theta_0 = 0.05$ rad
    \item \textbf{Zakłócenia zewnętrzne} -- losowa siła działająca na wahadło
    \item \textbf{Zmiana parametrów} -- +10\% masy wahadła (test odporności)
\end{enumerate}

\vspace{0.5em}
\textbf{Wskaźniki jakości:}
\begin{columns}
\begin{column}{0.5\textwidth}
\begin{itemize}
    \item $MSE_\theta$, $MSE_x$ -- błędy średniokwadratowe
    \item $IAE_\theta$ -- całkowy błąd bezwzględny
    \item $T_s$ -- czas regulacji
\end{itemize}
\end{column}
\begin{column}{0.5\textwidth}
\begin{itemize}
    \item $E_u$ -- energia sterowania
    \item $Max|\theta|$, $Max|x|$ -- maksymalne wychylenia
    \item $t_{comp}$ -- czas obliczeń
\end{itemize}
\end{column}
\end{columns}
\end{frame}

\begin{frame}{Warunki nominalne -- regulatory klasyczne}
\begin{columns}[c]
    \begin{column}{0.5\textwidth}
        \centering
        \includegraphics[width=\linewidth]{combined_nominal_classical.png}
        \small\textit{Przebieg kąta $\theta$}
    \end{column}
    \begin{column}{0.5\textwidth}
        \centering
        \includegraphics[width=\linewidth]{combined_nominal_pos_classical.png}
        \small\textit{Przebieg pozycji $x$}
    \end{column}
\end{columns}

\vspace{0.3em}
\small
\textbf{Obserwacje:} PD-LQR: $T_s = 0.2$ s (33\% szybszy od PD), ale wyższa energia ($E_u = 1.48$ vs $0.85$)
\end{frame}

\begin{frame}{Warunki nominalne -- regulatory zaawansowane}
\begin{columns}[c]
    \begin{column}{0.5\textwidth}
        \centering
        \includegraphics[width=\linewidth]{combined_nominal_advanced.png}
        \small\textit{Przebieg kąta $\theta$}
    \end{column}
    \begin{column}{0.5\textwidth}
        \centering
        \includegraphics[width=\linewidth]{combined_nominal_pos_advanced.png}
        \small\textit{Przebieg pozycji $x$}
    \end{column}
\end{columns}

\vspace{0.3em}
\small
\textbf{Obserwacje:} MPC: najniższa energia ($E_u = 0.56$), płynne sterowanie.\\
Fuzzy-LQR: szybka stabilizacja, ale wysoka energia ($E_u = 2.75$, +394\% vs MPC)
\end{frame}

\begin{frame}{Zakłócenia zewnętrzne -- model}
\begin{columns}
\begin{column}{0.55\textwidth}
\small
\textbf{Model zakłócenia:}
\[
w_k \sim \mathcal{N}\!\left(0,\;\sigma^2\right),\quad
\sigma^2=\frac{power}{T_s}
\]

\textbf{Wygładzanie:}
\[
F_{w,k}=\frac{1}{s}\sum_{i=0}^{s-1} w_{k-i}
\]

Symuluje losowe zakłócenia zewnętrzne (np. podmuchy wiatru) działające bezpośrednio na wahadło.
\end{column}

\begin{column}{0.45\textwidth}
\centering
\includegraphics[width=\linewidth]{wind_signal.png}
\end{column}
\end{columns}
\end{frame}

\begin{frame}{Zakłócenia -- regulatory klasyczne}
\begin{columns}[c]
    \begin{column}{0.5\textwidth}
        \centering
        \includegraphics[width=\linewidth]{combined_wind_classical.png}
        \small\textit{Przebieg kąta $\theta$}
    \end{column}
    \begin{column}{0.5\textwidth}
        \centering
        \includegraphics[width=\linewidth]{combined_wind_pos_classical.png}
        \small\textit{Dryf pozycji $x$}
    \end{column}
\end{columns}

\vspace{0.3em}
\small
\textbf{Obserwacje:} PD-LQR lepiej stabilizuje kąt ($Max|\theta| = 0.060$ vs $0.068$ rad) \\
i redukuje dryf ($Max|x| = 0.22$ vs $0.26$ m) przy niższej energii ($E_u = 11.7$ vs $15.4$)
\end{frame}

\begin{frame}{Zakłócenia -- regulatory zaawansowane}
\begin{columns}[c]
    \begin{column}{0.5\textwidth}
        \centering
        \includegraphics[width=\linewidth]{combined_wind_advanced.png}
        \small\textit{Przebieg kąta $\theta$}
    \end{column}
    \begin{column}{0.5\textwidth}
        \centering
        \includegraphics[width=\linewidth]{combined_wind_pos_advanced.png}
        \small\textit{Dryf pozycji $x$}
    \end{column}
\end{columns}

\vspace{0.3em}
\small
\textbf{Obserwacje:} Fuzzy-LQR: najlepsza precyzja ($Max|\theta| = 0.05$ rad), ale $E_u = 25.4$\\
MPC: największy dryf ($Max|x| = 0.41$ m), ale kontrolowana energia ($E_u = 12.4$)
\end{frame}

\begin{frame}{Zestawienie wyników -- warunki nominalne}
\centering
\small
\begin{tabular}{|l|c|c|c|c|c|}
    \hline
    \textbf{Wskaźnik} & \textbf{PD} & \textbf{PD-LQR} & \textbf{MPC} & \textbf{MPC-J2} & \textbf{Fuzzy} \\ \hline
    $MSE_\theta \times 10^{5}$ & 4.66 & \textcolor{Success}{\textbf{3.65}} & 6.19 & 5.15 & 9.32 \\ \hline
    $T_{s,\theta}$ [s] & 0.30 & \textcolor{Success}{\textbf{0.20}} & 1.20 & 0.30 & 0.80 \\ \hline
    $T_{s,x}$ [s] & 1.20 & 2.20 & 2.20 & 1.10 & \textcolor{Success}{\textbf{1.00}} \\ \hline
    $E_{u}$ & 0.85 & 1.48 & \textcolor{Success}{\textbf{0.56}} & 0.59 & 2.75 \\ \hline
\end{tabular}

\vspace{1em}
\textbf{Wnioski:}
\begin{itemize}
    \item PD-LQR: najszybsza stabilizacja kąta
    \item MPC: najniższe zużycie energii (o 62\% mniej niż Fuzzy-LQR)
    \item Fuzzy-LQR: najszybsza stabilizacja pozycji
\end{itemize}
\end{frame}

\begin{frame}{Zestawienie wyników -- zakłócenia zewnętrzne}
\centering
\small
\begin{tabular}{|l|c|c|c|c|c|}
    \hline
    \textbf{Wskaźnik} & \textbf{PD} & \textbf{PD-LQR} & \textbf{MPC} & \textbf{MPC-J2} & \textbf{Fuzzy} \\ \hline
    $MSE_\theta \times 10^{4}$ & 6.01 & 4.44 & 5.78 & 6.59 & \textcolor{Success}{\textbf{3.57}} \\ \hline
    $Max|\theta|$ [rad] & 0.068 & 0.060 & 0.062 & 0.065 & \textcolor{Success}{\textbf{0.051}} \\ \hline
    $Max|x|$ [m] & 0.26 & \textcolor{Success}{\textbf{0.22}} & 0.41 & 0.26 & 0.23 \\ \hline
    $E_{u}$ & 15.4 & \textcolor{Success}{\textbf{11.7}} & 12.4 & 17.3 & 25.4 \\ \hline
\end{tabular}

\vspace{1em}
\textbf{Wnioski:}
\begin{itemize}
    \item Fuzzy-LQR: najlepsza precyzja kątowa (minimalne wychylenia)
    \item PD-LQR: najlepsza ekonomia przy zakłóceniach, lepszy od MPC
    \item MPC: największy dryf pozycji (oszczędza energię kosztem pozycji)
\end{itemize}
\end{frame}

%======================
% Section 5: Odporność
%======================
\section{Analiza odporności}

\begin{frame}{Test odporności -- zmiana parametrów modelu}
\begin{columns}[c]
    \begin{column}{0.5\textwidth}
        \centering
        \includegraphics[width=\linewidth]{robustness_theta.png}
        \small\textit{Przebieg kąta (+10\% masy wahadła)}
    \end{column}
    \begin{column}{0.5\textwidth}
        \centering
        \includegraphics[width=\linewidth]{robustness_x.png}
        \small\textit{Przebieg pozycji (+10\% masy wahadła)}
    \end{column}
\end{columns}

\vspace{0.3em}
\small
\textbf{Obserwacje:} Wszystkie regulatory zachowały stabilność. Regulatory MPC wykazują \\
największą wrażliwość (model wewnętrzny różni się od rzeczywistego obiektu).
\end{frame}

\begin{frame}{Analiza wrażliwości na zmianę masy}
\begin{columns}
\begin{column}{0.55\textwidth}
\centering
\includegraphics[width=\linewidth]{robustness_sensitivity.png}
\end{column}
\begin{column}{0.45\textwidth}
\small
\textbf{Zakres testów:}\\
$-75\%$ do $+200\%$ masy nominalnej

\vspace{0.5em}
\textbf{Wnioski:}
\begin{itemize}
    \item Regulatory klasyczne (PD, LQR) -- płaska charakterystyka, niska wrażliwość
    \item MPC -- najwyższe $IAE_\theta$, wrażliwość na błąd modelu
    \item Wszystkie regulatory stabilne w całym zakresie
\end{itemize}
\end{column}
\end{columns}
\end{frame}

\begin{frame}{Złożoność obliczeniowa}
\centering
\begin{tabular}{|l|c|c|}
    \hline
    \textbf{Regulator} & \textbf{Czas [ms]} & \textbf{Względem PD} \\ \hline
    PD-PD & $< 0.01$ & $1\times$ \\ \hline
    PD-LQR & $0.02$ & $2\times$ \\ \hline
    Fuzzy-LQR & $0.05$ & $5\times$ \\ \hline
    MPC & $2.5$ & $250\times$ \\ \hline
    MPC-J2 & $2.8$ & $280\times$ \\ \hline
\end{tabular}

\vspace{1em}
\begin{itemize}
    \item Regulatory klasyczne i rozmyte: czas rzędu mikrosekund
    \item MPC: $\sim$2.5 ms (rozwiązywanie optymalizacji w każdym kroku)
    \item Wszystkie regulatory mogą pracować w czasie rzeczywistym ($\Delta t = 100$ ms)
\end{itemize}
\end{frame}

%======================
% Section 6: Podsumowanie
%======================
\section{Podsumowanie}

\begin{frame}{Główne wnioski}
\begin{enumerate}
    \item \textbf{Brak uniwersalnego regulatora} -- każdy ma swoje zalety i wady
    
    \item \textbf{Najwyższa precyzja: Fuzzy-LQR}
    \begin{itemize}
        \item Minimalne wychylenia kątowe ($Max|\theta| = 0.05$ rad)
        \item Wysoki koszt energetyczny ($E_u = 25.4$)
    \end{itemize}
    
    \item \textbf{Najlepsza ekonomia: MPC}
    \begin{itemize}
        \item Najniższa energia w warunkach nominalnych ($E_u = 0.56$)
        \item Płynne, przewidywalne sterowanie
    \end{itemize}
    
    \item \textbf{Uniwersalność: PD-LQR}
    \begin{itemize}
        \item Lepszy od MPC przy zakłóceniach (pozycja + energia)
        \item Niska złożoność obliczeniowa
    \end{itemize}
    
    \item \textbf{Wrażliwość na funkcję kosztu} -- MPC-J2 pokazuje znaczenie doboru wag
\end{enumerate}
\end{frame}

\begin{frame}{Realizacja celów pracy}
\begin{block}{Zrealizowane cele:}
\begin{itemize}
    \item[\checkmark] Środowisko symulacyjne w Pythonie
    \item[\checkmark] 5 zaimplementowanych regulatorów z optymalizacją parametrów
    \item[\checkmark] Wielokryterialna analiza porównawcza
    \item[\checkmark] Testy odporności na zakłócenia i zmianę parametrów
    \item[\checkmark] Kompleksowa dokumentacja wyników
\end{itemize}
\end{block}

\vspace{0.5em}
\textbf{Kierunki rozwoju:}
\begin{itemize}
    \item Algorytm swing-up (wprowadzanie z pozycji dolnej)
    \item Weryfikacja na stanowisku laboratoryjnym
    \item Porównanie z metodami uczenia ze wzmocnieniem
\end{itemize}
\end{frame}

\begin{frame}{Dziękuję za uwagę}
\centering
\vspace{2em}
{\Large\textbf{Pytania?}}

\vspace{2em}
\begin{tabular}{ll}
\textbf{Autor:} & Adam Sokołowski \\
\textbf{Opiekun:} & mgr inż. Robert Nebeluk \\
\textbf{Kierunek:} & Automatyka i Robotyka \\
\end{tabular}
\end{frame}

% Title slide again
\TitleSlide

\end{document}

